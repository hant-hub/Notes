\documentclass{report}
\usepackage[tmargin=2cm, rmargin=1in, lmargin=1in,margin=0.85in,bmargin=2cm,footskip=.2in]{geometry}
\usepackage{amsmath,amsfonts,amsthm,amssymb,mathtools}
\usepackage{enumitem}
\usepackage[]{mdframed}
\usepackage{tikz}

\title{\Huge{CRES 10}\\Day 2 Notes}
\author{\huge{Elijah Hantman}}
\date{}

\begin{document}
\maketitle
\newpage

{\huge Current Events \& Review}
\begin{mdframed}
    \begin{description}
        \item Israel began a ground invasion of Lebanon.
            Two ongoing wars, Gaza, and Lebanon. Over a thousand
            Lebanese have been killed in the last week, and thousands
            more have been injured.

            Israel's war efforts have been backed by the US.
            Every year for the past decade, the US has sent
            \$3.8 billion dollars to aid Israel. Earlier this
            year \$14 billion was sent, and last week an
            additional \$8.7 billion was added.

            The US pushes for the legitamacy of Israel as a
            country. This provides international justification
            and support for the Israeli government.
        \item {\large \textbf{Context Matters}}\\
            Context tells us what order and what arrangement of
            facts makes sense. The context is what stories are
            told, and whether those stories conform to, and/or
            predict reality.

            Isreali voices are far more common than Palestinian
            voices in mainstream media.

        \item Long-term war in the Democratic Republic of Congo.
            Congo is a major supplier of tin, tantalum, tungsten,
            cobalt, and gold. Vital elements in electronics of
            all kinds, batteries, integrated chips, etc.

            Global supply chains often rely on exploitation in
            countries like The Congo. Belgian colonization devastated the Congo, extracting
            massive amounts of resources and lives. A century
            of colonization and murder has created the
            circumstances for the current conflict.

            Despite having large amounts of valuable resources
            Africa and the Congo specifically are poor due to
            centuries of looting.

            The origin of African impoverishment also explains
            European wealth. Hannah Arendt famously argued that
            the technology and ideology that caused the 
            Holocaust originated from Colonialism and Imperialism.

            \begin{center}
            \fbox{Class note: The Holocaust was also inspired by
            US racism.}
            \end{center}

            \begin{center}
                \begin{mdframed}
                    Class note: 
                    \\
                    US racism likes to paint itself as innocent.
                    Defending whites, etc.

                    Also 9/11, started as a tragedy, related to the
                    Gulf War, quickly mutated into a nationalist
                    justification for invading Middle Eastern nations like
                    Afghanistan, Iraq, Iran, etc.

                \end{mdframed}
            \end{center}
            Pro Tips:
            \begin{enumerate}
                \item Critically engage with your own
                    responses.\\(Is my response to this
                    getting in the way of my learning?)
                \item Scale up. Scale Down.\\
                    (I didn't invent this problem, and I
                    don't believe it should exist. But is it
                    possible that I benefit from it? Or that
                    it shapes my life?)
                \item Read, re-read, and ask questions!\\
                    (Is my embarrassment at feeling ignorant
                    getting in the way of my curiosity?
                    Is my curiosity relevant to the subject
                    matter that we're discussing in class?)
            \end{enumerate}
            

    \end{description}

    \vspace{30}

\end{mdframed}
\pagebreak

{\huge What are our resistances to learning about race,
racism, and white supremacy?}
\begin{mdframed}
    {\large
    Two Definitions of Racism}
    
    \begin{description}
        \item "[T]he need to ascribe bone-deep features to
            people and then humiliate, reduce, and destroy
            them"\\
            - Coates, Between the World and Me, p.7

            \vspace{20}

            You are taking the things you see and transforming
            it into an invisible context. Something that is
            unchangeable, {\it essential}.
            
            
            \begin{center}
                \begin{mdframed}
                    Class Node: Not humiliating. Things that
                    are bone deep are not something to be
                    ashamed of.
                \end{mdframed}
            \end{center}

            For Coates, the ascription of bone-deep features
            is violence.
            
        \item "Racism, specifically, is the state sanctioned
            or extralegal production and exploitation of
            group-differentiated vulnerability to premature
            death"\\
            - Ruth Wilson Gilmore, Golden Gulag

            State sanctioned as in promoted and done by the
            state.\\
            Extra legal as in, unrelated or outside the
            purview of the law. This are things that the
            law, does not directly speak on.

            Production: manufacture or creation, requiring
            labor, materials, logistics, and larger systems.
            Production is organized, large scale, and requies
            mass cooperation.

            Exploitation: usage, or extraction of value from
            a given subject or object. Taking advantage
            of people for benefit.


            Do These definitions differ from other definitions
            of racism that you've encountered? If so, How?

    \end{description}

\end{mdframed}



\end{document}
