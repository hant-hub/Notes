\documentclass{report}
\usepackage[tmargin=2cm, rmargin=1in, lmargin=1in,margin=0.85in,bmargin=2cm,footskip=.2in]{geometry}
\usepackage{amsmath,amsfonts,amsthm,amssymb,mathtools}
\usepackage{enumitem}
\usepackage[]{mdframed}
\usepackage{tikz}

\title{\Huge{CRES 10}\\Policing Testimonial Part 3}
\author{\huge{Elijah Hantman}}
\date{}

\begin{document}
\maketitle
\newpage

\begin{center}
   \large Preface 
\end{center}
\begin{mdframed}
    In my original piece I focused on a specific kid I
    had a relationship with. I focused on the ways
    others were changing. In my new response I have tried
    to build a narrative which includes me as someone not
    fundamentally different from the kids who became cops.
    I have kept much of the same framing and message because
    I think it is a good message. People often describe their
    relationship with police in terms of privilege, oppression,
    brutality, and specific incidents. I wanted to show how
    many have a relationship with policing whose nature is that
    of \textit{recruitment}. And I also wanted to highlight
    how policing does not purely benefit the privileged and
    dehumanizes and separates us in the pursuit of \textit{recruitment}.
    In pursuit of that I tried to draw on more examples,
    the kids who ended up being police weren't always "the
    bad kids" as I said in my first testimonial, but some
    of them used to be the kinds of people I would be friends
    with. And as I reviewed and expanded my testimony I found 
    that the influence of policing was larger and more 
    omnipresent than I originally thought, so that is also 
    included as well.
\end{mdframed}
\begin{center}
    \large Testimonial 
\end{center}


My relationship with policing is not an uncommon one.
I grew up in a very white community and with my dad's side
being white I am often perceived as white by my peers.
Because of that I grew up in an environment where police
were not merely tolerated, but actively admired by many
of the kids around me. 
Several times a year the sheriff's department would 
come by and do shows with their dogs, or talk about drugs,
or just talk about being a cop. 
In sixth grade I ended up moving schools to a smaller and
more rural school. Here many of the parents were conservative
christians. There was a church right outside the school where
many parents would attend with their kids. When I got to highschool
many of the people around me at school were conservative Christians.
My teachers had kids in the military, as did my swim coach,
I had friends planning to join the military as well, and some
wanted to become cops. 

For the longest time, surrounded by all of this, I was
tentatively positive about police. The culture of my family
was such that I never really wanted to be a cop or a soldier
but being a white boy in a white community I was bombarded
on all sides by images of cops I was invited to project
myself onto. 

I would spend a day at school and see a school resource
officer. I would sometimes have assemblies with cops
telling us that the word "cop" was offensive and slang used
by criminals. At home shows geared towards my demographic
featured cops and soldiers as heroes, and aspirational figures.

I eventually began to be turned off the police by several
things. The first was the kids I knew who wanted to be cops or
soldiers. My brother came out as trans and I went out of my
way to try and learn about it to be a good brother. As I 
learned more and tried to be more caring I started to notice 
the ways in which the kids who wanted to be cops and soldiers 
were becoming cruel. People who had seemed so nice began to say
things which were sexist and bigoted. One of my best friends in
middle school started saying that women belong in the kitchen
and I couldn't believe that he would say that. 

I eventually found myself in english class arguing about how
trans people don't secretly control everything. I argued about
how systemic factors in poverty were real and not everything
was individual choice. The person I was arguing with wanted
to be a cop, and as far as I am aware might be one right now.

As I met more LGBTQ people, and made friends with people
with different perspectives I began to feel how omnipresent
police are even for people with ostensibly no relationship to
police. This omnipresence has changed the lives of everyone I
know. My sibling hates most of their school life because of the
casual bigotry he experiences, the kids I was once friends with
are now full blown conservatives shaped into the kind of 
uncritically fearful people who love police. And I find myself
increasingly only friends with the kinds of people who can
accept my family for who they are, and refuse to engage in the
bigotry and cruelty of policing. 

The masses of cops and prisons and wars have to draw their
manpower from somewhere. Nobody gets to escape the influence
of policing, even the ones who are supposed to benefit
still find their entire lives and social network shaped by
the implied prescence of policing. 

Now that I am older and more critical of my relationship with
policing than ever I am worried. The ideological package of police
has taken a lot of friendships from me, and I have young cousins
who are in communities just like the one I was in. Not all of them
are lucky enough to have friends or family who can push them
to be critical of police. Even now the existence of police forces
me to not only be critical of my own thoughts and influences,
but the ways relationships and family might be made hostile.

In elementry school I had a friend who's dad was a cop.
His dad was not a good person, he would berate his son for
trying on earings, and didn't care if he would lash out as long
as he wasn't "gay". In turn I eventually couldn't stay friends
with him, he would act out and hit me. In middle school I had
a friend who's dad was a church pastor. By the time we got to
high school he wanted to be a soldier and began saying to
girls who were ostensibly his friends that women belonged in the
kitchen. 

When I was a kid and I looked at police I felt mildly positive.
They were just professionals out to help people like fire fighters
or paramedics. Now I feel anger, I hate how I can't even be around
people who used to be my friends because of the machismo 
policing feeds off of. I hate how I have to worry about my
non-white friends and family. I hate how so many familiar faces
are now cops, and have dedicated their lives to opposing
the things that would make the people I care about safer and
happier.



\end{document}
