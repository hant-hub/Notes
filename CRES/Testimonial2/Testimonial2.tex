\documentclass{report}
\usepackage[tmargin=2cm, rmargin=1in, lmargin=1in,margin=0.85in,bmargin=2cm,footskip=.2in]{geometry}
\usepackage{amsmath,amsfonts,amsthm,amssymb,mathtools}
\usepackage{enumitem}
\usepackage[]{mdframed}
\usepackage{tikz}

\title{\Huge{CRES 10}\\Policing Testimonial 2}
\author{\huge{Elijah Hantman}}
\date{}

\begin{document}
\maketitle
\newpage

Over the course we have touched on many different ways
race and policing intersect. My current understanding of
these concepts places them within a context that is much
larger and more complicated than any single individual or
group.

Race as a concept and ideology has been deconstructed
during this course, as a social layer on top of the physical
reality. Race is not tied to biology or history or appearance
so much as it is a social othering, the relegation of some to
the bottom of society. Personally I understand the cause of 
this to not be any single person or group hating another, but
of many groups and people taking advantage of previous
othering.

When colonizing Africa the rhetoric of civilizing and savagery
worked well in convincing people to visit upon others horrific
crimes. When someone needs cheap labor it is easy to piggy back
off of the previous rhetoric to build an idea of a lazy
African who can only be civilized through hard labor. When trying
to win the votes of the Irish it was useful to align yourself
with them, and against the lazy savage Blacks. In that way I
see race as the product of many cruel, apathetic decisions.
Each step was wrong but each action made the following ones
easier until Race and racial identity was a bedrock upon which
so much stood. 

In this framing the role of police is that of control. It is the
means by which the rhetorical framework of Race is replicated.
Black people must be criminal, so over policing and targeted
laws ensure Black people fill up prisons. Black people are
lazy and poor, so regulations like redlining and sharecropping
exist which all but garuntee it. And the role of police in
all these systems is to be the physical violence which makes
the lines solid and meaningful.

As mentioned during lecture, one of the main benefits of whiteness
was the suppression and subjugation of minorities. Its not
necessarily about killing the non-white, but about exercising
power, and fetishing that exercise. That phenomenon is what
I experienced in the first testimonial, I was watching the
people around me be inducted into the worship and production
of Race and legitimacy. Through that lens you can see how 
Racism and Policing perpetuate themselves. The system of 
policing cultivates the ideologies policing is meant to 
enforce. And those who are fully inducted end up joining 
the police. 

When I was in school I was watching how kids are taught
about race, both officially and by culture at large. The
Cis-het White men, were educated into embracing race. They
ended up hating the non-straight, the non-white, the
non-cis people of the world. Some of them even ended up
believing that trans people shouldn't get healthcare even
as their friends came out as trans.
As children grew into adults they were seperated by ideology
and social construction to more easily fit into society.
They are made into vessels for the reproduction of Race
and Policing as constructs.

Each of these social constructions feed into each other.
When one of those kids I knew who idolized the police
saw police brutality, individualism would assuage his worries.
Homophobia explained why the world didn't seem all that good
despite all the people they idolized holding all the power,
classism explained why some should have so much but others
should have so little. Misogyny lets them square away the
questions about how women are represented to them. 

To them, policing is about comfort and immersion, and that 
comfort makes them into part of the system of policing. 
They are alienated from everything and everybody they 
would need to hate in order to become the arm of the law 
and enforce the systems which shaped them. And the worrying 
thing is how efficiently and pervasive the systems shaping 
people's beliefs and thoughts are.

I am lucky, I was more like those kids than not. The difference
was circumstantial.
Much of my family is non-white, and a few are scholars
of critical theory. Despite all that it was only the luck
of randomly finding resources that presented critical
ideas in a way I was receptive to. Because of this my 
relationship to policing is mostly a relationship to the kids 
who grew up to become Police rather than idolizing the Police
myself.

\end{document}
