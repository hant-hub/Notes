\documentclass{report}
\usepackage[tmargin=2cm, rmargin=1in, lmargin=1in,margin=0.85in,bmargin=2cm,footskip=.2in]{geometry}
\usepackage{amsmath,amsfonts,amsthm,amssymb,mathtools}
\usepackage{enumitem}
\usepackage[]{mdframed}
\usepackage{tikz}
\usepackage{minibox}

\title{\Huge{CRES 10}\\Day 9 Notes}
\author{\huge{Elijah Hantman}}
\date{}

\begin{document}
\maketitle
\newpage

\begin{itemize}
    \item Structural vs short term genocide
    \item Law vs Justice
        \begin{mdframed}
            Where does the langauge we use
            come from? How is it used and
            why is it deployed?

            Why do certain words stick more to
            certain contexts than others? What
            motivates choices in langauge and
            associations.
        \end{mdframed}
    \item History of Genocide
        \begin{enumerate}
            \item Most common genocide comes from
                UN. Goal was to prevent another
                Holocaust.
            \item Goal was to create a criminal
                context where intervention is
                justifiable and clear.
                \begin{center}
                    \minibox[frame]{
                        Signatory implies\\
                        agreement that a\\
                        convention should\\
                        exist. A ratifier\\
                        agrees to this\\
                        specific convention.
                    }
                \end{center}
            \item Anti colonial movements, end of many
                empires. US was looking to be an
                imperial power so it only ratified
                the convention under the convention
                that the US was immune to proscecution.

                \begin{mdframed}
                    The US participated in many
                    proxy wars where the military
                    committed acts which were at least
                    adjacent to genocide.

                    \vspace{15}

                    The US had in the past and at the
                    time still engaged in practices
                    which would fall under genocide.
                    The US did not wish to be held
                    accountable for these past and
                    ongoing crimes.
                \end{mdframed}
        \end{enumerate}
    \item International law is about protecting power,
        as much as it is about challenging power.

    \item One of the colonial powers that came
        out of the UN was Israel. Countries who
        practiced antisemitism on a systematic level
        for centuries applauded themselves for being
        better than the Nazis.

        They also found themselves coming up with a
        place to dump the Jews they expulsed from their
        own countries.

    \item UN definition of Genocide comes from the
        aftershock of a sudden genocide.

    \item Structural genocide is about legitimacy. It
        is often clad in claims of improvement,
        spreading civilization, etc.

        Improvement weaponized as a means of ordaining
        the destruction of culture and the conquest
        of land. Production as a measure by which
        the indigenous is determined to be undesireable.

        \pagebreak
    \item Structural genocides always create resistance
        and that resistance is used to further
        justify the genocide.

        \begin{mdframed}
            Foster care, and boarding schools are
            two institutions of genocide.

            Foster Care was similar to slavery,
            more so than the boarding schooslFoster care, and boarding schools are
            two institutions of genocide.

            Foster Care was similar to slavery,
            more so than the boarding schools
        \end{mdframed}
        \begin{mdframed}
            Some examples include
            \begin{itemize}
                \item HAMAS human sheilds
                \item Resisting arrest
                \item Rioting
                \item looting
                \item Follow the system
                \item etc.
            \end{itemize}
        \end{mdframed}
    \item All successfull resistance includes both
        violent and non-violent tactics.

        \begin{mdframed}
            MLK carried and owned many guns for
            self defense. The Black Panthers carried
            weapons but also engaged in nonviolent
            community action programs.
        \end{mdframed}

    \item Structural Genocide manifests as a systematic
        statistical difference in conditions which
        produce more suffering and earlier deaths.

        Often colonial states will blame the victims
        of structural genocide for their own poor
        condition, rather than ecnomic exploitation,
        the massive stealing of resources, discrimination,
        less opportunity, etc.

    \item Wealth is about time and the ability to safely
        explore and make mistakes. Nothing about that
        should be exclusive to a few. Structural
        genocide is about the ways in which people are
        deprived of safety, of exploration, of time.
        
    \item American Indian Movement fought the US government.
        They engaged in nonviolent action as well as
        violent action, in order to create homes and
        safety for multiple Native American Nations.

    \item Why genocide for only sudden genocide rather than
        structural genocide?
        \begin{mdframed}
            Legal terms are not the same as common
            usage. We can reappropriate the terms of
            law for our own usage.

            The law matters but it is not the only 
            means of change, nor is it the law which
            should decide what we view as justice.
        \end{mdframed}

\end{itemize}


\end{document}
