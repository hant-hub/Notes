\documentclass{report}
\usepackage[tmargin=2cm, rmargin=1in, lmargin=1in,margin=0.85in,bmargin=2cm,footskip=.2in]{geometry}
\usepackage{amsmath,amsfonts,amsthm,amssymb,mathtools}
\usepackage{enumitem}
\usepackage[]{mdframed}
\usepackage{tikz}
\usepackage{dirtytalk}

\title{\Huge{CRES 10}\\Day 3 Notes}
\author{\huge{Elijah Hantman}}
\date{}

\begin{document}
\maketitle
\newpage

{\large Why Military not Domestic?}
\begin{enumerate}
    \item Why does it look bad?
        Does it look bad?
    \item Geneva Covnention? Military
        not being used against civillians.
    \item Why do people want tanks?
        
        Terrorism?
        Think it is cool.

    \item Police are already militarized enough.

        Police departments can be highly military.

    \item Civil war era legislation, prevents usage of
        the military on US soil in order to prevent the
        government from putting down Confederacy like
        movements.

        Posse Comitatus Act
\end{enumerate}


Much of US identity is shaped by its fight against
fascism in the Second World War, however several
forms of Fascism were inspired by the US and have
taken root in US culture.

The US can never accept itself as having bad intentions
or core flaws.


\begin{mdframed}
    The US also controls what innocent means.
    The US is an arbiter of International Law, 
    and so shapes the definition of innocent, or polite,
    or upstanding, to benefit itself.

    \vspace{10}

    \say{Americans believe in the reality of 'race' as
    a defined, indubitable feature fo the natural world.
    Racism-the need to ascribe bone-deep features to people and
then humiliate, reduce, and destroy them-inevitably follows from
this inalterable condition. In this way, racism is rendered
as the innocent daughter of Mother Nature, and one is left
to deplore the Middle Passage or the Trail of Tears the way
one deplores the an earthquake, a tornado, or any other phenomenon
that can be case as beyond the handiwork of men.}

    -Coates, Between the World and Me


\end{mdframed}

Do These definitions differ?
\begin{mdframed}
    \begin{enumerate}
        \item delibrately broad.
        \item not intention based
        \item rarely is racism actually focused on material results
        \item Racism is sometimes just never clearly defined
    \end{enumerate}

    Racism in these definitions is irreducible. Racism
    does not come down to hate, it does not come down
    to prejudice, and it cannot be solved with education.

    Racism is defined in terms of effects upon entire groups.

    As a result, racism is not equated with hate or prejudice,
    it is not that those emotions aren't part of systemic
    definitions, its just that those feelings are not the
    defining features of what makes them systemic.

    The problem isn't the personal feelings of people, but the
    distribution and collection of power and wealth.

    This also means White people don't necessarily fix anything
    by befriending nonwhite people. Racism is a mode of production
    and a distribution of power and resources
\end{mdframed}

\begin{mdframed}
    \begin{itemize}
        \item Construction
            Something that is made in a given location
        \item Production
            The process of making something, often involving
            combining elements and materials.
        \item Constructed/Produced $\ne$ not real
            \\
            Things we make are real, but they are arbitrary
            they are more malleable than things we don't
            construct, but they are no less real.
        \item Racialization is the social construction and
            production of race.
            \\
            Construction is focusing on the outcomes of the
            process

            Production focuses on the process itself.
        \item Production involves not just the items themselves
            but all the hidden things. Tools, raw materials,
            shelter, food, the ideas which underly the products
            and tools, etc.

        \item Products then go on to make us, our time and
            the shape of our lives rest upon a foundation
            of tools, objects, ideas, and products.
             
    \end{itemize}
\end{mdframed}

\end{document}
