\documentclass[12pt]{article}
\usepackage{mathptmx}
\usepackage[tmargin=2cm, rmargin=1in, lmargin=1in,margin=0.85in,bmargin=2cm,footskip=.2in]{geometry}
\usepackage{amsmath,amsfonts,amsthm,amssymb,mathtools}
\usepackage{enumitem}
\usepackage[]{mdframed}
\usepackage{tikz}
\usepackage[backend=biber]{biblatex-chicago}
\addbibresource{Final.bib}

\usepackage{setspace}
\doublespacing

\title{\Huge{CRES 10}\\Final}
\author{\huge{Elijah Hantman}}
\date{}

\begin{document}
\maketitle
\newpage


\begin{mdframed}
    In light of the histories and contexts we’ve explored in this class, what do you think racial justice would look like, if it were to be realized? How would society transform? How would day-to-day life change? The point here is not just to think about what would be absent if racial justice were to be achieved, but also to consider what the achievement of racial justice would make present in the world. Would the world look and feel different? In what ways? For whom?
\end{mdframed}

The question of what society should look like is one of the oldest questions in the world. Many ideologies
prescribe specific visions of society, and many critical theories give tools to look at and understand what is
wrong with society but the question of what society ought to look like remains. I am going to argue that a world
with full racial justice would not only be a transformation of social norms but also an economic transformation,
and a transformation of the physical appearance of our society.

The aspect of our society most discussed in CRES 10 has been policing. There is a lot wrong with policing in our current society.
For one, the police are one of the visible and openly violent ways white supremacy exerts pressure on society. 
Police are known to kill unarmed Black people with little consequences. In Between the World and Me, Ta-Nehisi Coates\textcite{Coates} shares
a story about the killing of a friend of his during his time at Howard University. And later his meeting with
a Mother who’s kid had been murdered by a police officer in Chicago.
In addition, policing acts as a means of enforcing colonialism. In lecture\textcite{lecture} we’ve discussed police suppression of
Indigenous movements. In addition, according to Stuart Schraeder in Badges Without Borders\textcite{Badges} in the aftermath of the
second world war, the United States acted to disseminate its model of policing on a global scale. 
The concept of Policing as we know it is a key ingredient in the production of race and racism. As discussed
in Lecture 4, a part of the production of race is the systemic violence and disadvantage, and policing is one
of the most visible and widespread means of creating systemic violence.

Looking towards the future, what would a better system look like? Some easy first steps would include demilitarization.
A just society is not one which has a use for police armed like an invading army. Other easy changes include civilian 
oversight councils. More extensive changes might include incorporating forms of community regulation that predate police,
like constabulary etc. In addition, many jobs currently done by police would be better served via social workers or
other explicitly non violent specialists.

Policing is something I would consider expected to change in a racially just society, however something not as often
discussed is the ways the physical shape and structure of our cities and settlements would have to change.
Our current model of land usage and building has led to many detrimental side effects. According to
David Wallace-Wells in his book The Uninhabitable Earth\textcite{earth} not only are we on track for increasingly dangerous and extreme
weather events, but less nutritious food, increased disease, massive flooding, and in the worst case scenarios, the most
heavily populated regions on earth will become uninhabitable wastelands too hot for humans to survive.
Even in the short term our cities are architected in a fundamentally hostile manner. During lecture 10\textcite{lecture} we discussed
how one of the greatest injustices is the ways in which racialized peoples have their lives stolen, and one of the ways
this happens is via the negligence of industry. Highways are run through poor neighborhoods, forcing many into the streets
and polluting the homes of those who remain with fumes from cars and trucks. People are forced to live far from their
jobs and take underfunded transit or bike on incredibly dangerous roads. 
Even when unavoidable disasters happen, completely divorced from human actions, such as earthquakes or volcanic eruptions,
the amount of suffering is exacerbated. As discussed by Neil Smith in “There’s No Such Thing as a Natural Disaster”\textcite{smith}, the
things which make hurricanes so deadly, which makes earthquakes and tsunamis kill so many, is not their immense power or
unpredictability. It is possible to build safer places, to respond quickly and effectively to requests for aid, and to
create societal structures to support those affected. A large portion of that suffering is the result of racialization as
much as it is a result of natural processes.

What should we do instead? We could use more sustainable land management strategies. Indigenous peoples have lived on
their land for thousands of years and have developed a myriad of strategies for ensuring they do not over exploit the
land. Zooming out we need ways to hold all nations accountable for the damage they do to the global climate. The United
States will be one of the last places to be affected by the extreme events caused by climate change, yet there is no way
for countries like Honduras, Argentina, etc. to hold the United States to any kind of responsibility.
On a smaller level the physical design of our cities needs to change. Histories of Redlining, and building on top
of Black neighborhoods require active effort to correct. Not to mention walkable spaces, safer pedestrian infrastructure,
public transit, green areas, more breathable buildings, etc. All of these things are in a sense unrelated to Racial
Justice, however they all benefit those who are most disadvantaged, allowing them freedom of movement in spaces which
are beautiful and clean, things denied to the racialized poor.

Racial Justice is not something that can be isolated or separated from society as a whole. White supremacy and racialization
do not only hurt those they suppress but also build physical places into hostile ghettos. To truly build a just society
requires not just changes to our relationship with violence, but also how we travel, the appearance of our homes, and so
much more. Racism and racialization hurts all of us, and white supremacy is the enemy of everyone who wishes for a better
world regardless of race.


\printbibliography
\end{document}
