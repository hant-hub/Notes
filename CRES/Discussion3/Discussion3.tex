\documentclass{report}
\usepackage[tmargin=2cm, rmargin=1in, lmargin=1in,margin=0.85in,bmargin=2cm,footskip=.2in]{geometry}
\usepackage{amsmath,amsfonts,amsthm,amssymb,mathtools}
\usepackage{enumitem}
\usepackage[]{mdframed}
\usepackage{tikz}
                 
\title{\Huge{CRES 10}\\Discussion 3 Notes}
\author{\huge{Elijah Hantman}}
\date{}

\begin{document}
\maketitle
\newpage

\begin{description}
    \item {\large Reflection Testimonial Freewriting}
        \begin{mdframed}
            I think that my personal experience with
            policing is something that feels less sincere
            than what I would prefer from my writing.

            The truth is that Policing is not a large part
            of my life, and its not something I particularly
            care about, outside of the ways in which it opposes
            the freedoms and lives of people I care about. 

            I came into the class having already thought about
            policing, its role and the impacts its had for a
            long time. I already saw racism as a thing which
            requires no personal malice, and I already thought of
            it as something people do, rather than something
            people are.

            I think that the policing testimonial I gave is one which
            I still find myself tending towards. I am in an espeically
            privledged group, people for whom the police rarely attack,
            and people for whom the police recruit from. My testimonial
            is shaped by the ways in which that is true and made visible
            around me. 

            I find myself much more affected by the abhorent actions taken
            by police, more than anything else. It is enraging when yet
            another cop kills an innocent person, and it is maddening
            when there is a seemingly infinite argument about the fact
            that the murders committed by police are not execptions,
            but the point of policing. I find myself often burnt out, too
            many stories of horrific events, all surrounded by seemingly
            gormless centerists unable to square the reality with their
            impression of the police as a necessary and important part of
            society. 

            I think perhaps my testimonial falls victim to what is so easy for
            people like me, apathy. The world always feels like its drowing in
            blood and fire, and injustice and it can be so easy to disconnect
            when you aren't the one bleeding or burning, or suffocating. I want
            to care and help but it feels like an insurmountable task. I'm someone
            who gets nervous talking to cashiers, who can barely make eye contact
            and is scared of meeting new people. I feel utterly powerless to change
            things.

            I know I'm not powerless, I can organize, volunteer, donate. I can
            teach the things I learn here at university so even if I can't share
            the authority of my degree I can share the information. But it is terrifying
            and it feels like thowing a bucket of water into a forest fire.
        \end{mdframed}
    \item {\large Other Responses}
        \begin{itemize}
            \item Only one interaction with police. Extremely friendly and positive,
                strangely privledged.

                Muslim means policing of culture. People are made into police
                for their family and friends.

            \item
                No direct interactions. Father had many interactions.
                Dad tried to give the talk about racism, death of
                Treyvon Martin gave the impetus for getting the real
                details of Racism.

                Not afforded the ability to be ignorant. No escape.

            \item
                Protection of innocence. Only some are allowed to choose
                whether to learn about race and racism. 

            \item
                Assimilating into the US despite the US being the cause
                of fleeing your home.

                Giving up parts of yourself to live in a country that
                hates people like you and inflicted massive amounts
                of violence.

                Speaking a specific way, hiding language, hiding culture.

                Choosing to believe the police do their job. Want to be American
                Want to Assimilate.

                Always being critical, can't be Pakistani, will never be
                seen as American by some.

            \item
                Prisoner of war? Policing and prison abolition is an
                international issue. US police exchange weapons and
                training with police and militaries around the world.

            \item 
                Originally thought it was indirect. Most interactions are through
                others sharing their experiences.

                Parents were racially profiled.

                Mixed identities are strange, you are categorized both internally
                and externally. You are expected to fit in with family.

            \item 
                DC senior year. Culture shock, a lot of white students. So many
                white conservatives want to be cops and soldiers. 
        \end{itemize}
        \pagebreak
    \item {\large Can you Describe Policing?}
        \begin{mdframed}
            I think that fundamentally, policing is the usage of power
            to produce the invincible sovereign and suppress the subjects. 
            \vspace{20}
            \hline
            \vspace{20}
            I think this is a good definition because the core 
            of policing is both the exertion of state violence
            as well as the goal of separating the desireable
            from the undesireable.

            To be expand some inferences, not all policing is
            bad. Safety regulations save lives. Things like
            speed limits, fire safety, are all important.

            I don't exclude this from policing even though that
            is often not what people mean by policing. I think
            this definition is similar to power in that it is
            something that defines a state, and is something that
            is useful if dangerous.

            That doesn't mean however that Police are useful, they
            engage in a specific type of policing which I personally
            believe is harmful to people.
            
        \end{mdframed}
    \item {\large Responses}
        \begin{mdframed}
            \begin{itemize}
                \item US police punish people to "ensure a good and functioning society"
                \item Policing is the act of enforcing a widely accepted belief on others
                    in a manner that results in legal, physical, or finnanical consequences.

                \item Establishing controls over how one presents themselves or acts.
                    I think it is more about control over presentation and action.

                \item Broader perspective

                    System of control and punishment to convince us that
                    a militarized democracy is necessary.

                \item Instrument of colonial violence to maintain the
                    imperial core.

                \item Policing is Cyclical, through capitalism. 

                    Policing is the programmatic and cyclical
                    safeguarding of itself from those it
                    otherizes.

                \item Use of imperial core.

                    Policing is used to protect profit.

                \item Extra nuance: Grandma policing children
                    manners, politeness.
            \end{itemize}
            
        \end{mdframed}
    \item {\large}
\end{description}

\end{document}
