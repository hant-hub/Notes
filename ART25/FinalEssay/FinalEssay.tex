\documentclass[12pt]{report}
\usepackage[tmargin=2cm, rmargin=1in, lmargin=1in,margin=0.85in,bmargin=2cm,footskip=.2in]{geometry}
\usepackage{amsmath,amsfonts,amsthm,amssymb,mathtools}
\usepackage{enumitem}
\usepackage[]{mdframed}
\usepackage{tikz}
\usepackage{minibox}
\usepackage[style=apa]{biblatex}
\renewcommand{\familydefault}{\sfdefault}

\addbibresource{final.bib}

\title{\Huge{Art 25}\\Final Reflection}
\author{\huge{Elijah Hantman}}
\date{}

\linespread{2}

\begin{document}
\maketitle
\newpage

\center{\huge Theosophist Retrospective}

\begin{description}
    I was the lead programmer on Theosophist. In this reflection first I will cover the
    major design choices I personally made and why, then I will discuss regrets and thoughts on the final product.

\item {\large Inspiration}\\

    I was the one in my group who first proposed the topic. I had been watching a series of videos by the youtube 
    educator Kay Lack \parencite{lack} regarding text processing. I had also been watching videos regarding bible 
    scholarship from the youtube channel Paulogia \parencite{paul}. I had been thinking about interpretation, and 
    how we can build automatic systems to understand and react to writing. In my mind I imagined a game where you 
    would receive vague “prophecies” in the form of requirements, and then it would be your job to interpret them into 
    a poem or text of some kind, where you would then receive some kind of vague feedback. The entire point would have been 
    utilizing opaque and unclear systems to replicate the kinds of things people do when interpreting ambiguous or out of 
    context verses in religious texts.

\item {\large Medium}\\

    I do believe that there are many alternative ways to construct a game about poetry, with the right rules there is an
    absolutely beautiful game between two people spinning verses in person or over a zine. However, while I agree 
    with \cite{medium} that the medium is as important as the content when it comes to conveying a message, 
    I am not starting from a singular message. My goal is not to be a non-specific multi-medium artist, my goal at 
    university has been to become a knowledgeable and skilled  computer programmer. So I was never going to choose a 
    different medium, however the medium chosen does have some benefits. Many role playing games can be made from 
    chance and tables of prompts \parencite{rolerules}, however there are few mechanisms more adept and capable of 
    processing rules and delivering tables than a computer program. In addition, there is the added benefit of the 
    program being itself a tool for creating poems directly, if it were on pen and paper the game would simplify to 
    prompts, by using a computer program as a medium the entire activity can be self containted rather than split
    between multiple mediums and modes of activity.

\item {\large Engine}\\

    A key technical decision I made was the choice to not use a commercial game engine. I was inspired to build a 
    game from scratch by the series Handmade Hero on youtube by game programmer Casey Muratori. In addition, 
    rhetorically, I have a strong aversion to the idea of making a game that is unnecessarily esoteric, or temporary. 
    Personally I wanted to build a game that is interesting not just for its circumstances but also its construction 
    and craft as an object of labor. During the talk by Eddo Stern in class he mentioned that he does not release 
    most of his games outside of the exhibit where he displayed it. I do not blame him, however I cannot agree with 
    this style of game making. The idea of making a game just for a single exhibition or demo before throwing it away, 
    or the idea of building a game towards that end is one which repulses me personally. So the decision to take into 
    my hands the entirety of the technical body of the game was also a decision to make something with the obvious 
    and explicit effort of its craft made evident. Rather than splitting the effort between programmers working on 
    engines, myself, programmers working on tools and so on, all the work is in a single project, 3000 lines of code 
    which simply exist and are free for everyone to see and take in.

\item {\large Retrospective}\\

    Overall I think the game was good. I have many qualms with the writing and what ended up being the levels.
    The first level was my least favorite. I proposed the original idea based on a History of Mathematics class
    I was taking, and it would have revolved around the mythical Pythagorus and would poke at the myth that mathematics
    is fundamentally different from rhetoric or separate from culture. However in my desire not to micromanage the game 
    and allow others to take charge the vision I had of a glowing divine manifestation of the fetishization of Greek 
    culture ended up being a weird version of the professor from the movie Whiplash in a strange, inconsistent and 
    unclear environment, only the art and word choice hinting at the ancient Greek inspiration.

    To be clear I am very particular in my taste in art. I find no personal value in nonsense as the Dadaists did
    \parencite{sit}, nor do I connect with many of the tactics of the Situationists. I merely find that in giving
    control over the writing to a group member it has become something which I cannot relate to, something which
    is immensely frustrating in the way it echoes my vision but does not truly match it.

    I still believe that this idea has enormous potential yet to be tapped. I felt similar when looking over Fruit 
    of Law \parencite{fruitlaw}, the game itself is interesting however the format has so many more potential 
    expressions and complexities which could be explored. Near the end of the project I watched a video about Shel 
    Silverstein \parencite{attic}. As a child I also read Shel Silverstein’s books and one in particular has had
    a massive impact on me as a person. \textit{The Missing Piece Meets The Big O} \parencite{bigO} has had a profound
    impact on how I approach relationships, and now that I am investigating poetry it has returned. As I revisited
    these works a new goal for this concept took shape in my mind, I wanted to share this feeling, the joy of prose
    that is perfect in the moment, of reading a work that goes on to change your entire life.

    I do not think the project we made is a good piece of rhetoric. Some of my issues with what we made lie in
    the technical details, lack of polish, pause menus, cross-platform saves, etc. I also take issue with
    the writing, it isn't focused on anything in particular for either level in the game, it just is, and attempts
    to recreate a specific scenario. However, despite that I think there is a kernel of potential that the game
    is just competent enough to show glimpses of.

    I think it does open up a large space of aesthetic and rhetorical works. It could be used to comment on the 
    stress and desperation of real life communication, the search for the perfect words, or it could be used to
    show players how the world around them can be transformed into language, not to mention many other possible 
    rhetorical goals. My favorite level is one where you are asked to apologize for neglecting your partner
    for a week, and there are no good answers. It reminds me of writing to someone searching for the perfect
    words to make everything okay, and I think it is valuable to have experiences like that in art, not just
    in watching characters attempt poetry and fail, but in we ourselves grappling with the stark inadequacy and
    impotence of language. In this case the medium brings a new dimension which enhances the work \parencite{medium}.
    
\end{description}


\printbibliography
\end{document}
