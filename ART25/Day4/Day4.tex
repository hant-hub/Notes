\documentclass{report}
\usepackage[tmargin=2cm, rmargin=1in, lmargin=1in,margin=0.85in,bmargin=2cm,footskip=.2in]{geometry}
\usepackage{amsmath,amsfonts,amsthm,amssymb,mathtools}
\usepackage{enumitem}
\usepackage[]{mdframed}
\usepackage{tikz}
\usepackage{minibox}
\renewcommand{\familydefault}{\sfdefault}

\title{\Huge{Art 25}}
\author{\huge{Elijah Hantman}}
\date{}

\begin{document}
\maketitle
\newpage

\begin{description}
    \item {\large Mechanic is the Message} 
        \begin{mdframed}
            Using Train by Benda Romero
            as an example.
            \begin{itemize}
                \item Play as Train dispatchers
                \item Goal is to maximize number of people
                    in trains as possible.
                \item I think its a holocaust
                    game.
                \item The yellow from the yellow badges
                    from concentration camps.
                \item I was correct, that was kinda 
                    fucked up. I mean I'm not against
                    the game, but I don't know if I like
                    how it was placed into the class. I
                    feel like leaving it unspoken might
                    have been more taseful.
            \end{itemize}
            Unmanded By Molleindustria
            \begin{itemize}
                \item Unmaned drones
                \item Iraq, Afghanistan, etc.
                \item Through thermal cams people are nothing
                    but white specks that you shoot.
            \end{itemize}
        \end{mdframed}
    \item {\large Society of the Spectacle}
        \begin{mdframed}
            \begin{itemize}
                \item May '68
                    \begin{mdframed}
                        Global student action against
                        poor student treatment.
                    \end{mdframed}
                \item Opposition to the Vietnam War (1965-1973)
                    \begin{mdframed}
                        First televised war,
                        we could see what war actually
                        looked like. Knowledge of the war
                        could spread fast enough that
                        the government can't filter
                        it through traditional propaganda
                        instruments.
                    \end{mdframed}
                \item Civil Rights Movement (1954-1968)
            \end{itemize}
        \end{mdframed}
    \item {\large Situationist International}
        \begin{mdframed}
            \begin{itemize}
                \item 1957-1972
                \item Group of radical artists working
                    together.
                \item Anti-capitalist
                \item Libertarian Marxists
                \item Dadaist and Surrealists
                    \begin{mdframed}
                        About what can't be known.
                    \end{mdframed}
            \end{itemize}
            Strategies Used:
            \begin{itemize}
                \item Surrealists:
                    \begin{mdframed}
                        Chance based, Collage, Uncanny
                    \end{mdframed}
                \item Dada:
                    \begin{mdframed}
                        Nonsense, Humor, Trash, Performance
                    \end{mdframed}
                    \begin{center}
                        \minibox[frame]{
                            Bro this is kinda lame\\
                            Its like such a pretentious\\
                            take. The enlightenment claimed\\
                            objectivity but objectivity was\\
                            always a lie. The technologies\\
                            we've made aren't bad, flight\\
                            can be used to pollute and move\\
                            people back and forth fruitlessly.\\
                            It can also be used to put out fires\\
                            to map out the world, to move life\\
                            saving medicine, etc.\\
                            \vspace{10pt}

                            Its like Primitivism but without the\\
                            commitment.
                        }
                    \end{center}
                \item Both:
                    \begin{mdframed}
                        Games, Play, Aesthetic Intervention.
                    \end{mdframed}
            \end{itemize}
            \begin{mdframed}
                This stuff is truly more interesting to talk
                about than see or participate in. The oncept
                is way cooler than anything else.
            \end{mdframed}
            \begin{mdframed}
                I think I know why I am annoyed with this
                assignment. I am like Donald Knuth, or
                Tom7. I am not worried so much about it
                looking good, or seeming right, but about
                something being good, and being right. These
                interventionist things sound radical, and
                seem interesting. But they lack refinement,
                they are not the correct version of themselves.
            \end{mdframed}
            Eddo Stern
            \begin{itemize}
                \item Thinking about violence and embodiement.
                \item Tekken Torture tournament
                \item Get shocked when he takes damage.
                    \begin{mdframed}
                        I like this, but only because it is
                        dumb as hell, and funny. It is like
                        a shit post, but the pretentious
                        justifications are pretty lame. We have
                        literal fighting sports, MMA, boxing, etc.

                        Fighting games aren't violence in the same
                        way, and it seems to me that its a vary
                        immature understanding of games.
                    \end{mdframed}
            \end{itemize}
            {\large Situationist Beliefs}
            \begin{itemize}
                \item Social World increasingly mediated by
                    images
                \item Media saturation deprives people of
                    connective awareness of their own body
                    in society
                \item Media has been used to alienate us into
                    individual self contained units which
                    are easily shifted to the ends of
                    the status quo.
                \item Lived Experience is authentic
                \item Mediated Experience is less valuable
                    and less authentic
                \item They call the rituals of experience
                    Situations
                    \begin{mdframed}
                        Goal is to liberate everyday life
                        back into some notion of adventure
                        and belonging.

                        \vspace{10pt}

                        Moments of life constructed for
                        re-awakening and pursuing authentic
                        desires, not just ones constructed by
                        industrial capitalistic machinery.
                    \end{mdframed}
            \end{itemize}
        \end{mdframed}
\end{description}


\end{document}
