\documentclass{report}
\usepackage[tmargin=2cm, rmargin=1in, lmargin=1in,margin=0.85in,bmargin=2cm,footskip=.2in]{geometry}
\usepackage{amsmath,amsfonts,amsthm,amssymb,mathtools}
\usepackage{enumitem}
\usepackage[]{mdframed}
\usepackage{tikz}
\usepackage{listings}


\definecolor{codegreen}{rgb}{0,0.6,0}
\definecolor{codegray}{rgb}{0.5,0.5,0.5}
\definecolor{codepurple}{rgb}{0.58,0,0.82}
\definecolor{backcolour}{rgb}{0.95,0.95,0.92}

\lstdefinestyle{c_style}{
    language=C,
    backgroundcolor=\color{backcolour},   
    commentstyle=\color{codegreen},
    keywordstyle=\color{magenta},
    numberstyle=\tiny\color{codegray},
    stringstyle=\color{codepurple},
    basicstyle=\ttfamily\footnotesize,
    breakatwhitespace=false,         
    breaklines=true,                 
    captionpos=b,                    
    keepspaces=true,                 
    numbers=left,                    
    numbersep=5pt,                  
    showspaces=false,                
    showstringspaces=false,
    showtabs=false,                  
    tabsize=2
}

\lstdefinestyle{asm_style}{
    language=asm,
    backgroundcolor=\color{backcolour},   
    commentstyle=\color{codegreen},
    keywordstyle=\color{magenta},
    numberstyle=\tiny\color{codegray},
    stringstyle=\color{codepurple},
    basicstyle=\ttfamily\footnotesize,
    breakatwhitespace=false,         
    breaklines=true,                 
    captionpos=b,                    
    keepspaces=true,                 
    numbers=left,                    
    numbersep=5pt,                  
    showspaces=false,                
    showstringspaces=false,
    showtabs=false,                  
    tabsize=2
}




\title{\Huge{CSE 115A - Introduction to Software Engineering}}
\author{\huge{Elijah Hantman}}
\date{}

\begin{document}
\maketitle
\newpage

\begin{mdframed}
    Note: 

    Today is the day the projects are supposed to start,
    ideally we would have a sprint plan and TA meetings/stand up
    times ready to go, but it should be fine as long as we get it done
    within the week.

    This should be the first sprint, but since we ran a little late, we
    should assume less time.

    We also need to make presentations.
\end{mdframed}

{\huge Scrum Best Practices}

\begin{description}
    \item Review

        Roles
        \begin{itemize}
            \item Product Owner, communication between
                customer and team, generally responsible
                for prioritizing user stories and deciding
                what exactly each release will look like.
            \item Scrum Master, organize and fascilitate
                Scrum practices among the team. They also
                serve as an advocate for the team to management.
            \item The Team, the actual experts and people
                who are responsible for creating the final
                product and managing what exactly happens
                in each sprint.
        \end{itemize}
    \item Definitions of Done
        \begin{itemize}
            \item Agile emphasis
                \begin{itemize}
                    \item deliver actual functionality
                    \item avoid waste
                \end{itemize}
            \item Strict definition of progress
                \begin{itemize}
                    \item  user stories
                    \item tasks
                \end{itemize}
            \item Strict definition of completion
                \begin{itemize}
                    \item user stories
                    \item sprint tasks
                \end{itemize}
        \end{itemize}

        Each team creates their own definitions
        which \textbf{must} be applied consistently

        Example
        \begin{itemize}
            \item Code checked into repo
            \item Code reviewed for style and standards
            \item Code reviewed by team member
            \item External/public API documentation
            \item Unit tests
            \item Non-functional tests (perf, memory, usability, etc.)
            \item Regression tests
            \item static code analysis
            \item test coverage
        \end{itemize}

    \item Acceptance Criteria
        \begin{itemize}
            \item Is the thing what was requested?
            \item Criteria
                \begin{itemize}
                    \item objective criteria for user story
                        completion
                    \item Basis for functional testing
                    \item Basis for testing before release
                    \item Basis for BDD/ATDD
                        \begin{itemize}
                            \item BDD: Behavior Driven Development
                            \item ATDD: Acceptance Test Driven Development
                        \end{itemize}
                \end{itemize}
            \item User stories in Product Backlog
                \begin{itemize}
                    \item User story start of conversation
                    \item Product backlog contains all possible things anyone thought to ask for
                \end{itemize}
            \item User stories in Sprint Backlog
                \begin{itemize}
                    \item Commitment to actually create
                    \item should be known what is required to
                        complete and fulfill the user story.
                \end{itemize}
        \end{itemize}

    \item Team Working Agreements
        \begin{itemize}
            \item Logistics
                \begin{itemize}
                    \item Dev environments
                    \item Coding Style/standards
                \end{itemize}
            \item Work Patterns/Process
                \begin{itemize}
                    \item Definitions of done
                    \item collaboration
                \end{itemize}
            \item Product Design Patterns
                \begin{itemize}
                    \item UX/UI look and feel
                    \item architecture
                    \item Error handling.
                \end{itemize}
        \end{itemize}

    \item Avoiding Waste
        \begin{itemize}
            \item Unresolved problems grow
                \begin{itemize}
                    \item defect build up
                    \item Technical Debt: steep interest rate
                        \begin{mdframed}
                             As a side note: This is where
                             a preliminary design would be
                             helpful. It can allow for us to
                             discover the 'correct' way to
                             do something without necessarily
                             taking as much time as just trying
                             to build the right thing from the
                             start.
                        \end{mdframed}
                \end{itemize}
            \item Minimize time between defect insertion and
                correction
                \begin{itemize}
                    \item Goal is to make choices which allow
                        for things to be tested and finalized
                        as soon as possible so that fixes are
                        small and contained to single tasks
                        or only a couple sprints.
                \end{itemize}
        \end{itemize}
    \item Increment Strategies
        \begin{itemize}
            \item Conventional Approach
                \begin{itemize}
                    \item Layer by layer
                    \item Good when there are hard dependencies,
                        if one layer depends on the layer below existing
                        in its entirety.

                        Also usable when the system itself cannot
                        be easily broken apart into functional units,
                        like some hardware problems.
                    \item Example
                        \begin{itemize}
                            \item Start with Frontend
                            \item Then Server
                            \item Then Backend
                        \end{itemize}
                \end{itemize}
                \pagebreak
            \item Alternative Approach
                \begin{itemize}
                    \item Slice by Slice
                    \item Example
                        \begin{itemize}
                            \item Start with Feature 1
                            \item Then Feature 2
                            \item Then Feature 3
                        \end{itemize}
                    \item Each slice includes the full tech
                        stack required to make it work. For
                        a web app the Front end, and Backend
                        would be developed at the same time.
                    \item Benefits to slice by slice is that
                        it creates a testable feature at each
                        step, and it mirrors the way the customer
                        and product owner interact with the software
                \end{itemize}
            \item Considerations

                Try and minimize the number of stories which
                are in progress. Only fully completed stories
                are actually useful to the Product Owners and
                Customers.
        \end{itemize}
    \item Splitting User Stories
        \begin{itemize}
            \item User stories need to fit into a sprint
            \item Guideline
                \begin{itemize}
                    \item Size of user story should be <50\%
                        of a sprint
                \end{itemize}
        \end{itemize}

        How can we break apart large stories?
        \begin{itemize}
            \item Split by workflow
                \begin{mdframed}
                    Example: create an account in Banking App
                    \begin{itemize}
                        \item Checking
                        \item Savings
                        \item Line of Credit
                    \end{itemize}
                \end{mdframed}
                Ideally split and then prioritize based on
                usefulness and the frequency of usage.
            \item Split by Lifecycle
                \begin{mdframed}
                    Example: create an account in Banking App
                    \begin{itemize}
                        \item Create Account
                        \item Make Deposit
                        \item Withdraw Money
                    \end{itemize}
                \end{mdframed}

                General Patter: CRUD
                \begin{itemize}
                    \item Create
                    \item Read
                    \item Update
                    \item Delete
                \end{itemize}
            \item Split by Convenience Level
                \begin{mdframed}
                    Example: create an account in Banking App
                    \begin{itemize}
                        \item Enter data without autofill
                        \item Type in Data vs select from menu
                        \item Basic vs cool UI
                        \item Manual vs Automatic
                    \end{itemize}

                \end{mdframed}
                Start with basic tools and add the convenience
                and polish on top.
            \item Split by User Typicality
                \begin{mdframed}
                    Example: create an account in Banking App
                    \begin{itemize}
                        \item Novice, occasional, expert
                        \item single account owner vs multiple
                    \end{itemize}
                \end{mdframed}

                Add context to the user story to make
                it more specific.
                
        \end{itemize}

        What \textbf{Not} to do
        \begin{itemize}
            \item Do not split by architecture
                
                Not helpful to the user, leaves features
                which are not complete for demoing and testing.
            \item Do not split by function or critical
                qualities
                \begin{itemize}
                    \item cannot split off security
                    \item Data protection.
                \end{itemize}

                Leaves features which are broken and or
                actively harmful to users. Can also make
                adding the critical feature later more
                costly and complicated.
        \end{itemize}
    \item Misc Tips
        \begin{itemize}
            \item Defined Room
            \item Stand up same time and place every time
            \item Enforce working agreements
            \item Post Three Questions, should not need to be
                reiterated every time
                \begin{mdframed}
                    \begin{enumerate}
                        \item What did you do
                        \item What are you going to do
                        \item What is stopping you
                    \end{enumerate}
                \end{mdframed}
            \item Use a physical Scrum board in addition
                to whatever other tools.
                \begin{mdframed}
                    Not entirely sure why for this one? I
                    mean a scrum visualizer isn't all that
                    bad, and you shouldn't be updating the
                    scrum board in the meeting anyways.

                    Maybe put the scrum board in a public place.
                \end{mdframed}

            \item Use Synchonous Communication
                \begin{mdframed}
                    Not for everything, but it forces
                    an immediacy to the communication which
                    enhances the speed of communication and
                    increases accountability.
                \end{mdframed}
        \end{itemize}

    \item Agile Myths
        \begin{itemize}
            \item Understanding Problems
                \begin{itemize}
                    \item What can go wrong
                    \item How to identify and remove obstacles
                \end{itemize}
        \end{itemize}

        \begin{itemize}
            \item Agile means no plan
                \begin{mdframed}
                    Planning is still required,
                    and happens on multiple levels

                    The main characteristic is that the
                    plan is flexible and constantly is
                    updated with the new knowledge
                \end{mdframed}
            \item Agile means any process is fine
                \begin{mdframed}
                    Agile still requires well defined
                    processes and discipline
                \end{mdframed}
                \pagebreak
            \item Agile means rushing products out of the door
                \begin{mdframed}
                    While agile emphasizes functional intermediate
                    builds, it does not require poor engineering
                    or doing the fastest least effort thing
                    always.

                    Agile relies on good feedback which means
                    robust and fast processes for communication
                    and revision.
                \end{mdframed}
        \end{itemize}
    \item Scrum Smells
        \begin{itemize}
            \item Zero or more than 1 product owner.
                \begin{mdframed}
                    Product owner has the vison and controls
                    the direction. Zero or multiple either
                    lack or have conflicting visions and direction.
                \end{mdframed}
            \item The Scrum Task Master
                \begin{mdframed}
                    A scrum master who micro manages and directly
                    commands team members
                \end{mdframed}
            \item Commitment Phobia
                \begin{mdframed}
                    Avoidance on committing to compeleting tasks
                    or to actually making decisions
                \end{mdframed}
            \item Self unmanaged teams
                \begin{mdframed}
                    Teams that aren't managed externally
                    and have little internal control.
                \end{mdframed}
            \item Burn up/down charts which stagnate
                \begin{mdframed}
                    Stagnation shows that something else
                    may be wrong. Either everyone is blocked,
                    or people aren't updating the chart,
                    or even that something is wrong with the
                    tasks.

                    Sign to begin investigation into what
                    is happening exactly on the team.
                \end{mdframed}
            \item Urgent things crowd out important things
                \begin{itemize}
                    \item Important vs Unimportant
                        \begin{mdframed}
                            Whether a task contributes
                            to the goals of a system
                        \end{mdframed}
                    \item Urgen vs Not Urgent
                        \begin{mdframed}
                            Requires immediate attention
                        \end{mdframed}
                \end{itemize} 

                It is a problem if the unimportant urgent
                things cause the important non urgent things
                to not be done. A good Scrum process
                ensures that all important things are done
                regardless of urgency.

                It is also a benefit if things can be taken
                care of or forseen before they become urgent.
        \end{itemize}
    \item Scrum-but
        \begin{itemize}
            \item Scrum without key features
                \begin{itemize}
                    \item no stand ups
                    \item stand ups don't respect time
                    \item No permanent Scrum master or product owner
                    \item Sprint length varies
                    \item Fix bugs in stabalization sprint
                    \item No sprint review
                    \item No sprint retrospective
                \end{itemize}
        \end{itemize}

        Ask about alternatives, replacement systems.
    \item Technical Practicies

        Borrowed from XP (eXtreme Programming)
        \begin{itemize}
            \item Done Criteria
            \item Peer review / Pair Programming (lmao)
            \item Clean Code (yuck)
            \item TFD/TDD (mixed)
            \item Continuous Integration
            \item Version Control
            \item Test Coverage criteria
            \item Static Analysis Tools
        \end{itemize}

\end{description}


\end{document}

