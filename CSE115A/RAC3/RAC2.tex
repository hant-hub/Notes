\documentclass{report}
\usepackage[tmargin=2cm, rmargin=1in, lmargin=1in,margin=0.85in,bmargin=2cm,footskip=.2in]{geometry}
\usepackage{amsmath,amsfonts,amsthm,amssymb,mathtools}
\usepackage{enumitem}
\usepackage[]{mdframed}
\usepackage{tikz}
\usepackage{listings}


\definecolor{codegreen}{rgb}{0,0.6,0}
\definecolor{codegray}{rgb}{0.5,0.5,0.5}
\definecolor{codepurple}{rgb}{0.58,0,0.82}
\definecolor{backcolour}{rgb}{0.95,0.95,0.92}

\lstdefinestyle{c_style}{
    language=C,
    backgroundcolor=\color{backcolour},   
    commentstyle=\color{codegreen},
    keywordstyle=\color{magenta},
    numberstyle=\tiny\color{codegray},
    stringstyle=\color{codepurple},
    basicstyle=\ttfamily\footnotesize,
    breakatwhitespace=false,         
    breaklines=true,                 
    captionpos=b,                    
    keepspaces=true,                 
    numbers=left,                    
    numbersep=5pt,                  
    showspaces=false,                
    showstringspaces=false,
    showtabs=false,                  
    tabsize=2
}

\lstdefinestyle{asm_style}{
    language=asm,
    backgroundcolor=\color{backcolour},   
    commentstyle=\color{codegreen},
    keywordstyle=\color{magenta},
    numberstyle=\tiny\color{codegray},
    stringstyle=\color{codepurple},
    basicstyle=\ttfamily\footnotesize,
    breakatwhitespace=false,         
    breaklines=true,                 
    captionpos=b,                    
    keepspaces=true,                 
    numbers=left,                    
    numbersep=5pt,                  
    showspaces=false,                
    showstringspaces=false,
    showtabs=false,                  
    tabsize=2
}




\title{\Huge{CSE 115A - Introduction to Software Engineering}}
\author{\huge{Elijah Hantman}}
\date{}

\begin{document}
\maketitle
\newpage

Scrum + Engineering Practices is a paper by Williams,
Brown, et al. published by IEEE in September 2011 for
a conference.

The paper looks at several teams and details 'Flacid Scrum'
which refers to teams which use SCRUM but eschew other essential
development practices such as automation and testing.

The paper then examines four specific teams at Microsoft
and the specific development practicies they used.

\begin{enumerate}
    \item The teams used Planning Poker to come to collective
        estimates of difficulty and time required. The teams
        reported that it helped improve their confidence in
        their estimates being accurate and correct.
    \item The teams used Continuous integration, which is
        a practice of automating tests into the build pipeline
        which allowed for teams to work faster and detect problems
        sooner.
    \item Teams also attempted to practice Test Driven Development.
        Each team did something slightly different, but all attempted
        to write automatic unit tests either before or soon
        after writing code, and all teams saw the amount of their
        code tested rise up to nearly 90\%.
    \item Teams also used rigorous peer review. Senior developers
        would check the code and ensure that there weren't any
        stylistic errors or missed bugs.
\end{enumerate}

In the end the authors conclude that Scrum has improved the quality
of the software delivered, measured by the density of defects
in the resulting software, as well as increasing the total amount
of code produced.

Overall I think that the paper definitively shows that SCRUM
can work, although they admit in their limitations that
the results could be partially or largely due to other
factors such as familiarity with the code base and
possibly even just the nature of the projects themselves.

\end{document}

