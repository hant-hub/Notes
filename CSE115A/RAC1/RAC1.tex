\documentclass{report}
\usepackage[tmargin=2cm, rmargin=1in, lmargin=1in,margin=0.85in,bmargin=2cm,footskip=.2in]{geometry}
\usepackage{amsmath,amsfonts,amsthm,amssymb,mathtools}
\usepackage{enumitem}
\usepackage[]{mdframed}
\usepackage{tikz}
\usepackage{listings}


\definecolor{codegreen}{rgb}{0,0.6,0}
\definecolor{codegray}{rgb}{0.5,0.5,0.5}
\definecolor{codepurple}{rgb}{0.58,0,0.82}
\definecolor{backcolour}{rgb}{0.95,0.95,0.92}

\lstdefinestyle{c_style}{
    language=C,
    backgroundcolor=\color{backcolour},   
    commentstyle=\color{codegreen},
    keywordstyle=\color{magenta},
    numberstyle=\tiny\color{codegray},
    stringstyle=\color{codepurple},
    basicstyle=\ttfamily\footnotesize,
    breakatwhitespace=false,         
    breaklines=true,                 
    captionpos=b,                    
    keepspaces=true,                 
    numbers=left,                    
    numbersep=5pt,                  
    showspaces=false,                
    showstringspaces=false,
    showtabs=false,                  
    tabsize=2
}

\lstdefinestyle{asm_style}{
    language=asm,
    backgroundcolor=\color{backcolour},   
    commentstyle=\color{codegreen},
    keywordstyle=\color{magenta},
    numberstyle=\tiny\color{codegray},
    stringstyle=\color{codepurple},
    basicstyle=\ttfamily\footnotesize,
    breakatwhitespace=false,         
    breaklines=true,                 
    captionpos=b,                    
    keepspaces=true,                 
    numbers=left,                    
    numbersep=5pt,                  
    showspaces=false,                
    showstringspaces=false,
    showtabs=false,                  
    tabsize=2
}




\title{\Huge{CSE 115A - Introduction to Software Engineering}}
\author{\huge{Elijah Hantman}}
\date{}

\begin{document}
\maketitle
\newpage

%TODO(ELI): Make sure to change the apollo stuff if it becomes avalible

“Managing the development of large software systems" Winston Royce was published in
1987 as part of the 1987 international conference on Software Engineering.
Winston Royce stated that his goal was to share his understanding of the kinds of processes
which are required when delivering large software systems, based on his near decade
of experience in the aerospace industry. I tried to read about some of the Apollo systems
however the link was broken and I ultimately could not read them.

In his paper Royce makes a number of arguements.
One of the first arguements he makes is that there are two primary stages of delivering
software, Analysis, and Coding. To Royce Analysis comprises all of the labor required to
understand what needs to be made, things like researching the problem, selecting algorithms,
understanding what pieces of data need to be tracked and where they are used, all fall under
Analysis.

Along with the basic stages of Analysis and Coding, Royce adds a number of additional stages and
processes which seek to address problems in the basic model. He begins by adding standard steps,
defining clearly the requirements of the project, testing, and deployment/usage as discrete
stages with their own experts and processes. One of the stages he proposes which goes beyond the
standard is the Preliminary Design stage.

The goal of this new stage is to create a seed of the solution, this is going to inevitably
be incomplete, but the goal is to create a skeleton which makes the work of the analysts and
the final designers easier by reducing the uncertainty. Analysts can see what tasks need to
be done in what amount of time as they solve problems, designers and analysts can immediately
begin discussing the correct allocation of resources to each sub task without having to wait
for both analysis without an existing design and an official system design before iterating.

Many will credit Royce for creating a robust template for Waterfall development, however
Royce does not use the term Waterfall anywhere in his writing. He also does not state that
each stage cannot be happening concurrently and versions of each stage also tend to exist
in modern CI/CD pipelines, even those which are used by more agile teams.

I would argue that Royce's approach is not inconsistent with agile approaches even if it
cannot be done in the way he originally envisioned. The steps he describes are essential,
testing is a massive part of any professional development team, preliminary designs are useful
both on the micro level of individual features and tasks, as well as on the macro level for
broad architectual decisions which may influence the direction of the entire release.



\end{document}

