\documentclass{report}
\usepackage[tmargin=2cm, rmargin=1in, lmargin=1in,margin=0.85in,bmargin=2cm,footskip=.2in]{geometry}
\usepackage{amsmath,amsfonts,amsthm,amssymb,mathtools}
\usepackage{enumitem}
\usepackage[]{mdframed}
\usepackage{tikz}
\usepackage{listings}


\definecolor{codegreen}{rgb}{0,0.6,0}
\definecolor{codegray}{rgb}{0.5,0.5,0.5}
\definecolor{codepurple}{rgb}{0.58,0,0.82}
\definecolor{backcolour}{rgb}{0.95,0.95,0.92}

\lstdefinestyle{c_style}{
    language=C,
    backgroundcolor=\color{backcolour},   
    commentstyle=\color{codegreen},
    keywordstyle=\color{magenta},
    numberstyle=\tiny\color{codegray},
    stringstyle=\color{codepurple},
    basicstyle=\ttfamily\footnotesize,
    breakatwhitespace=false,         
    breaklines=true,                 
    captionpos=b,                    
    keepspaces=true,                 
    numbers=left,                    
    numbersep=5pt,                  
    showspaces=false,                
    showstringspaces=false,
    showtabs=false,                  
    tabsize=2
}

\lstdefinestyle{asm_style}{
    language=asm,
    backgroundcolor=\color{backcolour},   
    commentstyle=\color{codegreen},
    keywordstyle=\color{magenta},
    numberstyle=\tiny\color{codegray},
    stringstyle=\color{codepurple},
    basicstyle=\ttfamily\footnotesize,
    breakatwhitespace=false,         
    breaklines=true,                 
    captionpos=b,                    
    keepspaces=true,                 
    numbers=left,                    
    numbersep=5pt,                  
    showspaces=false,                
    showstringspaces=false,
    showtabs=false,                  
    tabsize=2
}




\title{\Huge{CSE 115A - Introduction to Software Engineering}}
\author{\huge{Elijah Hantman}}
\date{}

\begin{document}
\maketitle
\newpage

{\huge Agile Planning and Estimation Cont.}

\begin{description}
    \item User Story Format

        \begin{mdframed}
            As a [user] I want [goal] [for reason]
        \end{mdframed}

        Idea is to have a more direct summary of what
        is important. The user story informs an experience
        that the developers want to create rather than
        a feature or implementation.

    \item MoSCoW
        \begin{enumerate}
            \item Must Have
            \item Should Have
            \item Could Have
            \item Won't Have
        \end{enumerate}

        User stories should be sorted into these buckets
        in order to begin the process of determining what
        the product should be.

        MoSCoW is one scheme, other schemes can be simpler or
        more complicated depending on the team and the 
        product.

    \item Estimating Cost
        \begin{itemize}
            \item Development Effort required
                \begin{itemize}
                    \item Story points
                    \item Not measured in hours since
                        many parts of the project are vague
                        or unknown. The goal is to get a general
                        feel for the effort rather than exact
                        or even approximate estimates.
                \end{itemize}
            \item Story points are abstract, delibrately
                vague

                Points are allocated to small, medium, large, etc.
                depending on the team. Generally using fibbonacci
                numbers is common, the goal is to capture how
                more difficult tasks tend to take longer
                in a non linear fashion.
        \end{itemize}
    \item Planning Poker

        Game to determine and align everyone's estimations
        of effort.

        \begin{itemize}
            \item Discuss effort
            \item Each team member makes a secret estimate
            \item Check if every one matches
            \item repeat until everyone is 'close enough'
        \end{itemize}

        The ultimate goal is to spend more time upfront in
        order to prevent wasted work. Time spent planning
        to hopefully improve efficiency and give accurate
        costs for changes.

    \item Assigning Stories

        \begin{itemize}
            \item How many story points can reasonably
                be done in a sprint
            \item Should be revised based on how
                development progresses
        \end{itemize}
        
        \begin{mdframed}
            Note:

            Sprint goals are estimates not commitments,
            if a task is larger than expected it is okay
            to overrun a sprint.
        \end{mdframed}

        The end of a sprint should result in a working
        version of the product, meaning that user stories
        should be doable in a single sprint, at least in
        principle.


    \item Goal of Planning

        Prioritized list of user stories, estimated in points.
        What to do and estimates of how much work it will take.

        I mean sprint wise I would just push it into the next sprint
        but the professor seems to be like oh you already have a plan
        for each sprint and then it overruns so whatever.

    \item Backlogs

        Multiple backlogs, product backlog is every single story for the product.
        A portion of the product backlog is selected for the release, then a portion
        of the release backlog is allocated to each sprint. Each user story is broken
        down into tasks to be completed.

        Product Owners handle the product and release backlogs, the developers handle
        the sprints and specific tasks. However they also inform the Product Owners
        on what is and isn't possible.


    \item Sprint Plan

        More of a commitment than a release plan.
        The sprint plan involves actual time and resources.
        
        \begin{mdframed}
            Dude Idk why he keeps going back to
            me saying roll it over. I literally assumed
            we were planning the sprint, not that the
            release had been planned and we ran out of
            story points like bro.
        \end{mdframed}

        For a story to be done in a sprint it means that the
        entire feature should be present. This means both front
        and backend, enough to test and use the feature even if
        unpolished.

    \item Spring Planning Meeting

        \begin{itemize}
            \item Prioritize

                Analyze and develop goals based on current
                knowledge. Generally done by the PO with input
                from the team regarding what is and isn't possible.

            \item Planning

                How to achieve goals. Create tasks for the current
                sprint and goals, estimate the tasks in hours.

                This is mostly done by the team since it regards the
                technical details and final design work.

        \end{itemize}

        Estimations
        \begin{itemize}
            \item Units are in ideal hours

                \begin{itemize}
                    \item full knowledge and no surprises
                    \item Generally underestimation by 2 or more times
                \end{itemize}

            \item How much can each person do?
                \begin{itemize}
                    \item Assume conservative figure, 8-12 hours
                    \item Over time update with actual work completed.
                \end{itemize}
        \end{itemize}

        Possibilities
        \begin{itemize}
            \item Not enough time
                \begin{itemize}
                    \item reasses stories
                    \item update priorities
                    \item put down lower priority stories
                \end{itemize}

            \item Too much time
                \begin{itemize}
                    \item Unlikely
                    \item Add story midway
                    \item update priorities
                \end{itemize}
        \end{itemize}

    \pagebreak
    \item Expectations for Planning

        \begin{itemize}
            \item Task listing with time estimate and prioritized
            \item Team Roles
            \item Initial task assignments
            \item Initial task burn-up / burn-down chart
            \item Initial Scrum board set up
            \item Schedule of Scrum meetings
        \end{itemize}


    \item Managing the Sprint
        \begin{itemize}
            \item Individuals sign up for tasks
            \item Work remaining is estimated daily
            \item Any team can edit the backlog
            \item Revaluate and edit amount of work as
                things are discovered.
        \end{itemize}

    \item Daily Scrum
        \begin{itemize}
            \item Daily meeting
            \item 15 mins
                \begin{itemize}
                    \item Strict time
                    \item Follow up after for bigger issues
                    \item Stand-up, keep everything fast and short
                \end{itemize}
            \item Not for problem solving
                \begin{itemize}
                    \item Only team members, scrum master, and
                        product owner can talk
                    \item Anyone can attend
                \end{itemize}
            \item Goal is to avoid other meetings
        \end{itemize}

        Topics
        \begin{enumerate}
            \item What did you do yesterday?
            \item What will you do today?
            \item Is there anything in your way?
                \begin{mdframed}
                    Personally the first 2 could be
                    done async, and the third could be
                    done just between the people managing
                    both sides of the issue.
                \end{mdframed}
        \end{enumerate}

    \item Pitfalls
        \begin{itemize}
            \item Being late or missing the meeting has a penalty
                \begin{itemize}
                    \item if not present assumed nothing happened
                    \item disrespectful
                    \item can block people
                \end{itemize}
            \item Grandstanding

                Own up to whether you didn't get much done
                be honest and concise

            \item Over time
                \begin{itemize}
                    \item Stay in time
                    \item Big issues should be discussed
                        by involved team members after Scrum
                \end{itemize}
            \item Failure to commit to work
            \item Failure to update Scrum Board
        \end{itemize}

    \item Scrum Board
        
        Table which represents the current sprint progress.
        For each story there are tasks which are sorted into
        To-Do, In-Progress, and Done.

        As tasks are completed they move from the To-Do column
        to the In-Progress and eventually the Done column.

        As tasks are picked up the name of the person doing
        the task is noted.

        Only valuable if it is up to date.

    \item Sprint Post Mortem

        Evaluate product and process.

        Reflect on how things went and what went wrong
        or could have been done better, as well as what
        went well and what worked.

        If you missed the goal you need to replan the future
        sprints and reevaluate the time cost of each task.

        Sprints are never extended, they are just checkpoints
        and not deadlines. Unfinished work is not necessarily
        a failure to deliver.

    \item Sprint Review
        \begin{itemize}
            \item Demo of product at end of sprint
            \item Informal, less than 2 hour prep no slides
            \item Who
                \begin{itemize}
                    \item Whole team
                    \item Anyone interested
                \end{itemize}
            \item Show what actually exists and what is
                actually finished.
        \end{itemize}

    \item Sprint Retrospective
        \begin{itemize}
            \item After every sprint
            \item process improvement
            \item 15-30 mins
            \item Whole team
        \end{itemize}

        One way
        \begin{itemize}
            \item Start Doing
            \item Stop Doing
            \item Continue Doing
        \end{itemize}

        Goal is to improve the infrastruture and practicies
        by reviewing and reflecting on them.


    \item Initial Presentations
        Template is on Canvas.

        Can use any slides as long as all the same
        info is on the slides.
\end{description}


\end{document}

