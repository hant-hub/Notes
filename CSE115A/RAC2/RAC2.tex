\documentclass{report}
\usepackage[tmargin=2cm, rmargin=1in, lmargin=1in,margin=0.85in,bmargin=2cm,footskip=.2in]{geometry}
\usepackage{amsmath,amsfonts,amsthm,amssymb,mathtools}
\usepackage{enumitem}
\usepackage[]{mdframed}
\usepackage{tikz}
\usepackage{listings}


\definecolor{codegreen}{rgb}{0,0.6,0}
\definecolor{codegray}{rgb}{0.5,0.5,0.5}
\definecolor{codepurple}{rgb}{0.58,0,0.82}
\definecolor{backcolour}{rgb}{0.95,0.95,0.92}

\lstdefinestyle{c_style}{
    language=C,
    backgroundcolor=\color{backcolour},   
    commentstyle=\color{codegreen},
    keywordstyle=\color{magenta},
    numberstyle=\tiny\color{codegray},
    stringstyle=\color{codepurple},
    basicstyle=\ttfamily\footnotesize,
    breakatwhitespace=false,         
    breaklines=true,                 
    captionpos=b,                    
    keepspaces=true,                 
    numbers=left,                    
    numbersep=5pt,                  
    showspaces=false,                
    showstringspaces=false,
    showtabs=false,                  
    tabsize=2
}

\lstdefinestyle{asm_style}{
    language=asm,
    backgroundcolor=\color{backcolour},   
    commentstyle=\color{codegreen},
    keywordstyle=\color{magenta},
    numberstyle=\tiny\color{codegray},
    stringstyle=\color{codepurple},
    basicstyle=\ttfamily\footnotesize,
    breakatwhitespace=false,         
    breaklines=true,                 
    captionpos=b,                    
    keepspaces=true,                 
    numbers=left,                    
    numbersep=5pt,                  
    showspaces=false,                
    showstringspaces=false,
    showtabs=false,                  
    tabsize=2
}




\title{\Huge{CSE 115A - Introduction to Software Engineering}}
\author{\huge{Elijah Hantman}}
\date{}

\begin{document}
\maketitle
\newpage

The paper "The Danger of Architectural Technical Debt" By
Martini and Bosch, published 2015 at the IEEE/IFIP Conference
on Software architecture, is a study into Architectual technical
debt and specifically which specific actions are worst.

The paper explains technical debt as decisions made during software
development in the interest of meeting business requirements and
deadlines. Examples include violating the overall architecture
or just making suboptimal choices in general. Techical debt is important
because if technical debt is very bad it can slow development putting
more pressure on the development team which causes the debt to increase
faster and faster. This is a viscious cycle which can kill a project.

The paper covers 5 large international companies looking
at 5 Scandinavian sites over the course of 18 months. Overall
they list out several ways in which technical debt can be
incurred, such as failing to identify non functional requirements
early, violating architecture dependencies and creating new
dependencies which were not planned or anticipated, using either
synchronous or asynchronous functions in the wrong context resulting
in poor performance, and so on. 

The authors in addition distinguish the kinds of Technical Debt
which is most dangerous, accumulating techncial debt. They authors
use an example a company which failed to create a standard interface
to a database abstraction, this failure resulted in many modules
depend on a sub optimal interface which means that changes ripple
out to larger and larger portions of the codebase over time.

It is also noted that a common cause of vicious cycles is hidden
debt. If the technical debt is noted and consciously kept track
of it was much less likely to cause a viscious cycle.

However a major limitation of the
paper is that each company had wildly different policies and 
approaches to Agile development. Different policies can cause
technical debt to be worse or better, and can change how
technical debt can come about. The best ways to manage
technical debt is to first not create it, and second to keep
track and carefully manage what debt necessarily is created
during the development process.

\end{document}

