\documentclass{report}
\usepackage[tmargin=2cm, rmargin=1in, lmargin=1in,margin=0.85in,bmargin=2cm,footskip=.2in]{geometry}
\usepackage{amsmath,amsfonts,amsthm,amssymb,mathtools}
\usepackage{enumitem}
\usepackage[]{mdframed}
\usepackage{tikz}
\renewcommand{\familydefault}{\sfdefault}

\title{\Huge{Math 100}}
\author{\huge{Elijah Hantman}}
\date{}

\begin{document}
\maketitle
\newpage

Set Review Cont.
\begin{itemize}
    \item Cartesian Product

        \begin{gather}
            A \times B = \text{Set of all ordered pairs}\\
            = \{(a,b) | a \in A, b \in B \}
        \end{gather}

        Lemma:
        \begin{mdframed}
            \begin{gather}
                |A \times B| = |A| \times |B| 
            \end{gather}
        \end{mdframed}

    \item Could we make set arithmetic analogous to real arithmetic?

        \begin{gather}
            A^B \text{ or } \frac{A}{B} 
        \end{gather}

        The property of expontentials we want to preserve
        is:

        \begin{gather}
            A^{B_1 + B_2} = A^{B_1} \times A^{B_2}\\
            A^{\emptyset} = A
        \end{gather}

        $A/B$ only makes sense for some versions of A and B.


    \item How do we define Set Exponentiation for Arbitrary sets A and B?
        
        One possibility:

        \begin{mdframed}
            \begin{gather}
                A^B \coloneq \{f: B \to A \} 
            \end{gather}

            \begin{gather}
                A = \{1, 2\}\\
                A^B = \mathcal{P}(B)\\
                2^B = \mathcal{P}(B)
            \end{gather}

            This means that the powerset is a special
            case of set exponentiation.
        \end{mdframed}

        \begin{gather}
            A^{B_1 + B_2} = A^{B_1} \times A_{B_2}\\ 
            A^{B_1 + B_2} = \{f; B_1 + B_2 \to A \}\\
            = \{f_1: B_1 \to A \} \times \{f_2: B_2 \to A \}\\
            f \leftrightarrow (f_1, f_2)
        \end{gather}




\end{itemize}
\pagebreak
{\huge Chapter 2 - Logic}

\begin{mdframed}
    A statement is a declarative sentance which
    can be objectively determined to be True or False.
\end{mdframed}
\begin{description}
    \item Example

        \begin{enumerate}
            \item An integer 11 is divisible by 4

                False

            \item An integer 11 is big.

                Subjective, not falsifiable.
                Not a logical statement.

            \item An integer 11 is odd.

                True
        \end{enumerate}

    \item Logical Negation

        Given a statement P.

        $\neg P$ is a statement which is true if P is false,
        and false if P is true.

        \begin{tabular}{|c|c|}
            \hline
            P & $\neg P$\\
            \hline
            T & F\\
            \hline
            F & T\\
            \hline
        \end{tabular}

    \item Example

        P: an integer 11 is odd.\\
        $\neg P$: an integer 11 is not odd.

    \item Disjunction

        P, Q are statements.

        The disjunction of P and Q, $P \lor Q$
        is a statement which is true if P is true, or
        if Q is true.

        \begin{tabular}{|c|c|c|}
            \hline
            P & Q & $P \lor Q$\\ 
            \hline
            T & T & T\\
            \hline
            T & F & T\\
            \hline
            F & T & T\\
            \hline
            F & F & F\\
            \hline
        \end{tabular}
    \item Example

        P is a statement.

        \begin{displaymath}
            P \lor \neg P
        \end{displaymath}

        That statement is always true, meaning it is
        a tautology.
    \item Conjuction

        P, Q are statements.

        The conjunction of P and Q, $P \land Q$
        is a statement which is true only if both
        P is true, and Q is true.

        \begin{tabular}{|c|c|c|}
            \hline
            P & Q & $P \land Q$\\ 
            \hline
            T & T & T\\
            \hline
            T & F & F\\
            \hline
            F & T & F\\
            \hline
            F & F & F\\
            \hline
        \end{tabular}

    \item Implication

        P, Q statements.

        For P to imply Q, $P \implies Q$ means that
        if P is true, Q is true. If P is true and Q is false,
        the statement is false. Otherwise the statement is true.
    
    \pagebreak
    \item Example

        P: It is raining\\
        Q: I will stay home

        $P \implies Q$: If it is raining, I will stay home.

        If P is false the statement is trivially true.

        \begin{mdframed}
            Note: $P \implies Q$ is equivalent to
            $\neg (P \land \neg Q)$ or
            $\neg P \lor Q$
        \end{mdframed}

        \begin{tabular}{|c|c|c|}
            \hline
            P & Q & $P \implies Q$\\ 
            \hline
            T & T & T\\
            \hline
            T & F & F\\
            \hline
            F & T & T\\
            \hline
            F & F & T\\
            \hline
        \end{tabular} 

    \item Logical Identities 

        \begin{itemize}
            \item $P \implies Q$
                if P then Q

                P implies Q

                Q, if P.

                P only if Q.

                P sufficient for Q

                Q is necessary for P
        \end{itemize}
    
    \item What is the negation of an implication?
        
        \begin{gather}
            \neg (P \implies Q) \\ 
            = \neg (\neg P \lor Q)\\
            = P \land \neg Q\\
        \end{gather}

        \begin{tabular}{|c|c|c|c|}
            \hline
            P & Q & $P \implies Q$ & $\neg(P\implies Q)$\\
            \hline
            T & T & T & F\\
            \hline
            T & F & F & T\\
            \hline
            F & T & T & F\\
            \hline
            F & F & T & F\\
            \hline
        \end{tabular}

    \item Theorem

        \begin{gather}
           \neg (P \implies Q) \equiv P \land (\neg Q) 
        \end{gather}

        Following:

        \begin{gather}
           P \implies Q \equiv \neg P \lor Q 
        \end{gather}

        \pagebreak
    \item Definition
        \begin{mdframed}
            An open sentence is a declarative sentence
            which cantains variables.

            Each variable can assume any value in a given
            set, called the domain of the variables, and
            an open sentence becomes a statement when variables
            are replaced with values.
        \end{mdframed}

    \item Example

        \begin{gather}
            P(x): |x| = 3, x \in \mathbb{R} 
        \end{gather}

        The open sentence above only becomes objectively
        true or false when we assign a value to x.

        $P(x)$ is not a statement but $P(2)$ is a statement
        which is false, and $P(3)$ is a statement which is true.


\end{description}


\end{document}
