\documentclass{report}
\usepackage[tmargin=2cm, rmargin=1in, lmargin=1in,margin=0.85in,bmargin=2cm,footskip=.2in]{geometry}
\usepackage{amsmath,amsfonts,amsthm,amssymb,mathtools}
\usepackage{enumitem}
\usepackage[]{mdframed}
\usepackage{tikz}
\renewcommand{\familydefault}{\sfdefault}

\newtheorem{thm}{Theorem}[section]

\title{\Huge{Math 100}}
\author{\huge{Elijah Hantman}}
\date{}

\begin{document}
\maketitle
\newpage

Definition:
\begin{mdframed}
    Given $P \implies Q$, the converse
    is $Q \implies P$.
\end{mdframed}

Definition:
\begin{mdframed}
    The biconditional of P and Q is the statment
    \begin{displaymath}
        (P \implies Q) \land (Q \implies P)
    \end{displaymath}

    It is also written as

    \begin{displaymath}
        P \iff Q
    \end{displaymath}

    And uses the word iff or the phrase
    if and only if.

    Also in English, necessary and sufficient
    also means the same thing.

    This works because necessary means $Q \implies P$
    and sufficient means $P \implies Q$.

    Also means that P and Q are equivalent because
    the statement $P = Q$ will be tautalogical.
\end{mdframed}

\begin{center}
    Implication Truth Table\\
    \begin{tabular}{|c|c|c|c|c|}
        \hline
        P & Q & $P \implies Q$ & $Q \implies P$ & $P \iff Q$\\ 
        \hline
        T & T & T & T & T\\
        \hline
        T & F & F & T & F\\
        \hline
        F & T & T & F & F\\
        \hline
        F & F & T & T & T\\
        \hline
    \end{tabular}
\end{center}

\subsection{Tautologies and Contradictions}

Definition:
\begin{mdframed}
    A compound statement is a statement consisting
    of at least one statement involving at least one
    logical connective (and, or, not, implies, bicond).
\end{mdframed}

Definition:
\begin{mdframed}
    A tautology is a compound statement which is always
    true regardless of the truth values of component statements.
\end{mdframed}

Definition:
\begin{mdframed}
    A contradiction is a compound statement which is always
    false regardless of the truth values of component statements.
\end{mdframed}


Example:
\begin{mdframed}
    \begin{gather}
        P \land \neg P = F
    \end{gather}

    A simple contradiction.

    \begin{gather}
       P \lor \neg P = T 
    \end{gather}

    A simple tautology.


\end{mdframed}

Other Useful tautologies:
\begin{itemize}
    \item
        \begin{gather}
            P \land (P \implies Q) \implies Q
        \end{gather}

        This is a simple tautology from the definitions.
        If P implies Q and P is true, then Q must be true.
    \item 
        \begin{gather}
            (P \implies Q) \land (Q \implies R) \implies (P \implies R) 
        \end{gather}
\end{itemize}

\subsection{Logical Equivalence}

Definition:
\begin{mdframed}
    Given statements R, and S. R and S are
    logically equivlanet if they have the same
    truth value for all combinations of truth
    values of each of their component statements.

    Denoted: $R \equiv S$
    

    \begin{displaymath}
        R \equiv S
    \end{displaymath}

    When

    \begin{displaymath}
        R \iff S
    \end{displaymath}

    Is a tautology.
\end{mdframed}


\begin{thm}
   \begin{gather}
        (P \implies Q) \equiv (\neg P) \lor Q
   \end{gather}     
\end{thm}

\subsection{Fundamental Properties of Logical Equivalences}

\begin{enumerate}
    \item $\neg (\neg P) \equiv P$
    \item Commutivity:
        \begin{gather}
           P \lor Q \equiv Q \lor P\\ 
           Q \land P \equiv P \land Q
        \end{gather}
    \item Associativity:
        \begin{gather}
            (P \lor Q) \lor R \equiv P \lor (Q \lor R)\\ 
            (P \land Q) \land R \equiv P \land (Q \land R)
        \end{gather}
    \item Distributive Laws:
        \begin{gather}
            (P \lor Q) \land R \equiv (P \land R) \lor (Q \land R)\\
            (P \land Q) \lor R \equiv (P \lor R) \land (Q \lor R)
        \end{gather}
    \item DeMorgan's Law:
        \begin{gather}
            \neg (P \land Q) \equiv (\neg P) \lor (\neg Q)\\
            \neg (P \lor Q) \equiv (\neg P) \land (\neg Q)
        \end{gather}

        These are the same as the set theory version. It works
        since sets can also be defined as some logical statement
        where truth means it is in the set and false means
        it is excluded. Then the laws are literally the same.
\end{enumerate}

How do we use Equivalences?
\begin{enumerate}
    \item $\neg (P \implies Q) \equiv P \land (\neg Q)$
        We can show this is the same using De Morgan's law.
        And the tautology $P \implies Q \equiv \neg P \lor Q$

        \begin{gather}
            \neg (P \implies Q) \equiv \neg (\neg P \lor Q)\\ 
            P \land (\neg Q) \equiv P \land (\neg Q)
        \end{gather}
\end{enumerate}

Example:
\begin{mdframed}
    \begin{gather}
        ((\neg Q) \implies (P \land \neg P)) \equiv Q
    \end{gather}
\end{mdframed}

Example:
\begin{mdframed}
    Proof by Contradiction

    Goal:
    \begin{gather}
        P \implies Q
    \end{gather}

    \begin{gather}
        P \implies Q \equiv P \land (\neg Q) \implies (R \land \neg R)
    \end{gather}

    We get benefits by getting to assume more things.
    Rather than just assuming $P$ we assume $P$ and
    $\neg Q$. If we can reach a contradiction then we
    have shown $P \implies Q$.
    

    \begin{gather}
        P \land (\neg Q) \implies (R \land \neg R) \\
        \equiv \neg (P \land (\neg Q)) \lor (R \land \neg R)\\
        \equiv \neg (P \land (\neg Q)) \lor F\\
        \equiv \neg (P \land (\neg Q))\\
        \equiv \neg P \lor Q\\
        \equiv P \implies Q
    \end{gather}
\end{mdframed}

\subsection{Quantified Statements}

$P(x)$ is an open sentence.

Consider:
\begin{gather}
    \forall x \in S | P(x)\\
    \exists x \in S | P(x)
\end{gather}

These mean, for all x and there exists an x
respectively.


{\large Negation}
\begin{gather}
    \neg (\forall x \in S, P(x)) \equiv \exists x \in S, \neg P(x)\\
    \neg (\exists x \in S, P(x)) \equiv \forall x \in S, \neg P(x)
\end{gather}




\end{document}
