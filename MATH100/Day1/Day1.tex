\documentclass{report}
\usepackage[tmargin=2cm, rmargin=1in, lmargin=1in,margin=0.85in,bmargin=2cm,footskip=.2in]{geometry}
\usepackage{amsmath,amsfonts,amsthm,amssymb,mathtools}
\usepackage{enumitem}
\usepackage[]{mdframed}
\usepackage{tikz}
\renewcommand{\familydefault}{\sfdefault}

\title{\Huge{Math 100}}
\author{\huge{Elijah Hantman}}
\date{}

\begin{document}
\maketitle
\newpage

\begin{description}
   \item 
       {\Large Review/Intro }

       \begin{itemize}
           \item Midterm is take home, start when convenient.
           \item Practice writing papers
           \item Using \LaTeX
           \item One problem per lecture, due Sunday
           \item 2 Papers, normal math paper.
       \end{itemize}

       Goal is to practice writing, so the problems are not
       particularly advanced. Nor is it systematic, it is
       whatever problem is best to practice the given writing
       skill.
    \item
        {\Large Set Theory Notation}

        \begin{mdframed}
            Capital letters denote sets.

            Lower case letters denote elements.
        \end{mdframed}

        Example:
        \begin{displaymath}
         a \in A
        \end{displaymath}

        a is an element in the set A.

        \begin{displaymath}
        a \notin A 
        \end{displaymath}

        a is not contained in A.

        {\large How to Describe a Set?}
        \begin{enumerate}
            \item List all elements

                \begin{displaymath}
                 A = \{ -4, -2, 0, 2, 4 \}
                \end{displaymath}

            \item State properties of elements

                \begin{displaymath}
                    A = \{ x \in \mathbb{Z} | IsEven(x), |x| \le 4 \}
                \end{displaymath}
            
        \end{enumerate}

        {\large Empty Set}

        Denoted $\emptyset$, which is the greek letter 'phi'.
        Also denoted $= \{ \}$.

        \begin{mdframed}
            Note: $\{ \emptyset \}$

            A set whose single element is an empty set.
            This set has a size of 1. This is distinct from
            the empty set.

            This also implies $ \emptyset \in A $ is reasonable.
        \end{mdframed}

    \item {\large Cardinality}

        \begin{gather}
            |A| = \text{ the cardinality of the set A} 
        \end{gather}

        \begin{mdframed}
            There are multiple infinite hierarchies of cardinality,
            regarding sets whose definitions contain infinite but distinct
            quanities.
        \end{mdframed}

        The Cardinality is the number of elements.
        The empty set has a cardinality of 0, while the
        set containing the empty set has a cardinality of 1.
        
    \pagebreak
    \item {\large Subset}

        Let B, be a set.

        a set A is a subset of B if every element
        of A is an element of B.

        \begin{gather}
            A \subseteq B \iff {\forall a \in A | a \in B} 
        \end{gather}

        If $A \subseteq B$, then it is possible that $A = B$.

    \item {\large Proper Subset}

        A proper subset excludes the possibility that the
        subset is equal to the set.

        \begin{gather}
           A \subset B \implies A \neq B 
        \end{gather}
    
    \item {\large Example}

        \begin{gather}
            B = \{ \emptyset, \{ \emptyset \}, 1, 2, \{1, 2 \} \} 
        \end{gather}
        
        We can write:

        \begin{gather}
            \emptyset \in B, \quad \{\emptyset \} \subseteq B 
        \end{gather}

        Both mean the same thing, however the first directly states
        that the element is in B, and the second says that the
        set containing the element is a subset, which is equivalent.


        \begin{gather}
            1, 2 \in B \quad \{ 1, 2 \} \subseteq B \\ 
            \{ 1, 2 \} \in B
        \end{gather}

        All the above are true, however the first line
        refer to the same elements, but the second line refers
        to specifically the set which is an element of B.


    \item {\large Powerset}

        A: a set

        \begin{align}
            \mathcal{P}(A) &= \text{ The power set} \\ 
            &= \text{ the set of all subsets of A }
        \end{align}

        For Example:

        \begin{gather}
            A = \{1, 2, 3\}\\
            \mathcal{P}(A) = \{\emptyset, \{ 1 \}, 
                \{ 2 \}, \{ 3 \}, \{1, 2\}, \{2, 3 \},
            `\{3, 1\}, \{1, 2, 3\}\}
        \end{gather}
       
        \pagebreak
        {\large Lemma}
        \begin{mdframed}
            A is a set with a finite number of elements.

            Then:

            \begin{gather}
                | \mathcal{P}(A) | = 2 ^{|A|} 
            \end{gather}

            This follows since each element of the set can either
            be included or not included, meaning there are 2 variants
            per element, meaning for each element we double the number
            of subsets.

            This also follows from:

            \begin{gather}
                2^n = \sum_{k=0}^{n} \binom{n}{k} 
            \end{gather}

            This means we are counting the number of ways
            to choose k elements from n elements. We sum
            all of these choices for each size of subset.

            This can be directly shown via algebra since we
            have a closed form solution for $\binom{n}{k}$
        \end{mdframed}

    \item {\large Example}

        \begin{gather}
            A = \{\emptyset, \{ \emptyset \} \} \\
            \text{Then}\\
            \mathcal{P}(A) = \{\emptyset, \{ \emptyset \},
            \{ \{ \emptyset \} \},
        \{\emptyset, \{\emptyset \}\}\}
        \end{gather}

    \item {\large Set Operations}

        A, B, C are subsets of $U$

        \begin{enumerate}
            \item 
                $A \cup B$ are the elements of U which
                are in A or B.

                $\{ x \in U | x \in A \lor x \in B\}$

            \item $A \cap B$ are elements in U which
                are in both A and B.

                $\{x \in U | x \in A \land x \in B \}$

            \item $A - B$ The elements of U which are in
                A but not in B.

                $\{x \in U | x \in A \land x \notin B \}$

                Like in arithmetic, subtraction is not
                commutative.
            \item $A^{c} = \bar{A}$ The complement of A. It means
                all the elements of U which are not in A.

                $\{ x \in U | x \notin A \}$

                Two common notations, the bar or a small c.
        \end{enumerate}

        Some identities
        \begin{itemize}
            \item $A - B = A \cap \bar{B}$
            \item $\overline{A \cup B} = \bar{A} \cap \bar{B}$
            \item $\overline{A \cap B} = \bar{A} \cup \bar{B}$
                \begin{mdframed}
                    Note: The previous two identities are
                    corrollaries of de Morgan's Laws of logic.
                    They are also called de Morgan's Laws here.
                \end{mdframed}
        \end{itemize}

    \item {\large Indexed Sets }

        \begin{gather}
            \{A_1, A_2, ... , A_n\} = \{ A_i \}_{1}^{n}
            = \{ A_i \}_{i \in I}\\
            I = \{1, 2, ..., n \} \\
            A_1 \cap A_2 \cap ... \cap A_n = \bigcap_{i \in I} A_i\\
            A_1 \cup A_2 \cup ... \cup A_n = \bigcup_{i \in I} A_i\\
        \end{gather}

        Convenient for expressing infinite combinations of sets
        or infinite sets of sets. This is easy to denote if
        $I$ is an infinite set.

        \begin{gather}
            \{S_\alpha, S_\beta, S_\gamma, ... \} = \{S_\lambda \}_{\lambda \in I}\\ 
            I = \{\alpha, \beta, \gamma, ... \}
        \end{gather}


    \item Example

        \begin{gather}
            I = [1, \infty)\\
            \text{For each $r \in I$, consider a set}\\
            S_r = (-\frac{1}{r}, \frac{1}{r})\\
            \text{What is the intersection and union?}\\
            \bigcap_{r\in I}S_r,\quad \bigcup_{r\in I}S_r
        \end{gather}

        \begin{gather}
            \bigcap_{r \in I} S_r = \{x \in \mathbb{R} | x \in S_r, \forall r \in I \}\\
            = \{ 0 \}\\
            \text{For all $x \neq 0$ we can choose $r \ge 1$ such
            that $|x| > \frac{1}{r}$}
        \end{gather}

        This means that for all $x \neq 0$ it cannot be in the
        intersection.

        \begin{gather}
            \bigcup_{r \in I} S_r = (-1, 1) = S_1 
        \end{gather}

        In general, infinite union of open intervals is a collection
        of open intervals. However, infinite intersection of infinite
        open intervals may not be open.

        \begin{mdframed}
            One of the core concepts of topology apparently. Leads
            directly to topological spaces.
        \end{mdframed}
    
    \item Partition of Sets

        A is a set

        \begin{gather}
            \mathcal{S} = \{X_a \}_{a \in I}  
        \end{gather}

        A collection of nonempty subsets of A, indexed by I.

        Definition:
        \begin{mdframed}
            A collection S of subsets of A is
            a partition of A if 

            \begin{enumerate}
                \item $\bigcap_{a \in I} X_a = \emptyset$ All sets have
                    no elements in common, they are disjoint.

                \item $\bigcup_{a \in I} X_a = A$ The union of all
                    sets is equal to A.
            \end{enumerate}
        \end{mdframed}

        An equivalence relation is related to partitions,
        each relation implies a partition and each partition
        implies an equivalence relation.
    
    \item Example
        \begin{gather}
            A = \mathbb{Z}\\
            \text{Def: For $n \in \mathbb{Z}^+$, and $a, b \in \mathbb{Z}$}\\
            a \equiv b \mod n \iff n | a - b
        \end{gather}

        \begin{gather}
            n > 0 \text{ For $0 \leq r \leq n - 1$, let}\\ 
            [r]_n = \{ \text{all integers with remainder r after
            dividing by n} \}\\
            = \{k \in \mathbb{Z}^+ | k \equiv r \mod n \}\\
            = \{..., r - 2n, r - n, r, r + n, r + 2n, ... \}
        \end{gather}

        \begin{gather}
            [2]_3 = \{..., -11, -8, -5, -2, 2, 5, 8, 11, ... \}
        \end{gather}
        
        {\large Lemma}
        \begin{mdframed}
            \begin{gather}
                \{[0]_n, [1]_n, [2]_n, ... [n-1]_n \} 
            \end{gather}
            Is a partition of $\mathbb{Z}$.
        \end{mdframed}

        This is called a conguence class.

    \item {\large Example}

        For $n = 3$ then $0 \le r \le 2$
    
        That implies three subsets.

        \begin{gather}
            [0]_n = \{ ..., -6, -3, 0, 3, 6, ...\}\\
            [1]_n = \{ ..., -5, -2, 1, 4, 7, ...\}\\
            [2]_n = \{ ..., -4, -1, 2, 5, 8, ...\}
        \end{gather}

        Together, no element is shared, and they cover all
        elements in $\mathbb{Z}$.

        We can treat these sets as algebraic objects we can
        do arithmetic on.

        We can define division on these conguence classes which
        is contained to the sets. This is a finite field.

\end{description}




\end{document}
