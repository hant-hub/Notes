\documentclass{report}
\usepackage[tmargin=2cm, rmargin=1in, lmargin=1in,margin=0.85in,bmargin=2cm,footskip=.2in]{geometry}
\usepackage{amsmath,amsfonts,amsthm,amssymb,mathtools}
\usepackage{enumitem}
\usepackage[]{mdframed}
\usepackage{tikz}
\usepackage{listings}

\title{\Huge{CSE 120}\\Day 4 Notes}
\author{\huge{Elijah Hantman}}
\date{}

\begin{document}
\maketitle
\newpage

\begin{description}
    \item Review: Iron Law
        \begin{gather}
            E = \frac{Instructions}{Program} \times \frac{Cycles}{Instruction} \times \frac{Seconds}{Cycle}\\
            E = \frac{CPI * IC}{CR}
        \end{gather}

        \begin{mdframed}
            The CPI will vary per instruction type. This implies the average or global CPI
            will vary between programs with different compositions of instruction types.
        \end{mdframed}

    \item Relative Performance

        $Performance = \frac{1}{\textrm{Execution Time}}$

        Relative Speedup $= Performance_{old} / Performance_{new}$

    \item Instruction Set Architecture

        ISA's define:
        \begin{itemize}
            \item Defines a protocol between hardware/software
            \item Defines the state of the system (registers, memory, etc)
            \item Functionality of each instruction
            \item encoding of each instruction
        \end{itemize}

        ISA's don't define:
        \begin{itemize}
            \item How instructions are implemented in hardware.
            \item How fast or slow instructions are
            \item how much power each instruction will cost.
        \end{itemize}

        A good ISA creates a stable platform that allows software and hardware
        to evolve seperately.

    \item Classifying ISAs
        \begin{itemize}
            \item Basic differentiation: are the ALU operands coming from
                registers or RAM?

            \item Stack ISA
            \item Accumulator ISA
            \item Register-Memory ISA
            \item Register-Register/load-store ISA
        \end{itemize}


        \begin{mdframed}
            {\Large Why registers and RAM and not just Registers?}

            Registers are made out of individual flip flops,
            each one is significantly faster than memory or
            RAM.

            Registers are not memory dense and are expensive.
            In addition each additional register requires more
            complicated hardware and ISAs to encode and manage.

            Physical distance is also a limiter on register speed.
        \end{mdframed}

        \begin{mdframed}
            {\Large Register-memory ISA}

            Can operate on both registers or memory, always outputs to a register.

            Only accepts one operand from memory and one operand from a register.
            Generally the convention for ISAs is the register output is first,
            then the register source, then the Memory source.

            An example add might look like
        \end{mdframed}

    \begin{lstlisting}[language={[x86asm]Assembler}]
            add Rd, Rs, M;
    \end{lstlisting}

        \begin{mdframed}
            {\Large Register-Register ISA}

            Only operates with registers.

            Requires explicit movement of values to and from memory.
            The memory writes and reads are directly exposed to the
            software.
        \end{mdframed}
        
        The ISA type is based on \textbf{Only} the ALU's relationship
        with memory. They do not affect other memory or external operations.

        Generally Register-Memory structures would be used for CISC computers,
        and Register-Register ISA's were used for RISC computers.

        RISC computers tend to require greater numbers of instructions.
        In the early days of computing this was a big deal and led to
        CISC architectures becoming more dominant.


    \item Instruction Length and Format
        \begin{itemize}
            \item Fixed Length ISAs

                Address of next instruction is easy to compute, however
                the code density suffers due to common and rare instructions
                requiring the same number of bits.

                It also has the benefit of easy prefetching.
            \item Variable Length ISAs

                Generally gives better code density, fewer memory fetches for
                instructions, more common instructions can be shorter.

                This makes the Fetching and decoding of instructions more difficult.
                You have to decode each instruction before you can fetch the next
                instruction.

                Prefetching is difficult and complicated.
        \end{itemize}

    \item CISC vs RISC
        \begin{itemize}
            \item CISC (Complex Instruction Set Computer)

                The goal is to get more done by adding sophisticated instructions.
                Shift the complexity onto the hardware in order to save on software.

                This includes ISAs like x86, IBM 360, Motorola 68K, etc.

                \begin{lstlisting}[language={[x86asm]Assembler}]
                        MUL mem2 <= mem0 * mem1
                \end{lstlisting}

            \item RISC (Reduced Instruction Set Computer)

                Goal is to simplify ISA to allow for simple and fast hardware. It moves
                the complexity onto the Software to allow for less complicated hardware.

                Examples include MIPS, SUN, Sparc, RISC-V, etc.
                \begin{lstlisting}[language={[x86asm]Assembler}]
                        LOAD  reg0 <= mem0
                        LOAD  reg1 <= mem1
                        MUL   reg0 <= reg0 * reg1
                        STORE mem2 <= reg0
                \end{lstlisting}
        \end{itemize}

        Generally both CISC and RISC architectures have adapted for different
        use cases. RISC instruction sets for high performance have had to add
        additional instruction for hardware acceleration.

        CISC computers use the concept of micro-ops to achieve more of the
        speed and parallelism that RISC lends itself to.

        Due to how complicated ISAs and hardware has become, the only reliable
        way of benchmarking is repeated measurement.


        \begin{tabular}{| c | c |}
            \hline
            CISC & RISC\\
            \hline
            HW difficult to implement & HW easy\\
            \hline
            Multi-cycle complex instructions & Simple, sincle clock instructions\\
            \hline
            Small Code Size & Large Code Size\\
            \hline
            High CPI & Low CPI\\
            \hline
            Variable Length instructions & Fixed instructions\\
            \hline
        \end{tabular}

        Over time, the gap between register and memory speed has grown,
        making keeping data in the registers more and more important.

    \item Intel/AMD's take on CISC

        \begin{itemize}
            \item x86 is a CISC ISA
            \item Internally each CISC instruction is converted to RISC micro-ops

                A single instruction may become dozens of micro ops

            \item Ideally you get the best of both worlds

                Small code size with RISC like simplicity. 

                Decoding is till massively complicated. 
        \end{itemize}

\end{description}


\end{document}
