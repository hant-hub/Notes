\documentclass{report}
\usepackage[tmargin=2cm, rmargin=1in, lmargin=1in,margin=0.85in,bmargin=2cm,footskip=.2in]{geometry}
\usepackage{amsmath,amsfonts,amsthm,amssymb,mathtools}
\usepackage{enumitem}
\usepackage[]{mdframed}
\usepackage{tikz}

\title{\Huge{CSE 120}\\Day 2 Notes}
\author{\huge{Elijah Hantman}}
\date{}

\begin{document}
\maketitle
\newpage

\begin{description}
    \item Levels of Transformation\\
        Ultimately all computations have to
        eventually become voltages and moving
        electrons.
        \begin{mdframed}
            How do we conceptualize
            and solve problems?
            
            \vspace{10}

            Create Levels of abstraction
            Compilers, asm, ISA, components
            micro-ops, etc.

            We create space between us and the computer
            by grouping together large groups of operations
            into abstractions, which have higher level, more
            informationally dense interfaces.
        \end{mdframed}

    \item What is architecture?
        \begin{mdframed}
            Architecture is ISA + microarchitecture.

            \vspace{10}

            The ISA is the set of assembly instructions.
            Informs how to format and pass instructions to
            a CPU.

            Goal is to open black box of hardware interaction
            with ISA.

            ISA should be viewed as a protocol, or language,
            or format for communicating with a CPU.

            ISA's tend to stick around since it would require
            a major amount of retooling in order to change
            the underlying ISA. This is because the ISA is
            the CPU, program interface, so it is relied upon
            by both the CPU and all programs written for that
            platform.
        \end{mdframed}
    \item What is microarchitecture?
        \begin{mdframed}
            How do we break up individual instructions?
            How do we divy up work between arithmetic,
            logic, and memory management units.

            Can we run different parts of instructions in
            parallel? Can we reorder operations to reduce
            latency?

            The microarchitecture has enough information to
            build a logical diagram, which can then be used
            to create circuit diagrams, which can then be
            cut into a chip using lithography.

            Microarchitecture is made of ALUs, cores,
            buses, etc.

            Logic is made up of gates, registers, etc.

            Circuits are made up of transistors, capacitors,
            etc.

            Process Node refers to average size of transistors.
            This usually is calculated through a transistor
            density per $mm^2$.
        \end{mdframed}
    \item What is performance?
        \begin{mdframed}
            Performance is about how fast a machine can
            execute code. How fast it can read, process and
            output. Number of operations per unit of time.


            Efficiency is a sister concept which is about
            the usage of resources for a given output.
            In terms of hardware that would be cooling
            requirements and power consumption. Cost is
            also a part of efficiency.
        \end{mdframed}

\end{description}




\end{document}
