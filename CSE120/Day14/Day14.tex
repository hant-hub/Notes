\documentclass{report}
\usepackage[tmargin=2cm, rmargin=1in, lmargin=1in,margin=0.85in,bmargin=2cm,footskip=.2in]{geometry}
\usepackage{amsmath,amsfonts,amsthm,amssymb,mathtools}
\usepackage{enumitem}
\usepackage[]{mdframed}
\usepackage{tikz}
\usepackage{listings}


\definecolor{codegreen}{rgb}{0,0.6,0}
\definecolor{codegray}{rgb}{0.5,0.5,0.5}
\definecolor{codepurple}{rgb}{0.58,0,0.82}
\definecolor{backcolour}{rgb}{0.95,0.95,0.92}

\lstdefinestyle{c_style}{
    language=C,
    backgroundcolor=\color{backcolour},   
    commentstyle=\color{codegreen},
    keywordstyle=\color{magenta},
    numberstyle=\tiny\color{codegray},
    stringstyle=\color{codepurple},
    basicstyle=\ttfamily\footnotesize,
    breakatwhitespace=false,         
    breaklines=true,                 
    captionpos=b,                    
    keepspaces=true,                 
    numbers=left,                    
    numbersep=5pt,                  
    showspaces=false,                
    showstringspaces=false,
    showtabs=false,                  
    tabsize=2
}

\lstdefinestyle{asm_style}{
    language=asm,
    backgroundcolor=\color{backcolour},   
    commentstyle=\color{codegreen},
    keywordstyle=\color{magenta},
    numberstyle=\tiny\color{codegray},
    stringstyle=\color{codepurple},
    basicstyle=\ttfamily\footnotesize,
    breakatwhitespace=false,         
    breaklines=true,                 
    captionpos=b,                    
    keepspaces=true,                 
    numbers=left,                    
    numbersep=5pt,                  
    showspaces=false,                
    showstringspaces=false,
    showtabs=false,                  
    tabsize=2
}


\renewcommand{\familydefault}{\sfdefault}


\title{\Huge{CSE 120}\\Day 14 Notes}
\author{\huge{Elijah Hantman}}
\date{}

\begin{document}
\maketitle
\newpage

\begin{description}
    \item {\huge DM Cache and Cache Writes} 
    \item {\large Handling a Cache Miss}
        \begin{itemize}
            \item If the cache reports a hit
                \\
                Continue as usual
            \item If the cache reports a miss
                \\
                Cache miss has to be done in collaboration
                with processor control unit that initiates the
                memory access and refills the cache.
            \item Cache misses create stalls while waiting for
                memory.
            \item Cache misses increase CPI.
        \end{itemize}
    \item {\large Handling Cache Writes}
        \begin{itemize}
            \item Writing causes additional issues
            \item Assume the address is already loaded in cache
            \item If we write a new value to that address we
                can store the new data in the cache and avoid
                a memory access.
        \end{itemize}
    \item {\large Problems with Cache Writing?}
        \begin{itemize}
            \item Cache and memory are desynced.
            \item How can we ensure subsequent loads return the
                right value?
        \end{itemize}
    \item {\large Write-Through Caches}
        \begin{itemize}
            \item Solves by forcing all writes to update both
                cache and main memory.
            \item Forces all writes to access memory\\
                No cache advantage.
        \end{itemize}
    \item {\large Write Buffers}
        \begin{itemize}
            \item Buffer allows devices to work at average
                data rate rather than peak.
            \item Write buffers stores data before being committed
                to memory, amortizes writes.
        \end{itemize}
    \item {\large Write Back Caches}
        \begin{itemize}
            \item Memory is not updated until a block needs
                to be replaced
            \item The highest location of a block has the 
                correct data.
            \item Data only moves down the memory hierarchy when
                a block is evicted.
            \item Makes cache misses more expensive
        \end{itemize}
    \item {\large Discussion}
        \begin{itemize}
            \item Requries dirty bit to keep track of writes
            \item Write back is more popular
            \item penalty is only applied to next memory access
            \item Automatically handles collating writes
        \end{itemize}
    \item {\large Write Miss}
        \begin{itemize}
            \item Second scenario is we try to write
                to an address not contained in the cache
            \item When we update main memory, should we copy
                into cache?
        \end{itemize}
    \item {\large Write Around}
        \begin{itemize}
            \item Writes go directly into memory
            \item goes around cache.
            \item Good if data is used once.
        \end{itemize}
    \item {\large Allocate on Write}
        \begin{itemize}
            \item Loads new data into cache
            \item If the data is needed again it will be
                avalible.
        \end{itemize}
    \item {\large Unified vs Split Cache}
        \begin{itemize}
            \item Split Cache
                \begin{itemize}
                    \item One cache for instructions
                    \item one cache for data
                    \item Allows for accessing both instructions
                        and data at the same time
                \end{itemize}
            \item Unified Cache
                \begin{itemize}
                    \item Single cache with both program and code
                    \item More flexible
                    \item Lower miss rate
                    \item Allows for more entries and does not
                        enforce as strict a structure.
                \end{itemize}
        \end{itemize}
    \item {\large Measuring Cache Performance}
        \begin{itemize}
            \item If assume reasonable buffer depth and significantly
                faster buffer to lower memory, stalls are negligible
            \item Most write through implementations read and write
                stalls into the same miss rate and penalty.
            \item Memory stall cycles = Accesses * miss Rate * Miss Penalty
        \end{itemize}
    \item {\large Average Memory Access Times}
        \begin{itemize}
            \item Large caches have large hit times
            \item Average memory access times accounts for hit
                time.
            \item AMAT = Hit Time + Miss rate * Miss Penalty
        \end{itemize}
    \item {\large Associative Caches}
        \begin{itemize}
            \item Fully Associative
                \begin{itemize}
                    \item Allow a given block to map to any
                        entry
                    \item Requires all entries to be searched
                        at once
                    \item Comparator per entry is expensive
                \end{itemize}
            \item N-way set associative
                \begin{itemize}
                    \item Each set contains n entries
                    \item Block number determines set
                    \item Search all entries in a set at once
                    \item n comparators is less expensive (scales with n)
                \end{itemize}
            \item Side notes:
                \begin{itemize}
                    \item 1-way associative cache is the same
                        as direct mapped cache.
                    \item If n is equal to the number of entries
                        an n-way associative cache is equivalent
                        to a fully associative cache.
                \end{itemize}
        \end{itemize}
    \item {\large How much Associativeity}
        \begin{itemize}
            \item Increased associativity increases hit rate
                but has diminishing returns
        \end{itemize}
        \pagebreak
    \item {\large Replacement Policy}
       \begin{itemize}
           \item Direct mapped has no options
           \item Set associative
               \begin{itemize}
                   \item prefer non-valid entry
                   \item if all entries are valid choose using
                       some policy
               \end{itemize}
           \item Least Recently Used Policy (LRU)
               \\
               Chose the least used value\\
               Probably some linear feedback shift register 
       \end{itemize} 
    \item {\large}
\end{description}



\end{document}
