\documentclass{report}
\usepackage[tmargin=2cm, rmargin=1in, lmargin=1in,margin=0.85in,bmargin=2cm,footskip=.2in]{geometry}
\usepackage{amsmath,amsfonts,amsthm,amssymb,mathtools}
\usepackage{enumitem}
\usepackage[]{mdframed}
\usepackage{tikz}
\usepackage{listings}
\usepackage{minibox}


\definecolor{codegreen}{rgb}{0,0.6,0}
\definecolor{codegray}{rgb}{0.5,0.5,0.5}
\definecolor{codepurple}{rgb}{0.58,0,0.82}
\definecolor{backcolour}{rgb}{0.95,0.95,0.92}

\lstdefinestyle{c_style}{
    language=C,
    backgroundcolor=\color{backcolour},   
    commentstyle=\color{codegreen},
    keywordstyle=\color{magenta},
    numberstyle=\tiny\color{codegray},
    stringstyle=\color{codepurple},
    basicstyle=\ttfamily\footnotesize,
    breakatwhitespace=false,         
    breaklines=true,                 
    captionpos=b,                    
    keepspaces=true,                 
    numbers=left,                    
    numbersep=5pt,                  
    showspaces=false,                
    showstringspaces=false,
    showtabs=false,                  
    tabsize=2
}

\lstdefinestyle{asm_style}{
    language=asm,
    backgroundcolor=\color{backcolour},   
    commentstyle=\color{codegreen},
    keywordstyle=\color{magenta},
    numberstyle=\tiny\color{codegray},
    stringstyle=\color{codepurple},
    basicstyle=\ttfamily\footnotesize,
    breakatwhitespace=false,         
    breaklines=true,                 
    captionpos=b,                    
    keepspaces=true,                 
    numbers=left,                    
    numbersep=5pt,                  
    showspaces=false,                
    showstringspaces=false,
    showtabs=false,                  
    tabsize=2
}




\title{\Huge{CSE 120}\\Day 8 Notes}
\author{\huge{Elijah Hantman}}
\date{}

\begin{document}
\maketitle
\newpage

\begin{description}
    \item Implementing Logic for Control 
        \begin{mdframed}
            Consider the hypothetical
            
            Memwrite is 1 if instruction 0 and 2 and not 5.
            
            Build using combinatorial logic.

            A control unit will consume an instruction and
            output a signal for which logic needs to execute.
        \end{mdframed}

    \item Datapath Delay
        \begin{mdframed}
            Assume time for stages is 100ps for register
            read or write

            200ps for other stages

            The Critical path is the longest delay path
            the minimum clock period is determined by critical
            path.

            If the critical path is 800ps, the maximum
            clock frequency is $\frac{1}{800ps}$
        \end{mdframed}

    \item Single Cycle processor
        \begin{mdframed}
            Pros
            \begin{itemize}
                \item Single cycle makes hardware simpl
            \end{itemize}

            Cons
            \begin{itemize}
                \item Cycle time is set by worst case
                \item Hardware is underutilized

                    ALU and memory only used for a fraction
                    of a clock cycle

                    Not well amortized

                \item Best possible CPI is 1
                \item Floating point instructions don't work
            \end{itemize}
        \end{mdframed}
    \item Pipelining
        \begin{mdframed}
            Analogy is doing laundry.

            Washer takes 30mins

            Dryer takes 40mins

            Folding takes 20mins

            Need to follow order wash $\to$ dry $\to$ fold for
            each laundry load

            Each resource can only process one task at a time,
            however you can have each resource operate on
            a different task concurrently.

            It is possible to complete four loads of laundry
            in 210 mins.
        \end{mdframed}
        \begin{mdframed}
            The potential speedup is proportaional to the
            number of pipeline stages, and the pipeline rate
            is limited by the slowest pipeline stage.

            In future $n$ is the number of instructions and
             $k$ is the number of stages.
        \end{mdframed}

        \begin{mdframed}
            Pipelining only improves throughput, which means
            the speed of dependent or single instructions
            is unchanged.
        \end{mdframed}

        \begin{mdframed}
            Pipelining a type of parallelism, it is temporal
            parallelism, the same hardware resource is shared
            between different tasks across time, it is
            not directly subject to Little's law.

            Pipelining is instruction level parallelism, ILP.
        \end{mdframed}

    \item 5 Stage RISC Pipeline
        \begin{mdframed}
           5 stages 
           \begin{itemize}
               \item Instruction fetch
               \item instruction decode and register read
               \item execute operation or calculate address
               \item access memory operand
               \item write result back to register
           \end{itemize}

           Overlap instruction in different stages
           \begin{itemize}
               \item all hardware is used all the time
               \item Clock cycle is fast (common case is faster)
           \end{itemize}
        \end{mdframed}
        \begin{mdframed}
            For a pipeline the execution time is
            \begin{gather}
               f(n, k) = kc + (n-1)c 
            \end{gather}
            Where $c$ is the clock cycle period.
            This also assumes maximum pipeline utilization
            and no dependency chains which would disrupt
            instruction flow. 

            \begin{displaymath}
                f(n,k) = k + (n-1)
            \end{displaymath}
            Is the number of cycles required.
        \end{mdframed}
        \begin{center}
            \minibox[frame]{
                cool projects would be architecture\\
                simulators.
            }
        \end{center}
    \item Speedup
        \begin{mdframed}
            \begin{displaymath}
                \frac{unpipelined}{pipelined}
            \end{displaymath}

            This will vary both on the number of hazards,
            $k$, $n$, and other issues.

            Pipelining \textbf{only} improves throughput.
            Viewed from Iron law it can reduce cycle
            time but increase CPI.
        \end{mdframed}
    \item How do you implement stages?
        \begin{mdframed}
            \begin{enumerate}
                \item Seperate into discrete stages
                \item Additional registers are used to hold
                    data between stages.
                \item Each pipeline register adds a one cycle
                    delay
            \end{enumerate}

            We need to delay register write backs until the
            last stage, so we need to pass the destination
            register through all the pipeline registers
            until we are ready to write it.

            For control you can pipeline them just like data.
            Additional complexity but overall a difference of
            quantity not quality difficulty wise.
        \end{mdframed}
        \pagebreak
    \item Data Hazards
        \begin{mdframed}
            A hazard is anything that prevents the next
            instruction from executing on the following clock
            cycle due to interdependence between the execution
            of consecutive instructions.
            \begin{itemize}
                \item Structural Hazards

                    Two or more instructions in the pipeline
                    require the same hardware resource at the
                    same time.

                    Comes up often with floating point units
                    as well as architectures where instructions
                    and data share a resource.

                    Multiple pipeline paths are one of the reasons
                    instructions might have different timings.
                    In addition, it creates the opportunity for
                    data hazards in the memory access and writeback
                    stages.

                    Processors which break the instruction order
                    is known as Out of Order Processors.
                    This only happens with multiple pipelines.
            \end{itemize}
        \end{mdframed}
\end{description}


\end{document}
