\documentclass{report}
\usepackage[tmargin=2cm, rmargin=1in, lmargin=1in,margin=0.85in,bmargin=2cm,footskip=.2in]{geometry}
\usepackage{amsmath,amsfonts,amsthm,amssymb,mathtools}
\usepackage{enumitem}
\usepackage[]{mdframed}
\usepackage{tikz}
\usepackage{listings}

\title{\Huge{CSE 120}\\Day 5 Notes}
\author{\huge{Elijah Hantman}}
\date{}

\begin{document}
\maketitle
\newpage

\begin{description}
    \item {\large RISC-V User State}
        \begin{mdframed}
            \begin{itemize}
                \item 32 general purpose registers
                \item x0 is always set to zero
                \item In addition to the general purpose registers there is a
                    PC register which stores the address of the current instruction
                \item Every register is 64 bit in the RV64I variant
            \end{itemize}

            \begin{lstlisting}[language = {[x86asm]Assembler}]
            add x3, x2, x1
        \end{lstlisting}

        add is the opcode, x3 is the destination register,
        x2 is the source register, and x1 is the second source
        register.

        \begin{itemize}
            \item Instructions are fetched from memory
            \item instructions are then fed directly into the
                cpu, with a specific binary format.
        \end{itemize}

        The Instruction cycle
        \begin{enumerate}
            \item First the PC increments and fetches the instruction
            \item Processor then decodes the instruction
            \item Logically execute the instruction
            \item Access memory to write results if required
            \item Write back to register if required
        \end{enumerate}

        The cycle is usually abbreiviated $F \rightarrow D \rightarrow X \rightarrow M \rightarrow W$
        The important thing to note is that memory writes always happen before register writes.
        \end{mdframed}
    \item {\large RV64I Names}
        \begin{mdframed}
            During execution you are free to use general registers however you want.

            However registers generally have conventions for how they are usually used.
            
            The ABI defines how data structures or routines are accessed in machine code.

            \begin{mdframed}
                Some examples are
                \begin{enumerate}
                    \item x1 is the ra register for a return address
                    \item x2 is the sp register for the stack pointer
                    \item x3 is gp, or the global pointer for global data
                    \item etc.
                \end{enumerate}
            \end{mdframed}

            Registers are a finite resource that needs to be managed

            The Goal is to keep as much data in the registers for as long
            as possible. Eventually data will be spilled from registers to
            memory. Arrays are generally too large to be kept in memory.

            Assembly programmers need to balance between fully utilizing
            registers and taking advantage of the space memory provides.
        \end{mdframed}
    \item {\large Compiling}
        \begin{mdframed}
            \begin{lstlisting}[language=C]
                a = b + c;
            \end{lstlisting}

            Could be compiled to

            \begin{lstlisting}[language={[x86asm]Assembler}]
            add x1, x2, x3  # a = b + c
            \end{lstlisting}

            Most basic operations like, and, or, nor, xor, etc.
            can be compiled to a single instruction.


            \begin{lstlisting}[language=C]
                a = b + c + d - e;
            \end{lstlisting}

            More complicated code is broken apart into multiple instructions.

            \begin{lstlisting}[language={[x86asm]Assembler}]
                add x1, x2, x3 # a = b + c
                add x1, x1, x4 # a = a + d
                sub x1, x1, x5 # a = a - e
            \end{lstlisting}

            A smart programmer or compiler can pick fast instructions or optimize for size
            and pack multiple operations into single instructions depending on the goal
            and the architecture.
        \end{mdframed}
    \item {\large Numbers Representation}
        \begin{mdframed}
            \begin{itemize}
                \item Unsigned: 0 and absolute values only, no negatives.

                \item Signed: 0, positive and negative values

                \item In RISC-V all numbers are signed unless specified otherwise

                \item Signing convention is in 2's complement

                \item The word size is defined to be 32 bits.

                \item You can load a byte, half word, word, and double word directly.

                \item RISC-V is byte addressable, the smallest unit of data that can be loaded is
                    a single byte.
            \end{itemize}
        \end{mdframed}
    \item {\large RISC-V constants}
        \begin{itemize}
            \item constant == immediate == literal == offset
            \item there are immediate versions of many instructions
            \item RISC-V only uses a 12 bit representation for immediates
                    
                This means that the total range is limited
        \end{itemize}
    \item {\large Memory Transfer}
        \begin{itemize}
            \item Load always pulls from memory
            \item store always pushes data to memory
            \item all memory interactions in RISC-V happen through
                load and store instructions.

            \item floating-point loads and stores work with specialized
                floating point registers.
        \end{itemize}

        \begin{lstlisting}[language={[x86asm]Assembler}]
            ld dst, offset(base)
        \end{lstlisting}

        When loading from memory, effective addresses are
        computed. An effective address is an aligned byte
        address.

        The offset and base address are used for array operations,
        allowing for indexing into regions of memory.

        For a given load operation, it must be aligned to that
        memory load size. lb must be byte aligned, lh must be
        half-word aligned, lw must be word aligned, ld must
        be double word aligned.

        Misaligned data usually causes an exception on most RISC-V
        processors.

        For caching, most systems will force alignement for speed
        and efficiency. This divides memory into cache-lines. If
        memory is misaligned it could be split among multiple
        cache lines.

    \item {\large C Data load example}
        \begin{lstlisting}[langauge = C]
            a = b + *c;
        \end{lstlisting}

        C is a memory access so it always requires a load.

        \begin{lstlisting}[language = {[x86asm]Assembler}]
            ld x5, 0(x3)     # x5 = *c
            add x1, x2, x5   # a = b + x5
        \end{lstlisting}


    \item {\large Memory Storing}
        \begin{lstlisting}[language = {[x86asm]Assembler}]
            sd src, offset(base)
        \end{lstlisting}

        Offset value is 12-bit signed constant.\\
        Works identically to loading data but in reverse
    \item {\large C Data store example}
        \begin{lstlisting}[langauge = C]
            *a = b + c;
        \end{lstlisting}
        

        \begin{lstlisting}[language = {[x86asm]Assembler}]
            add x1, x2, x5   # x1 = b + c
            sd x1, 0(x3)     # *a = x1
        \end{lstlisting}
    \item {\large RISC-V Binary Format}
        \begin{mdframed}
            \begin{description}
                \item R type, non immediate
                    \begin{itemize}
                        \item first 7 bits encode additional opcode info
                        \item next 10 bits encode the two source operands
                        \item 3 bits are used for additional op code field
                        \item next 5 bits encode the destination operand
                        \item next 5 bits are for the destination operand
                        \item last 7 bits are for the opcode
                    \end{itemize}
                \item Immediate I type
                    \begin{itemize}
                        \item first 12 bits are instead used for the immediate
                    \end{itemize}
                \item Immediate S/B type
                    \begin{itemize}
                        \item The first 7 bits, as well as the return operand are
                            used for the immediate.

                        \item For only S/B types the immediate is a single value but
                            split between two bit fields.
                    \end{itemize}
                \item Immediate U/J type
                    \begin{itemize}
                        \item The first 20 bits are used for the immediate
                    \end{itemize}
            \end{description}

            When decoding the first step is to parse the opcode.
            The opcode will specify how to split the rest of the
            bits.

            The opcode can only specify 128 instructions directly,
            there are and additional 10 bits which can be used for
            the R type instructions.
        \end{mdframed}
\end{description}


\end{document}
