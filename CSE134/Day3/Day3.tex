\documentclass{report}
\usepackage[tmargin=2cm, rmargin=1in, lmargin=1in,margin=0.85in,bmargin=2cm,footskip=.2in]{geometry}
\usepackage{amsmath,amsfonts,amsthm,amssymb,mathtools}
\usepackage{enumitem}
\usepackage[]{mdframed}
\usepackage{tikz}
\usepackage{listings}


\definecolor{codegreen}{rgb}{0,0.6,0}
\definecolor{codegray}{rgb}{0.5,0.5,0.5}
\definecolor{codepurple}{rgb}{0.58,0,0.82}
\definecolor{backcolour}{rgb}{0.95,0.95,0.92}

\lstdefinestyle{c_style}{
    language=C,
    backgroundcolor=\color{backcolour},   
    commentstyle=\color{codegreen},
    keywordstyle=\color{magenta},
    numberstyle=\tiny\color{codegray},
    stringstyle=\color{codepurple},
    basicstyle=\ttfamily\footnotesize,
    breakatwhitespace=false,         
    breaklines=true,                 
    captionpos=b,                    
    keepspaces=true,                 
    numbers=left,                    
    numbersep=5pt,                  
    showspaces=false,                
    showstringspaces=false,
    showtabs=false,                  
    tabsize=2
}

\lstdefinestyle{asm_style}{
    language=asm,
    backgroundcolor=\color{backcolour},   
    commentstyle=\color{codegreen},
    keywordstyle=\color{magenta},
    numberstyle=\tiny\color{codegray},
    stringstyle=\color{codepurple},
    basicstyle=\ttfamily\footnotesize,
    breakatwhitespace=false,         
    breaklines=true,                 
    captionpos=b,                    
    keepspaces=true,                 
    numbers=left,                    
    numbersep=5pt,                  
    showspaces=false,                
    showstringspaces=false,
    showtabs=false,                  
    tabsize=2
}




\title{\Huge{CSE 134 - Embedded OS}}
\author{\huge{Elijah Hantman}}
\date{}

\begin{document}
\maketitle
\newpage

\begin{description}
    \item Review 
        \begin{itemize}
            \item Resource Manager
                \begin{itemize}
                    \item I/O
                    \item Virtualization
                    \item Memory (space sharing)
                    \item CPU (time sharing) (space sharing for multicores)
                \end{itemize}
            \item Hardware Abstraction
                \begin{itemize}
                    \item Interface
                    \item Kernel vs User
                        \begin{itemize}
                            \item Mediator (many users)
                            \item Security (untrusted code)
                        \end{itemize}
                    \item Hardware Specifics
                \end{itemize}
            \item CPU only touches memory
            \item I/O modes
                \begin{itemize}
                    \item Polling
                        \begin{mdframed}
                            Professor prefers the conceptualization
                            of waiting on input rather than flushing a buffer
                            or occasionally reading in the data. I think this is
                            a little incorrect when looking at how these things
                            tend to be implemented, like you don't just hang until
                            the entire message is brought through, you have to periodically
                            read data.

                            CPU intensive.
                        \end{mdframed}
                    \item Interrupts
                        \begin{mdframed}
                            Interrupts is used for low throughput low
                            latency applications. Things like keyboard or mouse input,
                            microphone samples, etc.

                            Requires Hardware support.
                        \end{mdframed}
                        \begin{mdframed}
                            Hardware signal causes CPU to execute interrupt handler located at
                            specific memory address. Kernel mode operation, forces kernel mode
                            and requires context switch. Many operations are dangerous and invalid
                            in this mode.

                            Modern interrupt handlers have a table of pointers to various handlers,
                            the specific hardware interrupt is a number which causes the cpu to read
                            from a specific entry in the table and jump to the correct interrupt handler.

                            Modern x86 has a register which holds the address of this table allowing
                            for it to be movable in memory.

                            Interrupts must be acknowledged before another interrupt can happen.

                            (This happens because two interrupts happening at once could wipe the
                            necessary information to correctly resume execution)
                        \end{mdframed}
                    \item DMA
                        \begin{mdframed}
                            Super high bandwidth comparitively. Used for things like GPU
                            interfacing, disk I/O, etc. Usually more performant because
                            of things like avoiding context switches and syscalls.

                            Requires Hardware support and is usually platform specific.
                        \end{mdframed}
                \end{itemize}
        \end{itemize}
    \item Concepts cont.
        \begin{itemize}
            \item Process
                \begin{mdframed}
                    Program in Execution, contains/lives in:
                    \begin{itemize}
                        \item Address Space
                        \item File Descriptor Table
                        \item CPU state (registers, stack, etc.)
                        \item Process state (Running, Zombie, Ready, Block, Quit, etc)
                    \end{itemize}
                \end{mdframed}
            \item Process Tree
                \begin{itemize}
                    \item Processes create Child Processes which share process space
                    \item Creates Tree structure
                \end{itemize}
            \item File Directory
                \begin{itemize}
                    \item Hierarchy of directories with leaves pointing at the files on disk
                    \item Also known as Inodes these are usually OS independent and are a
                        type of protocol
                \end{itemize}
            \item Special Files
                \begin{itemize}
                    \item Special files which work with the OS file API but
                        have special behavior
                    \item These are abstract I/O interfaces, things like networks
                        other programs, device brightness, etc.
                \end{itemize}
            \item I/O Protection Shell
                \begin{itemize}
                    \item Bits which tell the OS what specific users can do with a file
                    \item three sets, user, group, root
                    \item Controls read/write/execute
                \end{itemize}
            \item System Calls
                \begin{itemize}
                    \item Interface between user programs and Kernel.
                    \item Issued by user programs
                    \item Library routine will create an Exception (software interrupt)
                        and jump to the Syscall handler
                    \item Handler will do some work based on the system call api defined
                        by the OS, and will then return back when finished.
                        \begin{mdframed}
                            This is kinda true, may be subject to scheduling and time
                            sharing policies. A syscall is usually the end of a CPU Burst
                            and will allow the kernel to move the process into the BLOCKED
                            state.
                        \end{mdframed}
                \end{itemize}
            \item Operating Systems Structure
                \begin{itemize}
                    \item Micro Kernels
                        \begin{itemize}
                            \item Small number of processes
                            \item Mechanism in Kernal, policy outside
                            \item Minimize effects of bugs
                        \end{itemize}
                    \item Monolithic Systems
                        \begin{itemize}
                            \item Main program invokes requested service procedure
                            \item Set of procedures carry out system calls
                            \item Utility procedures help service procedures
                        \end{itemize}
                \end{itemize}
        \end{itemize}
\end{description}


\end{document}

