\documentclass{report}
\usepackage[tmargin=2cm, rmargin=1in, lmargin=1in,margin=0.85in,bmargin=2cm,footskip=.2in]{geometry}
\usepackage{amsmath,amsfonts,amsthm,amssymb,mathtools}
\usepackage{enumitem}
\usepackage[]{mdframed}
\usepackage{tikz}
\usepackage{listings}


\definecolor{codegreen}{rgb}{0,0.6,0}
\definecolor{codegray}{rgb}{0.5,0.5,0.5}
\definecolor{codepurple}{rgb}{0.58,0,0.82}
\definecolor{backcolour}{rgb}{0.95,0.95,0.92}

\lstdefinestyle{c_style}{
    language=C,
    backgroundcolor=\color{backcolour},   
    commentstyle=\color{codegreen},
    keywordstyle=\color{magenta},
    numberstyle=\tiny\color{codegray},
    stringstyle=\color{codepurple},
    basicstyle=\ttfamily\footnotesize,
    breakatwhitespace=false,         
    breaklines=true,                 
    captionpos=b,                    
    keepspaces=true,                 
    numbers=left,                    
    numbersep=5pt,                  
    showspaces=false,                
    showstringspaces=false,
    showtabs=false,                  
    tabsize=2
}

\lstdefinestyle{asm_style}{
    language=asm,
    backgroundcolor=\color{backcolour},   
    commentstyle=\color{codegreen},
    keywordstyle=\color{magenta},
    numberstyle=\tiny\color{codegray},
    stringstyle=\color{codepurple},
    basicstyle=\ttfamily\footnotesize,
    breakatwhitespace=false,         
    breaklines=true,                 
    captionpos=b,                    
    keepspaces=true,                 
    numbers=left,                    
    numbersep=5pt,                  
    showspaces=false,                
    showstringspaces=false,
    showtabs=false,                  
    tabsize=2
}




\title{\Huge{CSE 134 - Embedded OS}}
\author{\huge{Elijah Hantman}}
\date{}

\begin{document}
\maketitle
\newpage

\large{Operating Systems}
\begin{description}
    \item Why Operating System? 
        \begin{mdframed}
            \begin{itemize}
                \item Lots of types of hardware
                \item Different Instruction sets
                \item Possibly multiple programs on the same computer
                \item timesharing
                \item Concurrency
                \item Input/output devices
                    \begin{itemize}
                        \item Mice
                        \item keyboards
                        \item speakers
                        \item monitors
                        \item Networking
                        \item Printers
                        \item Disk
                    \end{itemize}
            \end{itemize} 
            An Operating System's job is to give a
            common layer for multiple types of programs to
            run on top of.

            The part of an Operating System which manages hardware
            access is called the Resource Manager
        \end{mdframed}
    \item What does an operating system do? 
        \begin{mdframed}
            They ensure the hardware is connected,
            they translate the raw output of hardware
            into understandable format.
        \end{mdframed}
    \item Where is the Software?
        \begin{mdframed}
            \begin{itemize}
                \item On top we have User Mode. This includes all applications,
                    games, etc.
                \item User Mode talks to Kernel mode software to interface with hardware.
                    Kernel mode is where the operating system lives, and is responsible
                    for directly handling hardware. This is also where device drivers
                    live, and any programs which directly edit memory, set permissions,
                    etc.
            \end{itemize}
        \end{mdframed}
        \begin{center}
            \begin{mdframed}
                Fun fact, for desktop systems,
                generally they have hardware support for
                kernel mode. ie: They allow editing special memory
                like what code is run when specific hardware things happen.
                as well as editing things like the virtual page tables used by the
                cpu to virtualize memory.
            \end{mdframed}
        \end{center}
    \item Resource Manager
        \begin{itemize}
            \item Time(cpu) and space(memory) sharing
            \item Split CPU time between multiple concurrent tasks
            \item Split memory between multiple concurrent processes
            \item Split I/O between multiple processes
        \end{itemize}
    \item Operating System is a Hardware Abstraction and Resource Manager
    \item Computer Hardware Review
        \begin{itemize}
            \item Two busses, memory and control
            \item Bus connects all hardware
            \item CPU has memory management unit to control the bus and
                manage the CPU cache and i/o.
            \item I/O covers all components except memory. CPU only
                has Memory management unit since the I/O is mapped
                into memory from the perspective of a program.
            \item One of the jobs of the MMU is to allow the CPU to
                interpret I/O as volitile memory
            \item CPU only interacts with I/O via memory as an intermediary
        \end{itemize}
    \item CPU Pipelining
        \begin{itemize}
            \item Pipelines allow for temporal parallelism on a hardware
                level.
            \item Can multiply the amount of work a CPU can do
            \item Increases Throughput without changing latency
            \item Pipelines can be linear or they can be "superscalar"
                and include spatial parallelism in addition to the
                temporal parallelism inherent to pipelines.
                \begin{mdframed}
                    Intel and AMD have a concept of ports which
                    can perform different types of instructions.
                    Instructions are also decomposed into micro instructions
                    which are then split between ports and executed.
                \end{mdframed}
        \end{itemize}
    \item Memory Hierarchy
        \begin{itemize}
            \item We use a hierarchy because memory speed is inversely
                related to its density. To get advantages of fast memory
                like SRAM and the large size of DRAM, Magnetic Disk, Magnetic Tape,
                we use both kinds and move the memory we need into the fast
                memory and move memory out to disk to save space.
            \item RAM $\to$ Random Access Memory
                \begin{mdframed}
                    Can be read anywhere at all time, can be written to and
                    read from. Usually volitile.
                \end{mdframed}
            \item ROM $\to$ Read Only Memory
            \item EEPROM $\to$ Eletronically Enabled Programable Read Only Memory
        \end{itemize}
    \item Multithreaded and Multicore Chips
        \begin{itemize}
            \item Often cores share L2 and L3 cache.
            \item On GPUs groups of 16-32 cores will share memory, as well
                as the global memory shared by all cores.
                \pagebreak
                \begin{mdframed}
                    Fun fact, Intel initially used a shared L2 cache design,
                    and AMD originally used a split L2 design.

                    AMD usually turned out faster. But Intel had the benefit
                    of using a large L2 cache which reduced the complexity of
                    synchronizing the caches and ensuring memory was not corrupted.

                    MESI, MSI, etc.

                    AMD is faster for many different programs which do not have
                    shared memory. Intel is faster for using multiple cores working
                    on the same memory.
                \end{mdframed}
        \end{itemize}
    \item Disk
        \begin{itemize}
            \item Magnetic disks, come in a large stack. Single write head for each
                surface, ie: 2 per disk.
            \item Segment on the first surface is where the hardware will look for
                instructions on start up.
            \item Supernode controls manages mapping each disk to a number for
                reads and writes.
        \end{itemize}
    \item Device Drive
        \begin{itemize}
            \item Software which interfaces with some hardware.
            \item Complicated usually
                \begin{mdframed}
                    Converts read commands to commands the specific hardware understands.

                    For a HDD it converts memory locations into arm movements and disk
                    number.
                \end{mdframed}
            \item Different Modes
                \begin{itemize}
                    \item Polling
                        \begin{mdframed}
                            Write information into memory and occasionally check.
                            Works both ways, software might write memory and
                            hardware occasionally reads.

                            This is often used for Games to read Input, every
                            frame the game will check for input and then update
                            it's internal tracking.
                        \end{mdframed}
                    \item Interrupt
                        \begin{mdframed}
                            Send special signal to the CPU which
                            forces CPU to execute specific code to handle
                            input.
                        \end{mdframed}
                    \item DMA
                        \begin{mdframed}
                            Directly map device information into memory.
                            Have some kind of mapping between memory addresses
                            and the device directly.

                            This is for things like GPUs or sometimes memory
                            mapped I/O which is a Userland abstraction.
                        \end{mdframed}
                \end{itemize}
            \item I/O via Polling 
                \begin{itemize}
                    \item Driver issues commands
                    \item Driver keeps checking on device until it is ready
                    \item Big Use of CPU
                    \item Called Programmed I/O, not common anymore
                \end{itemize}
            \item Interrupt I/O
                \begin{itemize}
                    \item CPU is working and doesn't check
                    \item Device sends signal to CPU which causes
                        it to run special Interrupt handler
                    \item Driver then Sends commands and handles input
                        before returning to normal execution
                    \item Requires CPU hardware support, usually
                        the return address is kept in special register
                    \item Requires special handling to store register state
                \end{itemize}
            \item DMA I/O
                \begin{itemize}
                    \item The Device directly writes into memory
                    \item CPU interaction is kept to minimum
                \end{itemize}
        \end{itemize}
\end{description}


\end{document}

