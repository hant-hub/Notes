\documentclass{report}
\usepackage[tmargin=2cm, rmargin=1in, lmargin=1in,margin=0.85in,bmargin=2cm,footskip=.2in]{geometry}
\usepackage{amsmath,amsfonts,amsthm,amssymb,mathtools}
\usepackage{enumitem}
\usepackage[]{mdframed}
\usepackage{tikz}
\usepackage{listings}


\definecolor{codegreen}{rgb}{0,0.6,0}
\definecolor{codegray}{rgb}{0.5,0.5,0.5}
\definecolor{codepurple}{rgb}{0.58,0,0.82}
\definecolor{backcolour}{rgb}{0.95,0.95,0.92}

\lstdefinestyle{c_style}{
    language=C,
    backgroundcolor=\color{backcolour},   
    commentstyle=\color{codegreen},
    keywordstyle=\color{magenta},
    numberstyle=\tiny\color{codegray},
    stringstyle=\color{codepurple},
    basicstyle=\ttfamily\footnotesize,
    breakatwhitespace=false,         
    breaklines=true,                 
    captionpos=b,                    
    keepspaces=true,                 
    numbers=left,                    
    numbersep=5pt,                  
    showspaces=false,                
    showstringspaces=false,
    showtabs=false,                  
    tabsize=2
}

\lstdefinestyle{asm_style}{
    language=asm,
    backgroundcolor=\color{backcolour},   
    commentstyle=\color{codegreen},
    keywordstyle=\color{magenta},
    numberstyle=\tiny\color{codegray},
    stringstyle=\color{codepurple},
    basicstyle=\ttfamily\footnotesize,
    breakatwhitespace=false,         
    breaklines=true,                 
    captionpos=b,                    
    keepspaces=true,                 
    numbers=left,                    
    numbersep=5pt,                  
    showspaces=false,                
    showstringspaces=false,
    showtabs=false,                  
    tabsize=2
}




\title{\Huge{CSE 134 - Embedded OS}}
\author{\huge{Elijah Hantman}}
\date{}

\begin{document}
\maketitle
\newpage

\begin{itemize}
    \item What is operating system
        \begin{mdframed}
            Abstraction and resource manager. Provides clean beatiful
            APIs for ugly hardware.

            Time sharing + space sharing, (CPU + memory)
        \end{mdframed}
    \item Computer Architecture
        \begin{mdframed}
            CPU, Memory, I/O.

            I/O includes printers, screens, disk, any thing
            that isn't memory or CPU which the computer can interact
            with.


            What is an interrupt? What is DMA?

            Interrupts are hardware signals which force the CPU to execute
            specific instructions. They are used because sometimes we need
            to notify the CPU of an event.

            Different modes. Polling (I/O request), Interrupts (Interrupt pins),
            DMA (direct memory access).

            \begin{itemize}
                \item DMA allows for no CPU usage, high throughput.
                    Large data, ex: screens, GPUs, etc.
                \item Polling, requires no hardware support but requires busy-waiting,
                    and is blocking.
                \item Interrupts do not require busy waiting but are low throughput,
                    force a context switch, and requires specific hardware.

                    Software can also use interrupts, these are called system calls,
                    and force the CPU to begin running kernel code in kernel mode.
            \end{itemize}

            CPU pipelining, improves throughput but not latency. Allows for temporal
            parrellelism which allos heterogenous tasks to occur simultaneously.

            Use a buffer?! Idk man she doesn't explain very well.
        \end{mdframed}
        \begin{mdframed}
           Handling interrupts 

           Switches to kernel mode, uses a table to jump to interrupt
           handler. It then runs the interrupt code, which promises to save
           the CPU state (registers + memory). Then it runs in kernel mode until
           eventually it switches back into the process and restores the CPU
           state. This is also when the CPU scheduler will run for timesharing.
           I/O syscalls will cause the thread to be blocked and it will wait in a
           queue until it is ready to run again.
        \end{mdframed}
    \item Processes
        \begin{itemize}
            \item Process is a program in execution
                \begin{mdframed}
                    Data, memory, file descriptors.

                    Binary program, instruction stream.
                    Status (Running, Ready, Blocked), stack, which is part
                    of the memory of the program.

                    It also has other resources which in Linux are usually represented by file
                    descriptors.

                    Memory means an address space, which is a mapping from the addresses
                    emitted by the running process to physical addresses which can be
                    interpreted by the physical memory.

                    Address space = location in memory.
                \end{mdframed}
            \item What is a thread?
                \begin{mdframed}
                    Threads are CPU execution states. Threads have their own stacks but
                    share an address space with their parent process.

                    Thread context switches are cheaper since it doesn't have to switch
                    the page table.

                    Running to blocked when I/O (polling or interrupt)

                    Blocked to Ready when I/O finishes (polling or interrupt)

                    Ready to Running when Scheduled

                    Running to Ready when non blocking request?

                    Running to Ready when preemption (round robin)
                \end{mdframed}
            \item What is a context switch
                \begin{mdframed}
                    When the CPU changes from a running process to another
                    process. It also includes switches from User space to Kernel mode.
                \end{mdframed}
            \item Scheduler Metrics
                \begin{mdframed}
                    Utilization
                    \begin{displaymath}
                        U = 1 - (\prod u_i)
                    \end{displaymath}
                    
                    Waiting Time: Queue waiting time

                    This is the amount of time spent waiting in the
                    ready queue.

                    Turnaround time: the time from the start of a process
                    to its termination. It is the sum of all the waiting times
                    of each process.

                \end{mdframed}
            \item Interprocess Communication
                \begin{mdframed}
                    Race condition is a condition where two processes
                    can both access some shared resource and can cause
                    unwanted behavior.

                    Critical region is the region of code where the process
                    or thread is accessing the shared resource.

                    Mutex and semaphores are best solution alongside disabling interrupts.
                    Disabling interrupts is dangerous because it give user space too much power.

                    Mutex is a single integer. Semaphores are an incrementing integer.

                \end{mdframed}
            \item Deadlock
                \begin{itemize}
                    \item Mutual exclusion
                    \item Hold and wait
                    \item Non preemption
                    \item circular wait
                \end{itemize}
                \begin{mdframed}
                    Detect and recover,
                    Dynamic avoidance, 
                    prevention.

                    Prevention, invalidate any of the four conditions.
                    Allow multiple access, allow request everything,
                    allow preemption, numeric order.

                    Resouce allocation graph is a way to detect circular wait
                    conditions.
                \end{mdframed}
        \end{itemize}
\end{itemize}


\end{document}

