\documentclass{report}
\usepackage[tmargin=2cm, rmargin=1in, lmargin=1in,margin=0.85in,bmargin=2cm,footskip=.2in]{geometry}
\usepackage{amsmath,amsfonts,amsthm,amssymb,mathtools}
\usepackage{enumitem}
\usepackage[]{mdframed}
\usepackage{tikz}
\usepackage{listings}


\definecolor{codegreen}{rgb}{0,0.6,0}
\definecolor{codegray}{rgb}{0.5,0.5,0.5}
\definecolor{codepurple}{rgb}{0.58,0,0.82}
\definecolor{backcolour}{rgb}{0.95,0.95,0.92}

\lstdefinestyle{c_style}{
    language=C,
    backgroundcolor=\color{backcolour},   
    commentstyle=\color{codegreen},
    keywordstyle=\color{magenta},
    numberstyle=\tiny\color{codegray},
    stringstyle=\color{codepurple},
    basicstyle=\ttfamily\footnotesize,
    breakatwhitespace=false,         
    breaklines=true,                 
    captionpos=b,                    
    keepspaces=true,                 
    numbers=left,                    
    numbersep=5pt,                  
    showspaces=false,                
    showstringspaces=false,
    showtabs=false,                  
    tabsize=2
}

\lstdefinestyle{asm_style}{
    language=asm,
    backgroundcolor=\color{backcolour},   
    commentstyle=\color{codegreen},
    keywordstyle=\color{magenta},
    numberstyle=\tiny\color{codegray},
    stringstyle=\color{codepurple},
    basicstyle=\ttfamily\footnotesize,
    breakatwhitespace=false,         
    breaklines=true,                 
    captionpos=b,                    
    keepspaces=true,                 
    numbers=left,                    
    numbersep=5pt,                  
    showspaces=false,                
    showstringspaces=false,
    showtabs=false,                  
    tabsize=2
}




\title{\Huge{CSE 134 - Embedded OS}}
\author{\huge{Elijah Hantman}}
\date{}

\begin{document}
\maketitle
\newpage

\begin{description}
    \item Review 
        \begin{itemize}
            \item Purpose
                \begin{mdframed}
                    Hardware Abstraction\\
                    Resource Manager
                \end{mdframed}
            \item What is Time Sharing
                \begin{enumerate}
                    \item CPU
                    \item I/O Resources
                    \item Memory Bus??
                \end{enumerate}
            \item What is Space Sharing
                \begin{enumerate}
                    \item RAM (Random Access Memory)
                    \item ROM (Read Only Memory)
                    \item EPROM (Electrongic Programmable Read Only Memory)
                \end{enumerate}
            \item What is a trap instruction?
                \begin{mdframed}
                    Software Interrupt, switch from User to Kernel
                    Mode.
                \end{mdframed}
            \item What is an interrupt?
                \begin{mdframed}
                    It is a hardware component which causes
                    the CPU to change modes and run an interrupt handler
                \end{mdframed}
            \item What is the CPU pipeline?
                \begin{mdframed}
                    Instruction level temporal parrallelism.
                    Fetch, Decode Execute.

                    Fetch, Decode, ALU, Memory, Write back

                    FDEMW
                \end{mdframed}
            \item What is Virtual Memory?
                \begin{mdframed}
                    Virtual memory is a hardware feature which translates
                    memory reads and writes to a physical memory address.

                    We 'emulate' a memory address space which looks identical
                    regardless of the underlying physical memory utilization.
                \end{mdframed}
            \item What is Process?
                \begin{mdframed}
                    Program in Execution, Code and memory and cpu state.
                    The combination of these elements of computation form a 
                    process.

                    In modern Operating systems Code refers to an instruction stream,
                    memory refers to some address space which is mapped to some physical
                    locations on the physical memory component, state refers to the
                    registers as well as some information about I/O and data transfer.
                     
                    READY, RUNNING, BLOCKED. All are orthogonal execpt that ready cannot
                    go to blocked.

                    File descriptors.

                    Process permissions, ID, ring level, etc.

                    I guess she wants that Virtual memory is an abstraction over physical
                    memory. I think that is unclear, she should ask "what does it mean to
                    virtualize memory?" Which would have been much better.
                \end{mdframed}
            \item What is an I-Node?
                \begin{mdframed}
                    Data structure which makes up the hierarchy of the filesystem.
                    Contains all relevant metadata of a file except for its filename. 

                    The filename is stored in the inode of the directory which points
                    at the file.

                    A directory is a table which contains i-nodes that point at it's contents.
                    Essentially there is a massive table, of i-nodes, which point at each other,
                    and at different locations in memory or disk. These locations are called
                    Block Addresses.
                \end{mdframed}
        \end{itemize}
\end{description}


\end{document}

