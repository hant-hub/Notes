\documentclass{report}
\usepackage[tmargin=2cm, rmargin=1in, lmargin=1in,margin=0.85in,bmargin=2cm,footskip=.2in]{geometry}
\usepackage{amsmath,amsfonts,amsthm,amssymb,mathtools}
\usepackage{enumitem}
\usepackage[]{mdframed}
\usepackage{tikz}

\title{\Huge{ECE 30}}
\author{\huge{Elijah Hantman}}
\date{}

\begin{document}
\maketitle
\newpage

{\huge Engineering Principles of Electronics}

\begin{description}
    \item Introduction 
        \begin{mdframed}
            Taught by Paul Vroomen\\
            Office Hours in E2-280\\
            Office Hours: Wed. 10am - 12pm\\
            Email: pvroomen@ucsc.edu
        \end{mdframed}
        \begin{mdframed}
            TA Office Hours:
            \begin{itemize}
                \item Baskin, Thursdays 12am-2pm
                \item BE-230, Mondays 10am-12am
                \item idk, Fridays 10:45am-12:45pm
            \end{itemize}
        \end{mdframed}
        \begin{mdframed}
            \begin{itemize}
                \item Class Website -> Canvas
                \item Important Announcements
                \item Everything basically
            \end{itemize}
        \end{mdframed}
    \item Overview of Course
        \begin{mdframed}
            \begin{itemize}
                \item Four Applications of Engineering
                    Principles in mobile phone and computers.
                    \begin{enumerate}
                        \item Accelerometers
                        \item Touchscreens
                        \item LTE, 5G communications
                        \item Data Processing Chips
                    \end{enumerate}
                \item For each we will review physics + how to
                    apply principles.
            \end{itemize}
        \end{mdframed}
        \begin{mdframed}
            \begin{itemize}
                \item Textbook required
                \item Older text for comp sci
                \item e-Text is avalible on most book websites
                \item Syllabus lists which sections are assigned
                    per lecture
            \end{itemize}
        \end{mdframed}
        \begin{mdframed}
            \begin{itemize}
                \item Recommended textbook
                \item Basic engineering principles
                \item Handouts from sections of the book will
                    be used
            \end{itemize}
        \end{mdframed}
        \pagebreak
        \begin{mdframed}
            {\huge Grading}\\
            {\large Quizzes}
            \begin{itemize}
                \item 4 quizzes
                \item Most material covered via quizzes
                \item 10 multiple choise questions +
                    bonus question
                \item Open book 
                \item Avalible from 2pm to midnight
                \item 15 mins once started
            \end{itemize}
            {\large Exams}
            \begin{itemize}
                \item Midterm 1 hour
                \item Solve problems from lectures 1-14
            \end{itemize}
            \begin{itemize}
                \item Final 3 hours
                \item Questions from full course
            \end{itemize}
            {\large Homework}
            \begin{itemize}
                \item 8 sets total
                \item Solve questions using principles discussed
                    in class
                \item Must be submitted by midnight on following
                    Sunday
                \item May work in teams of up to 3
                \item Must list names of other students in team
            \end{itemize}
            {\large Late Submissions}
            \begin{itemize}
                \item HW submissions are due by midnight
                \item 50\% late penalty (2\% of final)
                \item $>$24 hours late will not be graded
                \item Some exceptions
                    \begin{itemize}
                        \item Serious illness or emergency
                        \item Personal, family, or other crisis
                        \item Problems should be reported
                            immediately
                    \end{itemize}
            \end{itemize}
            {\large Breakdown}
            \begin{itemize}
                \item Quizzes 20\%
                \item Homework 30\%
                \item Midterm Exam 20\%
                \item Final 30\%
            \end{itemize}
        \end{mdframed}
        \pagebreak
    \item Concepts, Symbols, Quantities, and Units
        \begin{mdframed}
            {\large What does "Engineering Principles of 
            Electronics" Means?}
            \begin{itemize}
                \item About leverage and Application
                \item About Practices
                    \begin{mdframed}
                        ex: For integrated circuites through
                        trial and error we have learned we need
                        an ISO1 clean room to ensure acceptable
                        quality.

                        ISO1 designates a maximum number of
                        particles per $m^3$.
                        \begin{itemize}
                            \item 10 $>$ 0.1 um
                            \item 2 $>$ 0.2 um
                            \item 0 $>$ 0.3 um
                        \end{itemize}
                    \end{mdframed}
                \item Physics is why a device works, Engineering
                    Principles are how we leverage physical
                    laws to create a device
            \end{itemize}

            {\large Princples}
            \begin{itemize}
                \item Principles $\to$ Concepts $\to$ 
                    Symbols $\to$ Quantities
                    $\to$ Units
                \item Measuring Acceleration $\to$
                    Velocity $\to$ v $\to$ 28
                    $\to$ $m/s$
                \item Concepts are codified in equations
                   \begin{displaymath}
                        v = \frac{\Delta x}{\Delta t}
                   \end{displaymath}

                   Where v is they symbol for velocity,
                   x for distance, and t for time. The captial
                   Delta means "Change of". This can be represented
                   via

                   \begin{displaymath}
                        \Delta x = x_f - x_0
                   \end{displaymath}

                   Where $x_f$ is the final distance and 
                    $x_0$ is the initial distance.

                \item The laws of physics are concepts which are
                    made clear through equations, using symbols
                    which we have tied to the real world via
                    consensus and agreement.

                    Each law represents repeated observations
                    which have been summarized into a single
                    relationship.
            \end{itemize}
        \end{mdframed}

        \begin{mdframed}
            {\large Units}
            \begin{itemize}
                \item Units are arbitrary and created by Humans
                \item Units gain value through consensus
                    \begin{mdframed}
                        Units are tools for communication and
                        organization!
                    \end{mdframed}
                \item Units provided detail about concepts and
                    can often solve problems by simply matching
                    units.
                \item Converting Units is a key skill
            \end{itemize}
            \begin{mdframed}
                Example
                \begin{itemize}
                    \item 2.4 miles 1min 45s per 100 yds
                    \item 112 miles at 18.5 mph
                    \item transition times 10mins, 5min
                \end{itemize}
                \begin{itemize}
                    \item 112 miles at 18.5 mph $=$ 6.05
                \end{itemize}
            \end{mdframed}
        \end{mdframed}
\end{description}
\end{document}
