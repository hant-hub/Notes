\documentclass{report}
\usepackage[tmargin=2cm, rmargin=1in, lmargin=1in,margin=0.85in,bmargin=2cm,footskip=.2in]{geometry}
\usepackage{amsmath,amsfonts,amsthm,amssymb,mathtools}
\usepackage{enumitem}
\usepackage[]{mdframed}
\usepackage{tikz}
\usepackage{siunitx}
\renewcommand{\familydefault}{\sfdefault}
\usepackage{minibox}

\title{\Huge{ECE 30}\\Day 14 Notes}
\author{\huge{Elijah Hantman}}
\date{}

\begin{document}
\maketitle
\newpage

\begin{description}
    \item {\large Agenda} 
        \begin{mdframed}
            \begin{itemize}
                \item Midterm Overview
                \item Simple Harmonic Motion Review
                \item Traveling Waves
            \end{itemize}
        \end{mdframed}
    \item {\large Midterm Review}
        \begin{mdframed}
            \begin{itemize}
                \item Some did really good, some did really really
                    bad. Hopefully I'm closer to the good side.
                \item I can't think of any difficult problems. even
                    the bonus problem seemed pretty chill.
                \item Uniform Electric field problem
                    \begin{mdframed}
                        Make sure to pause and understand the dynamics
                        before you start.

                        This is a constant acceleration problem,
                        the orthogonal velocity stays the same.
                        Remember Coulomb's law, and usetilize the
                        provided constants to reduce the problem
                        to a dynamics problem.

                        Answer was $-3.5 \times 10^{13} \si{m/s^2}$, which I think I got correct.
                    \end{mdframed}
                    \begin{mdframed}
                        For the second part its just Newtonian dynamics.
                        Things to remember is that the initial x veclocity
                        stays constant.
                    \end{mdframed}
                \item Circuit Review
                    \begin{mdframed}
                        Open switches should be interpreted as completely
                        seperate circuits. 

                        The first current was 0.05A. And the voltage
                        was 15V.
                    \end{mdframed}
                    \begin{mdframed}
                        The charge on capacitors in series combines such
                        that it acts like one large capacitor. We end
                        up getting an equivalent capacitance of 3 $\si{\mu F}$.
                        Which ends up giving a total charge of $45 \si{\mu C}$
                    \end{mdframed}
                \item For the first spring problem
                    \begin{mdframed}
                        Should have gotten 275 N/m which I think
                        I got correct.
                    \end{mdframed}
                    \begin{mdframed}
                        When asked to draw the force,
                        k is the slope, so we should have
                        A point from the origin to,
                        $(2, -5.5)$
                    \end{mdframed}
                    \begin{mdframed}
                        For calculating the work, remember the
                        force is not constant, therefore you have
                        to integrate. I did integrate and it
                        comes out to the area of a triangle which
                        is kinda fun.
                    \end{mdframed}
                    \begin{mdframed}
                        The graph one is easy. Its just the sum
                        of a bunch of areas so its chill.
                    \end{mdframed}
                    \begin{mdframed}
                        Displacement is the integral of velocity,
                        so you can just numerically integrate using
                        basic area functions.
                    \end{mdframed}
                \item Friction problem
                    \begin{mdframed}
                        I ended up saying that the coefficient of
                        static friction is the tangent of the
                        critical angle. 

                        Garunteed 100\% so thats epic
                    \end{mdframed}
                \item Bonus
                    \begin{mdframed}
                        I think I got this right since I noticed
                        both things he pointed out, ie: using
                        $(\pi - \theta)$ rather than  $\theta$
                        directly. As well as there being  $2n$
                        capacitors instead of $n$ capacitors.
                    \end{mdframed}
            \end{itemize}
        \end{mdframed}
        \pagebreak
    \item {\large Harmonic Motion quick Review}
        \begin{mdframed}
            We can imagine simple harmonic motion as arising
            from vectors spinning in circles.

            Each circle has two waves that can be derived,
            a sine and cosine wave. The radius is the magnitude
            of the wave, and the frequency is the number of
            revolutions per unit time.

            For springs we had an equation:

            \begin{displaymath}
                x = A cos(\omega t)
            \end{displaymath}

            With the following relationships:

            \begin{displaymath}
                \omega = \frac{2\pi}{T} = 2\pi f
            \end{displaymath}

            Where $T$ is the period, and $f$ is the
            frequency. $\omega$ is scaled by $2\pi$ 
            because we are using radians and $2\pi$ is
            one revolution.

            \begin{mdframed}
                $\omega$ is the angular frequency in
                $\si{Rad/s}$. Which is arguably
                just $\si{1/s}$ or Hertz($\si{Hz}$)
            \end{mdframed}

            \begin{mdframed}
                $f$ is the frequency, which is the
                number of times a full cycle happens
                per second. It uses  $\si{Hz}$ without
                caveats.
            \end{mdframed}
            
            \begin{mdframed}
                $\lambda$ is the usual symbol for wavelength,
                which is measured in meters in the case
                of space waves and seconds for oscilitory
                motion.
            \end{mdframed}
            
            Some other equations:
            \begin{gather}
               v = -A\omega sin(\omega t)\\ 
               a = A \omega^2 cos(\omega t)
            \end{gather}
            
            
        \end{mdframed}
\end{description}


\end{document}
