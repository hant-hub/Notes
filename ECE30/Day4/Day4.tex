\documentclass{report}
\usepackage[tmargin=2cm, rmargin=1in, lmargin=1in,margin=0.85in,bmargin=2cm,footskip=.2in]{geometry}
\usepackage{amsmath,amsfonts,amsthm,amssymb,mathtools}
\usepackage{enumitem}
\usepackage[]{mdframed}
\usepackage{tikz}
\renewcommand{\familydefault}{\sfdefault}
\usepackage{minibox}

\title{\Huge{ECE 30}}
\author{\huge{Elijah Hantman}}
\date{}

\begin{document}
\maketitle
\newpage
\begin{description}
    \item {\large Agenda}
        \begin{itemize}
            \item Uniformly Accelerated Motion
            \item Mass and Newton's First Law
            \item Force and Newton's Second Law
            \item Weight
            \item Equilibrium and Newton's Third Law
        \end{itemize}
    \item {\large Uniformly Accelerated Motion}
        \begin{mdframed}
            \begin{center}
                \minibox[frame]{
                    $
                    |v| \triangleq s
                    $
                    speed is the magnitude of the
                    velocity.
                }
            \end{center} 

            Average Acceleration is defined as
            \begin{displaymath}
                \vec{a}_{avg} = \frac{\Delta \vec{v}}{\Delta t}
            \end{displaymath}

            Using the same limiting process we can define
            instantaneous acceleration to be:
            \begin{displaymath}
                \vec{a} = \frac{d\vec{v}}{dt}
            \end{displaymath}

            Acceleration is the change in velocity over time,
            there are thing that corrospond to increasing
            derivatives of position but they are not usually
            relevant.

            \begin{center}
                \minibox[frame]{
                    If we plot the components of\\
                    velocity rather than position we\\
                    are in what is called a "Phase Space".\\
                    Any depiction of the state of a system\\
                    which does not corrospond to the literal\\
                    space it occupies is a "Phase Space".
                }
            \end{center}

            We also get an equation from this.
            \begin{displaymath}
                \vec{v}_f = \vec{a} \Delta t + \vec{v}_0
            \end{displaymath}

            For most situations Acceleration is taken to be
            constant. This is true for gravitational acceleration,
            in a grounded reference frame. It is not true for
            things like orbits or the movements of celestial
            bodies.

            Constant acceleration produces parabolic motion.
            This arises from integrating acceleration twice
            to get position against time.

            \begin{displaymath}
                \vec{x} = \frac{1}{2}\vec{a}t^2
                + \vec{v}_0 t + \vec{x}_0
            \end{displaymath}

            This also gives the equations:
            
            \begin{displaymath}
                \vec{v} - \vec{v}_0 = \vec{a}t
            \end{displaymath}
            \begin{displaymath}
                \vec{v}^2 = \vec{v}_0^2 + 2\vec{a}\Delta x
            \end{displaymath}
            \begin{displaymath}
                \Delta x = \vec{v}_0 t + \frac{1}{2}\vec{a}t^2
            \end{displaymath}
            \begin{displaymath}
                \Delta x = \frac{1}{2}(\vec{v}_0 + \vec{v})t
            \end{displaymath}
            \begin{displaymath}
                \Delta x = \vec{v}t - \frac{1}{2}\vec{a}t^2
            \end{displaymath}
            \begin{center}
                \minibox[frame]{
                    Note:\\
                    For these equations everything is\\
                    component wise. These are technically\\
                    vector quantities but they act as\\
                    scalars in this case.
                }
            \end{center}
        \end{mdframed}
    \item {\large Mass and Newton's First Law}
        \begin{mdframed}
             Mass is a property of matter which is constant.
             In SI units it is measured with kg.

             Mass has a property called \textit{inertia} which
             resists acceleration. We can thus deduce that
             a body's velocity will not change unless acted
             upon. This is Newton's First law.

             Things like Gravity and Friction are considered
             external actors which is why things on earth slow
             down or stop.
        \end{mdframed}
    
\end{description}






\end{document}
