\documentclass{report}
\usepackage[tmargin=2cm, rmargin=1in, lmargin=1in,margin=0.85in,bmargin=2cm,footskip=.2in]{geometry}
\usepackage{amsmath,amsfonts,amsthm,amssymb,mathtools}
\usepackage{enumitem}
\usepackage[]{mdframed}
\usepackage{tikz}
\renewcommand{\familydefault}{\sfdefault}
\usepackage{minibox}

\title{\Huge{ECE 30}\\Day 17 Notes}
\author{\huge{Elijah Hantman}}
\date{}

\begin{document}
\maketitle
\newpage

\begin{description}
    \item {\large Agenda} 
        \begin{itemize}
            \item Review Homework 6
            \item Ampere - Maxwell Law
            \item Faraday's Law
        \end{itemize}
    \item {\large Homework 6 Overview}
        \begin{mdframed}
            Circular motion happend when centripedal
            force is experienced and the speed is constant.

            Given a particle moving in uniform circular motion,
            \begin{displaymath}
                |a_c| = \frac{v^2}{r}
            \end{displaymath}
            
            This is a requirement and helpful for solving the questions.
        \end{mdframed}
    \item {\large Ampere}
        \begin{mdframed}
            Three facts:
            \begin{enumerate}
                \item $dB \propto r, ds$
                \item $|dB| \propto \frac{1}{r^2}$
                \item $|dB| \propto I, |ds|$
                \item $|dB| \propto sin(\theta)$
            \end{enumerate}

            We can combine into:

            \begin{displaymath}
                d\vec{B} = \frac{\mu_0}{4\pi} \frac{I d\vec{s} \times \vec{r}}{r^2}
            \end{displaymath}

            Therefore:

            \begin{displaymath}
                \vec{B} = \frac{\mu_0 I}{4\pi} \int \frac{d\vec{s} \times \vec{r}}{r^2}
            \end{displaymath}

            This results in:

            \begin{displaymath}
                \vec{B} = \frac{\mu_0 I}{2\pi a}
            \end{displaymath}

            Where $a$ is the orthogonal distance to the conductor.
            
            This means the magnetic field is only a function of
            the distance from the conductor.
        \end{mdframed}
        \pagebreak
        {\large Ampere - Maxwell}
        \begin{mdframed}
           Since the magnetic field is a function of distance,
           at each point along a conductor the magnetic
           field is radially symmetric.

           \begin{displaymath}
               |\vec{B}| = \frac{\mu_0 I}{2\pi a}
           \end{displaymath}

           Note that, $\mu_0$ is the permeability of free
           space, it is different than $\epsilon_0$ which
           is for electric fields.

           Consider $ds$ as a small length along the magnetic
           field line. 

           That means $\vec{B}$ and $d\vec{s}$ are parallel.

           Therefore:

           \begin{displaymath}
                \vec{B} \cdot d\vec{s} = Bds
           \end{displaymath}
           
            Since $B$ is constant for a given loop,
             \begin{gather}
                 \oint \vec{B} \cdot d\vec{s} = \oint Bds = \frac{\mu_0 I}{2\pi a} \oint ds\\
                 = \mu_0 I
            \end{gather}

            Ampere's Law:

            \begin{displaymath}
                \oint \vec{B} \cdot d\vec{s} = \mu_0 I
            \end{displaymath}

            This means that the line integral above
            applies to all closed loops. The magnetic
            field in a closed loop is entirely determined by the
            current flowing through the loop.
        \end{mdframed}
        \begin{mdframed}
            Consider a Capacitor. The surface of the wire
            leading up to the Capacitor must follow
            Ampere's Law.

            If we imagine a surface which passes between the plates of the
            capacitor, we find that:

            \begin{displaymath}
                \oint \vec{B} \cdot d\vec{s} = 0
            \end{displaymath}

            Because $I$ is zero between the plates.

            To fix this Maxwell said there was a current
            flowing between the plates.

            \begin{displaymath}
                I_d \triangleq \epsilon_0 \frac{d \phi_E}{dt}
            \end{displaymath}

            He called this "displacement current".
            $\phi_E$ is called Electric Flux.

            The flux of a field is the field multiplied
            by the area.

            \begin{displaymath}
                \phi_E = \int \int \vec{E} \cdot d\vec{A}  
            \end{displaymath}

            Whre $d\vec{A}$ is an area vector. It is a small
            area with a direction and size.

             \begin{displaymath}
                 d\vec{A} = d\vec{x} \times d\vec{y}
            \end{displaymath}

            Where $dx$ and $dy$ are perpendicular.
            
            
            Example:
            \begin{mdframed}
                For a flux through a circle we get:

                \begin{gather}
                   \int \int E dA\\ 
                   = E \int \int dA\\
                   = \pi r^2 E
                \end{gather}
            \end{mdframed}
            
            The Ampere Maxwell Law:

            \begin{displaymath}
                \oint \vec{B} \cdot d\vec{s} = 
                \mu_0(I + \epsilon_0 \frac{d\phi_E}{dt})
                = \mu_0 (I_c + I_d)
            \end{displaymath}
            
            We can see this means that if an electric field is
            changing it will cause a magnetic field.
            This is the fourth Maxwell Equation.
        \end{mdframed}
    \item {\large Faraday's Law}
        \begin{mdframed}
           If a changing Electric field creates a magnetic field,
           could a changing magnetic field create an electric field?

           If you hold a magnet still near a loop of wire
           the voltage will stay at zero.

           If you move the magnet into the loop 
           the voltage will become negative.

           If you move the magnet out of the loop
           the voltage becomes positive.

           The electric field will move to counteract
           the changing magnetic field.


           A moving Magnet induces a voltage in the
           loop.

           Be can derive Faraday's Law:

           \begin{displaymath}
               \oint E \cdot d\vec{s} = \frac{-d\phi_B}{dt} 
           \end{displaymath}

           The change in the magnetic flux over time
           is the inverse of a line integral of the Electric
           field which encompasses the surface.
           

           Important:
           \begin{enumerate}
               \item A changing magnetic field causes a changing
                   electric field.
               \item The Direction of $\vec{E}$ induced by
                   $\vec{B}$ will always act to reduce $\vec{B}$
           \end{enumerate}

           The x means the magnetic field is traveling into
           the page.
        \end{mdframed}
\end{description}


\end{document}
