\documentclass{report}
\usepackage[tmargin=2cm, rmargin=1in, lmargin=1in,margin=0.85in,bmargin=2cm,footskip=.2in]{geometry}
\usepackage{amsmath,amsfonts,amsthm,amssymb,mathtools}
\usepackage{enumitem}
\usepackage[]{mdframed}
\usepackage{tikz}
\renewcommand{\familydefault}{\sfdefault}
\usepackage{minibox}

\title{\Huge{ECE 30}}
\author{\huge{Elijah Hantman}}
\date{}

\begin{document}
\maketitle
\newpage

{\large Dot and Cross Product (cont)}
\begin{itemize}
    \item Dot Product (An inner product in the space $\mathbb{R}^n$)
        \begin{mdframed}
           Measures how "aligned" two vectors are. Has two
           equivalent definitions.

           \begin{displaymath}
               \vec{a} \cdot \vec{b} = 
               |\vec{a}||\vec{b}|cos(\theta)
           \end{displaymath}
           Where $\theta$ is the angle between the vectors. 
           This definition makes clear that the dot products
           is positive when the vectors have a small angle,
           zero when orthogonal, and negtive when opposing.

           Another equivalent definition is as follows:
           \begin{displaymath}
                \vec{a} \cdot \vec{b} =
                \sum \vec{a}_i  \vec{b}_i
           \end{displaymath}

           For 2D:

           \begin{displaymath}
                \vec{a} \cdot \vec{b}
                = a_1  b_1 + a_2  b_2
           \end{displaymath}

           Some Algebraic properties:
           \begin{itemize}
               \item $a \cdot b = b \cdot a$
               \item $a \cdot b \in \mathbb{R}$ 
           \end{itemize}
           And for Unit vectors
           \begin{itemize}
               \item $i \cdot i = 1$
               \item $i \cdot j = 0$
               \item $i \cdot -i = -1$
           \end{itemize}

           This also implies the dot product is linear. This
           is also obvious from the second definition.
           
           The dot product is not reversible since it maps pairs
           of vectors to scalars.
        \end{mdframed}
    \item Cross Product (Analogous to an outer product)
        \begin{mdframed}
            Only works in 3 dimensions, in 2D usually can be
            substituted with per-product.

            The cross product is defined as:
            \begin{displaymath}
                a \times b = |a||b|sin(\theta)w
            \end{displaymath}
            Where $w$ is a direction orthogonal to both
            $a$ and $b$. $w$ follows a right hand rule, which
            is an arbitrary choice.

            \begin{center}
                \minibox[frame]{
                    In Geometric algebra, the outer product,\\
                    in this context, actually has the same\\
                    magnitude as the cross product but is\\
                    instead defined as the plane which\\
                    contains $a$ and $b$. This is equivalent\\
                    because orthogonal vectors can be used\\
                    to define planes.
                }
            \end{center}
            \pagebreak

            Some Properties:
            \begin{itemize}
                \item $a \times b = -b \times a$
            \end{itemize}
            Table of unit vectors\\
            \begin{tabular}{|c|c|c|c|}
                \hline
                &i & j & k\\
                \hline
                i& 0& k& -j\\
                \hline
                j& -k& 0& i\\
                \hline
                k& j& -i& 0\\
                \hline
            \end{tabular}
            

        \end{mdframed}
\end{itemize}

{\large Agenda}
\begin{itemize}
    \item Displacement
    \item Instantaneous Acceleration
    \item Uniformly Accelerated Motion
\end{itemize}

{\large Displacement}
\begin{mdframed}
   Our goal is to describe particle motion in 2D 
   space. We need:
   \begin{itemize}
       \item Reference point
       \item Unit vectors
   \end{itemize}

   We can use a vector to describe a particle's position, with
   each component being the projection of the particle's position
   onto a real number line aligned with our unit vectors.

   The \textbf{displacement} is defined as the difference
   between this vector at two points in time. The average
   velocity is defined as the ratio between the change in
   time and the displacement.

   \begin{gather}
       \Delta\vec{x} = \vec{x}_f - \vec{x}_0\\
       \vec{v} = \frac{\Delta\vec{x}}{t_f - t_0}
   \end{gather}

   This implies the following relation:
   \begin{displaymath}
       \vec{x}_f = \vec{x}_0 + \Delta t \vec{v} 
   \end{displaymath}

   For a continuous path this method will have error. We can
   imagine finer and finer samples, with smaller and smaller
   $\Delta t$ values. We can express this via a limit.

   \begin{displaymath}
       v = \lim_{\Delta t \to 0} \frac{x_f - x_0}{t_f - t_0}
   \end{displaymath}
   
   We can rewrite position as a function over time.

   \begin{displaymath}
       v = \lim_{\Delta t \to 0} \frac{p(t_f) - t(t_0)}{t_f - t_0}
   \end{displaymath}
   We can see this is equivalent to a derivative which is
   defined as:

   \begin{displaymath}
        \frac{d}{dt}p(t) = \lim_{t \to 0} \frac{p(x + t) - p(x)}{t}
   \end{displaymath}
   We can see these are identical since $t_f = t_0 + \Delta t$
   and thus can be easily substituted to get us the derivative
   formula.


   \begin{center}
       \minibox[frame]{ 
           Velocity is the derivative of position with respect
           to time.
       }
   \end{center}
\end{mdframed}
\begin{mdframed}
    As a side note: You can imagine velocity as a conversion
    between time and displacement. Due to acceleration this
    conversion doesn't remain constant but you can always
    use it to calculate a change in position.
\end{mdframed}

\end{document}
