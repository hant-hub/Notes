\documentclass{report}
\usepackage[tmargin=2cm, rmargin=1in, lmargin=1in,margin=0.85in,bmargin=2cm,footskip=.2in]{geometry}
\usepackage{amsmath,amsfonts,amsthm,amssymb,mathtools}
\usepackage{enumitem}
\usepackage[]{mdframed}
\usepackage{tikz}
\renewcommand{\familydefault}{\sfdefault}
\usepackage{minibox}
\usepackage{siunitx}

\title{\Huge{ECE 30}\\Day 8 Notes}
\author{\huge{Elijah Hantman}}
\date{}

\begin{document}
\maketitle
\newpage

\begin{description}
    \item {\large Agenda} 
        \begin{itemize}
            \item Electric Fields
            \item Work and Energy
            \item Electric Potential Energy
        \end{itemize}
    \item {\large Electric Fields}
        \begin{mdframed}
            What are Ions?\\
            ions are atoms which have had electrons stripped
            from them, this causes them to be electrically
            charged. The vast majority of atoms are neutral
            normally but we can delibrately charge them for
            etching.

            The human blood has extra ions. This means human
            fingers are electrically charged.
        \end{mdframed}
        \begin{mdframed}
            Charges effect their environment. Many forces like
            gravity also have seemingly non-local effects. This
            can be modeled by imagining all of space has a charge
            value, this is called a field.
        \end{mdframed}
        \begin{mdframed}
            The value of the electric field is the force a charged
            particle would feel at the given point in space. This
            defines a function from a position to a force which
            creates a vector field.

            \begin{displaymath}
                \vec{\epsilon} \triangleq \frac{\vec{F_e}}{q_0}
            \end{displaymath}

            Where $\vec{F_e}$ is the Coulomb force exerted on
            a particle with charge $q_0$.
            The electro static force is measured in force
            per unit charge. $\si{N/C}$
        \end{mdframed}
        \begin{mdframed}
            An electric field is represented via electric field
            lines. The field lines are always drawn as arrows which
            describe the movement of a positive charge. There are
            some additional rules.

            \begin{enumerate}
                \item Field lines are tangent to the electic field
                    at all points.
                \item Direction is same as the force on a positive
                    test charge.
                \item The number of lines passing through a surface
                    perpendicular to the lines is proportional to
                    the magnitude of the electrical field.
            \end{enumerate}

            Something to note is that the electric field is
            conservative, which means it is the gradient of
            some potential function. This is intuitive if
            you imagine creating a surface where the
            charges represent the height, and everything
            is continuous. This would be a potential
            function, and the gradient would be the vector
            field.
        \end{mdframed}
        \pagebreak
    \item {\large Work and Energy}
        \begin{mdframed}
           Work is defined as: 
           \begin{displaymath}
                W = F \times \Delta x
           \end{displaymath}
           For a constant foce in the direction of
           $\Delta x$. Only the component
           of work in the direction of $\Delta x$
           contributes to the work.
           And has units of $\si{Nm}$. Or Joules ($\si{J}$)

           Therefore we can also write work as:
           \begin{displaymath}
                W = \vec{F} \cdot \Delta \vec{x}
           \end{displaymath}
           
           Over a curved path you can compute work using a
           line integral.

           \begin{displaymath}
               W = \int_S \vec{F} \cdot d\vec{x}
           \end{displaymath}

           This is the more general form which works for non
           constant force and a non linear path. This 
           can be analytically solved via a parameterization.

            \begin{displaymath}
                W = \int_a^b f(x(t), y(t), z(t)) 
                \sqrt{(x'(t))^2 + (y'(t))^2 + (z'(t))^2} dt
            \end{displaymath}

            This comes from the formula for arc length multiplied
            by the force at each point. This requires being able
            to parameterize the line which is usually possible
            but may be difficult in general.

            Work can also be written as:
            \begin{displaymath}
                W = \int_a^b m v dv
            \end{displaymath}

            By integrating the above integral we get:

            \begin{displaymath}
            W = \frac{1}{2}mv^2 {\bigg |_a^b}
            \end{displaymath}
            Which is by definition the difference in kinetic
            energy of a particle since kinetic energy is written:

            \begin{displaymath}
                K_e = \frac{1}{2} mv^2
            \end{displaymath}

            Which means:

            \begin{displaymath}
                W = \Delta K_e
            \end{displaymath}
        \end{mdframed}
        \begin{mdframed}
            What if there is no change in speed?

            \vspace{10pt}

            Since $W = F_y \Delta y = mg(y - y_0)
            = mgh$, we can see that work is the potential
            energy when the speed doesn't change.

            Work is the change in potential energy. This works
            because forces like gravity and electro static forces
            are conservative.

            Energy is defined as the capacity for doing work.
            Therefore to perform work a body must have energy.
        \end{mdframed}
        \pagebreak
    \item {\large Electrical Energy}
        \begin{mdframed}
            Electric Potential:
            \begin{gather}
                W = \int_a^b F dx 
                = -\int_a^b q \epsilon dx
            \end{gather}
        \end{mdframed}
\end{description}


\end{document}
