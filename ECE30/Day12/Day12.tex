\documentclass{report}
\usepackage[tmargin=2cm, rmargin=1in, lmargin=1in,margin=0.85in,bmargin=2cm,footskip=.2in]{geometry}
\usepackage{amsmath,amsfonts,amsthm,amssymb,mathtools}
\usepackage{enumitem}
\usepackage[]{mdframed}
\usepackage{tikz}
\usepackage{circuitikz}
\renewcommand{\familydefault}{\sfdefault}
\usepackage{minibox}

\title{\Huge{ECE 30}\\Day 12 Notes}
\author{\huge{Elijah Hantman}}
\date{}

\begin{document}
\maketitle
\newpage

\begin{description}
    \item {\large Agenda} 
        \begin{itemize}
            \item Review Quiz 2
            \item RC circuits
            \item Touchscreens
        \end{itemize}
    \item {\large Quiz 2 Review}
        \begin{mdframed}
            Notes on answers I think I got incorrect.
            \begin{itemize}
                \item 3 point flexure is about brittle materials
                    not breaking
                \item I think I got this right, but capacitors
                    in parallel are independent.
                \item Close reading, should have noticed
                    the path is from B to A, and that
                    the path goes with the field which
                    means negative work happens
                \item Remain repelled but the distance
                    will decrease (Not sure if I got this
                    wrong, I was a little confused about
                    the quiz wording.)
            \end{itemize}
            I'm a little
            annoyed but whatever. Also the professor goes
            over like a ton of the incorrect answers but
            not in a helpful way just like he feels
            he has to rexplain this to us over and over 
            again.
        \end{mdframed}
    \item {\large RC circuits}
        \begin{mdframed}
            An RC circuit is a circuit with both resistors and
            capacitors in it.

            \begin{circuitikz}
                \draw (0,0) to[capacitor] (3,0)
                (3,0) to[resistor] (3,3)
                (3,3) to[switch] (0,3)
                (0,3) to[battery] (0,0);
            \end{circuitikz}

            As the capacitor charges the current and voltage accross
            the resistor drops.

            Based on the loop rule:

            \begin{displaymath}
                V - iR - v_c(t) = 0
            \end{displaymath}

            This means that:

            \begin{displaymath}
                V - v_c(t) = iR
            \end{displaymath}
            \begin{displaymath}
                \frac{V - v_c(t)}{R} = i(t)
            \end{displaymath}
            
            This means, as the charge increases in the
            capacitor, the current drops, until the capacitor
            matches the voltage of the battery and the
            current is zero.

            This also means the voltage across the resistor
            will settle at zero.

            This also means that initially when the capacitor
            has zero charge, the circuit will act
            as if the capacitor didn't exist.

            This leads the rule of thumb that initially,
            capacitors act like wires and don't
            affect the current flow. And that when
            fully charged capacitors do not allow for current
            to flow and can be removed when thinking about
            current and voltage without altering
            the circuit.

            In addition:

            \begin{displaymath}
                v_c(t) = \frac{q(t)}{C}
            \end{displaymath}

            \begin{displaymath}
                V - \frac{dq_c}{dt}R - \frac{q_c}{C} = 0
            \end{displaymath}
            
            \begin{displaymath}
                \frac{cV - q_c}{RC} = \frac{dq_c}{dt}
            \end{displaymath}

            \begin{displaymath}
                \frac{dt}{RC} = \frac{1}{CV - q}dq
            \end{displaymath}

            \begin{displaymath}
                \int_0^t \frac{dt}{RC} = \int_0^{q(t)}\frac{1}{CV-q}dq
            \end{displaymath}

            \begin{displaymath}
                \frac{t}{RC} = -ln(\frac{cv - q}{cv}) 
            \end{displaymath}
            
            \begin{displaymath}
                q(t) = CV(1-e^{\frac{-t}{RC}})     
            \end{displaymath}

            This implies:

            \begin{displaymath}
                Q_{max} = CV
            \end{displaymath}

            and:

            \begin{displaymath}
                i(t) = \frac{dq}{dt}
                = \frac{CVe^{\frac{-t}{RC}}}{RC}
            \end{displaymath}
            \begin{displaymath}
                i(t) = \frac{V}{R} e^{\frac{-t}{RC}}
            \end{displaymath}

            This means the maximum current is $\frac{V}{R}$ which
            looks like Ohm's law. However it exponentially
            decays to zero as the capacitor charges.
        \end{mdframed}
    \item {\large Touchscreen}
        \begin{mdframed}
            Grid of conductive line, one set sensing the other
            driving lines. At each intersection they form a 
            capacitor.

            The driving lines are energized one by one and
            each sensing line is checked.

            When you touch the screen the ions in your blood
            are positively charged. Your finger creates a
            tiny parallel capacitor which changes the rate
            the capacitor charges and discharges. This is
            detectable and allows for calculating the point
            of contact.

            This increases the capacitance of the screen which
            is detectable.
        \end{mdframed}
\end{description}




\end{document}
