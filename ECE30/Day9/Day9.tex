\documentclass{report}
\usepackage[tmargin=2cm, rmargin=1in, lmargin=1in,margin=0.85in,bmargin=2cm,footskip=.2in]{geometry}
\usepackage{amsmath,amsfonts,amsthm,amssymb,mathtools}
\usepackage{enumitem}
\usepackage[]{mdframed}
\usepackage{tikz}
\usepackage{siunitx}
\renewcommand{\familydefault}{\sfdefault}
\usepackage{minibox}

\title{\Huge{ECE 30}}
\author{\huge{Elijah Hantman}}
\date{}

\begin{document}
\maketitle
\newpage

\begin{description}
    \item {\large Agenda}
        \begin{itemize}
            \item Electric Potential Cont.
            \item Potential Difference - Voltage
            \item Capacitance
            \item Charge Storage
        \end{itemize}

    \item {\large Electric Potential Cont.}
        \begin{mdframed}
           If bringing a charge from infinity, the potential
           energy is equal to $\frac{k_e Q q_0}{r}$ for any
           $r$.
        \end{mdframed}
    \item {\large Voltage}
        \begin{mdframed}
            If the initial distance is not infinity, then
            \begin{displaymath}
                \frac{\Delta E_p}{q_0} = \Delta V
            \end{displaymath}

            In other words, voltage is the difference in potential
            energy.
        \end{mdframed}
    \item {\large Capacitance}
        \begin{mdframed}

            If we create a voltage difference between two
            metal plates, a uniform electric field forms
            between them.

            The constant of proportionality was defined as
            \begin{displaymath}
                C \triangleq \frac{q}{\Delta V}
            \end{displaymath}

            Both $q$ and  $\Delta V$ are defined to be positive,
            which implies $C$ is always positive. 

            $C$ is measured in  $\si{C/V}$ which is also known
            as a Farad $\si{F}$ which is a very large
            unit. Most capacitors are measured in
            $\si{\mu F}$ or  $\si{mF}$
            
            \begin{displaymath}
                \Delta V = \int_A^B \vec{\epsilon} \cdot d\vec{r}
                = \epsilon d
            \end{displaymath}
            
            We saw by experiment $E \propto \frac{q}{A}$ where
            $A$ is the area of the plates.

            The factor of proportionality is $\frac{1}{\epsilon_0}$
            where  $\epsilon_0$ is the permitivity of free space.
            Therefore we have the equation:

            \begin{displaymath}
                E = \frac{q}{\epsilon_0 A}
            \end{displaymath}

            Therefore :
            \begin{displaymath}
                \Delta V = \frac{qd}{\epsilon_0 A}
            \end{displaymath}
            
            For a smaller area you require less voltage for the
            same charge.

            Since:
            \begin{displaymath}
                C = \frac{q}{\Delta V}
            \end{displaymath}
            Then:
            \begin{displaymath}
                C = \frac{\epsilon_0 A}{d}
            \end{displaymath}

            Capacitance decreases with distance, and increases
            with area. This is useful for accelerometers and
            touch screens.
        \end{mdframed}
        \begin{mdframed}
            Capacitors with dialectric.

            \vspace{10pt}

            A dialectric is a material that doesn't conduct,
            ex: rubber, glass, concrete, etc.


            A voltmeter is like an open switch which doesn't allow
            charge to flow.

            We find that adding a dialectric increases the capacitance.
            The capacitance increases linearly with permitivity of the
            dialectric material.

            Some common materials used have permitivity values:
            \begin{itemize}
                \item Vacuum $1.0$ 
                \item Glass $5.60$
                \item Rubber $6.70$
            \end{itemize}
        \end{mdframed}
    \item {\large Energy Storage}
        \begin{mdframed}
            We create a potential difference between the plates
            of a capcitor, the electric field between the
            plates can do work. It can be used as a battery.

            We know:
            \begin{displaymath}
                W = \Delta E_p = \Delta V_{q_0}
            \end{displaymath}
            
            When we connect a voltage electrons move and
            deposit on the sides of the capcitors. As
            the power begins to flow more and more electrons
            move and the voltage climbs until it eventually matches
            the circuit.

            Assuming zero resistance the Voltage difference is
            linear with time.

            

        \end{mdframed}
\end{description}


\end{document}
