\documentclass{report}
\usepackage[tmargin=2cm, rmargin=1in, lmargin=1in,margin=0.85in,bmargin=2cm,footskip=.2in]{geometry}
\usepackage{amsmath,amsfonts,amsthm,amssymb,mathtools}
\usepackage{enumitem}
\usepackage[]{mdframed}
\usepackage{tikz}
\usepackage{siunitx}
\renewcommand{\familydefault}{\sfdefault}
\usepackage{minibox}

\title{\Huge{ECE 30}\\Day 20 Notes}
\author{\huge{Elijah Hantman}}
\date{}

\begin{document}
\maketitle
\newpage

\begin{description}
    \item {\large Agenda} 
        \begin{itemize}
            \item Review Q.7 of Quiz 3
            \item Struction of the Atom
                \begin{itemize}
                    \item Rutherford Orbital Model
                    \item The Bohr Model
                    \item De Broglie Waves
                \end{itemize}
        \end{itemize}
    \item {\large Review}
        \begin{mdframed}
            \begin{itemize}
                \item The current is clockwise. I was correct last time I think.
                    turns out it depends on the face we are looking at,
                    could be solved with "parity" but whatever.
            \end{itemize}
        \end{mdframed}
    \item {\large Structure of the Atom}
        \begin{mdframed}
            \begin{itemize}
                \item Rutherford said that the nucleus must be tiny since
                    some alpha particles were deflected, but most
                    weren't.
                \item Since alpha particles are positively charged the nucleus
                    must have a very strong positive charge.
                \item Therefore Rutherford postulated that the atom
                    was electrons orbiting a positively charged nucleus.
            \end{itemize}

            If the atom is an orbital then the following holds true.

            \begin{displaymath}
                F_c = F_e
            \end{displaymath}

            The centripedal force balances out

            \begin{displaymath}
                F_c = m_e \frac{v^2}{r}
            \end{displaymath}

            And from Coulombs law

            \begin{displaymath}
                F_e = k \frac{e^2}{r^2}
            \end{displaymath}

            Where $e$ is the charge of an electron.
            
            From the first equality we know:

            \begin{displaymath}
                m_e \frac{v^2}{r} = k \frac{e^2}{r^2}
            \end{displaymath}
            
            Therefore

            \begin{displaymath}
                m_e v^2 = k \frac{e^2}{r}
            \end{displaymath}
            
            We can therefore calculate velocity:

            \begin{displaymath}
                v = \frac{e}{\sqrt{4 \pi \epsilon_0 m_e r}}
            \end{displaymath}
            
            The total energy is:

            \begin{displaymath}
                E = \frac{1}{2} mv^2 - \frac{e^2}{4\pi \epsilon_0 r}
            \end{displaymath}
            
            In other words, it is the velocity minus the potential
            energy of the electron. The negative comes from the fact
            the electron and proton have opposite charges. The
            $e$ is the magnitude of the charge not the sign.
        \end{mdframed}
        \begin{mdframed}
            If we substitude for $v$

            \begin{displaymath}
                E = \frac{e^2}{8 \pi \epsilon_0 r} - \frac{e^2}{4 \pi \epsilon_0 r}
            \end{displaymath}
            \begin{displaymath}
                E = \frac{-e^2}{8 \pi \epsilon_0 r}
            \end{displaymath}
            
            Since the energy is negative it means that orbiting an atom
            the electron is stable. It requires energy to break
            an electron away from an atom and it releases energy
            to bring an atom into orbit around an atom.
            
            This also implies:

            \begin{displaymath}
                r = \frac{-e^2}{8 \pi \epsilon_0 E}
            \end{displaymath}

            Since we can ionize hydrogen we can measure $E$
            directly. Since the other constants are also
            measurable we can calculate the size of a hydrogen
            atom.

             \begin{displaymath}
                 r = 5.3 \times 10^{-11} \si{m}
            \end{displaymath}
            
            This derivation matched experimental measurements.
        \end{mdframed}
        Problems
        \begin{mdframed}
            If electron is moving around an atom it should be
            generating electro magnetic waves. It should
            slowly be losing energy until it collapses in
            on itself.

            Gasses held at low pressure release a spectrum of
            light when heated.
        \end{mdframed}
        \begin{mdframed}
            French Physicist De Broglie proposed that all
            matter was made of waves. To test this he
            shot electrons through two small slits. It was
            well known that waves will create interferece
            patterns when passed through a similar setup.

            When firing electrons and protons through these
            slits they form interference patterns, both
            when fired one at a time and en masse.

            When we detect which slit the electron passed
            through the interference patterns disappeared.

            This seemed to imply that particles move as
            waves until they are measured.
        \end{mdframed}
        \begin{mdframed}
            At a similar time Albert Einstein was working on
            the Photoelectric effect, in which light dislodges
            electrons from around atoms. He could only explain
            this if light came in discrete particles "photons".

            All of this implies that matter is both a wave
            and a particle at the same time.
        \end{mdframed}
        \begin{mdframed}
           Hydrogen's spectra had a band at 6,536 Angstroms,
           \begin{mdframed}
               An Angstrom is defined:
               \begin{displaymath}
                   A = 10^{-10} \si{m}
               \end{displaymath}
           \end{mdframed}

           There are other bands of increasing density up to
           3,646 \AA.

           Light appeared to only be emittable in specific
           bands.
        \end{mdframed}
        \pagebreak
        \begin{mdframed}
           In 1928 De Broglie proposed that particles have
           the properties of waves, and that they obey
           the same wave properties as light:
           \begin{displaymath}
                \lambda = \frac{h}{P}
           \end{displaymath}
           Where $h$ is planck's constant and  $P$ is momentum.
           This is the same equation as for electromagnetic
           waves.

           \begin{displaymath}
               h \triangleq \text{Planck's Constant } 6.625 \times 10^{-34} \si{Jhz}
           \end{displaymath}
           \begin{displaymath}
               P \triangleq \text{Momentum } m \times v 
           \end{displaymath}

           Planck Postulated that:
           \begin{enumerate}
               \item An oscillating entity at atomic dimensions
                   can only have energies given by:
                   \begin{displaymath}
                        E = nhf
                   \end{displaymath}
                   Where $f$ is the oscillating frequency,
                    $n$ is a positive integer (quantum number),
                    and $h$ is Planck's Constant. 
                \item  Oscillators only radiate energy in
                    discrete quanta if $n$ changes:

                    If  $n$ changes by 1 :
                    \begin{displaymath}
                        \Delta E = \Delta nhf
                    \end{displaymath}

                    for $\Delta n = 1$:

                     \begin{displaymath}
                        \Delta E = hf
                    \end{displaymath}

                    This means that $hf$ is the smallest
                    change in energy possible.
           \end{enumerate}
        \end{mdframed}
        \begin{mdframed}
            Danish scientist Nils Bohr used these observations
            by Planck to revise the model of the Atom. His
            model is called The Bohr Model.
        \end{mdframed}
        \begin{mdframed}
            De Broglie: wavelength is $\lambda_e = \frac{h}{m v_e}$ 

            We know from Rutherford:

            \begin{displaymath}
                v = \frac{e}{\sqrt{4 \pi \epsilon_0 m r}}
            \end{displaymath}

            Therefore:

            \begin{displaymath}
                \lambda_e = \frac{h}{e}\sqrt{\frac{4 \pi \epsilon_0 r}{m}}
            \end{displaymath}
            
            But $r = 5.3 \times 10^{-11} \si{m}$ (from Rutherford). 

            Therefore we know all the constants on the left
            side of the equation. We then get that:

            \begin{displaymath}
                \lambda_e = 33 \times 10^{-11} \si{m}
            \end{displaymath}
            
            The circumference of a Hydrogen Atom is
            $2\pi r \approx 33 \times 10^{-11} \si{m}$
            
            So the wavelength of an electron at energy level
            $n = 1$ is equal to the circumference of it's
            orbit.

            This implies an electron in an atom is a
            standing wave.
            
        \end{mdframed}
\end{description}


\end{document}
