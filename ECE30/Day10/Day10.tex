\documentclass{report}
\usepackage[tmargin=2cm, rmargin=1in, lmargin=1in,margin=0.85in,bmargin=2cm,footskip=.2in]{geometry}
\usepackage{amsmath,amsfonts,amsthm,amssymb,mathtools}
\usepackage{enumitem}
\usepackage[]{mdframed}
\usepackage{tikz}
\usepackage{siunitx}
\renewcommand{\familydefault}{\sfdefault}
\usepackage{circuitikz}
\usepackage{minibox}

\ctikzset{bipoles/length=.6cm}
\newcommand\esymbol[1]{\begin{circuitikz}
\draw (0,0) to [#1] (1,0); \end{circuitikz}}

\title{\Huge{ECE 30}\\Day 10 Notes}
\author{\huge{Elijah Hantman}}
\date{}

\begin{document}
\maketitle
\newpage

\begin{description}
    \item {\large Agenda} 
        \begin{itemize}
            \item Finish Lecture 9
            \item Electric Current
            \item Ohm's Law
            \item Terms and Symbols for Electric Circuits
        \end{itemize}
    \item {\large Lecture 9 Cont.}
        \begin{mdframed}
           When a capacitor is connected to a potential
           difference, $\Delta v$, $q$ (charge in the plates)
           changes.

           Initially $q = 0\si{C}$ and it eventually reaches
           $Q = cv$.
           
           \begin{displaymath}
                dw = \Delta v dq
           \end{displaymath}

           The total Work done to charge a capacitor:

           \begin{displaymath}
                W = \int_0^Q \frac{q}{c} dq
                = \frac{1}{c} \int_0^Q q dq
           \end{displaymath}
           \begin{displaymath}
                W = \frac{Q^2}{2c} \si{J}
           \end{displaymath}

           We also know:

           \begin{displaymath}
                c = \frac{Q}{\Delta v}
           \end{displaymath}

           Therefore:

           \begin{displaymath}
                W = \frac{1}{2}c \Delta v^2
           \end{displaymath}
           
           
           For a given capacitance, energy stored is
           equivalent to the square of the voltage difference,
           and directly proportional to the square of the
           charge.
        \end{mdframed}
    \item {\large Electric Current}
        \begin{mdframed}
            \begin{displaymath}
                \vec{E} = \frac{\vec{F}}{q}
                = \frac{m\vec{a}}{q}
            \end{displaymath}
            
            Let $q = \overline{e}$ (charge of an electron)

            Then $\vec{a} =\frac{-\overline{e} \vec{E}}{m}$ 

            $\vec{E}$ in copper is effectively constant

            We also know that  $\vec{a} = \frac{d\vec{v}}{dt}$
            We can expand using the chain rule to:

             \begin{displaymath}
                \vec{a} = \frac{dv}{dx} \cdot \frac{dx}{dt}
                = \frac{\vec{v}dv}{dx}
            \end{displaymath}

            Therefore:

            \begin{displaymath}
                \vec{a}dx = \vec{v}dv
            \end{displaymath}

            \begin{displaymath}
                a \int_{x_0}^{x} dx = \int_{v_0}^v v dv
            \end{displaymath}

            Therefore:

            \begin{displaymath}
                a(x-x_0) = \frac{v^2}{2} - \frac{v_0^2}{2}
            \end{displaymath}

            Letting $v_0 = 0$ and $x_0 = 0$ then:

            \begin{displaymath}
                2ax = v^2
            \end{displaymath}
            \begin{displaymath}
                v = \sqrt{2ax}
                = \sqrt{\frac{2eEx}{m}}
            \end{displaymath}
        \end{mdframed}
        \begin{mdframed}
            When electrons flow they collide with atoms,
            which causes the atoms to vibrate converting
            some of the electrical energy into thermal
            energy.

            For a conductor $\overline{e}$ collides with $w$
            atoms and give off kinetic energy  $K = \frac{1}{2}mv^2$
            and the conductor will heat up.

            We define:  $i = \frac{dq}{dt}$
            where  $i$ uses units of  $\si{C/s} \triangleq A$
            or Amps. Amps are units of Current.

            Amps are large! Usually $\si{mA}$ or  $\si{nA}$,
            or  $\si{\mu A}$ are used.
        \end{mdframed}
        {\large Resistance}
        \begin{mdframed}
            $\vec{E}$ in a conductor is uniform, and
            $\Delta v = \int_{s_1}^{s_2}\vec{E} \cdot d\vec{s}$
            whre $\vec{s}$ is displacement in the direction
            of the field.

            Therefore:

             \begin{displaymath}
                 \Delta v = E(s_2-s_1) = EL
            \end{displaymath}

            Where $L$ is the length of the conductor. 

            Therefore, as $\Delta v$ increases $E$ increases.
            Since $\vec{E} = \frac{\vec{F}}{q}$
            the electrons in a conductor are accelerated more
            as  $\Delta v$ increases.

            Therefore  $\Delta v \propto i = \frac{dq}{dt}$
            We therefore have a factor of propotionality, written
            $R$ for resistance.

        $R \triangleq \frac{V}{i}$ with units called Ohms($\si{\Omega}$) 
        Which comes from the name Ohms' Law.
        \end{mdframed}
    \item {\large Circuit Symbols}
        \begin{mdframed}
            Battery: \esymbol{battery} The voltage decreases from
            the short to the long side. In this case it goes from right to left.
            measured in Volts (V)\\
            Capacitor: \esymbol{capacitor} Non directional. Measured in Farads (F)\\
            Resistor: \esymbol{resistor} A component which
            has a specific resistance, usually higher than
            the material the circuit is made out of. Measured in Ohms ($\Omega$)\\
            Voltmeter: \esymbol{voltmeter}  $R = \infty$\\
            Ammeter: \esymbol{ammeter} $R = 0$\\
            Conductor: \esymbol{short} carries current. Arrows sometimes
            added to indicate the flow of positive charge.\\
            Switch: \esymbol{switch} 
        \end{mdframed}

\end{description}


\end{document}
