\documentclass{report}
\usepackage[tmargin=2cm, rmargin=1in, lmargin=1in,margin=0.85in,bmargin=2cm,footskip=.2in]{geometry}
\usepackage{amsmath,amsfonts,amsthm,amssymb,mathtools}
\usepackage{enumitem}
\usepackage[]{mdframed}
\usepackage{tikz}
\renewcommand{\familydefault}{\sfdefault}
\usepackage{minibox}

\title{\Huge{ECE 30}}
\author{\huge{Elijah Hantman}}
\date{}

\begin{document}
\maketitle
\newpage

\begin{description}
    \item {\large Agenda}
        \begin{mdframed}
            \begin{itemize}
                \item Young's Modulus and 3 point Flexure
                \item Photolithography
                \item DRIE
                \item How an Accelerometer Works
            \end{itemize}
        \end{mdframed}
    \item {\large Harmonic Motion}
        \begin{mdframed}
            Harmonic Osscilators are described by
            a cosine function.

            \begin{displaymath}
                x = Acos(\omega t)
            \end{displaymath}

            We describe $\omega$ in terms of the frequency and
            period of the osscilator.
             \begin{displaymath}
                f = \frac{1}{T}
            \end{displaymath}

            Where $f$ is the frequency and $T$ is the period of the
            osscilator.

            \begin{displaymath}
                T = 2\pi \sqrt{\frac{m}{k}}
            \end{displaymath}

            Derivation:
            \begin{mdframed}
                \begin{gather}
                   F = -kx\\ 
                   F = ma\\
                   a = \frac{dv}{dt} = \frac{d^2x}{dt}\\
                   \frac{d^2x}{dt} = \frac{-kx}{m}\\
                   x = A cos(\omega t)\\
                   \frac{dx}{dt} = -A\omega sin(\omega t)\\
                   \frac{d^2x}{dt} = -A\omega^2 cos(\omega t)\\
                   -A\omega^2 cos(\omega t) = \frac{-kAcos(\omega t)}{m}\\
                   -Am\omega^2 cos(\omega t) = -kAcos(\omega t)\\ 
                   m\omega^2 = k\\
                   \omega^2 = \frac{k}{m}\\
                   \omega = \sqrt{\frac{k}{m}}
                \end{gather}

                To get the period we have to multiply
                by $2\pi$. This is because the period
                of an unmodified cosine function is
                 $2\pi$ and the constant  $\omega$ multiplies
                 with this  $2\pi$ period.
            \end{mdframed}
        \end{mdframed}
    \item {\large Young's Modulus}
        \begin{mdframed}
            Measures the resistance of a solid to elastic
            deformation.

            Imagine a beam which is L units long, y units
            tall and z units across. We elastically deform
            the object by $\Delta L$.

            \begin{gather}
                \textrm{Stress} = \frac{F}{A}\\
                \textrm{Strain} = \frac{\Delta L}{L}\\
                Y = \frac{F/A}{\Delta L/L}
            \end{gather}

            All materials have a Young Modulus.

            \begin{itemize}
                \item Steel $= 20\times 10^{10} N/m^2
                    = 140 GPa$
            \end{itemize}

            \begin{displaymath}
                1 \frac{N}{m^2} = 1Pa
            \end{displaymath}
            
        \end{mdframed}
    \item {\large 3 Point Flexure}
        \begin{mdframed}
            \begin{center}
                \minibox[frame] {
                    Two points are fixed\\
                    One point is flexed
                }
            \end{center}
            
            Start with a beam of length L.
            Fix the ends and apply a force F on
            the middle. The crosssection is $d$ high
            and  $b$ across.

            Measuring from the bottom surface of the beam
            it will bend down a distance $\delta$.

            It can be shown: 
            \begin{displaymath}
                \delta = \frac{F}{y} \frac{L^3}{4bd^3}
            \end{displaymath}
        \end{mdframed}
    \item {\large Photolithography}
        \begin{mdframed}
            \begin{itemize}
                \item A very pure silicon wafer is grown in a
                    lab.
                \item We then coat the waver in a photoresist
                    material.
                \item We then take a mask which blocks light.
                \item The light is passed through the mask
                    and focused down to a microscopic structure.
                \item The light passing through the mask develops
                    the photoresist, removing it from the surface.
                \item In additive lithography we then deposit
                    material into the gaps in the photoresist.
                \item In subtractive lithography we etch into
                    the material through the gaps left in the
                    photoresist using an ionized gas which
                    reacts with the silicon but not the
                    resist.
                \item At the end we remove the photoresist and are
                    left with a silicon structure.
            \end{itemize}
        \end{mdframed}
        \pagebreak
    \item {\large DRIE}
        \begin{mdframed}
            \begin{itemize}
                \item 
                    We begin with two silicon wafers with an
                    insulator between them.
                \item We etch into the silicon the anchor points
                \item We take some glass and deposit metal
                    onto it
                \item We then flip the silicon over and bond
                    it to the glass deposits using
                    Anodic bonding
                \item We then etch away the back of the
                    silicon until we remove the full
                    insulator
            \end{itemize}
            Deep Reactive Ion Etch system.
            \begin{itemize}
                \item Gas is let into a chamber
                \item We apply current to ionize the gas
                \item The gas is pushed onto the wafer
                \item The back of the wafer is covered with
                    ionized super cold helium.
                \item The gases are then pumped out the bottom
                    of the chamber.
                \item The etching and masking phases repeat to
                    create deep features into the material.
            \end{itemize}
        \end{mdframed}
\end{description}


\end{document}
