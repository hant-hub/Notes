\documentclass{report}
\usepackage[tmargin=2cm, rmargin=1in, lmargin=1in,margin=0.85in,bmargin=2cm,footskip=.2in]{geometry}
\usepackage{amsmath,amsfonts,amsthm,amssymb,mathtools}
\usepackage{enumitem}
\usepackage[]{mdframed}
\usepackage{tikz}
\renewcommand{\familydefault}{\sfdefault}
\usepackage{minibox}

\title{\Huge{ECE 30}\\Day 19}
\author{\huge{Elijah Hantman}}
\date{}

\begin{document}
\maketitle
\newpage

\begin{description}
    \item {\large Agenda} 
        \begin{itemize}
            \item Review Quiz 3
            \item Finish Modulation of Radio Waves
            \item Structure of The Atom
        \end{itemize}
    \item {\large Quiz 3 Review}
        \begin{mdframed}
           \begin{enumerate}
               \item I think I messed up cause I'm silly
                   Was supposed to divide 0.5 by 2, and get 0.25
               \item I don't remember what I answered. Odds are I got
                   this one wrong, cause I know I missed 3.
               \item This one I think I got correct. It really
                   should specify better the direction of current.
               \item I definitely got this one correct. I remember my answers.
               \item I got this one correct. I understood the
                   orthogonality part of magnetic fields.
               \item XD this one I definitely answered correctly
                   It was such a red herring style question
               \item I got this one wrong I think. I must have
                   forgotten to flip the direction for
                   the current. Idk The professor was getting
                   messed up, we definitely need diagrams for
                   this XD.
               \item I definitely messed this one up cause
                   I remember the answer I gave. It rotates
                   in the magnetic field.
               \item This was directly from lecture super easy
               \item Both fields. I got this one, it makes
                   sense since both fields are changing.
               \item Easy, straight from the definition of
                   magnetic forces. Definitely got this
                   one correct.
           \end{enumerate} 
        \end{mdframed}
    \item {\large Modulation}
        \begin{mdframed}
            Given some modulating signal $m(t)$. 
            We multiply it by $v(t)$ which is much higher
            frequency.

            By multiplying by $v(t)$ we get a signal which
            is easier to detect.


            The Frequency of $RF(v(t))$ signal  $>>$ the frequency
            of  $m(t)$. We can therefore filter out  $v(t)$ to
            leave  $m(t)$.

            When recieving, we filter the RF, and get back
             $m(t)$.
        \end{mdframed}
        \begin{mdframed}
            For a digital signal $m(t)$ will not be
            sinusoidal. It will instead by some series
            of high and low voltages in a fixed time.

            The simplest technique is to use Amplitude
            Shift Keying (ASK). It uses the same
            multiplicative idea. Essentially
            we can transform $m(t)$ to be between
            two non zero amplitudes.

            This has the downside of requiring massively
            high frequencies in order to have enough resolution
            to seperate each bit quickly.

            In WIFI, or Satellite Quadrature Amplitude
            Modulation is used (QAM).

            Two signals are used, each a quarter cycle
            out of phase. By blending between the two
            signals you can indicate a one or zero.
        \end{mdframed}
    \item {\large Frequency Shift Keying}
        \begin{mdframed}
            We instead tie two frequencies to a one
            and a zero, and modulate between them to indicate
            digital data.

            This is more reliable since the amplitude can remain
            high.

            It is used by LTE, and 5G technologies.
        \end{mdframed}
        \pagebreak
    \item {\large Structure of The Atom}
        \begin{mdframed}
            Observations by 1890s:
            \begin{itemize}
                \item Atoms contain electrons.
                \item The charge of electrons is negative and has been directly measured.
                \item Electrons were super light.
                \item Atoms are neutral (putting aside chemistry)
            \end{itemize}

            In the Early 20th Century the Thompson Model
            was created (1898). He modeled atoms as a ball
            of positive charge, with electrons embedded in
            its surface.

            When experimenting Rutherford folloed up in 1911
            by experimenting with firing $\alpha$ particles
            at ultra thin gold foil. He then measured
            how many particles made it through the foil.
            He saw that some did make it through, but some
            particles actually bounced back.

            Rutherford concluded that the nucleus of the atom
            must be tiny, and that it must have a powerful
            electric field. This means Atoms must be mostly
            empty space.

            Rutherford then developed an orbital model
            based on these conclusions.

            As a side note:
            \begin{mdframed}
                An electron volt is voltage multiplied by
                the charge of an electron. It is equivalent
                to a Joule in that they measure the same thing,
                however an Electron volt is nicer for
                physics since it uses units of electrons.
            \end{mdframed}
            
            Assumes that electrons are bodies orbiting
            and that the centripedal force is equal
            to the electrostatic force.`
            
        \end{mdframed}
\end{description}


\end{document}
