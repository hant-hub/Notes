\documentclass{report}
\usepackage[tmargin=2cm, rmargin=1in, lmargin=1in,margin=0.85in,bmargin=2cm,footskip=.2in]{geometry}
\usepackage{amsmath,amsfonts,amsthm,amssymb,mathtools}
\usepackage{enumitem}
\usepackage[]{mdframed}
\usepackage{tikz}
\renewcommand{\familydefault}{\sfdefault}
\usepackage{minibox}

\title{\Huge{ECE 30}\\Day 21 Notes}
\author{\huge{Elijah Hantman}}
\date{}

\begin{document}
\maketitle
\newpage

\begin{description}
    \item {\large Agenda} 
        \begin{mdframed}
            \begin{itemize}
                \item The Bohr Postulate
                \item Atomic Energy Levels
                \item Intro to Solid State Physics
            \end{itemize}
        \end{mdframed}
    \item {\large The Bohr Postulate}
        \begin{mdframed}
            An electron can circle a nucleus if its orbit contains
            an integral number of waves.

            \begin{displaymath}
                nx = 2 \pi r_n
            \end{displaymath}

            Where $n \triangleq$ quantum number and
             $r_n$ is the radius of the orbit from the nucleaus.
            

             Substituting for $\lambda$ we get:

             \begin{displaymath}
                 \frac{nh}{e} \sqrt{\frac{4\pi \epsilon_0 r_n}{m}} = 2 \pi r_n
             \end{displaymath}
             
             Therefore:

             \begin{displaymath}
                 r_n = \frac{n^2 h^2 \epsilon_0}{\pi m_e e^2}
             \end{displaymath}
             
             Also the energy levels in terms of orbit radius.

             \begin{displaymath}
                 E_n = \frac{-e^2}{8 \pi \epsilon_0 r_n}
             \end{displaymath}

             Substituting:

             \begin{displaymath}
                 E_n = \frac{-me^4}{8 \epsilon_0^2 h^2} \frac{1}{n^2}
             \end{displaymath}
             

             \begin{displaymath}
                E_n \triangleq \text{ Energy Level of the Hydrogen Atom}
             \end{displaymath}
             
             Electrons away from an atom can be any energy level.

             As it approaches an atom it is constrained to specific
             discrete energy levels. Calculating $E_1$ we get
             -13.6eV.

             \begin{mdframed}
                 Note: 1eV = $1.588\times 10^{-19}$J    
             \end{mdframed}
             
             Above $n = 2$ The energy becomes less negative, which means
             the electron is less attracted. In addition the higher
             the energy levels the closer the energy levels are to each other.

             Because of the large gaps between energy levels for low
             values of $n$ electrons are very stable and require
             tens of electron volts per electron to ionize.
        \end{mdframed}
        \pagebreak
        \begin{mdframed}
           Bohr: Discrete proton emitted when an electron moves
           from one energy level to another, releasing the difference
           in energy as electromagnetic radiation.

           From Planck: Photon energy $E = hf$

            \begin{displaymath}
                f = \frac{me^4}{8\epsilon_0^2 h^3} (\frac{1}{n_f^2} - \frac{1}{n_s^2})
           \end{displaymath}

           And for Photons, $\lambda = \frac{C}{f}$.
           Therefore:

           \begin{displaymath}
                \frac{1}{\lambda} = \frac{f}{C}
                = \frac{me^4}{8\epsilon_0^2 C h^3} (\frac{1}{n_f^2} - \frac{1}{n_s^2})
           \end{displaymath}
           
           So the spectrum of light emissions from H gas
           fits perfectly with the empirical measurements.
           
        \end{mdframed}

    \item {\large Solid State Physics}
        \begin{mdframed}
           \begin{displaymath}
               E_p = -\frac{1}{4 \pi \epsilon_0} \frac{q |e|}{x}
           \end{displaymath}
            
            This is the potential energy of an electron.

            $q \triangleq$ Positively charged ion (nucleus)

            $e \triangleq$ Charge of he electron

            $x \triangleq$ Distance between q and e.
        \end{mdframed}
        \begin{mdframed}
            As an electron is brought near to the nucleus
            the energy will go deeper negative. However
            because of the discretization of energy
            levels it will drop to the lowest energy level it
            can.

            In a lattice the potential energy of the electrons
            follows a periodic, wave like distribution, with
            many holes with slight bumps between ions.


            What happens to Bohr Energy Levels under these
            conditions?

            Schroedinger's equation describes the probability distribution
            for an electron under some arbitrary situation. Analytically
            unsolvable.

            We know:
            \begin{displaymath}
                p = mv
            \end{displaymath}
            And: 
            \begin{displaymath}
                K_e = \frac{1}{2}mv^2
            \end{displaymath}

            We know from De Brogli and Planck:

            \begin{displaymath}
                p = \frac{h}{\lambda}                
            \end{displaymath}
            
            Combining:

            \begin{displaymath}
                E = \frac{p^2}{2m}
            \end{displaymath}
        
        \end{mdframed}
        \pagebreak
        \begin{mdframed}
            Which implies:

            \begin{displaymath}
                E = (\frac{hk}{2 \pi})^2 \frac{1}{2m}
            \end{displaymath}

            Where $k = \frac{2\pi}{\lambda}$ 
            
            
            
             
        \end{mdframed}
\end{description}


\end{document}
