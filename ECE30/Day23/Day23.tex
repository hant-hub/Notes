\documentclass{report}
\usepackage[tmargin=2cm, rmargin=1in, lmargin=1in,margin=0.85in,bmargin=2cm,footskip=.2in]{geometry}
\usepackage{amsmath,amsfonts,amsthm,amssymb,mathtools}
\usepackage{enumitem}
\usepackage[]{mdframed}
\usepackage{tikz}
\renewcommand{\familydefault}{\sfdefault}
\usepackage{minibox}

\title{\Huge{ECE 30}\\Day 23 Notes}
\author{\huge{Elijah Hantman}}
\date{}

\begin{document}
\maketitle
\newpage

\begin{description}
    \item {\large Agenda} 
        \begin{itemize}
            \item Energy Bands
                \begin{itemize}
                    \item Insulators
                    \item Conductors
                    \item Semiconductors
                \end{itemize}
            \item P and N Doped Silicon
                \begin{itemize}
                    \item The MOSFET
                \end{itemize}
        \end{itemize}

    \item Energy Bands
        \begin{mdframed}
            Conductors are when the outer electron
            energy level is not empty.
            This allows electrons to move between
            atoms in a lattice when electric field
            is provided.
        \end{mdframed}
        \begin{mdframed}
            \begin{enumerate}
                \item For electrons to move in a field,
                    must have empty avalible states
                    and allowed avalible states.

                    When a band is full, electrons need
                    sufficient energy to jump to next
                    allowed band, otherwise they
                    cannot move.
                \item Each band has a limit on the number
                    of electrons. For a structure with
                    N atoms, s bands can hold 2N electrons,
                    and p bands can hold 6N electrons.
            \end{enumerate}

            Conductor:
            \begin{mdframed}
                Avalible states in partially filled bands.

                When a voltage is applied electrons are
                accelerated resulting in a current.

                For Sodium metal (11 electrons).

                We have a 1S layer with two electrons,
                a 2S band with two electrons,
                2P band with six electrons,
                3S band with a single electron.

                There is space in the outer band and
                there are electrons in the outer band,
                the material must be a conductor.
            \end{mdframed}

            Insulator:
            \begin{mdframed}
                No avalible spaces in the upper band,
                if a voltage is applied no electrons can move.

                If the voltage is high enough the electrons
                can be stripped off.

                ex: Helium. (2 electrons)

                First band 1S has two electrons which is full,
                the next band 2S is completely empty.

                The first layer has no free states, and the
                outer band has no free electrons, therefore
                Helium is an insulator.
            \end{mdframed}
        \end{mdframed}

        \begin{mdframed}
            Semiconductor:
            \begin{mdframed}
                In some ways a midpoint between Conductors
                and Insulators. All semiconductors are group
                four elements.

                Known semiconductors are Silicon, Germainium,
                Carbon, etc.

                If we graph Energy levels against
                Inter atomic distance we find something
                interesting. When they are far apart
                they have normal energy levels. As they
                form a crystal the 1S level becomes a
                band, like normal, but the 2S and 2P
                levels merge before separating around
                the table crystal lattice. This means
                that the energy bands are extremely
                close. We end up with four electrons
                in a new valence band, and no
                electrons in the conduction band.

                Since the valence and conduction band
                are so close we can use a normal voltage to
                force electrons from the valence to the
                conduction band without completely
                ionizing the material.

                The energy gap between the valence
                and conduction bands is known as $E_g$.

                Here are the values for various
                semiconductors.

                Silicon 1.1eV

                Carbon 6eV

                Germainium 0.7eV
            \end{mdframed}

            This fact was used by Brattain, Bardeen,
            and Shockley to create the first
            Field Effect Transistors with Germainium.

            Shockley was involved in the creation of Intel
            and Fairchild semiconductors which shaped
            computing for decades.

            All three were working on semiconductors
            at Bell labs which was the R\&D department
            of AT\&T.

        \end{mdframed}
    \item Minor Notes:
        \begin{mdframed}
            Holes: Normally we show the flow of positive current,
            however in real life only electrons exist which are
            negative. However mathematically the states where electrons
            could be, but aren't move just like electrons and can
            be treated as "positive" particles.

            These missing electrons are positively charged, and disappear
            when meeting an electron.

            Electron holes move in the opposite direction as
            electrons.

            Holes behave exactly the same as electrons but
            with opposite charge. Which is extremely helpful
            for calculation.
        \end{mdframed}
    \item Doped Silicon
        \begin{mdframed}
            All the group three atoms have one less electrons
            in their outer shell, and all the group five
            atoms have one additional electron in their
            outer shell.

            By inserting group three or group four electrons
            in the semiconductor lattice, we can change the
            charge of the structure.

            N-type silicon is created with Phosphorus usually
            and adds a free electron which hovers just below the
            conduction band meaning it is really easy to conduct.

            Around $10^{-2}$eV below conduction band.

            When electrons move because they are free no holes move.
            These are called majority carriers. N-type semiconductors
            are easily excited into conducting electrons.

            When doping with a group three element we get P-type
            semiconductors which have opposite properties. Usually
            Boron or Arsenic is used.

            To create these doped semiconductors gaseous group 3 or 4
            elements are diffused into Silicon at extremely high temperature,
            using photo lithography to control which portions are
            doped using which element. Alternate method is ion implantation
            where you ionize the gas first.

        \end{mdframed}
        \pagebreak
    \item MOSFET (Metal Oxide Semiconductor Field Effect Transistor)
        \begin{mdframed}
            Using a P type substrate with two  
            N type portions embedded on the surface. One is the
            drain and on is the source.

            We then cover the surface with Silicon dioxide which
            is a dielectric, and on top of the Dielectric layer
            we put some metal which is usually doped polysilicon,
            this is called the Gate.

            The Gate is connected to a variable voltage source,
            which we can control. The substrate is connected
            to ground.

            The drain is connected to a resistor and then ground.
            The source is connected to ground.

            If the gate is at zero nothing conducts, if we 
            apply a voltage to the gate, a negative charge
            is created between the source and drain creating
            a channel allowing for the source and drain
            to conduce electrically.

            If we reverse the n and p silicon we get the opposite
            behavior.
        \end{mdframed}
\end{description}


\end{document}
