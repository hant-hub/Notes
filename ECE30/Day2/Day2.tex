\documentclass{report}
\usepackage[tmargin=2cm, rmargin=1in, lmargin=1in,margin=0.85in,bmargin=2cm,footskip=.2in]{geometry}
\usepackage{amsmath,amsfonts,amsthm,amssymb,mathtools}
\usepackage{enumitem}
\usepackage[]{mdframed}
\usepackage{tikz}
\usepackage{minibox}

\title{\Huge{ECE 30}}
\author{\huge{Elijah Hantman}}
\date{}

\renewcommand{\familydefault}{\sfdefault}

\begin{document}
\maketitle
\newpage

\begin{description}
    \item Agenda 
        \begin{itemize}
            \item HW Set 1 Posted
            \item Vectors and Scalars
            \item Reference Frames, vector Components, and Vector Sum
            \item Unit Vectors
            \item Vector Dot and Cross Products
        \end{itemize}
\end{description}
{\huge Vectors and Scalars}
\begin{description}
    \item What is a Vector? 
        \begin{mdframed}
            In real life many quantities cannot be directly
            mapped to a real number. Even in one dimension, the
            sign indicates the direction.
        \end{mdframed}
        {\large Velocity}
        \begin{mdframed}
            \begin{displaymath}
                v = \frac{\Delta x}{\Delta t}
                = \frac{x_f - x_0}{t_f - t_0}
            \end{displaymath}

            Velocity is a representation of how 
            displacement or position changes with time.
            It is a correlation with time.

            In a context of calculus it is a kind of numerical
            integration. Velocity is the first derivative of
            position with respect to time.

            \begin{displaymath}
                v = \frac{d}{dt}x
            \end{displaymath}

            Both velocity and displacement are vector quantities.
            Time is arguably vector but in classical physics is
            only a scalar.
            
            \begin{center}
                \minibox[frame]{
                    Vectors are quantities with both direction
                    and magnitude. They are denoted $\vec{v}$
                }
            \end{center}

            \begin{center}
                \minibox[frame]{
                Scalars are magnitudes only. They are written
                as $s$
                }
            \end{center}

        \end{mdframed}
    \item Reference Frames
        \begin{mdframed}
           For vectors to make sense we need a defined start point.
           In application things like compasses, maps, gyroscopes, etc.
           can be used to define shared reference frames for navigation
           or calculations.

           Reference frames are arbitrary, they are useful because they
           are relevant or shared.

           \begin{center}
               \minibox[frame]{ 
                   Reference frames are a social construct lol
               }
           \end{center}

           Reference frames are required to define direction.
           It is helpful to imagine our reference frames as composed
           of orthogonal components, of course this is only relevant
           when conceptualizing a single reference frame or in translating
           between reference frames.

           Approaches to Vectors.
           \begin{itemize}
               \item Polar. The magnitude and direction are stored
                   as two seperate values. ie: Angle and size.
                   This is useful in navigation or rotationally
                   symmetric contexts.
               \item Component. The vector is decomposed into
                   multiples of some unit vector, which encodes
                   the relative magnitudes along each unit vector.
                   Unit vectors are the base of this system.
                   Component vectors are useful for combining
                   vectors or for using matricies.
           \end{itemize}

           Formulas:
           \begin{enumerate}
               \item Magnitude:
                   \begin{displaymath}
                    |\vec{v}| = \sqrt{\sum v_i^2}
                   \end{displaymath}
                   It is sometimes denoted:
                   \begin{displaymath}
                       v = \sqrt{\sum \vec{v}_i^2}
                   \end{displaymath}
                   
               \item Direction:\\
                   For 2D
                   \begin{displaymath}
                       tan(\theta) = \frac{v_2}{v_1}
                   \end{displaymath}
                   For 3D
                   \begin{displaymath}
                        (x, y, z) = 
                        (cos(\theta)sin(\phi), cos(\theta)cos(\phi), sin(\theta))
                   \end{displaymath}
                    
                   \begin{center}
                       \minibox[frame]{
                           For 2D direction is equivalent to
                           mapping a circle to a line.\\
                           For 3D direction is equivalent to
                           mapping the surface of a sphere to
                           a plane.
                       }
                   \end{center}
               \item Summation:
                   \begin{gather}
                        \vec{a} = (a_1, a_2, ... a_i)\\
                        \vec{b} = (b_1, b_2, ... b_i)\\
                        \vec{a} + \vec{b}
                        = (a_1 + b_1, a_2 + b_2, ... a_i + b_i)
                   \end{gather}
                   
                   It can also be visualized by imagining laying
                   each vector tip to tail, and the vector from the
                   origin to the tip of the last vector is the sum.

                   \begin{center}
                       \minibox[frame]{
                           This property of summing nicely along components
                           is one of the properties required for linearity\\
                           and therefore linear algebra.
                       }
                   \end{center}
                                       
               \item Polar Conversions in 2D
                   \begin{gather}
                       \vec{v} = (v_1, v_2) = (\theta, v)\\ 
                       v_1 = |\vec{v}|cos(\theta)\\
                       v_2 = |\vec{v}|sin(\theta)\\
                       v = \sqrt{v_1^2 + v_2^2}\\
                       \theta = atan2(v_2/v_1)
                   \end{gather}

                   Note: atan2 depends on the sign of the
                   components.
           \end{enumerate}
        \end{mdframed}

    \item Unit Vectors
        \begin{mdframed}
            We can conceptualize a vector as a sum of 
            components. This can be represented algebraicly
            using symbols. One of the most famous vector
            quantities are complex numbers. $a + bi$

            By convention three dimensional vectors in physics
            are written as:
            \begin{displaymath}
                a\vec{i} + b\vec{j} + c\vec{k}
            \end{displaymath}

            This ties neatly into quaternions which are written
            \begin{displaymath}
                a\vec{i} + b\vec{j} + c\vec{k} + d
            \end{displaymath}

            A unit vector corresponds to a single step along
            an orthogonal direction in our reference frame.
            Unit vectors define a reference frame and are
            arbitrary.
            \begin{center}
                \minibox[frame]{
                    As an aside: we have already encountered
                    two different unit vectors.\\

                    Polar coordinates use a degree which steps\\
                    around a circle as one unit, and a unit which
                    moves away from the origin as a second vector.\\

                    Component coordinates use $\hat{x}$ and $\hat{y}$\\
                    It should be noted that Polar coordinates are nonlinear\\
                    and do not share many properties with components.
                }
            \end{center}
            
            In the abstract, the way a linear operation works on the
            unit vectors define the way it works on the entire
            vector space.

            \begin{center}
                \minibox[frame]{
                    It should be noted that unit vectors in physics
                    are unitless.
                }
            \end{center}
        \end{mdframed}
\end{description}
\end{document}
