\documentclass{report}
\usepackage[tmargin=2cm, rmargin=1in, lmargin=1in,margin=0.85in,bmargin=2cm,footskip=.2in]{geometry}
\usepackage{amsmath,amsfonts,amsthm,amssymb,mathtools}
\usepackage{enumitem}
\usepackage[]{mdframed}
\usepackage{tikz}
\renewcommand{\familydefault}{\sfdefault}
\usepackage{minibox}

\title{\Huge{ECE 30}\\Day 16 Notes}
\author{\huge{Elijah Hantman}}
\date{}

\begin{document}
\maketitle
\newpage
\begin{description}
    \item {\large Agenda} 
        \begin{mdframed}
            \begin{itemize}
                \item Sinusoidal Traveling Waves - Complete
                \item Magnitism Introduction
                \item Magnetic Force
            \end{itemize}
        \end{mdframed}
    \item {\large Sinusoidal Traveling Waves}
        \begin{mdframed}
            By definition: 
            \begin{displaymath}
                v = \frac{\lambda}{T}
            \end{displaymath}

            therefore:

            \begin{displaymath}
                y(x,t) = A sin(2\pi (\frac{x}{\lambda} - \frac{t}{T})
            \end{displaymath}

            The Wave number $k \triangleq \frac{2\pi}{\lambda}$:

            We know  $\omega = \frac{2\pi}{T}$:

             \begin{displaymath}
                y(x,t) = A sin(kx - \omega t)
            \end{displaymath}

            In general: 
            \begin{displaymath}
                y(x,t) = A sin(kx - \omega t + \phi)
            \end{displaymath}
        \end{mdframed}
    \item {\large Magnetism}
        \begin{mdframed}
            There exists a magnetic field, written
            with $\vec{B}$ which is measured in
            Teslas (T).

            The direction of $\vec{B}$ is the direction
            of a compass needle held at that point.

            Magnetic fields always flow from North
            poles to South poles. Similar to Electric
            fields which always flow from Positive
            to Negative.

            We can define $\vec{B}$ at any point in space
            by the magnetic force exerted on a charged
            particle moving ad $\vec{v}$

            Magnetic force is written $\vec{F_B}$
        \end{mdframed}
        \pagebreak
    \item {\large Magnetic Forces}
        \begin{mdframed}
           We can empirically determine: 
           \begin{enumerate}
               \item $|\vec{F_B}| \propto q_1 |\vec{v}|$
               \item If $\vec{v}$ is parallel to  $\vec{B}$
                   then  $\vec{F_B} = 0$
               \item The direction of $\vec{F_B}$ is
                   perpendicular to both  $\vec{v}$
                   and  $\vec{B}$
               \item $\vec{F_B}$ on a positive charge is opposite
                   $\vec{F_B}$ on a negative charge.
               \item $|\vec{F_B}| \propto |\vec{v} \times \vec{B}|$
           \end{enumerate}

           These properties imply that:

           \begin{displaymath}
                \vec{F_B} = k(\vec{v} \times \vec{B})
           \end{displaymath}
           
           Magnetic fields use the right hand rule. We curl
           our right hand from $\vec{v}$ to  $\vec{B}$, and
           our thumb points in the direction of $\vec{F_B}$.
           The standard cross product uses this rule. This
           only applies if $q$ is positive. 

           Therefore:

           \begin{displaymath}
                \vec{F_B} = q(\vec{v} \times \vec{B})
           \end{displaymath}

           This is one of the cases which motivated the
           creation of the cross product.

           We also get the identity:

           \begin{displaymath}
            |\vec{F_B}| = q |\vec{v}| |\vec{B}| sin(\theta)
           \end{displaymath}

           This comes from the definition of the cross product.
           
           $\vec{B}$ on a current carrying conductor can
           has the following properties according to
           experiments

           \begin{enumerate}
               \item $d\vec{B} \propto \vec{r}, d\vec{s}$
           \end{enumerate}

           
        \end{mdframed}
\end{description}



\end{document}
