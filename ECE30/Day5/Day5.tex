\documentclass{report}
\usepackage[tmargin=2cm, rmargin=1in, lmargin=1in,margin=0.85in,bmargin=2cm,footskip=.2in]{geometry}
\usepackage{amsmath,amsfonts,amsthm,amssymb,mathtools}
\usepackage{enumitem}
\usepackage[]{mdframed}
\usepackage{tikz}
\renewcommand{\familydefault}{\sfdefault}
\usepackage{minibox}

\title{\Huge{ECE 30}}
\author{\huge{Elijah Hantman}}
\date{}

\begin{document}
\maketitle
\newpage

\begin{description}
    \item {\large Agenda}
        \begin{itemize}
            \item Newton's Laws
            \item Hooke's Law and Oscillatory Motion
            \item Young's Modulus
            \item Three point beam springs
        \end{itemize}
    \item {\large Newton's Laws}
        \begin{enumerate}
            \item A Body in Motion Will Stay in Motion
                \begin{mdframed}
                    Frames of reference mean that
                    constant velocity is interchangable
                    with being at rest.
                \end{mdframed}
                \begin{mdframed}
                    A definition of Mass:
                    \begin{mdframed}
                        That property of an object which
                        specifies how much resistance an
                        object has to its velocity being
                        changed.
                    \end{mdframed}
                    In other words:
                    \begin{mdframed}
                        Mass is a measure of inertia, or
                        resistance to acceleration.
                    \end{mdframed}
                \end{mdframed}
                \begin{mdframed}
                    Force is defined as:
                    \begin{displaymath}
                        F = ma
                    \end{displaymath}
                    Or in other words:
                    \begin{mdframed}
                        Force is what changes the motion of
                        an object.
                    \end{mdframed}
                \end{mdframed}
            \item Body subjected to multiple forces is accelerated
                in the direction of the resultant force.
                \begin{mdframed}
                    This implies the relationship
                    \begin{displaymath}
                        F \propto m
                    \end{displaymath}
                    And it implies:
                    \begin{displaymath}
                        F \propto a
                    \end{displaymath}

                    Therefore, force can be written as:
                    \begin{displaymath}
                        \vec{F} = m\vec{a}
                    \end{displaymath}

                    \begin{center}
                        \minibox[frame]{
                            As a sidenote: Force is an arbitrary\\
                            unit. Its one of those units we made\\
                            up to explain things.
                        }
                    \end{center}

                    \begin{displaymath}
                        \vec{F} = \sum_{i = 1}^N \vec{F}_i 
                    \end{displaymath}

                    These are vector sums which also means that
                    the final sum can be decomposed into component
                    wise sums.
                \end{mdframed}
                \pagebreak
                \begin{mdframed}
                   What Units is Force in?

                   \vspace{10pt}

                   Mass is measured in kg, and acceleration in
                   $m/s^2$, Force must be measured in $kg  \frac{m}{s^2}$.
                   This unit is known as a Newton.
                \end{mdframed}
                \begin{mdframed}
                    Weight vs Mass

                    \vspace{10pt}

                    Weight is defined as the force exerted on a body
                    via gravity, which is equal to the force a body
                    exerts on the planet. This means the units for
                    weight is actually Newtons. Weight is only measured
                    in kg because the gravitational acceleration is
                    reasonably constant on Earth.

                    \begin{displaymath}
                        W = mg
                    \end{displaymath}
                \end{mdframed}
                \begin{mdframed}
                    Equilibrium:

                    \vspace{10pt}

                    This is the state where the net force is
                    zero.

                    \begin{displaymath}
                        \sum_{i=1}^N \vec{F}_i = 0
                    \end{displaymath}
                    
                    When the above equation is not true, that
                    is known as Non-Equilibrium.
                \end{mdframed}
            \item Every Action has an Equal and Opposite Reaction
                \begin{mdframed}
                    If a body acts on anther, both bodies experience force
                    in opposite directions from the point of contact.
                \end{mdframed}
        \end{enumerate}
    \item Hooke's Law
        \begin{mdframed}

            The force a spring exerts is proportional to how
            much it is stretched or compressed.
            \begin{displaymath}
                F \propto -x
            \end{displaymath}
            
            Where $x$ is the displacement of the length of a spring
            from its resting length.

            Further investigation revealed that $F$ is equal to
            the displacement multiplied by some constant.

            \vspace{10pt}



            The force of a spring is equal to a constant times the 
            displacement of the spring from its resting length.
            \begin{displaymath}
                F = kx
            \end{displaymath}
            
            The spring constant $k$ has units,  $N/m$ which is required
            for the total equation to have correct units.
        \end{mdframed}
    \item Harmonic Oscillation
        \begin{mdframed}
            \begin{gather}
                p(t) = x_0 cos(t \frac{2\pi}{T})
            \end{gather}
            Where $T$ is the time between peaks.


        \end{mdframed}

\end{description}



\end{document}
