\documentclass{report}
\usepackage[tmargin=2cm, rmargin=1in, lmargin=1in,margin=0.85in,bmargin=2cm,footskip=.2in]{geometry}
\usepackage{amsmath,amsfonts,amsthm,amssymb,mathtools}
\usepackage{enumitem}
\usepackage[]{mdframed}
\usepackage{tikz}
\usepackage{siunitx}
\renewcommand{\familydefault}{\sfdefault}
\usepackage{minibox}

\title{\Huge{ECE 30}\\Day 7 Notes}
\author{\huge{Elijah Hantman}}
\date{}

\begin{document}
\maketitle
\newpage

\begin{description}
    \item {\large Agenda} 
        \begin{itemize}
            \item Review Quiz 1
            \item Electric Charge
            \item Coulomb's Law
            \item Superposition
        \end{itemize}
    \item {\large Quiz 1 Review}
        \begin{mdframed}
            Lost points on a silly pedantic thing.
            When asking if velocity is equal don't assume
            there is a conversion to magnitude.

            Everything else was fine, just expect the
            pedantic next time.
        \end{mdframed}
    \item {\large Electric Charge}
        \begin{mdframed}
            Amber and Cat fur can be used to make amber repel. 
            Glass and silk also work.

            If you touch them together they repel and
            remain??

            When brought near after rubbing amber and cat
            fur. It collapses and repels.

            Conclusions:
            \begin{enumerate}
                \item Amber and Cat fur generated a negative
                    charge
                \item Glass and silk generate a positive
                    charge
                \item Like charges Repel
                \item Opposite Charges Attract
                \item Charges flow through conductive
                    materials
                \item Charges cannot flow through insulators
            \end{enumerate}

            The positive and negative assignment was arbitrary.

            Set up an experiment called Coulomb Experiment
            to test his hypotheses.

            Set up two hanging masses, with opposite charges.
            To measure the strength of the force he tied
            both masses to pulleys attached to masses.
            By measuring the mass required for the masses
            to hang without touching in equilibrium you
            can measure the force exerted by the charge.

            He came to some conclusions.
            \begin{enumerate}
                \item $F_e \propto \frac{1}{d^2}$
                \item $F_e \propto \frac{q_1 q_2}{d}$
            \end{enumerate}

            Introduced a constant of proportionality.
            \begin{displaymath}
                F_e = k_e \frac{q_1 q_2}{d^2}
            \end{displaymath}
            
            Which became known as Coulomb's Law.

            The Units for $K_e$ must be  $d^3 m/q^2$.
            The world eventually decided on Coulomb for
            measuring charge. So  $K_e$ has units of
            $\si{Nm^2/C^2}$

            An alternative formulation is known.
            \begin{displaymath}
                k_e = \frac{1}{4\pi e_0}
            \end{displaymath}
            Where $e_0$ is the permitivity of free space.
            
        \end{mdframed}
        \pagebreak
    \item {\large Superposition}
        \begin{mdframed}
            Net force is still the sum of all forces acting
            on a body. This applies to Coulomb as well as
            Newton.

            \begin{displaymath}
                q_1, q_2, q_3, q_4
            \end{displaymath}
            \begin{displaymath}
                \vec{F} = \vec{F_{21}} + \vec{F_{31}}
                + \vec{F_{41}}
            \end{displaymath}
            
        \end{mdframed}

\end{description}



\end{document}
