\documentclass{report}
\usepackage[tmargin=2cm, rmargin=1in, lmargin=1in,margin=0.85in,bmargin=2cm,footskip=.2in]{geometry}
\usepackage{amsmath,amsfonts,amsthm,amssymb,mathtools}
\usepackage{enumitem}
\usepackage[]{mdframed}
\usepackage{tikz}
\usepackage{siunitx}
\usepackage{circuitikz}
\renewcommand{\familydefault}{\sfdefault}
\usepackage{minibox}

\title{\Huge{ECE 30}\\Day 11 Notes}
\author{\huge{Elijah Hantman}}
\date{}

\begin{document}
\maketitle
\newpage

\begin{description}
    \item {\large Agenda} 
        \begin{itemize}
            \item Series and Parallel Circuits
            \item Kirchoff's Rules
            \item Electric Circuits
        \end{itemize}
    \item {\large Series and Parallel Circuits}
        \begin{mdframed}
            \begin{circuitikz} \draw
                (6,2) to[battery] (0,2)
                (0,0) to[resistor] (2,0)
                (2,0) to[resistor] (4,0)
                (4,0) to[resistor] (6,0)
                (6,0) to[short] (6,2) to[short] (3,2)
                (3,2) to[short] (0,2) to[short] (0,0);
            \end{circuitikz} 

            \begin{displaymath}
                V = iR_1 + iR_2 + iR_3
                = v_1 + v_2 + v_3
            \end{displaymath}
            
            \begin{displaymath}
                V = i(R_1 + R_2 + R_3)
                = i R_{eq}
            \end{displaymath}

            Resistors in series are equivalent to 
            a resistor with the sum of their resistances.
            Resistance adds in series.

            \begin{circuitikz} \draw
                (0,0) to[capacitor] (3,0)
                (3,0) to[capacitor] (6,0)
                (6,0) to[short] (6,2)
                (6,2) to[battery] (0,2)
                (0,2) to[short] (0,0);
            \end{circuitikz}
            
            The first capacitor charges to $Q_1$, and its
            other half charges to $-Q_1$, the next plate ends
            up charging to $Q_2$ and its other plate charges
            to $-Q_2$. Since $Q_2$ gets its charge from the
            first capacitor $Q_1 = Q_2$.

            The charge is the same when capacitors are in series.

            \begin{displaymath}
                v = v_1 + v_2
                = \frac{Q}{C_{eq}} = \frac{Q_1}{C_1} + \frac{Q_2}{C_2}
            \end{displaymath}

            Therefore:

            \begin{displaymath}
                \frac{1}{C_{eq}} = \frac{1}{C_1} + \frac{1}{C_2}
            \end{displaymath}
            
            Capacitors in series have smaller capacitance.
        \end{mdframed}
        \pagebreak
        \begin{mdframed}
            \begin{circuitikz}\draw
                (0,0) to[resistor] (3,0)
                (0,2) to[resistor] (3,2)
                (0,4) to[resistor] (3,4)
                (0,6) to[battery] (3,6)
                (0,0) to[short] (0,6)
                (3,0) to[short] (3,6);
            \end{circuitikz}

            \begin{displaymath}
                v_1 = v_2 = v_3 = v
            \end{displaymath}
            \begin{displaymath}
                i = i_1 + i_2 + i_3
            \end{displaymath}

            \begin{displaymath}
                \frac{v_1}{r_1} + \frac{v_2}{r_2} + \frac{v_3}{r_3} 
                = i = \frac{v}{r_{eq}}
            \end{displaymath}

            Dividing using the first equation:

            \begin{displaymath}
                \frac{1}{r_{eq}} = \frac{1}{r_1} + \frac{1}{r_2} + \frac{1}{r_3}
            \end{displaymath}

            Parallel resistors reduce the overall resistance. Mirror to
            Capacitors in series.
            
            
            \begin{circuitikz}\draw
                (0,0) to[capacitor] (3,0)
                (0,2) to[capacitor] (3,2)
                (0,4) to[battery] (3,4)
                (0,0) to[short] (0,4)
                (3,0) to[short] (3,4);
            \end{circuitikz}
             
            The equivalent charge is the sum of the charges
            of each capacitor.

            \begin{displaymath}
                Q_{eq} = Q_1 + Q_2
            \end{displaymath}

            The voltage differences must be the same.

            \begin{displaymath}
                v_1 = v_2 = v
            \end{displaymath}

            The total charge:

            \begin{displaymath}
                Q = Q_1 + Q_2
            \end{displaymath}
            
            \begin{displaymath}
                c_{eq} v = c_1v_1 +c_2v_2                 
            \end{displaymath}
            \begin{displaymath}
                c_{eq} = c_1 + c_2
            \end{displaymath}
            
            Capacitors in parallel add. 
        \end{mdframed}
    \item {\large Kirchoff's Rules}
        \begin{mdframed}
           \begin{enumerate}
               \item The algebraic sum of currents flowing
                   into any junction in a circuit
                   is always zero. (every current flowing
                   in is balanced by current flowing out).
                   \begin{mdframed}
                       Currents are a vector field with
                       a divergence equal to zero. Similar
                       to water.
                   \end{mdframed}
               \item Algebraic of all potential differences
                   (voltage) in a closed circuit loop
                   is always zero.
                   \begin{mdframed}
                       The voltage lift across a battery
                       must equal the voltage drop across
                       the circuit. A closed circuit cannot
                        elevate or reduce the total
                        potential across the entire circuit.
                   \end{mdframed}
               \item 
           \end{enumerate}  
        \end{mdframed}
\end{description}


\end{document}
