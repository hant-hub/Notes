\documentclass{report}
\usepackage[tmargin=2cm, rmargin=1in, lmargin=1in,margin=0.85in,bmargin=2cm,footskip=.2in]{geometry}
\usepackage{amsmath,amsfonts,amsthm,amssymb,mathtools}
\usepackage{enumitem}
\usepackage[]{mdframed}
\usepackage{tikz}
\renewcommand{\familydefault}{\sfdefault}
\usepackage{minibox}

\title{\Huge{ECE 30}\\Day 22 Notes}
\author{\huge{Elijah Hantman}}
\date{}

\begin{document}
\maketitle
\newpage

\begin{description}
    \item {\large Agenda} 
        \begin{mdframed}
            \begin{itemize}
                \item Review of Midterm makeup
                \item Schrodinger Equation Thoughts
                \item Energy Bands
                    \begin{itemize}
                        \item Insulators
                        \item Conductors
                        \item Semiconductors
                    \end{itemize}
                \item P and N Doped Silicon
                    \begin{itemize}
                        \item MOSFET
                    \end{itemize}
            \end{itemize}
        \end{mdframed}
    \item {\large Midterm Makeup Review}
        \begin{mdframed}
           lol didn't need to take it 
        \end{mdframed}
    \item {\large Schrodinger Thoughts}
        \begin{mdframed}
            \begin{itemize}
                \item No physical way to measure Bohr energy
                    shells, due to Uncertainty principal (Heisenberg)
                \item Schrodinger applied Maxwell's equations to matter
                    \begin{itemize}
                        \item Conservation of Energy
                        \item De Brogli Hypothesis
                    \end{itemize}
                \item Resultant $\Psi^2$ is the most detailed
                    information we can actually measure.
                \item We say that $\Psi^2$ is physical reality,
                    and Bohr orbits are an abstraction
            \end{itemize}

            \begin{displaymath}
                K + U = E
            \end{displaymath}
            \begin{displaymath}
                K = \frac{-h^2}{2m} \frac{d^2 \Psi(x)}{dx^2}
            \end{displaymath}
            \begin{displaymath}
                U = U(x)\Psi(x)
            \end{displaymath}
            \begin{displaymath}
                E = E \Psi(x)
            \end{displaymath}
            
            Where $\Psi$ is the disturbance in space
            due to the matter wave, and 
             $\Psi^2 dx$ is the probability of finding
             the electron in a given length $dx$. 

             This is the Time independent Single
             Dimension Schrodinger Equation, since
             it only considers a single spatial
             dimension.
        \end{mdframed}
    \item {\large Energy Bands}
        \begin{mdframed}
            In a regular lattice you get alternating
            bands of forbidden and allowed states
            for electrons.
            

            For example, a Sodium Lattice:
            \begin{itemize}
                \item Energy Shells for one atom
                \item 1S, 2S, 2P, 3S are Bohr energy
                    levels.
                    \begin{mdframed}
                        Note: since energy levels are actually
                        probability distributions over
                        the energy levels, the distribution
                        depends on temperature.
                    \end{mdframed}
                \item Each S level accomodates two electrons,
                    this comes from the spin of electrons,
                    so electrons get two additional states
                    $+\frac{1}{2}$ and  $-\frac{1}{2}$.
                \item The P levels accomodate 6 electrons.
                    This is because a P shell has multiple
                    lobes which can allow for up to six
                    electrons.
                \item Higher energy levels allow for more possible
                    shapes and more possible electrons in the same
                    energy level.
                \item This combined means a Sodium atom has
                    11 electrons to match the protons,
                    That means each level has 2, 2, 6,
                    1 electrons respectively.
            \end{itemize}

            Sodium is a conductor because the shells are close enough
            to be jumped betwen in a lattice, allowing electrons to flow.
        \end{mdframed}
\end{description}


\end{document}
