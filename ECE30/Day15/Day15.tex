\documentclass{report}
\usepackage[tmargin=2cm, rmargin=1in, lmargin=1in,margin=0.85in,bmargin=2cm,footskip=.2in]{geometry}
\usepackage{amsmath,amsfonts,amsthm,amssymb,mathtools}
\usepackage{enumitem}
\usepackage[]{mdframed}
\usepackage{tikz}
\renewcommand{\familydefault}{\sfdefault}
\usepackage{minibox}

\title{\Huge{ECE 30}\\Day 15 Notes}
\author{\huge{Elijah Hantman}}
\date{}

\begin{document}
\maketitle
\newpage

\begin{description}
    \item {\large Agenda} 
        \begin{mdframed}
            \begin{itemize}
                \item Traveling Waves
                \item Sinusoidal Traveling Waves and the Wave Function
                \item Magnetism - Intro
                \item Magnetic Force
            \end{itemize}
        \end{mdframed}
    \item Homework 5 Overview
        \begin{mdframed}
            its chill, its just voltage changes so its cool. 
        \end{mdframed}
    \item {\large Traveling Waves}
        \begin{mdframed}
            It begins as a single pulse in a medium like
            a rope or a field.

            Over time the pulse will travel down
            the medium.

            The velocity of propogation or the speed of the
            traveling wave is $\frac{\lambda}{T}$. Here it
            is the wavelength in space divided by the wavelength
            in time. It also implies:

            \begin{displaymath}
                v = \lambda f
            \end{displaymath}

            The shape of a traveling wave does not change as it
            moves.

            A given particle in space will be in the same position
            every $T$ seconds. Or  $x - vt$ at  $t = 0$.
            
            The wave is described as a function of both distance and
            time.

            \begin{displaymath}
                y(x,t) = y(x-vt, 0)
            \end{displaymath}
            
            In general:

            \begin{displaymath}
                y(x,t) = f(x-vt)
            \end{displaymath}

            Here the sign of the velocity encodes the
            direction of wave movement.

            \begin{displaymath}
                y(x,t) \triangleq \textrm{Wave Function}
            \end{displaymath}
            
            One example wave function could be:
            \begin{displaymath}
                y(x,t) = \frac{1}{(x-3t)^2 + 1}
            \end{displaymath}

            Wave functions can be any symmetric-ish function
            with a phase shift applied.
        \end{mdframed}
        \pagebreak
    \item {\large Sinusoidal Traveling Waves}
        \begin{mdframed}
            
            \begin{displaymath}
                y(x,0) = Asin(ax)
            \end{displaymath}

            Here $a$ is a conversion factor from distance to
            radians. It has units  $\frac{1}{m}$.

            When $\frac{\lambda}{2}$  $A sin(ax) = 0$

            \begin{displaymath}
                \frac{a\lambda}{2} = \pi
            \end{displaymath}
            \begin{displaymath}
                a = \frac{2\pi}{\lambda}
            \end{displaymath}
            
            therefore:

            \begin{displaymath}
                y(x,0) = A sin(\frac{2\pi}{\lambda} x)
            \end{displaymath}
            
            We know that $y(x,t) = y(x-vt, t)$, therefore:

             \begin{displaymath}
                y(x,t) = A sin(\frac{2\pi}{\lambda} (x-vt))
            \end{displaymath}
            
            
        \end{mdframed}
\end{description}


\end{document}
