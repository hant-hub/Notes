\documentclass{report}
\usepackage[tmargin=2cm, rmargin=1in, lmargin=1in,margin=0.85in,bmargin=2cm,footskip=.2in]{geometry}
\usepackage{amsmath,amsfonts,amsthm,amssymb,mathtools}
\usepackage{enumitem}
\usepackage[]{mdframed}
\usepackage{tikz}

\title{\Huge{Stats 130}\\Day 8 Notes}
\author{\huge{Elijah Hantman}}
\date{}

\begin{document}
\maketitle
\newpage

\begin{description}
    \item {\Large Poisson}
        \begin{mdframed}
            A Poisson random variable counts
            the number of occurances of an event.
            It has no upper bound.

            \begin{displaymath}
                f_X(x) = \frac{e^{-\lambda} \lambda^{x}}{x!} , x = 0, 1, 2, ...
            \end{displaymath}

            Here $\lambda$ is the is the intensity of the process
            and is proportional to the number of expected occurances
            in a given time frame.

            $f_X(x)$ is greater than 0 by defninition. $\forall x$

             \begin{displaymath}
                \sum_{x=0}^{\infty}f_X(x) =
                \sum_{x=0}^{\infty} \frac{e^{-\lambda} \lambda^{x}}{x!} =
                e^{-\lambda} \sum_{x=0}^{\infty} \frac{\lambda^{x}}{x!} =
                e^{-\lambda}e^{\lambda} = 1
            \end{displaymath}
            
            This shows the Poisson is a probabiilty function, and
            is denoted $X \sim Pois(\lambda)$
            
        \end{mdframed}
        \begin{mdframed}
            Example: The number of customers that order a pizza
            in a 30 min interval is a Poisson random variable
            with $\lambda = 5$. What is the probability that
            between 6 and 6:30 three or more customers will
            order pizza?

             \begin{gather}
                 f_X(x) = \frac{e^{-\lambda} \lambda^x}{x!}\\
                 f_X(x) = \frac{e^{-5} 5^x}{x!}, x = 0, 1, 2\\
                 Pr(X \ge 3) = 1 - (Pr(X = 0) + Pr(X = 1) + Pr(X = 2))\\ 
                 = 1 - e^{-5}(1 + 5, + \frac{25}{2}) = 0.875348
            \end{gather}
        \end{mdframed}
        \pagebreak
    \item {\Large Continuous Random Variables}
        \begin{mdframed}
            Reimagine the building power and water
            example. If we no longer require the
            demand to be a discrete integer value
            we now have a continuous random variable.
            There are an infinite number of elements
            in the Sample Space, and they cannot be traversed
            in a countable order.
        \end{mdframed}
    \item {\Large Cumulative Distribution Function}
        \begin{mdframed}
            With continuous random variables, the idea of
            $X = x$ is unhelpful, since  $Pr(X = x) 
            | X : S \to D \subseteq \mathbb{R}$ is always
            equal to zero or some infinitesimal. For continuous
            values it is more helpful and tractable to work with
            ranges.

            A CDF is defined as 
             \begin{displaymath}
                F_X(x) = Pr(X \le x)
            \end{displaymath}

            A CDF can be made for any random variable whether
            continuous or discrete. The capital letters are
            also part of the notation.

            Some properties of CDFs are:
            \begin{itemize}
                \item $0 \le F_X(x) \le 1, \forall x$
                \item  $F_X(x)$ is non decreasing.\\
                    $x_1 < x_2 \implies {X \le x_1} \subset {X \le x_2}$
                    therefore,
                    $Pr(X\le x_1) \le Pr(X\le x_2)$ 
                    which implies\\
                    $F_X(x_1) \le F_X(x_2)$ 
                \item $\lim\limits_{x \to -\infty} F_X(x) = 0$, 
                    $\lim\limits_{x \to +\infty} F_X(x) = 1$ 

                \item For $b > a$,  $Pr(X \in (a, b]) = Pr(a < X \le b)$\\
                     $= Pr(X \le b) - Pr(X \le a) = F_X(b) - F_X(a)$ 
            \end{itemize}

            If $F_X(x)$ is continuous for all  $x$, then $X$ is
            a continuous random variable.


            Imagine a discrete random variable with probability
            function and CDF as follows.
            \begin{gather}
               f_X(x) = 
               \begin{cases}
                   \frac{1}{k} & x = 1, 2, 3 ..., k\\ 
                   0 & \textrm{otherwise}
               \end{cases}\\
               F_X(x) = Pr(X \le x) = \sum_{i \le x} Pr(X = i) =
               \begin{cases}
                   0 & x<1\\ 
                   \frac{\lceil x \rceil}{k} & 1 \le x < k\\
                   1 & x \le k
               \end{cases}
            \end{gather}

            This CDF is not continuous which implies the random
            variable must be discrete.
        \end{mdframed}
        \pagebreak
    \item {\Large Probability Density Function}
        \begin{mdframed}
            When $X$ is a continuous random variable,
             $F_X(x)$ is both continuous and differentiable.
             Therefore we can calculate the derivative
             of  $F_X(x)$.

             The Probability Density Function or PDF, is
             defined as,

             \begin{displaymath}
                f_X(x) = F_X'(x)
             \end{displaymath}

             A PDF is not defined for discrete random variables
             since the CDF is not continuous or differentiable.
             
             In addition the lower case "f" is used since
             the discrete meaning of a probability function
             is undefined for continuous variables. This
             means only one possible meaning is valid at 
             a time which is unambiguous.

             \vspace{10}

             Properties:
             \begin{itemize}
                 \item 
                     \begin{displaymath}
                        F_X(x) = \int_{-\infty}^{x} f_X(t)dt \quad \forall x
                     \end{displaymath}
                \item
                    \begin{gather}
                       Pr(a < x \le b) = F_X(b) - F_X(a)\\ 
                       = \int_{-\infty}^{b}f_X(t)dt -
                       \int_{-\infty}^{a} f_X(t)dt\\
                       = \int_a^b f_X(t)dt
                    \end{gather}
                \item 
                    \begin{displaymath}
                    f_X(x) \ge 0, \forall x
                    \end{displaymath}
                \item 
                    \begin{displaymath}
                    Pr(X \in A) = \int_A f_X(t)dt
                    \end{displaymath}
                \item
                   \begin{displaymath}
                       \int_{-\infty}^{\infty} f_X(t)dt
                       = \lim\limits_{x \to +\infty} F_X(x)
                       = 1
                   \end{displaymath}
             \end{itemize}
        \end{mdframed}
    \item {\Large Examples}
        \begin{mdframed}
            Utility problem. Let $X$ be the water
            demand and suppose the probability of an
            event is proportional to the area of the event.

            \begin{displaymath}
                F_X(x) = Pr(X \le x)
                \frac{(150 -1) \times (x - 4)}{(150 -1) \times (200-4)}
                = \frac{x-4}{196}
            \end{displaymath}

            The bottom is the area of the sample space,
            the top is the area that represents the
            interval we are interested in.
            
            The pdf of this function is
            \begin{displaymath}
                f_X(x) =  
                \begin{cases}
                    \frac{1}{196} & 4 \le x \le 200\\ 
                    0 & \textrm{otherwise}
                \end{cases}
            \end{displaymath}

            The pdf is constant, which is a distribution known
            as a uniform distribution. A uniform distribution
            is notated $X \sim Unif(a,b)$
        \end{mdframed}
        \begin{mdframed}
            Consider a random variable $X$ with pdf
             \begin{displaymath}
                f_X(x) = 
                \begin{cases}
                    cx & 0 < x < 4\\ 
                    0 & \textrm{otherwise}
                \end{cases}
                ,c > 0
            \end{displaymath}

            Determine the value of $c$. Calculate  $Pr(1 \le X \le 2)$.
            The integral of  $f_X(x)$ must equal 1.

             \begin{displaymath}
                \int_{-\infty}^{+\infty}f_X(t)dt =
                \int_{0}^{4}ct dt =
                \frac{ct^2}{2}\bigg\rvert_{0}^{4} =
                8c
            \end{displaymath}

            Therefore $c = \frac{1}{8}$
            
            \begin{displaymath}
                Pr(1 \le X \le 2) = \frac{3}{16}
            \end{displaymath}
        \end{mdframed}
    \item {\Large }
\end{description}


\end{document}
