\documentclass{report}
\usepackage[tmargin=2cm, rmargin=1in, lmargin=1in,margin=0.85in,bmargin=2cm,footskip=.2in]{geometry}
\usepackage{amsmath,amsfonts,amsthm,amssymb,mathtools}
\usepackage{enumitem}
\usepackage[]{mdframed}
\usepackage{tikz}
\renewcommand{\familydefault}{\sfdefault}

\title{\Huge{Math 181}\\Day 12 Notes}
\author{\huge{Elijah Hantman}}
\date{}

\begin{document}
\maketitle
\newpage

\begin{description}
    \item {\large Euclid and The Elements}    
        \begin{mdframed}
            \begin{itemize}
                \item Most printed book after the Bible.
                \item Large influence on many mathematicians
                \item Influence on US Presidents
                    \begin{mdframed}
                        President Garfield published a proof
                        of pythagorean theorem. He was a teacher
                        before and studied in depth.
                    \end{mdframed}
                    \begin{mdframed}
                        Abraham Lincoln also studied The Elements.
                        He largly self studied, but one of the
                        texts he studied was Euclid's elements.
                        
                        Story of how he went and memorized large
                        portions of Euclid's elements before studying
                        law. May be a tall tale but he definitely 
                        tried to train himself in mathematics.
                    \end{mdframed}
                \item Foundational plane geometry
                \item Foundational Number theory
            \end{itemize}
                
            Very famous yet we don't have much of the context
            and few have actually read the text.
        \end{mdframed}
    \item {\large What do we know of Euclid and The Elements?}
        \begin{mdframed}
            \begin{itemize}
                \item We know very little
                \item Most of our primary fragments come from
                    Egypt. 
                \item First fragment is dated to 100AD, we could
                    at most push back to 100BC. Full copies stop
                    at around 300AD.
            \end{itemize}

            What about Euclid himself?
            \begin{itemize}
                \item The text begins with defining
                    lines and points immediately.
                \item He continues defining terms and then gives
                    out his famous five postulates.
                    \begin{mdframed}
                        \begin{enumerate}
                            \item To draw a straight line from
                                any point to any point
                            \item To produce a finite straight line
                                continuously in a straight line
                            \item To describe a circle with any 
                                center and distance
                            \item All right angles are equal
                            \item If two lines are parallel
                                they do not meet.
                        \end{enumerate}
                    \end{mdframed}
                \item Some more axioms
                    \begin{enumerate}
                        \item Equality is transitive
                        \item Adding to both sides of
                            an equality maintains the equality
                        \item If you subtract from both sides
                            of an equality the equality is
                            maintained
                        \item If two lines are the same, they
                            have the same length
                        \item The whole is greater than the part.
                    \end{enumerate}
                \item He quickly starts proving various propositions
                    culminating in the Pythagorean theorem in the
                    first volume.
                \item It is not known why Euclid wrote his Elements.
                    It seems unlikely to be a textbook. It seems
                    likely that it was a compendium or collection
                    of what the Greeks knew of geometry.
                \item Euclid came before Archimedes, around
                    300 BC. He also came before Plato (247 BC),
                    after Alexander the Great, and Ptolomy.
                    \begin{mdframed}
                        One text refers to him as Euclid of
                        Alexandria. Which was an Egyptian port
                        city.
                    \end{mdframed}
            \end{itemize}
        \end{mdframed}
        \pagebreak
    \item {\large Familiar Mathematics}
        \begin{mdframed}
            The first 6 books of Euclid's Elements
            are about plane geometry.

            Books 7-9 are about Number Theory.
            \begin{mdframed}
                One result from here is an algorithm for
                finding the greatest common factor
                between two integers.
            \end{mdframed}

            Book 10 concerns irrational numbers. Culminates in
            a proof of the irrationality of $\sqrt{2}$

            Books 11-13 concerns 3D geometry and the last book
            is about the platonic solids. However the last
            book may be a later addition.
        \end{mdframed}
    \item {\large Euclid's Number Theory}
        \begin{mdframed}
            Euclid took a very geometric approach to all math.
            He used his postulates to build up number theory.

            He also liked to use four terms for his examples of
            sequences. It was a literary convention at the time.
        \end{mdframed}
        {\large Euclid's Proposition 35}
        \begin{mdframed}
            If we have a sequence of arbitrary length. 

            \begin{displaymath}
                a_1, a_2, a_3, a_4, .... a_{n+1}
            \end{displaymath}

            They are each in "continued proportion".
            This means they are a geometric series, or
            have a geometric relationship.

            \begin{displaymath}
                \frac{a_{i+1}}{a_i} = r
            \end{displaymath}

            Next Euclid describes subtracting the first term
            from the last and second term.

            \begin{displaymath}
                a_{n+1} - a_1
            \end{displaymath}
            \begin{displaymath}
                a_2 - a_1
            \end{displaymath}

            Then he says:

            \begin{displaymath}
                \frac{a_2 - a_1}{a_1} = 
                \frac{a_{n+1} - a_1}{a_1 + a_2 + ... a_n}
            \end{displaymath}

            Which more formally is:

            \begin{displaymath}
                \frac{a_2 - a_1}{a_1} = 
                \frac{a_{n+1} - a_1}{\sum_{i=1}^n a_i}
            \end{displaymath}
            
            We usually write geometric series as:
            \begin{displaymath}
                a_n = a_0 r^{n}
            \end{displaymath}

            This is equivalent to the previous definition.

            \begin{displaymath}
                y = \frac{r^{n+1}-1}{r-1}
            \end{displaymath}
            
            This is found by setting the series equal to $y$ then
            multiplying by  $r$ and taking the difference
            between  $y$ and  $y \cdot r$. 
        \end{mdframed}
        \pagebreak
        \begin{mdframed}
           \begin{enumerate}
               \item Is the formula correct?
                   \begin{gather}
                       \frac{a_2-a_1}{a_1} = \frac{a_{n+1}-a_1}{\sum^n a_i}\\ 
                       \frac{a_1 * r - a_1}{a_1} = \frac{a_1 r^{n} - a_1}{a_1\frac{r^n-1}{r-1}}\\
                       \frac{a_1 * r - a_1}{a_1} = \frac{(a_1r^{n} - a_1)(r-1)}{a_1(r^n-1)}\\
                       r-1 = \frac{a_1(r^{n} - 1)(r-1)}{a_1(r^n-1)}\\
                       r-1 = r-1
                   \end{gather}
                   This shows the formulas are the same, and it is correct.
                   Both express the relationship between the ratio between
                   two terms, and the sum of the sequence.

               \item How does it compare to other formulas
                   for geometric series?
               \item How to make sense of the proof?
           \end{enumerate}

           Euclid is not thinking like Nichomachus. Nichomachus
           thought geometrically but with a leaning towards
           abacus and calculative methods.
        \end{mdframed}
\end{description}



\end{document}
