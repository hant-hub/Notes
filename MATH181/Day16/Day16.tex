\documentclass{report}
\usepackage[tmargin=2cm, rmargin=1in, lmargin=1in,margin=0.85in,bmargin=2cm,footskip=.2in]{geometry}
\usepackage{amsmath,amsfonts,amsthm,amssymb,mathtools}
\usepackage{enumitem}
\usepackage[]{mdframed}
\usepackage{tikz}
\renewcommand{\familydefault}{\sfdefault}

\title{\Huge{Math 181}\\Day 16 Notes}
\author{\huge{Elijah Hantman}}
\date{}

\begin{document}
\maketitle
\newpage

\begin{description}
    \item What are some large numbers? 
        \begin{enumerate}
            \item The grains of sand needed to fill up
                the universe.
            \item Number of electrons/atoms in the Earth
            \item The number of atoms in a standard
                quantity of matter. $6.22\times 10^{22}$
            \item Tree(3) where the Tree function is
                the maximum number of Trees you can create
                with a specific coloring.
            \item The Monster Group.
            \item Super permutations
            \item Graham's Number
        \end{enumerate}
    \item {\large Archimedes on Grains of Sand}
    \item Split into sections
        \begin{enumerate}
            \item Cover Letter And Introduction to the Problem
            \item Geometry for Estimating the Size of the Universe
            \item Numerical Estimates
            \item Describes System of Numerals From Earlier
                Work
        \end{enumerate}
    \item {\large Numerical Estimates}
        \begin{mdframed}
            Archimedes takes the Universe to be a sphere centered
            on the Earth, which has an outer edge passing through
            the center of the Sun.

            Archimedes uses four numerical estimates which were
            famous enough at the time that the King would be
            somewhat familiar with them.

            \begin{enumerate}
                \item The perimeter of the Earch is about
                    3,000,000 Stadia.
                    \begin{mdframed}
                        1 Stadia $\approx$ 150 meters
                    \end{mdframed}
                \item The Diameter of the Earth is greater
                    than that of the moon, and the Sun is greater
                    than the Earth
                \item The diameter of the sun is around
                    30 times the diameter of the Moon.
                \item The diameter of the Sun is greater than the
                    side of a regular 1000-gon incribed in a
                    great circle inscribed on the Universe sphere.
            \end{enumerate}

            What do we need?
            \begin{enumerate}
                \item We need the distance from the Earth
                    to the Sun, to calculate the volume of the
                    Universe.
            \end{enumerate}

            Next time the calculation, today the answer is
            1000 times the diameter of the Earth. Very
            convenient so that was nice for Archimedes.

            Archimedes refrains from exceeding 1000 so that
            the King can continue to follow along with Standard
            Greek Numerals.
        \end{mdframed}
        \begin{mdframed}
            The next estimate is the size of a grain of sand.
            Archimedes takes the approach of assuming the number
            of grains of sand in a sphere with a diameter of
            the $1/40$th the smallest unit the Greeks had, contained 
            1000 grains of sand.

            A fingerbreadth(daktylos) is around 19.3mm.

            Using the formula for the volume of a sphere,
            we find 1000 grains of sand per $\frac{4}{3\cdot 80^3}$ cubic
            Daktylos.

            We then can use the size of a grain of sand to convert the volume
            of the Universe to sand.
        \end{mdframed}
\end{description}


\end{document}
