\documentclass{report}
\usepackage[tmargin=2cm, rmargin=1in, lmargin=1in,margin=0.85in,bmargin=2cm,footskip=.2in]{geometry}
\usepackage{amsmath,amsfonts,amsthm,amssymb,mathtools}
\usepackage{enumitem}
\usepackage[]{mdframed}
\usepackage{tikz}
\renewcommand{\familydefault}{\sfdefault}

\title{\Huge{Math 181}\\Day 21 Notes}
\author{\huge{Elijah Hantman}}
\date{}

\begin{document}
\maketitle
\newpage

\begin{description}
    \item Review 
        \begin{mdframed}
            Table of whole number pythagorean triples.
            The largest has 5 digits which requires
            insight into how to find Pythagorean triples.
        \end{mdframed}
        \begin{mdframed}
            Changed how we view the history of Mathematics.
            We can't know anything more about this tablet for sure,
            however we can make guesses about it.
        \end{mdframed}
    \item Theories
        \begin{mdframed}
            \begin{enumerate}
                \item It is a trig table
                    for $tan^2(\theta)$ or  $\frac{1}{cos^2(\theta)}$ 
                    depending on how you intepret the first column
                \item Negebaur\\
                    Generate as follows:
                    \begin{mdframed}
                        if p and q take on whole number
                        values subject to conditions
                        \begin{enumerate}
                            \item $p > q > 0$
                            \item p and q have no non-trivial common
                                divisor.
                            \item p and q are not both odd.
                        \end{enumerate}
                        then:
                        \begin{gather}
                            x = p^2 - q^2 = s\\
                            y = 2pq = D\\
                            z = p^2 + q^2 = l
                        \end{gather}
                        
                        This will produce all Reduced Pythagorean triples
                        exactly once.
                    \end{mdframed}
                \item The table is a table of reciprocal pairs
                    $x,y$ with  $x*y = 1*60^n$ for some $n$.

                    The $x,y$ would have been listed in the
                    missing part of the table.
                    \begin{mdframed}
                        \begin{gather}
                            s' = \frac{s}{l} = \frac{x-y}{2}\\ 
                            l' = \frac{l}{l} = 1\\
                            d' = \frac{d}{l} = \frac{x+y}{2}
                        \end{gather}
                        Alongside some scaling so that
                        $s$ and $d$ are coprime.
                    \end{mdframed}
            \end{enumerate}
        \end{mdframed}
    \item Evidence
        \begin{mdframed}
            All the theories are consistent with the numbers
            present in the table and are mathematically sound.
        \end{mdframed}
        \begin{mdframed}
            \begin{itemize}
                \item The first interpretation seems natural
                    however when inspecting the angle measurements
                    it seems strange to pick the range

                    It also seems strange since we have no other
                    records of Babylonians using trigonometry.
                \item The p and q values seem to be random

                    This is strange since Babylonians don't
                    really need to calculate these values,
                    and we don't have records of them
                    doing abstract values.
                \item The third theory seems good.

                    This seems evidenced, since Babylonians
                    use $x$ and  $\frac{1}{x}$ in math problems
                    often.
            \end{itemize}
        \end{mdframed}
\end{description}


\end{document}
