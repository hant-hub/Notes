\documentclass{report}
\usepackage[tmargin=2cm, rmargin=1in, lmargin=1in,margin=0.85in,bmargin=2cm,footskip=.2in]{geometry}
\usepackage{amsmath,amsfonts,amsthm,amssymb,mathtools}
\usepackage{enumitem}
\usepackage[]{mdframed}
\usepackage{tikz}
\renewcommand{\familydefault}{\sfdefault}

\title{\Huge{Math 181}\\Day 18 Notes}
\author{\huge{Elijah Hantman}}
\date{}

\begin{document}
\maketitle
\newpage

\begin{description}
    \item {\large Babylonian Mathematics} 
    \item Notation
        \begin{mdframed}
           \begin{itemize}
               \item Babylon appears often in the Old Testement
                   as an origin of the Jewish people.
               \item Appeared sometimes in Greek literature,
                   which discussed Hanging Gardens of Babylon
                   as a wonder of the world.
               \item Modern Archeology has uncovered mathematical
                   theory and ancient artifacts. In some ways
                   it surpassed Greek Mathematics.
                   \begin{mdframed}
                       History of Mathematics was revolutionized
                       by Nordebour?? who uncovered much
                       of the features and known discoveries of
                       Babylonian mathematics.
                   \end{mdframed}
           \end{itemize} 
        \end{mdframed}
    \item What is the Cuniform Numeral System?
        \begin{mdframed}
            \begin{itemize}
                \item Positional Numerals (not official just for this class)
                    \begin{itemize}
                        \item Two Basic symbols
                        \item Vertical Triangle and line indicates 1
                        \item Horizontal Triangle and line indicates 10
                            \begin{mdframed}
                                These symbols were formed by pressing
                                a reek stylus into wet clay.

                                Once the clay dried it could be very well preserved.
                            \end{mdframed}
                        \item The symbols were used to build the
                            numbers 1 to 60 by grouping symbols.
                        \item There are some conventions for
                            grouping symbols, generally you do rows
                            of three symbols with leftovers centered.
                        \item for the Tens, it has a similar structure,
                            except you do diagonals which follow the slope
                            of the triangle when you need to add
                            another row.
                        \item All Other numbers between 1 and 59
                            are written (10s) then (1s).
                            ie: (10-50) (1-9)
                            \begin{mdframed}
                                The numbers 1-59 are known as
                                number signs, or graphemes in linguistics.

                                Beyond 59 numerical phrases are used
                                which combine to form a larger number.
                            \end{mdframed}
                        \item Each digit in a large number are written
                            left to right, largest digit to smallest.
                            Each digit represents 60 times the value
                            of the next.
                            ie:
                            62 = 1, 2
                        \item A late addition added a blank space symbol,
                            but in general you figure out ambiguous values
                            from context.
                        \item Same algorithm as always for calculating the
                            digits. Divide by 60, fractional is the ones
                            place, repeat for all digits. Multiply all
                            fractions by 60, and then you have your digits.
                    \end{itemize}
            \end{itemize}
        \end{mdframed}
    \item {\large History}
        \begin{mdframed}
            \begin{itemize}
                \item Babylonian writing Appears around 
                    3000 BC
                \item Babylonian positional numerals
                    appear around 2000BC
                \item Babylonian scribes had multiple different
                    systems but the positional system was
                    the latest system they used.
                \item Most of the tablets with positional
                    numerals were dated to between 2000 and
                    1600 BC. From 330 BC to 1 AD it was mostly
                    used in math and Astronomy rather than
                    everyday use.
                \item Very interesting properties such as arbitrarily
                    large numbers, but it didn't survive. Greek
                    astronomical works used some Babylonian numerals
                    to limited degrees.
            \end{itemize}
        \end{mdframed}
\end{description}


\end{document}
