\documentclass{report}
\usepackage[tmargin=2cm, rmargin=1in, lmargin=1in,margin=0.85in,bmargin=2cm,footskip=.2in]{geometry}
\usepackage{amsmath,amsfonts,amsthm,amssymb,mathtools}
\usepackage{enumitem}
\usepackage[]{mdframed}
\usepackage{tikz}
\renewcommand{\familydefault}{\sfdefault}

\title{\Huge{Math 181}\\Day 15 Notes}
\author{\huge{Elijah Hantman}}
\date{}

\begin{document}
\maketitle
\newpage

\begin{description}
    \item {\large Archimedes}\\ 
        Archimedes The Sand Reckoner
        \begin{mdframed}
            Unclear whether Archimedes would have seen
            this work as anything important.
        \end{mdframed}
    \item {\large What do we know about Archimedes?}
        \begin{mdframed}
            \begin{itemize}
                \item Lived in Syracuse (Greek Island)
                \item His father was known to be an astronomer
                    \begin{mdframed}
                        This is very thin evidence wise.
                        We don't have any census or names
                        for this time.
                    \end{mdframed}
                \item Contemporaneous with a Roman invasion
                    of Syracuse
                    \begin{mdframed}
                        The dramatic tale is that Archimedes
                        came up with many inventions to protect
                        the city. And eventually the city fell
                        and Archimedes was killed by a Roman
                        soldier.

                        Highly unlikely to be anything close to
                        accurate. The invasion was real and he
                        likely was involved, and probably died
                        around the same time period. It seems
                        very unlikely that he had many of the
                        inventions he was credited with.
                    \end{mdframed}
                \item Mathematical decoration on his grave
                \item Archimedian property
                    \begin{mdframed}
                        Can always find a rational number between
                        any real number and zero.
                        \begin{displaymath}
                            \forall r \in \mathbb{R}
                        \end{displaymath}
                        \begin{displaymath}
                            \exists q \in \mathbb{Q}
                        \end{displaymath}
                        \begin{displaymath}
                            0 < q < r
                        \end{displaymath}
                    \end{mdframed}
                \item Integral Calculus?
                    \begin{mdframed}
                        He discovered formulas for the volume
                        of a sphere and circle using similar
                        reasoning to integral calculus.
                    \end{mdframed}
                \item Approximation of $\pi$
                     \begin{mdframed}
                        He comes up with a range which correctly
                        bounds $\pi$ to three digits.
                    \end{mdframed}
                \item Bouyancy
                    \begin{mdframed}
                        He likely did not invent the concept
                        of Bouyancy but is often attributed
                        with a story where he invents the
                        fact that objects displace water
                        equal to their volume when submerged.

                        This story also coins the term Eureka
                        as the exclaimation Archimedes shouted
                        when he came to a realization.
                    \end{mdframed}
                \item Unpopular until after his Death?
                    \begin{mdframed}
                        It is claimed his work was
                        so ahead of its time that he
                        wasn't given proper praise.

                        It is also reasonable to think that
                        the number of people who could read
                        and understand Archimedes was small.

                        Certainly by the first sources we have
                        he was lauded as the pinnacle of
                        mathematics.
                    \end{mdframed}
            \end{itemize}
        \end{mdframed}
        \pagebreak
    \item The Sand Reckoner
        \begin{mdframed}
            \begin{itemize}
                \item Universe means the sphere centered on the
                    Earth and extending to the sun.
                \item Letter to King Gelon. Multiple times
                    translated.
                    \begin{mdframed}
                        King who ruled part of Sicily. We
                        have good historical records of this
                        person. Not as celebrated but does
                        have good evidence and dates.
                    \end{mdframed}
                    \begin{mdframed}
                        Relationship is maybe a little
                        condescending. Slightly informal,
                        may be a translation issue.
                        Also it is notably theatrical, literary
                        would be the preferred term of
                        Classicists. Assumes the King is knowledgable
                        about the current mathematics and the current
                        working mathematicians/astronomers.
                    \end{mdframed}
                    \begin{mdframed}
                        We should ask who Archimedes grew up around.
                        Is he just comfortable with Kings and nobility?
                        Or is he a common person who is showing his
                        background.

                        He is framing things as if he is the best
                        and most knowledgable. The sentence about 
                        Zeuxippus could be translated as he worked
                        with Zeuxippus or as if he was demonstrating
                        he was better than Zeuxippus.
                    \end{mdframed}
                \item He is stating his goal to calculate
                    the amount of sand which could fit
                    in the universe, and that there are
                    numbers significantly larger than that.
                \item He mentions that Aristarchus of Samos
                    proposed a theory of Heliocentrism
                    in which the Earth orbits the sun and
                    the universe is many times larger than
                    what was previously thought.
                    \begin{mdframed}
                        Some outside knowledge is that
                        Aristarchus had a good method to
                        calculate the size of the Sun. He
                        was off by several orders of magnitude
                        due to poor measurment tools, but was
                        able to realize that the sun is several
                        dozen times larger than the Earth. And
                        therefore he proposed a Heliocentric
                        model.
                    \end{mdframed}
            \end{itemize} 
        \end{mdframed}
    \item How many Grains of Sand can fit in the Universe?
        \begin{mdframed}
            We need at least a few assumptions.
            \begin{enumerate}
                \item Size of a grain of sand
                \item Size of the Universe
            \end{enumerate}

            Archimedes basically makes up the size of a grain
            of sand by taking a small unit of volume.
            To calculate the size of the Universe he is going
            to use astronomical data and geometry to estimate
            the size of the universe.

            \vspace{10pt}

            The second problem he is tackling is:
            \begin{mdframed}
                How to write down the answer?
            \end{mdframed}

            Usual greek numerals only reach 999.
        \end{mdframed}
    \item Why is Archimedes Answering These Questions?\\
        What genre of text is this?
        \begin{mdframed}
            Not his usual fare. Archimedes was known for one
            upping mathematicians in well known difficult problems.
            These questions are not particularly famous.

            \begin{mdframed}
                Before the internet most math was in the form
                of textbooks and academic papers. Textbooks
                were for learning and giving background.

                Euclid's Elements seem to fall into either
                textbook writing or Encyclopedia.

                Archimedes letters were research letters which
                assumed large amounts of background.
            \end{mdframed}

            \begin{itemize}
                \item Seems like Archimedes is pushing back against
                    incorrect views.
                \item Seems like he is playing with the
                    math, for himself?
                \item Interested in extreme sizes.
                    \begin{mdframed}
                        Calculus things, infintesimals, vs
                        the universe.
                    \end{mdframed}
                    \begin{mdframed}
                        He mentions he gave more detail about
                        his number system in a letter we do
                        not have.
                    \end{mdframed}
                \item Math communication?
                    \begin{mdframed}
                        Philosophical and practical reasons
                        to see the usefulness of exceedingly
                        large numbers.

                        Mathematics for broader audiences
                        in more literary terms.

                        The goal is to express how cool and awesome
                        mathematicians are to think about these things.
                        It also is a crossover between philosophy
                        and mathematics via formal logic.
                    \end{mdframed}
            \end{itemize}
        \end{mdframed}
\end{description}


\end{document}
