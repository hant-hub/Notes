\documentclass{report}
\usepackage[tmargin=2cm, rmargin=1in, lmargin=1in,margin=0.85in,bmargin=2cm,footskip=.2in]{geometry}
\usepackage{amsmath,amsfonts,amsthm,amssymb,mathtools}
\usepackage{enumitem}
\usepackage[]{mdframed}
\usepackage{tikz}
\renewcommand{\familydefault}{\sfdefault}

\title{\Huge{Math 181}\\Day 8 Notes}
\author{\huge{Elijah Hantman}}
\date{}

\begin{document}
\maketitle
\newpage

\begin{description}
    \item {\large How did the ancient Greeks add,
        subtract, etc.?}
        \begin{mdframed}
           The obvious answer is the same way we
           do today. Using a table and algorithms.
           Using pencil and paper, using calculators.

           These are obviously incorrect answers.
        \end{mdframed}
        \begin{mdframed}
            For small computations fingers and mental math
            work fine. However larger computations still
            require a different approach.
        \end{mdframed}
        \begin{mdframed}
           Larger numbers used an abacus. Modern abacus worked
           via moving beads and using them as a counter. Greeks
           used a different kind of Abacus.

           \vspace{10pt}

           We knew they had sophisticated abilities to do
           mathematics, they had calculated many digits of
           $\pi$ and various astronomical constants.

           We don't have any explicit sources for how an
           ancient Greek abacus worked, only passing references
           in other texts. One historian in the Roman court
           mentions in passing that "for those men in the king's
           court are exactly like counters on a reckoning 
           board...".

           There was also a reference in Aristophenes' play,
           The Wasp, which mentions a reckoning board and
           pebbles.

           \vspace{10pt}

            The Darius Vase has a depiction of what is
            believed to be a man doing a calculation using
            a reckoning board and recording the results
            in his other hand. Believed to be a tax collector
            with a wax tablet in his left hand as a semi-permanent
            way of writing. He has a table which may have pebbles
            for computation, or coins the second man is puting on
            the table.

            A zoomed in view shows various symbols on the table.
            It could be coins or it could be some calculation. If
            it was a calculation it looks like there might be
            various categories that you add pebbles to? This would
            be around the time the system covered in class was
            popularized. The first few numerals are believed to
            be Attic Numerals which have the structure.

            \begin{itemize}
                \item $10,000$
                \item $1,000$ 
                \item $100$
                \item $10$
                \item ?
                \item Obol
                \item $\frac{1}{4}$ Obol
                \item $\frac{1}{8}$ Obol
            \end{itemize}

            The last three aren't numerals, but are symbols for
            currency. The unknown one could either be 5 or
            a Drachma, its ambiguous.
       \end{mdframed}
        \begin{mdframed}
            Out best guess was that they had a marker which
            kept track of the count, moving from one side to the
            other increases the value. This is just a guess
            from literary reference.

            One issue too is that the table appears to be wood,
            so the reckoning boards likely didn't 
            survive to be found via archeology.
        \end{mdframed}
        \begin{mdframed}
            Some artifacts seem like they could be reckoning
            boards.
        \end{mdframed}
        \begin{mdframed}
            We have a number of what are believed to be Marble
            reckoning boards. However they are mostly rubble
            which was reused as part of a building.
        \end{mdframed}
        \begin{mdframed}
            The Salamis Tablet, is the best artifact we have.
            Its is still woefully incomplete. We don't know how
            historically it was used, we can only speculate. It
            could have just been a game board rather than a
            reckoning board, it is just too incomplete and
            isolated. It is fairly large, and has three
            sets of numerals which is strange if it is
            a reckoning board.
        \end{mdframed}
        \begin{mdframed}
            Several theories of where Greek mathematics came
            from. Older theories attributed it to Pythagorus
            and Egypt, new theories argue that it arose
            from everyday mathematical work like surveying
            and accounting. One piece of evidence was Nichomachus
            and his work which echos work on a Reckoning board.
        \end{mdframed}
\end{description}


\end{document}
