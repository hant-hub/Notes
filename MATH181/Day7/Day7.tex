\documentclass{report}
\usepackage[tmargin=2cm, rmargin=1in, lmargin=1in,margin=0.85in,bmargin=2cm,footskip=.2in]{geometry}
\usepackage{amsmath,amsfonts,amsthm,amssymb,mathtools}
\usepackage{enumitem}
\usepackage[]{mdframed}
\usepackage{tikz}
\renewcommand{\familydefault}{\sfdefault}

\title{\Huge{Math 181}\\Day 7 Notes}
\author{\huge{Elijah Hantman}}
\date{}

\begin{document}
\maketitle
\newpage

\begin{description}
    \item {\large Mediterranian Mathematics} 
        \begin{mdframed}
            \begin{itemize}
                \item Greek math is undoubtably important
                    even if it is not the only
                    place in the ancient world which made
                    significant contributions.
                \item Greece was inhabited for thousands of
                    years.
                \item Ancient Greece included much of the
                    coast of the mediterannian, including parts
                    of Asia Minor, Africa, Spain, Italy, etc.
                \item Ancient Greece was defined not by government
                    but by culture, langauge, religion, and dress.
                \item Ancient Greece was made up of hundreds of small
                    city states which were governed independently.
                    Some example city states are:
                    \begin{itemize}
                        \item Athens
                        \item Sparta
                        \item Corinth
                        \item etc.
                    \end{itemize}
            \end{itemize}
        \end{mdframed}
    \item {\large Greek Numerals}
        \begin{mdframed}
            How do they work? 
            \begin{itemize}
                \item Letters have values
                    \begin{itemize}
                        \item $\alpha = 1$
                        \item $\beta = 2$
                        \item $\gamma = 3$
                        \item $\delta = 4$
                        \item $\epsilon = 5$
                        \item $\digamma = 6$
                        \item $\zeta = 7$
                        \item $\eta = 8$
                        \item $\theta = 9$
                    \end{itemize}
                    Tens:
                    \begin{itemize}
                        \item $\iota = 10$
                        \item $\kappa = 20$
                        \item $\lambda = 30$
                        \item $\mu = 40$
                        \item $\nu = 50$
                        \item $\xi = 60$
                        \item $O = 70$
                        \item $\pi = 80$
                        \item qoppa$ = 90$
                    \end{itemize}
                    Hundreds:
                    \begin{itemize}
                        \item $\rho = 100$ 
                        \item $\sigma = 200$
                        \item $\tau = 300$
                        \item $\upsilon = 400$
                        \item $\phi = 500$
                        \item $\chi = 600$
                        \item $\psi = 700$
                        \item $\omega = 800$
                        \item sampi$ = 900$
                    \end{itemize}
            \end{itemize}
            They borrow three characters from the Venetician
            alphabet to fill out their values.
            \begin{itemize}
                \item First you pick one value from each
                    group, ones, tens, hundreds.
                \item Then you order them from largest to
                    smaller with the exception of numbers
                    in the teens.
                \item For example
                    \begin{itemize}
                        \item $\rho \mu \alpha = 100 + 40 + 1 = 141$
                        \item $\beta \iota = 2 + 10 = 12$
                    \end{itemize}
            \end{itemize}

            Structural Features
            \begin{itemize}
                \item Similar to our modern system but different
                    than the Roman system.
                \item Known as a cipher system, which means
                    there is one character for each power
                    of ten.
                \item Largest number is $999$.
                \item System is additive rather than
                    positional.
                    \begin{mdframed}
                        Western/Arabic/Hindu numeral system is positional which
                        means we share symbols between values.
                    \end{mdframed}
            \end{itemize}

            Some extensions were added over time to allow
            expression of numbers beyond 999.
            \begin{enumerate}
                \item For $\alpha$ through $\theta$ 
                    you can add a hasta mark ,$\alpha$ to
                    multiply the value by a thousand.
                    This allows for a maximum of 
                    $9999$
                \item A different system to extend the numbers
                    even further was used. $M$ was equal to
                    ten thousand "myriad"
                    \begin{mdframed}
                        In classical literature, myriad
                        means both, 10,000 and also a
                        large number.
                    \end{mdframed}
                \item $M^x$ is equal to 10,000 times a number
                    from the second or third column.
                \item This gives all values up to
                   1 million. 
            \end{enumerate}

            Over a thousand years the means of writing
            each character changed drastically.

            To distinguish numbers they used a combination
            of context, additional symbols, like dots,
            leaves, squiggles, lines, etc.

            They also used different symbols for the start
            and stop of a line.

            The modern convention is an overline to indicate
            a number.

            Too much has been lost to know for sure how these
            numerals came about and why the system which won
            did. We do know they were in use around 500 BC
            but other than that little is known.
        \end{mdframed}
        \begin{mdframed}
            Greeks used numbers in a similar way to today
            where they can serve as ordinals or cardinals,
            order and amount.
        \end{mdframed}
        \begin{mdframed}
            Most written text we have is not literary,
            but stone inscriptions in churches, and buildings.
            Many mathematical or astronomical texts are copies
            of copies and we don't have the original text.

            During the reign of Alexander the Great many Greek
            texts made their way to Egypt which had a good
            climate for preserving texts.
        \end{mdframed}
        \begin{mdframed}
            Where did they come from?
            \begin{mdframed}
                Old theory was that it was based on the
                systems of venetcian peoples and
                Ancient Jews.
            \end{mdframed}
            This theory became outdated since both Jews and
            Venetican peoples did not use their alphabets in
            a similar way until after the Greeks.
            \\
            Alternatives
            \begin{itemize}
                \item People in Turkey invented the system and
                    it spread out
                \item Could be borrowed from Egyptian system
                    \begin{mdframed}
                        While using completely different
                        symbols, the later Egyptian
                        system was structurally very
                        similar to the Greek system.
                    \end{mdframed}
            \end{itemize}
        \end{mdframed}
\end{description}


\end{document}
