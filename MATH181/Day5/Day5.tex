\documentclass{report}
\usepackage[tmargin=2cm, rmargin=1in, lmargin=1in,margin=0.85in,bmargin=2cm,footskip=.2in]{geometry}
\usepackage{amsmath,amsfonts,amsthm,amssymb,mathtools}
\usepackage{enumitem}
\usepackage[]{mdframed}
\usepackage{tikz}
\renewcommand{\familydefault}{\sfdefault}

\title{\Huge{Math 181}}
\author{\huge{Elijah Hantman}}
\date{}

\begin{document}
\maketitle
\newpage

\begin{description}
    \item Historical Biography
        \begin{itemize}
            \item Famous: Leonard Euler
            \item Not-Famous: E. J. Edmunds
        \end{itemize}
        \begin{itemize}
            \item 
                \begin{mdframed}
                    Life facts are important, and the source of those
                    facts are important.
                \end{mdframed}
            \item 
                \begin{mdframed}
                    How facts are used is important, and why
                    facts are preserved.
                \end{mdframed}
        \end{itemize}
    \item {\large What did Euler Do?}
        \begin{itemize}
            \item Found/Discovered/Invented the constant $e$ 
            \item Invented sigma summation, function notation,
                radicals, $\pi$, etc.
            \item Euler's Formula
                \begin{displaymath}
                    e^{i\phi} = cos(\phi)+isin(\phi)
                \end{displaymath}
            \item Euler's Characteristic
            \item Other Euler's Constant
                \begin{displaymath}
                    \gamma
                \end{displaymath}
        \end{itemize}
    \item {\large Who was Euler?}
        \begin{mdframed}
            \begin{itemize}
                \item Born in Basel, Switzerland 1707
                \item Buried in Saint Petersburg, and lived
                    much of his life in Berlin.
                \item 1720 began studies at University of
                    Basel. (13 years old)
                \item 1726 wrote a dissertation
                    \begin{mdframed}
                        Wrote original research on the
                        propagation of sound. He was 19
                        years old.
                    \end{mdframed}
                \item 1727 Moves to St. Petersburg
                    \begin{mdframed}
                        He was wanted by Peter the Great,
                        and generally found money and work as
                        a scholar.
                    \end{mdframed}
                    \begin{mdframed}
                        Euler does a ton of work and his work
                        is still being published today. Currently
                        over 80 large volumes have been released.
                    \end{mdframed}
                \item 1738 Eye Problems
                    \begin{mdframed}
                        In St. Petersburg Euler lost vision
                        in one of this eyes and so most images
                        of him are from only one side.
                    \end{mdframed}
                \item 1741 Leaves Russia for Berlin,
                    Frederick the Great hires him at Berlin
                    Academy.
                    \begin{mdframed}
                        Very tumuoltuous times, purges and
                        political issues.
                    \end{mdframed}
                \item 1766 Returns to St. Petersburg
                \item 1783 Dies of Natural Causes
                    \begin{mdframed}
                        Final years can't read or see anymore.
                        Has to hire an assistant to read and
                        write for him and still manages to
                        produce many mathematical works.
                    \end{mdframed}
            \end{itemize}
            \pagebreak
            Other Works by Euler
            \begin{itemize}
                \item Basel Problem
                    \begin{displaymath}
                        \sum_{n=1}^{\infty} \frac{1}{n^2}
                    \end{displaymath}
                    Turns out:
                    \begin{displaymath}
                        \sum_{n=1}^{\infty} \frac{1}{n^2}
                        = \frac{\pi^2}{6}
                    \end{displaymath}
                    Also Proved
                    \begin{displaymath}
                        \int_0^1 \frac{sin(ln(x))}{ln(x)}
                        = \frac{\pi}{4}
                    \end{displaymath}
                    Using:
                    \begin{displaymath}
                        \frac{sin(ln(x))}{ln(x)}
                        = \frac{ln(x)-\frac{ln(x)^3}{3!} +
                        \frac{ln(x)^4}{4!}...}{ln(x)}
                    \end{displaymath}
                    
                \item Proved Fermat number is not prime. 
                    \begin{displaymath}
                        2^{(2^5)}+1
                    \end{displaymath}
                    Which has 10 digits.
                    
                \item K\"onigsburg Bridge Problem
                    \begin{mdframed}
                        Seven bridges in a city, can you
                        find a path which watches each bridge
                        only once. Turns out to be impossible
                        and Euler came up with a good arguement.
                    \end{mdframed}
            \end{itemize}
            {\large What did other people think of Euler}
            \begin{itemize}
                \item Voltaire thought Euler to be simple and
                    devoutly religious man. Boring outside
                    of Math.
            \end{itemize}
        \end{mdframed}
    \item E. J. Edmunds
        \begin{mdframed}
            \begin{itemize}
                \item Edmunds was a Free Person of Color
                    in New Orleans
                \item Born Before the Civil War
                \item Unlike Euler who was so well known it
                    makes it hard to figure out why we know what
                    we know, Edmunds was recently rediscovered.
                    \begin{mdframed}
                        Recently rediscovered by Sian Zelbo in
                        Colombia University Dissertation.
                    \end{mdframed}
                \item Born 1851 in New Orleans
                \item Free Person of Color\\
                    Father is a clerk/salesman in dry goods
                    company.
                    \begin{mdframed}
                        Family does well financially, and owns
                        a home in Trem\'e. Father frequently
                        travels to Europe for work. Likely
                        spoke French.
                    \end{mdframed}
                \item 1861 Civil War Breaks out (10 years old)
                \item 1862 Occupied by Union Soldiers
                    \begin{mdframed}
                        Both bad and good for Edmunds. Issues
                        with occupation, but well positioned
                        as a non-slaveholding wealthy family.
                    \end{mdframed}
                \item 1870 Attended Fillmore School ("Whites only
                    school")
                    \begin{mdframed}
                        He is fairly light skinned for a person
                        of color, but it is unclear whether
                        that affected his academic career.
                    \end{mdframed}
                \item 1871 Goes to Paris and takes the exams
                    to be admitted for the E\'cole Polytechnique.
                    \begin{mdframed}
                        E\'cole started out as an elite military
                        school, with major math and science
                        departments. Becomes a leading school
                        and is a gateway to the top of French
                        analysis.

                        \vspace{10pt}

                        Other mathematicians from E\'cole
                        \begin{itemize}
                            \item Cauchy
                            \item Poincaire
                            \item Poisson
                            \item etc.
                        \end{itemize}

                        \vspace{10pt}

                        Edmunds is accepted with a 133/144,
                        and is the bottom of his class.
                    \end{mdframed}
                \item 1873 Passes exams, Set to do Artillery
                    Training
                \item 1875 Decides to leave school and moves
                    back to New Orleans
                    \begin{mdframed}
                        1874 is undocumented, and he likely
                        works at a University but it isn't
                        known.
                    \end{mdframed}
                \item Publishes more than 200 short publications
                    in Mathematics.
                \item To be Continued XD
            \end{itemize}
        \end{mdframed}
\end{description}


\end{document}
