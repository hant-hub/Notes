\documentclass{report}
\usepackage[tmargin=2cm, rmargin=1in, lmargin=1in,margin=0.85in,bmargin=2cm,footskip=.2in]{geometry}
\usepackage{amsmath,amsfonts,amsthm,amssymb,mathtools}
\usepackage{enumitem}
\usepackage[]{mdframed}
\usepackage{tikz}
\renewcommand{\familydefault}{\sfdefault}

\title{\Huge{Math 181}\\New Light on Plimpton}
\author{\huge{Elijah Hantman}}
\date{}

\begin{document}
\maketitle
\newpage

\begin{itemize}
    \item Direct response to Sherlock Holmes in Babylon
        \begin{mdframed}
            I kinda disagree with her here.
            She seems to be arguing that
            the Sherlock article was about using
            mathematical wits to deduce answers to
            historical mysteries. I read it more
            about how math and history together
            can shed a lot of light and bring up many
            interesting questions but ultimately
            mathematics cannot solve history by itself.

            I kinda see it as in the same category as the
            write up of Eratosthenes' calculation of the
            size of the Earth, its mostly mathematics
            but it isn't discounting the history, merely
            focusing on a different aspect of the
            investigation.
        \end{mdframed}
    \item Babylonian ideas of geometry were different. Modern
        understanding of an idealized triangle is equilateral,
        while babylonians were more shaped like the marks
        they made in writing.
    \item We must first and foremost be historians, examining
        historical and archeological context before we
        dive into mathematics.
    \item Three major interpretations
        \begin{enumerate}
            \item Some kind of Trigonometric table

                Columns 2 and 3 are the short sides
                and diagonals of right triangles,
                and the first column is a square or
                reciprocal square of a trigonometric
                function.
            \item Neugebauer and Aaboe argued that the table
                was generated using $x = p^2 - q^2$ 
                alonside some specific restrictions.
            \item Bruins interpreted the tablet
                as deriving from reciprocal
                pairs.
        \end{enumerate}
    \item Claim: We cannot choose one of these
        theories with mathematics alone, all three are
        equally explanatory of the mathematics.
    \item Evidence: We know ancient Mesopotamia commonly dealt
        with pythagorean triples outside the context of
        trigonometric tables.
    \item Evidence: Other tables use a type of shorthand
        which is unique to the historical context of 
        Larsa.
    \item Claim: Neugebauer is likely incorrect since under
        his theory the table would be formatted counter to
        the styles and conventions of all other tablets uncovered.
\end{itemize}
{\large Trigonometry}
\begin{itemize}
    \item The linguistics and conception of what various geometric
        primitives are shapes how Mesopotamians went about
        calculating and working with these quantities.
    \item Claim: The table could not be about trigonometry
    \item Evidence: Circles were not thought about in
        terms of rotating radii, or angles, so the concept
        of a table of angles and various circle properties
        doesn't make historical sense.
    \item Note:
        \begin{mdframed}
            Mesopotamians knew you could generate a circle via
            rotating radius, but they just didn't conceptualize
            circles as fundamentally defined by their
            radius, but rather their area and Circumference.

            They did have concepts of angles, but they
            were used in different contexts which
            would make a trigonometric table anachronistic.
        \end{mdframed}
\end{itemize}
{\large Reciprocal Pairs}
\begin{itemize}
    \item Reciprocal pairs do have good evidence of being
        important to Ancient Mathematics.
    \item Fits the tabular expectations of decreasing order
        and of being calculable using known methods
        the Ancient Mesopotamians had access to.
    \item Doesn't explain on its own what the missing
        column might be.
    \item To answer we need to actually translate and
        decode the headings at the top of the document.
    \item We should be adding ones to each entry
    \item So this is the most likely
\end{itemize}

{\large Author?}
\begin{itemize}
    \item Mesopotamia has little interest in individual
        authors
    \item Most likely male, all known female scribes lived
        more north.
    \item Unlikely a professional mathematician, since at
        the time academics were not "professionalized".
    \item Unlikely just an educated member of the
        merchant or ruling class. There are no examples of
        this throughout Mesopotamian history, which indicates
        it was socially not considered.
    \item Likely some worker who used literacy and mathematics
        for his job.
    \item The methods used were taught to scribes which could
        indicate this was a trainee or teacher.
    \item Familiar with formats used by administrators of
        Larsa which indicates he was some kind of professional
        bureaucratic scribe.
    \item This lines up with if he was a teacher, as all known
        ancient teachers also had careers in temple
        administration.
    \item unlikely to have been written for the administration,
        too similar to school mathematics or teacher problems.
\end{itemize}


\end{document}
