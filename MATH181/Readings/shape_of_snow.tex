\documentclass{report}
\usepackage[tmargin=2cm, rmargin=1in, lmargin=1in,margin=0.85in,bmargin=2cm,footskip=.2in]{geometry}
\usepackage{amsmath,amsfonts,amsthm,amssymb,mathtools}
\usepackage{enumitem}
\usepackage[]{mdframed}
\usepackage{tikz}
\renewcommand{\familydefault}{\sfdefault}

\title{\Huge{Math 181}}
\author{\huge{Elijah Hantman}}
\date{}

\begin{document}
\maketitle
\newpage

{\large Author}
\begin{itemize}
    \item B. J. Mason (Basil John Mason??)
    \item Meteorologist
    \item Renowned for founding meteorological office
    \item highly influential
    \item President of many scientific bodies
\end{itemize}

{\large Document}
\begin{itemize}
    \item Snowflakes are an extremely interesting thing to study
        near infinite variety yet clear patterns
    \item Most important and notable feature is hexagonal
        symmetry
    \item Some of the earliest obervations come from
        Ancient China where hexagonal crystals were noted
    \item A poem from 500 A.D. references the six sided
        nature of snowflakes
    \item After Kepler in 1665 Robert Hooke uses a microscope
        and records several sketches of snowflakes. revealed
        that the branches were parallel demonstrating errors
        in previous work
    \item As time went on more shapes and theories were discovered
        and proposed. Past works focused exclusively on hexagonal
        plate shapes but pyramids and hybrid crystals also
        can form.
    \item A few years after in a report by Friedrich
        Martens are weather conditions linked to the kinds of
        snowflakes that form
    \item Most literature is on categorization and description,
        Kepler's attempt to explain the growth and reason
        was nearly unique for 300 years.
    \item Kepler discussed many important observations but ultimately
        lacked the knowledge and was unconvinced of any particular
        reason snowflakes were shaped as they are.
    \item Kepler's work was foundational to Crystallography as
        it related non symmetric shapes being packed together
        to more regular symmetry and patterns
    \item The spheres Kepler thought of he did not link
        to atoms, and until the end of the 19th century
        did x-ray diffraction confirm this for chemists.

        Both concepts developed in relative isolation.
\end{itemize}

{\large Modern Explanation}
\begin{itemize}
    \item Water molecules bond to form hexagonal rings
    \item This paper is kinda old, 1966 based on the
        translation
    \item The theory at the time is that water molecules
        find it easier to bond to the sides rather than
        the flat face, so it forms a hexagonal plate.
        At extremely low temperatures the water molecules
        can crystalize directly on the face to form
        hollow columns and pyramids
\end{itemize}

\end{document}
