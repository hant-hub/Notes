\documentclass{report}
\usepackage[tmargin=2cm, rmargin=1in, lmargin=1in,margin=0.85in,bmargin=2cm,footskip=.2in]{geometry}
\usepackage{amsmath,amsfonts,amsthm,amssymb,mathtools}
\usepackage{enumitem}
\usepackage[]{mdframed}
\usepackage{tikz}
\renewcommand{\familydefault}{\sfdefault}

\title{\Huge{Math 181}\\Written In Stone Notes}
\author{\huge{Elijah Hantman}}
\date{}

\begin{document}
\maketitle
\newpage

\begin{itemize}
    \item 4000 BCE, Iraq
    \item published research in Historia Mathematica
    \item Arguement
        \begin{mdframed}
            Babylonians constructed trigonometric
            table using only exact ratios of
            triangles, no need for angles.
        \end{mdframed}
    \item sidenote:
        \begin{mdframed}
            The approach of tying triangles to circles
            ie: angles, dates back to Hipparchus
            of Nicaea who died after 127 BCE.

            He used table of chords to calculate the orbits
            of the Moon and Sun.
        \end{mdframed}
    \item Claim: Babylonian trigonometry conceptualized
        triangles as portions of a rectangle rather than
        inscribed into a circle.
        \begin{mdframed}
            This is claimed to be partially a result of the
            Babylonian base 60 numeral system, which allows
            for many exact ratios and for exact trigonometry
            rather than approximations.
        \end{mdframed}
    \item Claim: The ratio of short side to long side
        was important as a measure of steepness.
\end{itemize}
{\large Tablet}
\begin{itemize}
    \item Claim: 1964 Derek J. Solla Price discovered
        pattern and claims there was meant to be
        38 rows.
    \item Claim: Historian Friburg conjectured missing
        columns should be the remaining pythagorean
        ratios.
    \item Claim: The ratios follow a pattern of moving
        from a square to an more and more eccentric
        rectangle
\end{itemize}


\end{document}
