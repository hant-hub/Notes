\documentclass{report}
\usepackage[tmargin=2cm, rmargin=1in, lmargin=1in,margin=0.85in,bmargin=2cm,footskip=.2in]{geometry}
\usepackage{amsmath,amsfonts,amsthm,amssymb,mathtools}
\usepackage{enumitem}
\usepackage[]{mdframed}
\usepackage{tikz}
\renewcommand{\familydefault}{\sfdefault}

\title{\Huge{Math 181}}
\author{\huge{Elijah Hantman}}
\date{}

\begin{document}
\maketitle
\newpage

{\large Author}
\begin{itemize}
    \item Stefano Gulizia
    \item At the University of Milan History Dept.
    \item Focus on Aristotelian natural philosophy
    \item Smallish number of citations only 53 in total
        on google scholar
\end{itemize}

{\large Abstract}
\begin{itemize}
    \item Layered ontology is important to Kepler
    \item layered ontology developed to deal with
        disciplinary and religious crisis
    \item Explicit belief in Platonism, part of
        geometric constructivism movement
    \item Mysterium Cosmographicum reframes Kepler
        in context of courtly bricolage (construction)
    \item shows that later work De nive sexangula has
        congnitive practicies which are ludic in style
        \begin{mdframed}
            A little lost here but this makes sense,
            On The Six Cornered Snowflake has a lot of
        \end{mdframed}
    \item Kepler's essay on cristallography is an
        epistemic improvement on natural jokes.
\end{itemize}

{\large idk man}
\begin{itemize}
    \item Kepler worked in the midst of sectarian conflict
        and that ended up reflected in his methodology
    \item Kepler's methods were both about nesting solids
        and bringing together multidisciplinary knowledge
    \item Kepler's Mysterium Cosmographicum (1596) introduced
        hypothesis about the solar system, with each
        nested solid seperating the planets.
    \item Both the Mysterium and the Six Cornered Snowflake
        demonstrate play and combines Lutheran theology,
        natural light doctrine, and causal regressus method.
    \item Both defend cosmological neatness and homogeneity
    \item Kepler's writing has a thread of trying to communicate
        the joy of seeking and solving problems rather
        than just expressing his discoveries
    \item The Six Cornered Snowflake has Kepler following threads
        of platonism, spontaneous generation, and divine
        law givers in the shape of snowflakes and honeycomb
    \item His prose is far from the precision of standard
        mathematics
\end{itemize}

\end{document}
