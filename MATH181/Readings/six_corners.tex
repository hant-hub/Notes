\documentclass{report}
\usepackage[tmargin=2cm, rmargin=1in, lmargin=1in,margin=0.85in,bmargin=2cm,footskip=.2in]{geometry}
\usepackage{amsmath,amsfonts,amsthm,amssymb,mathtools}
\usepackage{enumitem}
\usepackage[]{mdframed}
\usepackage{tikz}
\renewcommand{\familydefault}{\sfdefault}

\title{\Huge{Math 181}\\The Six Cornered Snowflake}
\author{\huge{Elijah Hantman}}
\date{}

\begin{document}
\maketitle
\newpage

\begin{itemize}
    \item Written as a gift for a friend
    \item Goal is to find something trivial to discuss
    \item Question: Why do Snowflakes have six corners?
        \begin{mdframed}
            \begin{itemize}
                \item Cannot be because they were in
                    that shape before they froze,
                    water vapor is shapeless
                \item It must be because of some
                    force or agent
            \end{itemize}
        \end{mdframed}
    \item Question: Why are Bee Hives Hexagonal?
        \begin{mdframed}
            \begin{itemize}
                \item Also Bee Hives can be described
                    as constructed of multiple rhombi
                \item Can we construct more shapes out of
                    rhombi?
                \item It should be noted that Rhombi, can
                    tile all of space, and that Honeycombs
                    can be extended to fill all of space.
            \end{itemize}
        \end{mdframed}
    \item Question: Pomagranate seeds under pressure form
        rhomboid shapes, why?
        \begin{mdframed}
            \begin{itemize}
                \item Pomagranate seeds are soft
                \item Pomagranate seeds only form rhomboid
                    shapes under pressure
                \item When you compress soft balls,
                    they also form rhombic dodecahedrons
                    \begin{mdframed}
                        They can also form cubes and any
                        other 3d tiling, however they
                        form rhombic dodecahedrons when
                        they are sufficiently "jumbled",
                        if they are held in a square grid
                        before being compressed they form
                        cubes.
                    \end{mdframed}
            \end{itemize}
        \end{mdframed}
    \item Claim: When you arrange Circles on a plane there
        are two efficient tilings, one is such that
        each sphere has four neighbors, the other is
        so each sphere has six neighbors.
        \begin{mdframed}
            If you pack spheres as tight as possible,
            and layer them on top of each other, you will
            either get square layers with four neighbors
            or triangular (hexagon) layers with six
            neighbors.

            For the square arrangement, each sphere will
            lie on top of a single other sphere.

            In this case when subjected to pressure, each neighbor
            contact will become a flat side and so the spheres
            will become cubes.
        \end{mdframed}
        \begin{mdframed}
            If instead we use a hexagonal or triangular
            tiling we end up 12 neighbors. When compacted
            this leads to a Dodecahedron.

            Kepler states that this is the tightest possible
            but I don't think he proves this.
        \end{mdframed}
        \begin{mdframed}
            The intermediate case, where either the
            layers themselves are four neighbor tilings and
            the layers are combined such that each sphere
            lies on top of four spheres, or

            When we use a hexagonal tiling for the layers
            and then rest each sphere directly on top
            of a lower sphere,

            We end up with hexagonal prisms in the first
            case, and we end up with Dodecahedrons again
            in the second case.
        \end{mdframed}
    \item Construction of stacks
        \begin{mdframed}
            If we construct a pyramid of stacked spheres
            it will naturally fall into the hexagonal
            ie: dodecahedron tiling. This can be easily
            seen by constructing such a pyramid, and then
            peeling off a side of the pyramid to form layers.
            Each sphere in one layer will be touching four spheres
            in the lower layer.
        \end{mdframed}
    \item Claim: The closest pack in three dimensions could
        not exist without the square and the triangle together.
        \begin{mdframed}
            He claims that when spheres are packed together
            in nature they slide off each other until they are
            in the dodecahedron pattern, where they are
            compressed into dodecahedrons. This explains the
            pomagranate seeds mentioned above.
        \end{mdframed}
    \item Question: If packing of space is why Pomagranates are
        rhomboids, then why are honeycombs rhomboids?
        \begin{mdframed}
            Kepler claims God imprinted the pattern onto bees
            as it is the best pattern for them.

            \begin{mdframed}
                I guess that's fair Kepler lmao
            \end{mdframed}
        \end{mdframed}
\end{itemize}

\begin{itemize}
    \item Claim: Plane surface can only be tiled by
        the triangle, square, and hexagon.
        \begin{mdframed}
            The hexagon has the largest area of the
            three, and that makes it valuable for bees
            and pomagranates and so on.
        \end{mdframed}
    \item Claim: Volume can only be tiled by cubes
        and rhomboids
        \begin{mdframed}
            The rhomboids have greater volume than the cubes.

            This doesn't make sense, why should bees care about
            tiny amounts of space, each cell would be larger
            if it were round, and the waste would be between
            the cells.
        \end{mdframed}
        \begin{mdframed}
            A second reason is that Bees construct their
            cells via two bees pushing against a wall. A
            sphere would prevent that, and make the construction
            both more fragile, and more difficult to construct.

            The gaps would also allow for air and other things
            to enter the hive which is undesireable.

            \begin{mdframed}
                Here Kepler says "cold" and talks as if bees
                are building a house like a person and they
                are preventing drafts. Its kinda fair but
                definitely not scientific.
            \end{mdframed}
        \end{mdframed}
    \item Question: If Rhomboids are so cool, why are most
        leaves five sided?
        \begin{mdframed}
            Idk man, I guess maybe its a wonder of nature.

            \begin{mdframed}
                This is mostly to contrast with the arguement
                for six corners, which is about utility and
                filling space, while five sides is unknown
                and therefore we can just wonder at its beauty
            \end{mdframed}
        \end{mdframed}
    \item Both Dodecahedrons and Icosahedrons contain the golden
        ratio
        \begin{mdframed}
            Icosahedrons can be constructed with "golden rectangles"
            where their height to their width is the golden ratio.
        \end{mdframed}
        \begin{mdframed}
            Dodecahedrons are formed out of pentagons which have
            a ton of ratios that happen to be the golden ratio,

            ex: many diagonal lines in a pentagon are to
            the edges the golden ratio.
        \end{mdframed}
        \begin{mdframed}
            Side note: the golden ratio is great for polynomial
            algebra as $\phi^2 = \phi + 1$ so any polynomial
            can be reduced to a linear system.
        \end{mdframed}
    \item Observation: When frost forms, it creates a 
        hexagonal shape.
        \begin{mdframed}
            Is cold just the negation of heat?

            \begin{mdframed}
                Kepler being religious kind a echoes everywhere.
                I think he thought that even if there was
                a reason it was one which was put in
                place knowingly by a god. idk tho.
            \end{mdframed}
        \end{mdframed}
    \item Observation: Hoarfrost is also hexagonal
        \begin{mdframed}
            Question for another time, since tackling
            the flow of hot and cold air is extremely
            intimidating.
        \end{mdframed}
    \item Question: Why are Snowflakes flat?
        \begin{mdframed}
            Layers of water vapour are not cleanly
            deliniated and it makes little sense for
            only a single layer to freeze while the other layers
            remain gas.
        \end{mdframed}
    \item Claim: For the sake of investigation, we assume
        snowflakes form some kind of structure formed by
        three rods, as it falls it lands on three of the
        rods which then buckle and the whole structure flattens
        into a six cornered snowflake
        \begin{mdframed}
            Kepler acknowledges this is a big assumption
            but it is what he is using to motivate the rest
            of his reasoning.

            He is also only testing whether it is true last
            so he doesn't ruin his present.
        \end{mdframed}
    \item Observation: Octohedrons can be formed by taking
        three lines and orienting them at right angles.
        This gives six points which can be connected to
        form an Octohedron.
    \item Observation: A drop is the unit by which
        water vapour condenses. This makes sense since
        a drop of water doesn't break apart on its
        own but rather hold its shape.
    \item Assumption: Water vapour in the air is packed
        together in some tiline, either the cubic or
        Dodecahedron patterns
        \begin{mdframed}
            This assumption would mean that the vapour would
            freeze starting from the gaps, inwards towards
            the centers. This forms six lines of freezing
            in an octohedron pattern
        \end{mdframed}
    \item Question: Why would the water vapour enter this
        pattern?
        \begin{mdframed}
            Water vapour is gaseous so it isn't forced
            or held together in this shape.

            Perhaps it is because the cubic pattern is more
            rotationally symmetric, the rhombic pattern
            has a difference between square tilings
            and hexagonal tilings depending on how
            you slice them.
        \end{mdframed}
    \item Problem: Not all snowflakes are the same
        \begin{mdframed}
            Each snowflake can have different sizes and
            shapes. While all hexagonal they can't be
            formed by a uniform grid.

            In addition there isn't a clear force which
            guides water vapour into a cubic tiling.
        \end{mdframed}
    \item Problem: Cannot be formed by chance
        \begin{mdframed}
            The pattern is clear enough that it would
            be incredibly unlikely to only ever see
            six sided snowflakes rather than five,
            seven, eight, or so on.
        \end{mdframed}
    \item Claim: Since each snowflake is individual in its
        size and all have similar shape then it must be
        something to do with the center drop.
        \begin{mdframed}
            This kinda seems a little presumptuous to me
            but Kepler seems sure that by illustrating
            issues with both uniform tiling and random
            meeting then it must be some kind of growth from
            the center
        \end{mdframed}
    \item Random Kepler pontificating
        \begin{mdframed}
            He argues with an imaginary person who claims
            that the three dimensions needed to form an
            octohedron to collapse into a snowflake are
            similar to animals, but that it doesn't really
            make sense unless we stretch the logic
            to absurdity.

            He then argues that the reason we have a top
            and bottom, left and right, front and back,
            comes from functional principals and the fact
            that each dimension of our bodies is required
            for us to function at all.
        \end{mdframed}
    \item Kepler Gives up
        \begin{mdframed}
            Kepler admits defeat in looking for a scientific
            or geometric reason. He says that the shape
            of a snowflake does not make it more durable
            or longer lasting. Therefore, since
            god made the world, it must be for its
            aesthetics, to "adorn".

            Furthermore we can't attribute snowflakes to
            individual souls seeking beauty because they are
            so short lived and numerous. And they again
            all have six corners.
        \end{mdframed}
    \item Kepler Does Theology
        \begin{mdframed}
            Kepler argues that the world of spirits,
            I guess kinda like angels, are made up of 
            regular geometric figues. So an octohedron
            is the natural way to think of a snowflake rather
            than a hexagon.

            \begin{mdframed}
                I think it is interesting that Kepler is wrong
                here, hexagons are the correct way to view
                a snowflake since they form from hexagonal
                hydrogen bonds between water molecules.
            \end{mdframed}

            He then argues that cubes are the most fundamental
            solid, with octohedrons being closely related.

            He then says the difference between a cube and
            octohedron is that the former points out and the
            later points in. Also an octohedron has fewer
            corners.

            Taken together the octohedron is prefferable
            for a snowflake because it is freezing in a pattern
            moving inwards towards a center rather than outwards.

            \begin{mdframed}
                Here Kepler is backwards I think, I might
                be reading him incorrectly though.
            \end{mdframed}
        \end{mdframed}
    \item Kepler ends by revisiting his assumption
        \begin{mdframed}
            Kepler revisits his assumption that snowflakes
            come from octohedrons and concludes he must be
            incorrect. This comes after observing
            a snowflake which is shaped like a star and
            had a tiny spine on the back which held it up.

            He concluded that perhaps snowflakes must be
            formed in a flat plane.

            He then comes up with multiple reasons why this should
            be based on the properties of a hexagon,
            before dismissing them one after another,
            concluding with perhaps chemists could find
            the answer where geometry has failed.
        \end{mdframed}
\end{itemize}


\end{document}
