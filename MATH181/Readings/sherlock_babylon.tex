\documentclass{report}
\usepackage[tmargin=2cm, rmargin=1in, lmargin=1in,margin=0.85in,bmargin=2cm,footskip=.2in]{geometry}
\usepackage{amsmath,amsfonts,amsthm,amssymb,mathtools}
\usepackage{enumitem}
\usepackage[]{mdframed}
\usepackage{tikz}
\renewcommand{\familydefault}{\sfdefault}

\title{\Huge{Math 181}\\Sherlock Holmes in Babylon}
\author{\huge{Elijah Hantman}}
\date{}

\begin{document}
\maketitle
\newpage

\begin{itemize}
    \item Goal: Trace research of Otto Neugebauer and
        Abraham Sachs.
    \item Note: Author does not claim to be a historian,
        this is a tertiary source referencing the
        work of historians on primary sources
    \item Claim: Tablet was made 3,700 years ago,
        in the city of Nippur.
    \item He walks through the process of conjecture
        and validation.
    \item Missing multiplication tables
    \item Claim: due to practice of omitting trailing
        zeros, multiplication tables serve as tables
        of recipricals which multiply to give one.
    \item The values in the table can be used to perform
        floating precision multiplies and divides.
    \item Evidence: inscribed cylinder with copies of both
        reciprocal and multiplication tables. May have
        been used as a device for doing computations.
    \item Claim: Plimpton tablet could not be arithmetic
    \item Evidence: Tablet contains statistically unusually
        high amount of prime numbers, which indicates some
        kind of mathematics beyond accounting.
    \item Evidence: By combining columns specific numeric
        relationships become obvious, ie: one is the sum
        of two squares, and the other is the difference.
    \item Evidence: By looking at where the pattern breaks
        down is it possible for these to be errors by the
        scribe? Looking at each error they can be shown to
        be either one or two symbols away from a correct value
        which indicates that our notion of correct values is
        accurate and also that the possibility of the discrepencies
        being a copying error is not unlikely.
\end{itemize}
{\large Purpose?}
\begin{itemize}
    \item Claim: You can use the values of each column to create
        a right triangle.
    \item Arguements:
        \begin{mdframed}
            No evidence these relations would be of interest
            to Babylonians. However there is evidence
            they were aware of how to solve quadratic equations.
        \end{mdframed}
    \item Neugabauer and Sachs used a slightly different ratio
        because they believe the tablet has leading ones
        but its disputed. Both ratios can equally be used
        to characterize the tablet entries.
    \item Common school problem in Babylon to solve equation
        $x - x^R = d$ where  $x^R$ is the reciprocal of
         $x$.
     \item Specific algorithm taught in other tablets
         could be done using ratios in Plimpton tablet.
     \item This is an alternative interpretation, where
         the tablet has nothing to do with triangles but instead
         with solving numerical puzzles.
\end{itemize}

\end{document}
