\documentclass{report}
\usepackage[tmargin=2cm, rmargin=1in, lmargin=1in,margin=0.85in,bmargin=2cm,footskip=.2in]{geometry}
\usepackage{amsmath,amsfonts,amsthm,amssymb,mathtools}
\usepackage{enumitem}
\usepackage[]{mdframed}
\usepackage{tikz}
\renewcommand{\familydefault}{\sfdefault}

\title{\Huge{Math 181}\\Philosophy of Mathematics}
\author{\huge{Elijah Hantman}}
\date{}

\begin{document}
\maketitle
\newpage

\begin{description}
    \item Author 
        \begin{mdframed}
            Written by Paolo, Mancousu, A philosopher

            Specialized in Philosophy of methematics and
            history

            Many published works

            Some translatation work

            Several thousand citations across his work
        \end{mdframed}
\end{description}

{\large Philosophy of Mathematics and Mathematical Practice
in the Early Seventeenth Century}
\begin{itemize}
    \item p. 8 Traditional focus is on impacts of Galilean
        revolution of physics
    \item p. 8 Continuity hypothesis, that the history
        of mathematics is unbroken from the Greeks to the
        modern day.
    \item p. 9 Continuity hypothesis is kinda contentious,
        however not particularly relevant here
    \item p.9 The goal isn't to find specific predecessors
        to discoveries or advancements but to trace
        methodology and principles
    \item p. 9 Pure mathematics made massive leaps in
        the 17th century
        \begin{mdframed}
            \begin{itemize}
                \item Algebra
                \item Analytic Geometry
                \item Projective Geometry
                \item Probability theory
                \item Calculus
                \item usage of continuous and
                    infinite methods like calculus
            \end{itemize}
        \end{mdframed}
    \item p.9 More popular to see the 17th century as
        a radical departure from Greek methods
    \item p. 10 Is mathematics an Aristotelian science?
        Also, in general how does the philosophy of
        mathematics compare or
        contrast with ancient Greek ideas.
\end{itemize}
{\large The Quaestio De Certitudine Mathematicarum}
\begin{itemize}
    \item p. 11 Aristotle said that to posses scientific
        knowledge we need to know how we know otherwise
        it is useless.
    \item p. 11 Four causes Formal, material, efficient,
        and final.
    \item p. 12 Only some lines of reasoning are valid
        for Aristotle. Reasoning from effects to causes
        is weaker than reasoning from causes
        to effects. Since the ground is not wet
        it did not rain today is a demonstration of a
        fact. It did not rain today therefore the
        ground is not wet is a demonstration of a 
        reasoned fact.
    \item p. 12 Aristotle gave three levels of proof which
        were integrated into 17th century thought. The
        fact and reasoned fact were the first two
        and were for most sciences. The last was
        defined as giving both cause and effect
        simultaneously and was believed to be
        the main method for mathematical reasoning.
    \item p. 12 Some argue that mathematics cannot be
        this highest level of demonstration, but are
        nonetheless still clearer than any scientific
        knowledge.
        \begin{mdframed}
            Mathematics is fundamentally not causal, and therefore
            demonstrations are not scientific.

            It is debatable whether any scientific field rises
            to an Aristotellian science especially since Aristotle
            requires all types of demonstrations be possible.
        \end{mdframed}
    \item p. 13 An example of how mathematics is not causal.
        \begin{mdframed}
            Euclid's proof that the sum of a triangle's internal
            angles is always 180 degrees. The proof uses
            some additional external angles, however these
            angles do not cause, and are not caused by the
            triangle's internal angles.

            The triangle adding to 180 degrees is simply true
            regardless of the angles or geometry we do around
            it.
        \end{mdframed}
    \item p. 16 Famous treatsie on Explicit Mathematical
        finitism. Explicitly defined mathematics as dealing
        with finite and bounded entities, and used that to
        delineate between mathematics and the other sciencies.
    \item p. 17 Biancani argues that mathematical concepts
        exist in the minds of humans and God, therefore they
        are "real".
    \item p. 17 Biancani's arguement is a Chritianization of
        Aritotelian ideas about Platonism and ideal
        concepts of spheres and shapes.
        \begin{mdframed}
            Paolu is careful to note here that the explicit
            religiousity of these ideas indicates we should
            be skeptical when they attribute Galilean ideas
            to Platonism.
        \end{mdframed}
    \item p. 17 Biancani also chooses to argue that mathematical
        demonstrations can be causal. He argues from Greek
        thinkers, as well as by arguing that while mathematics
        does not fulfill some of Aristotle's methods of
        scientific knowing it does fulfill many of the
        classical thoughts about mathematics.
    \item p. 18 Biancani uses Euclid's demonstration of how
        to construct an Equilateral triangle over a line
        segment to argue that mathematical demonstrations
        can be causal, as the definitions cause the
        construction to be valid.
    \item p. 18 Biancani was representative of a movement
        of Aritotelians to attempt to adapt to modern
        mathematics. It eventually petered out as math
        grew so far beyond the bound of traditional syllogism
        and logic that it wasn't useful as a framework.
    \item p. 19 Biancani was well recieved and cited by
        multiple philosophers and histories of mathematical
        thought.

        Despite this, his views were not embraced by scholarship.
    \item p. 20 Biancani's views placed mathematics on a high
        pedestal, above all sciences as it was the only one
        to produce both certain proof and causal scientific
        knowledge. Not to mention the objects it dealt with
        were direct godly objects rather than the material
        and uncertain.
    \item Next section is about Gassendi and Barrow. Barrow's
        goal was to show mathematics was a real science.
    \item p. 20 Barrow argued that axioms were developed
        through scientific induction.
    \item p. 20 Barrow also argues that Bancani was wrong and
        mathematical figures could exist in reality, and
        are as real as thought experiements in physics.
    \item p. 21 Barrow argues that mathematical demonstrations
        are entirely Causal, and that Axioms are the cause
        and the intermediary facts therefore cause their
        conclusions.
    \item p. 21 Barrow also points towards defitions as
        causing their resulting attributes. Things like
        triangles having internal angles equal to 180 degrees
        is caused by their definition and the definition of
        angles.
    \item p. 23 Barrow begins with the belief that mathematics
        are the best and most perfect science, since
        they produce both scientific demonstrations, as well
        as certainty beyond any science.
    \item p.23 Gassendi argued that no Aristotelian science
        existed. Much of Barrows arguementation seems
        to be pointed at refuting Gassendi.
        \begin{mdframed}
            Gassendi quoted extensively fro Pereyra
            whom Barrow also explicitly refuted
            in his lectures. 

            Gassendi's main intellectual challenge was
            his status as an important and well known
            philosopher
        \end{mdframed}
    \item The primary object of dicussion was the certainty,
        the scientific rigor, and the classical
        conformity of mathematics to the Greek ideal
        of science and knowledge.
\end{itemize}
{\large The Quaestio and Mathematical Practice}
\begin{itemize}
    \item p. 25 Proof by Contradiction
        \begin{mdframed}
            Piccolomini argues that mathematical propositions
            could not be proven by causal proofs. Euclid's propositions
            I.35 and I.36 imply each other. If they were causal
            then they would cause themselves which is parodoxical.

            This is a broad arguement that biconditional theorems
            must not be causal since if they were they would be
            parodoxical and recursive.
        \end{mdframed}
    \item p. 25 Barozzi replied that although some parts of mathematics
        must not be causal, some parts may be and that some relationships
        not being causal does not invalidate causality as a framework
        for understanding math.
    \item p. 25 Both Barozzi and Piccolomini seem to agree that
        proofs by contradiction are acausal and do not
        conform to causal principles.
    \item p. 26 Both Pereyra and Biancani also agree on
        this point
    \item p. 26 As a consequence direct proofs were seen
        as superior to proofs by contradiction.
    \item p. 26 Commentator Rivaltus on the work of Archimedes
        takes the position that proofs by contradiction are
        equal to any other proof. This is because the requirement
        for a proper demonstration is about knowing where the
        knowledge came from rather than knowing what caused
        the fact.
        \begin{mdframed}
            Rivaltus says that the intermediate steps of the
            proof don't cause the conclusion, but cause us
            to know the conclusion.
        \end{mdframed}
    \item p. 27 Aristotle had explicitly argued that proof by
        Contradiction was lesser than direct proof which made
        Rivaltus' positino unpopular
    \item p. 27 Guldin was a philosopher who argued against
        Rivaltus. He appealed to the authority of
        Biancani to argue that direct proofs are valued more
        than contradiction. He also started a program to remove
        proof by contradiction from Euclidean geometry.
\end{itemize}
{\large Proofs By Superposition}
\begin{itemize}
    \item p. 28 Proof by superposition is when you have two figures
        and as part of the proof you show equality between
        each part of one figure to show it is the same as
        another figure.
    \item p. 29 Superposition was common in Greek mathematics
        but was only used rarely. In the 17th century
        there was a motion to treat it as lesser in the same
        way proof by contradiction was.
    \item p. 30 Peletier argued that the propositions that
        use superposition should be instead taken as axioms.
    \item p. 30 Many commentators on Euclids work in the
        late Renaissance argued that congruency should be
        considered invalid
        \begin{mdframed}
            It was seen as below the dignity of geometry
            to imply displacement or to mechanically compare
            figures when overlayed
        \end{mdframed}
\end{itemize}

\end{document}
