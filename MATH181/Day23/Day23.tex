\documentclass{report}
\usepackage[tmargin=2cm, rmargin=1in, lmargin=1in,margin=0.85in,bmargin=2cm,footskip=.2in]{geometry}
\usepackage{amsmath,amsfonts,amsthm,amssymb,mathtools}
\usepackage{enumitem}
\usepackage[]{mdframed}
\usepackage{tikz}
\renewcommand{\familydefault}{\sfdefault}

\title{\Huge{Math 181}\\Day 23 Notes}
\author{\huge{Elijah Hantman}}
\date{}

\begin{document}
\maketitle
\newpage

\begin{description}
    \item How to study Counting Words in the History
        of Mathematics?
        \begin{mdframed}
           Ancient Greek has a similar structure to 
           English. 

           English uses opaque words for zero through 10, then
           for 13-19 the prefix teen is used. 11 and 12 is
           odd, but 20 and above uses a system of prefixes.
        \end{mdframed}
        \begin{mdframed}
            Santa Barbara Native language had two bases,
            10 and 4.

            Had subtractive principle, some numbers
            are spoken as less than some fixed number.

            The closest thing to this in English is time.
        \end{mdframed}
    \item Formalizing Counting Words
        \begin{mdframed}
            Existing framework known as Generative Grammer.
        \end{mdframed}
    \item Hurford
        \begin{enumerate}
            \item Formal Grammer for counting
            \item Linguistic rules for converting numbers
                into words
            \item Recognizing well formed words
        \end{enumerate}

        \begin{mdframed}
            As an example, 400 as four hundred vs. twenty
            twenty.

            In english the second is not well formed since
            it does not follow standard convention.
        \end{mdframed}
        What is a formal Grammer?
        \begin{mdframed}
            A formal grammer consists of:
            \begin{enumerate}
                \item Terminal symbols
                \item Non Terminal Symbols
                \item A distinguished non-terminal symbol
                    called the Start State
                \item A set of rules of the form
                    $NT \to$ some symbols in the grammer
            \end{enumerate}
        \end{mdframed}
        \begin{mdframed}
            Examle:

            \begin{enumerate}
                \item Terminal symbols: $\{a, b\}$
                \item Non Terminal Symbols: $\{S\}$ 
                \item Rules: $\{S \to a|S, S \to b\}$
            \end{enumerate}

            To produce a valid word in the grammer,
            we begin at the start state, then select
            rules to apply until we run
            out of rules.

            ex:

            \begin{gather}
               S \quad (S \to aS)\\ 
               aS \quad (S \to aS)\\
               aaS \quad (S \to b)\\
               aab\\
            \end{gather}
        \end{mdframed}
        \begin{mdframed}
            The Language of a formal grammer is all
            words in terminal symbols produced from S by
            repeatedly applying rules. (Somtimes known as
            production rules)

            We can produce trees from the process of generating
            words in the following way.

            The root begins with the start symbol, at each
            level all avalible rules produce branches on this
            tree. The path from the root to a leaf is the
            order of which rules are applied to generate
            a specific word.

            We can also construct a tree with a start symbol
            at the root, and each level are the symbols
            produced by applying some rule. For example
            the word aab in the above language would
            have a tree S, then a, S, then a, S, then b.
        \end{mdframed}
    \item Formal Grammer For Numbers in English
        \begin{mdframed}
            \begin{enumerate}
                \item Terminal Symbols $\{I, X\}$ (one, ten)
                \item Non Terminal Symbols $\{N, P, M\}$ 
                    Which stand for Number, Phrase, and Multiply
                \item Production Rules:
                    \begin{itemize}
                        \item $N \to I$
                        \item $N \to I N$
                        \item $N \to P$
                        \item $N \to P N$
                        \item $P \to N M$
                        \item $M \to X$
                        \item $M \to N M$
                    \end{itemize}
                \item Notes: $N$ is our start symbol
            \end{enumerate}

            For example:
            \begin{gather}
               N\\ 
               I N\\
               \quad I N\\
               \quad \quad I N\\
               ...
            \end{gather}

            \begin{gather}
               N\\ 
               P \quad N\\
               N \quad M\\
               I \quad N \quad M\\
               I \quad I \quad X\\
            \end{gather}
        \end{mdframed}
        \pagebreak
    \item Interpretation
        \begin{mdframed}
            Each node in a tree produced by the grammer
            has a value determined as follows.

            \begin{itemize}
                \item $I = 1$
                \item $X = 10$
                \item $N = x + y$
                \item $N = x$
                \item $M = x ^ y$
                \item $M = x$
                \item $P = x \cdot y$
            \end{itemize}
        \end{mdframed}
    \item The Plan
        \begin{enumerate}
            \item Convert Trees to Words
            \item Rules for Recognizing Words
            \item Rules for Generating Words
        \end{enumerate}
        \begin{mdframed}
            We can start categorizing trees with names. 
            A tree with a $N$ root, and  $I$ and three
            as children, that is four.

            We can continue like this.

            One Hundred is a tree with a phrase, a
            left child of one, and a right child
            of hundred.

            A hundred is a tree with a multiply,
            with a two on one side and a ten on
            the other.
        \end{mdframed}
        \begin{mdframed}
            How could we analyze the same trees for Ancient Greek?

            The tree for Ancient Greek is very structurally
            similar to English. Some languages are not
            similar.
        \end{mdframed}
    \item Example Mixtec, parts of Southern Mexico Used pre contact in Mexico
        \begin{mdframed}
            \begin{itemize}
                \item Terminal Symbols
                    \begin{itemize}
                        \item 1
                        \item 10
                        \item 15
                        \item 20
                    \end{itemize}
                \item Non Terminal Symbols
                    \begin{itemize}
                        \item N (number)
                        \item P (phrase)
                        \item M (multiply)
                    \end{itemize}
                \item Production Rules
                    \begin{itemize}
                        \item $N \to 1$
                        \item $N \to 1\quad N$
                        \item $N \to P$
                        \item $N \to P \quad N$
                        \item $P \to N \quad M$
                        \item $M \to 10$
                        \item $M \to 15$
                        \item $M \to 20$
                        \item $M \to N \quad M$
                    \end{itemize}
            \end{itemize}            
        \end{mdframed}
        \begin{mdframed}
            It is a fallacy to think cultures without highly developed
            mathematics have simple or primitive linguistic
            number systems. 

            Our ability to speak numbers linguistically is fundamental
            to our linguistic ability and doesn't substantially evolve
            over time.
        \end{mdframed}
\end{description}


\end{document}
