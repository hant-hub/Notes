\documentclass{report}
\usepackage[tmargin=2cm, rmargin=1in, lmargin=1in,margin=0.85in,bmargin=2cm,footskip=.2in]{geometry}
\usepackage{amsmath,amsfonts,amsthm,amssymb,mathtools}
\usepackage{enumitem}
\usepackage[]{mdframed}
\usepackage{tikz}
\renewcommand{\familydefault}{\sfdefault}

\title{\Huge{Math 181}\\Day 10 Notes}
\author{\huge{Elijah Hantman}}
\date{}

\begin{document}
\maketitle
\newpage

\begin{description}
    \item Left Off with work of Nichomachus. 
        \begin{mdframed}
            \begin{itemize}
                \item Nichomachus claims mathematics came
                    from Pythagoras which originated in
                    Egypt. Born from religious ideas
                    of everything as numbers.
                \item Nichomachus is presenting ideas
                    from much later, ideas came from
                    working with abbaci, and accounting
                    work.
            \end{itemize}
        \end{mdframed}
        \begin{mdframed}
            Earliest math that we can document comes from
            Hippocrates of Chios (470-421 BC)

            \begin{mdframed}
                Side note: different Hippocrates than the
                Hippocratic oath. Same name, different people.
            \end{mdframed}

            Did some geometric work which we still have and
            can date, a lot of work after that.
        \end{mdframed}
    \item {\large Plato's Meno}
        \begin{mdframed}
            It was a socratic dialogue from around 385 BC that
            discusses math. The main discussion is about
            where virtue comes from, in the form of a discussion
            between Socrates and the politician Meno.

            \begin{itemize}
                \item Socrates claims that we merely recall
                    things that we already know
                \item Meno asks for a demonstration
                \item Socrates calls over one of Meno's
                    slaves, and Socrates starts to teach
                    the slave geometry.
                    \begin{mdframed}
                        He walks the slave boy through
                        a geometric explanation of multiplication.
                        The slave gets the correct answer first.

                        The slave gets the incorrect answer,
                        and Socrates points out he was only
                        asking questions and not actually
                        teaching. They also dunk on the literal
                        slave for not knowing mathematics.

                        Socrates continues walking through the problem
                        with the slave. He leads the boy to the conclusion
                        that the scaling factor must be between 2 and 4
                        feet. And prompts the slave to answer 3, which
                        is also incorrect.

                        \begin{mdframed}
                            Socrates is such an asshole to the literal slave
                            child with no education.
                        \end{mdframed}

                        Socrates points out that his
                        questioning has prompted the slave
                        to seek knowledge. 

                        Socrates then endeavors to show the
                        correct answer was inside the Slave
                        all along by asking leading questions.

                        \begin{mdframed}
                            Bruh, 90\% is Socrates talking and the
                            other 10\% is other people saying
                            "sure plato", "certainly". Nobody
                            ever gets to be smart when Socrates
                            is around.
                        \end{mdframed}
                    \end{mdframed}
                \item Meno goes full coward mode and gives in
                    and licks Socrates boots
            \end{itemize}
            Generally considered to be a fictional account
            Plato was using to make points.
        \end{mdframed}
        \pagebreak
    \item Mathematics of Hippocrates
        \begin{mdframed}
            A lot of focus on Squares and constructing
            Squares.
        \end{mdframed}
        \begin{mdframed}
            Start with inscribing a square in to a 
            circle. A second circle is created which
            passes through two adjacent points on the square
            with a diameter equal to the side of the inscribed
            square.

            The question was "What is the area of the second
            circle that is outside the first?". The question
            is also known as the Lune of Hippocrates.


        \end{mdframed}
\end{description}


\end{document}
