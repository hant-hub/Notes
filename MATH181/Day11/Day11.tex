\documentclass{report}
\usepackage[tmargin=2cm, rmargin=1in, lmargin=1in,margin=0.85in,bmargin=2cm,footskip=.2in]{geometry}
\usepackage{amsmath,amsfonts,amsthm,amssymb,mathtools}
\usepackage{enumitem}
\usepackage[]{mdframed}
\usepackage{tikz}
\renewcommand{\familydefault}{\sfdefault}

\title{\Huge{Math 181}}
\author{\huge{Elijah Hantman}}
\date{}

\begin{document}
\maketitle
\newpage

\begin{description}
    \item Area of a lune cont. 
        \begin{mdframed}
            Hippocrates argued:
            \begin{enumerate}
                \item Area $AB^2 = 2 AC^2$ 
                    \begin{mdframed}
                        This equates the area of the
                        inscribed square to a square with
                        side length equal to the diameter
                        of the circle.

                        This follows from the Pythagorean theorem.
                    \end{mdframed}
                \item The halfcircle $ACB = 2 \cdot AEC$
                    \begin{mdframed}
                        The large circle is twice the size
                        of the large circle.
                    \end{mdframed}
                \item $\frac{1}{2}$ of the small circle, is
                    $\frac{1}{4}$ of the large circle.
                    Which means  $AEC = AC$ Where $AC$ 
                    is a quadrant of the large circle.
                \item Since the quadrant and the half circle
                    overlap, by finding the area of the
                    quadrant which does not overlap the
                    half circle, we can find the area of
                    the lune.
                    \begin{mdframed}
                        The triangle ACD where D is the
                        center of the large circle has side
                        lengths $1$. This means the 
                        area is  $1^2 \times \frac{1}{2}$
                        which is $\frac{1}{2}$
                    \end{mdframed}
            \end{enumerate}
            Greeks thought the way you should reason about the world
            is to start from agreed upon facts and deduce consequences.
            This is the start of axioms and proofs.

            Hippocrates appears to use the following axioms:
            \begin{itemize}
                \item Pythagorean Theorem (Probably misnamed)
                    \begin{mdframed}
                        Comes up lots when surveying or 
                        in agriculture and land management.
                    \end{mdframed}
                \item Angle on a circle's diameter is
                    right
                    \begin{mdframed}
                        This comes from the parallel postulate,
                        it means the final triangle is
                        easy to compute the area becuase
                        the interior angle is right.
                    \end{mdframed}
                \item Similar circles take up similar fractions
                    of similar squares.
                    \begin{mdframed}
                        Both circles are some fraction
                        of a square. They take up the same
                        fraction of a square which has side
                        length equal to its diameter.
                    \end{mdframed}
            \end{itemize}
        \end{mdframed}
    \item  {\large How did we know Hippocrates did it?}
        \begin{mdframed}
           We don't have original manuscripts. 
           We have nothing from his lifetime.
           We have several degrees removed summaries of
           his work.

           We have work from Simplicius (490-560AD) around the
           time Christianity took off. Hippocrates lived
           around (470-421 BC) which is nearly 1000 year
           difference.

           Simplicius is actually just quoting Eudemus'
           History of Geometry (370-300BC). Simplicius was
           mostly discussing Aristotle and quoted Eudemus
           which mentioned Hippocrates incidently. Still
           nearly 100 years after, and Eudemus likely did not
           have access to Hipocrates original work.

           Simlicius does not distinguish quotes, however
           textual analysis shows more archaic Greek which
           indicates a quotation.
        \end{mdframed}
        \pagebreak
        \begin{mdframed}
            Some Circumstantial evidence:
            \begin{enumerate}
                \item Hippocrates is dated to time when people
                    started writing single authored texts
                    \begin{mdframed}
                        Earlier texts were inventories,
                        tax records, text which was not
                        attributed to any individual.
                    \end{mdframed}
                \item Lots of math after him.
                    \begin{mdframed}
                        Unlike Pythagoras we see mathematics
                        immediately after Hippocrates, where
                        people were building on his ideas.
                    \end{mdframed}
                \item Hippocrates is Obscure
                    \begin{mdframed}
                        Unlike Pythagoras who was mythologized
                        in many ways, Hippocrates was relatively
                        unknown and unlikely to be overly
                        exaggerated or mythologized.
                    \end{mdframed}
            \end{enumerate}
            \begin{mdframed}
                I think its funny how the professor prounounces
                Eudemus differently every time.
            \end{mdframed}
        \end{mdframed}
\end{description}


\end{document}
