\documentclass{report}
\usepackage[tmargin=2cm, rmargin=1in, lmargin=1in,margin=0.85in,bmargin=2cm,footskip=.2in]{geometry}
\usepackage{amsmath,amsfonts,amsthm,amssymb,mathtools}
\usepackage{enumitem}
\usepackage[]{mdframed}
\usepackage{minibox}
\usepackage{tikz}
\usetikzlibrary{arrows}
\usepackage{pgfplots}
\pgfplotsset{compat=newest}
\renewcommand{\familydefault}{\sfdefault}

\definecolor{uuuuuu}{rgb}{0.26666666666666666,0.26666666666666666,0.26666666666666666}
\definecolor{qqqqff}{rgb}{0,0,1}

\title{\Huge{Math 181}}
\author{\huge{Elijah Hantman}}
\date{}

\begin{document}
\maketitle
\newpage

\begin{description}
    \item {\large Problem 1}\\ 
        Attached on Back.
    \item {\large Problem 2}
        \begin{mdframed}
            George A. Miller was born Lynnville
            Pensylvania, on the 31st of July, 1863.
            He got his undergraduate in Muhlenberg College,
            and his PhD at Cumberland University. 
            In 1895 he got his passport, and according to
            newspapers studied at the Universities of Leipzig
            and Paris. He taught at multiple American universities throughout
            the 1890s, before settling into the University of Illinois in
            1906. In 1900 he won the International Mathematics Prize
            and wrote more than 800 artiles.
            There he became very wealthy via investments.
            One of this mathematical contributions was a history of
            abstract groups in 1927. Group theory would go on to
            be a massively important field in abstract algebra
            and have connections to other disciplines like physics.
            He retired in 1931, and passed away in 1951 leaving
            almost a million dollars to the University of Illinois.
        \end{mdframed}
        \begin{center}
            \minibox[frame]{
                As a side note: the census is impossible\\
                to read, the resolution is both too low, and the cursive\\
                strokes too light to make out properly. In addition,\\
                initials are used making it difficult to tell who\\
                each row refers to.
            }
        \end{center}
    \item {\large Problem 3}
        \begin{mdframed}
            I would be interested to know several things about
            Miller. One is what he won the International Mathematics
            Prize for. I presume there are original documentation
            of all the winners of the prize, as well as newspapers,
            published announcements, and so on that can be found.

            Another interesting thing I am curious about has to do
            with the paper on Abstract groups. I am curious if he
            had any other contributions to Group theory,
            abstract algebra, and so on. His published work should
            still be avalible via libraries and archives of mathematical
            papers. There are also some citations of other work of
            his in his own paper which could lead to answers.
        \end{mdframed}
    \item {\large Problem 4}
        \begin{mdframed}
            The problem presented is as such: 
            \begin{mdframed}
                Solve and Discuss:
                \begin{displaymath}
                    \begin{cases}
                        x^2 + y^2 = a^2\\ 
                        log(x) + log(y) = n
                    \end{cases}
                \end{displaymath}
            \end{mdframed}

            This is very vague, so I set out a couple of goals
            to orient my work.
            \begin{enumerate}
                \item Combine both equations into one.
                \item Write $a$ in terms of $n$ and vice versa.
                \item Write $x$ and $y$ in terms of only
                    $a$ and $n$.
            \end{enumerate}

            First Lets begin by rewriting the second
            equation to remove the logarithms.
            \begin{gather}
                log(x) + log(y) = n\\ 
                log(xy) = n\\
                e^(log(xy)) = e^n\\
                xy = e^n 
            \end{gather}
            We already get a very nice simplification which
            is begging for further manipulation.
        \end{mdframed}
        \begin{mdframed}
            Next lets solve for $y$ and plug it into the
            first equation.

            \begin{gather}
                xy = e^n\\
                y = \frac{e^n}{x}\\ 
                y^2 = \frac{e^{2n}}{x^2}
            \end{gather}

            Now to substitute.

            \begin{gather}
                x^2 + y^2 = a^2\\ 
                x^2 + \frac{e^{2n}}{x^2} = a^2\\
                x^4 + e^{2n} = a^2 x^2\\
                x^4 - a^2 x^2 = -e^{2n}
            \end{gather}

            We have already acomplished the first goal,
            equation 11 combines both constraints into a single
            statement.

            \vspace{10pt}

            To write $a$ in terms of $n$ we need to do a
            substitution. We set $u = x^2$ and substitute.
            \begin{gather}
                u^2 - a^2 u = -e^{2n}    
            \end{gather}
            This is clearly a quadratic, so we first put
            the quadratic into standard form, then solve using
            the quadratic formula.

            \begin{gather}
                u^2 - a^2 u + e^{2n} = 0 
            \end{gather}

            The roots of this quadratic therefor are:
            \begin{gather}
                u = \frac{a^2 \pm \sqrt{a^4 - 4e^{2n}}}{2} 
            \end{gather}

            Undoing our substitution of $u$ we can obtain
            the possible values of $x$.

            \begin{gather}
                x = \sqrt{\frac{a^2 \pm \sqrt{a^4 - 4e^{2n}}}{2}}     
            \end{gather}

            We can eliminate all negative values of $x$ by noticing
            that the second equation is only valid when $x$ and
            $y$ are positive, because logarithms are only defined
            on $\mathbb{R}^+$. This means only positive solutions
            are valid.

            \vspace{10pt}

            The number of solutions will depend on both $a$
            and $n$. This can be seen clearly by graphing both
            equations.

            \begin{center}
                \begin{tikzpicture}
                    \begin{axis}[ 
                        xlabel=$x$,
                        ylabel={$y$},
                        axis equal image
                        ] 
                        \addplot [domain=0.3:10, smooth, thick]{e/x}; 
                        \draw (axis cs:0,0) circle [radius=(2*e)^0.5];
                    \end{axis}
                \end{tikzpicture}
            \end{center}

            Based on this graph we can see that there are either,
            one, two, or no solutions. Lets see if we can find the
            values of $a$ and $n$ that yield a unique solution. These
            are the values for which our formula above only has a
            single value, which means that:

            \begin{displaymath}
                a^4 - 4e^{2n} = 0
            \end{displaymath}

            This would remove the term under the radical thus
            simplifying the formula to yield only a single
            value. All that's left is solving for $a$.

            \begin{gather}
                a^4 - 4e^{2n} = 0\\ 
                a^4 = 4e^{2n}\\
                a^2 = 2e^n\\
                a = \sqrt{2e^n}
            \end{gather}

            This also means we can write $n$ as:

            \begin{gather}
                n = ln(\frac{a^2}{2})
            \end{gather}

            If we are interested in any solutions at all,
            we simply replace the equality with an inequality:

            \begin{gather}
                a^4 - 4e^{2n} \ge 0\\ 
                a^4 \ge 4e^{2n}\\
                a^2 \ge 2e^n\\
                a \ge \sqrt{2e^n}
            \end{gather}
        \end{mdframed}
        \pagebreak
        \begin{mdframed}
            To solve the last goal all we need to do is plug
            our solution back into one of the equations and solve
            for $y$.
            \begin{gather}
                x^2 + y^2 = a^2\\
                y = \sqrt{a^2-x^2}\\
                y = \sqrt{a^2 - \frac{a^2 \pm \sqrt{a^4 - 4e^{2n}}} {2}}\\
                y = \sqrt{a^2 - \frac{a^2}{2} \mp \frac{\sqrt{a^4-4e^{2n}}}{2}}\\
                y = \sqrt{\frac{a^2 \mp \sqrt{a^4-4e^{2n}}}{2}}
            \end{gather}
            This gives something which is equivalent to the formula
            we obtained for $x$. This makes sense since both equations
            are symmetric about the $y = x$ axis.
        \end{mdframed}
    \item {\large Problem 5}
        \begin{mdframed}
            Given a triangle ABC, we can bisect two of its
            angles with lines. Prove that if the segments
            of each line inside the triangle are of equivalent
            length the entire triangle is isosceles.

            \begin{center}
                
                \scalebox{0.5}{ 
                    \begin{tikzpicture}[line cap=round,line join=round,>=triangle 45,x=1cm,y=1cm]
                        \begin{axis}[
                            x=1cm,y=1cm,
                            axis lines=middle,
                            axis line style = {draw=none},
                            ymajorgrids=false,
                            xmajorgrids=false,
                            xmin=-8,
                            xmax=8,
                            ymin=0,
                            ymax=11,
                            yticklabel = \empty,
                            xticklabel = \empty,
                            xtick={},
                            ytick={},]
                            \clip(-9.703709200000004,-9.009582400000001) rectangle (20.50945840000001,7.578670599999998);
                            \draw [line width=0.8pt] (-3,1)-- (0,7);
                            \draw [line width=0.8pt] (0,7)-- (3,1);
                            \draw [line width=0.8pt] (3,1)-- (-3,1);
                            \draw [line width=0.8pt] (3,1)-- (-1.583592135001262,3.8328157299974768);
                            \draw [line width=0.8pt] (-3,1)-- (1.583592135001262,3.8328157299974768);
                            \begin{scriptsize}
                                \draw [fill=qqqqff] (-3,1) circle (2.5pt);
                                \draw[color=qqqqff] (-2.746306000000001,1.7003090999999986) node {$A$};
                                \draw [fill=qqqqff] (3,1) circle (2.5pt);
                                \draw[color=qqqqff] (3.244791200000002,1.7003090999999986) node {$B$};
                                \draw [fill=qqqqff] (0,7) circle (2.5pt);
                                \draw[color=qqqqff] (0.24924260000000076,7.176043099999998) node {$C$};
                                \draw[color=black] (-0.9103245999999998,4.1482842999999985) node {$l$};
                                \draw[color=black] (1.1511282000000012,4.1482842999999985) node {$m$};
                                \draw[color=black] (0.12040180000000071,1.8935702999999986) node {$n$};
                                \draw [fill=uuuuuu] (-1.583592135001262,3.8328157299974768) circle (1.5pt);
                                \draw[color=uuuuuu] (-1.3290572,4.405965899999998) node {$L$};
                                \draw [fill=uuuuuu] (1.583592135001262,3.8328157299974768) circle (1.5pt);
                                \draw[color=uuuuuu] (1.8275424000000016,4.405965899999998) node {$M$};
                                \draw[color=black] (1.0867078000000012,3.2463986999999985) node {$r$};
                                \draw[color=black] (-0.33054099999999953,2.3767232999999983) node {$s$};
                            \end{scriptsize}
                        \end{axis}
                    \end{tikzpicture}           
                }
            \end{center}

            In the figure above we are asked to prove that if
            the lines AM and BL are the same length, then the
            triangle is isosceles.
            \begin{mdframed}
                To begin the proof we circumscribe a circle
                with points A, B, and M defining the circle.

                \begin{center}
                    
                    \scalebox{0.5}{
                        \begin{tikzpicture}[line cap=round,line join=round,>=triangle 45,x=1cm,y=1cm]
                            \begin{axis}[
                                x=1cm,y=1cm,
                                axis lines=middle,
                                axis line style = {draw=none},
                                ymajorgrids=false,
                                xmajorgrids=false,
                                xmin=-8,
                                xmax=8,
                                ymin=-6,
                                ymax=11,
                                yticklabel = \empty,
                                xticklabel = \empty,
                                xtick={-16,-15,...,23},
                                ytick={-9,-8,...,12},]
                                \clip(-16.48988932463552,-9.548123376486267) rectangle (23.12340201686295,12.20117837284389);
                                \draw [line width=0.8pt] (-5,1)-- (0,9);
                                \draw [line width=0.8pt] (0,9)-- (5,1);
                                \draw [line width=0.8pt] (5,1)-- (-5,1);
                                \draw [line width=0.8pt] (5,1)-- (-2.4271869638936536,5.116500857770154);
                                \draw [line width=0.8pt] (-5,1)-- (2.4271869638936536,5.116500857770154);
                                \draw [line width=0.8pt] (0,0.7372545026683103) circle (5.006898760347375cm);
                                \begin{scriptsize}
                                    \draw [fill=qqqqff] (-5,1) circle (2.5pt);
                                    \draw[color=qqqqff] (-4.665026237621052,1.9177706525295326) node {$A$};
                                    \draw [fill=qqqqff] (5,1) circle (2.5pt);
                                    \draw[color=qqqqff] (5.343875732459052,1.9177706525295326) node {$B$};
                                    \draw [fill=qqqqff] (0,9) circle (2.5pt);
                                    \draw[color=qqqqff] (0.3183089204779031,9.899553236264289) node {$C$};
                                    \draw[color=black] (-1.7932737736318236,5.169608001458507) node {$l$};
                                    \draw[color=black] (2.0920383835300735,5.169608001458507) node {$m$};
                                    \draw[color=black] (0.14938230494912497,2.1711605758226997) node {$n$};
                                    \draw [fill=uuuuuu] (-2.4271869638936536,5.116500857770154) circle (1.5pt);
                                    \draw[color=uuuuuu] (-2.0888953508071855,5.845314463573619) node {$L$};
                                    \draw [fill=uuuuuu] (2.4271869638936536,5.116500857770154) circle (1.5pt);
                                    \draw[color=uuuuuu] (2.767744845645186,5.845314463573619) node {$M$};
                                    \draw[color=black] (1.7541851524725172,4.15604830828584) node {$r$};
                                    \draw[color=black] (-0.8219457343413492,2.9735619995843945) node {$s$};
                                    \draw[color=black] (-2.3422852741003526,4.662828154872174) node {$t$};
                                \end{scriptsize}
                            \end{axis}
                        \end{tikzpicture}
                    }
                \end{center}

                Lets imagine our circle intersects the line BL
                at a point between B and L.

                \begin{center}
                    \scalebox{0.5}{
                        \begin{tikzpicture}[line cap=round,line join=round,>=triangle 45,x=1cm,y=1cm]
                            \begin{axis}[
                                x=1cm,y=1cm,
                                axis lines=middle,
                                axis line style = {draw=none},
                                ymajorgrids=false,
                                xmajorgrids=false,
                                xmin=-8,
                                xmax=8,
                                ymin=-5,
                                ymax=9,
                                yticklabel = \empty,
                                xticklabel = \empty,
                                xtick={-16,-15,...,23},
                                ytick={-9,-8,...,12},]
                                \clip(-16.48988932463552,-9.548123376486268) rectangle (23.12340201686295,12.201178372843888);
                                \draw [line width=0.8pt] (-5,1)-- (-5,9);
                                \draw [line width=0.8pt] (-5,9)-- (5,1);
                                \draw [line width=0.8pt] (5,1)-- (-5,1);
                                \draw [line width=0.8pt] (5,1)-- (-5,4.507810593582122);
                                \draw [line width=0.8pt] (-5,1)-- (-0.5555555555555552,5.444444444444445);
                                \draw [line width=0.8pt] (0,0.4444444444444448) circle (5.0307695211874535cm);
                                \begin{scriptsize}
                                    \draw [fill=qqqqff] (-5,1) circle (2.5pt);
                                    \draw[color=qqqqff] (-4.665026237621052,1.9177706525295326) node {$A$};
                                    \draw [fill=qqqqff] (5,1) circle (2.5pt);
                                    \draw[color=qqqqff] (5.343875732459052,1.9177706525295326) node {$B$};
                                    \draw [fill=qqqqff] (-5,9) circle (2.5pt);
                                    \draw[color=qqqqff] (-4.665026237621052,9.899553236264289) node {$C$};
                                    \draw[color=black] (-4.158246391034718,5.5074612325160635) node {$l$};
                                    \draw[color=black] (-0.2729342338728204,4.958449732047535) node {$m$};
                                    \draw[color=black] (0.14938230494912497,2.1711605758226997) node {$n$};
                                    \draw [fill=uuuuuu] (-5,4.507810593582122) circle (1.5pt);
                                    \draw[color=uuuuuu] (-4.665026237621052,5.254071309222897) node {$L$};
                                    \draw [fill=uuuuuu] (-0.5555555555555552,5.444444444444445) circle (1.5pt);
                                    \draw[color=uuuuuu] (-0.23070257999062582,6.183167694631175) node {$M$};
                                    \draw[color=black] (0.3605405743600976,3.902658384992673) node {$r$};
                                    \draw[color=black] (-2.1311270046893798,3.269183576759756) node {$s$};
                                    \draw[color=black] (-2.3845169279825473,4.3672065776968125) node {$t$};
                                    \draw [fill=uuuuuu] (-3.5572912430182484,4.001735687462694) circle (1.5pt);
                                    \draw[color=uuuuuu] (-3.2291500056264377,4.7472914626365625) node {$O$};
                                \end{scriptsize}
                            \end{axis}
                        \end{tikzpicture}

                    }
                \end{center}
                The arc from M to B, must be greater than the arc
                from O to M. We know this to be true because the
                angles LAM and MAB are equal, and the angle
                formed by OAM is contained within the angle
                LAM. This means MAB is larger than or equal to
                OAM.

                This line of reasoning works because for inscribed
                and tangent angles, the conversion between arc length
                and angle is linear. 
                The arc O to M is equal to the arc O to A. This is
                true because the line BL which splits both bisects
                the angle ABM. This means both are equivalent arc
                length.
                Using this fact, we can conclude that the
                distance B to L is greater than A to M.
                However this contradicts the fact that
                B to L and A to M are equal by definition.

                This means the point O cannot be between
                B and L.
            \end{mdframed}
            \begin{mdframed}
                If the point O is instead further from B than
                L.
                \begin{center}
                    \scalebox{0.5}{
                        \begin{tikzpicture}[line cap=round,line join=round,>=triangle 45,x=1cm,y=1cm]
                            \begin{axis}[
                                x=1cm,y=1cm,
                                axis lines=middle,
                                axis line style = {draw=none},
                                ymajorgrids=false,
                                xmajorgrids=false,
                                xmin=-8,
                                xmax=8,
                                ymin=-6,
                                ymax=11,
                                yticklabel = \empty,
                                xticklabel = \empty,
                                xtick={-16,-15,...,23},
                                ytick={-9,-8,...,12},]
                                \clip(-16.48988932463552,-9.548123376486268) rectangle (23.12340201686295,12.201178372843888);
                                \draw [line width=0.8pt] (-5,1)-- (5,7);
                                \draw [line width=0.8pt] (5,7)-- (5,1);
                                \draw [line width=0.8pt] (5,1)-- (-5,1);
                                \draw [line width=0.8pt] (5,1)-- (1.25,4.75);
                                \draw [line width=0.8pt] (-5,1)-- (5,3.7698396494843345);
                                \begin{scriptsize}
                                    \draw [fill=qqqqff] (-5,1) circle (2.5pt);
                                    \draw[color=qqqqff] (-4.665026237621052,1.9177706525295326) node {$A$};
                                    \draw [fill=qqqqff] (5,1) circle (2.5pt);
                                    \draw[color=qqqqff] (5.343875732459052,1.9177706525295326) node {$B$};
                                    \draw [fill=qqqqff] (5,7) circle (2.5pt);
                                    \draw[color=qqqqff] (5.343875732459052,7.914665503801149) node {$C$};
                                    \draw[color=black] (0.48723553600668124,3.9448900388748673) node {$l$};
                                    \draw[color=black] (4.457011000932967,4.493901539343396) node {$m$};
                                    \draw[color=black] (0.14938230494912497,2.1711605758226997) node {$n$};
                                    \draw [fill=uuuuuu] (1.25,4.75) circle (1.5pt);
                                    \draw[color=uuuuuu] (1.585258536943739,5.5074612325160635) node {$L$};
                                    \draw [fill=uuuuuu] (5,3.7698396494843345) circle (1.5pt);
                                    \draw[color=uuuuuu] (5.343875732459052,4.493901539343396) node {$M$};
                                    \draw[color=black] (3.7390728849356605,3.8604267311104787) node {$r$};
                                    \draw[color=black] (0.3183089204779031,2.2556238835870883) node {$s$};
                                \end{scriptsize}
                            \end{axis}
                        \end{tikzpicture}

                    }
                \end{center}

                We can instead of using a circle which touches
                points A, B, and M, can instead use a circle
                that touchs A, B, and L.
                \begin{center}
                    \scalebox{0.5}{
                        \begin{tikzpicture}[line cap=round,line join=round,>=triangle 45,x=1cm,y=1cm]
                            \begin{axis}[
                                x=1cm,y=1cm,
                                axis lines=middle,
                                axis line style = {draw=none},
                                ymajorgrids=false,
                                xmajorgrids=false,
                                xmin=-8,
                                xmax=8,
                                ymin=-6,
                                ymax=11,
                                yticklabel = \empty,
                                xticklabel = \empty,
                                xtick={-16,-15,...,23},
                                ytick={-9,-8,...,12},]
                                \clip(-16.48988932463552,-9.548123376486268) rectangle (23.12340201686295,12.201178372843888);
                                \draw [line width=0.8pt] (-5,1)-- (5,7);
                                \draw [line width=0.8pt] (5,7)-- (5,1);
                                \draw [line width=0.8pt] (5,1)-- (-5,1);
                                \draw [line width=0.8pt] (5,1)-- (1.25,4.75);
                                \draw [line width=0.8pt] (-5,1)-- (5,3.7698396494843345);
                                \draw [line width=0.8pt] (0,-0.25) circle (5.153882032022076cm);
                                \begin{scriptsize}
                                    \draw [fill=qqqqff] (-5,1) circle (2.5pt);
                                    \draw[color=qqqqff] (-4.665026237621052,1.9177706525295326) node {$A$};
                                    \draw [fill=qqqqff] (5,1) circle (2.5pt);
                                    \draw[color=qqqqff] (5.343875732459052,1.9177706525295326) node {$B$};
                                    \draw [fill=qqqqff] (5,7) circle (2.5pt);
                                    \draw[color=qqqqff] (5.343875732459052,7.914665503801149) node {$C$};
                                    \draw[color=black] (0.48723553600668124,3.9448900388748673) node {$l$};
                                    \draw[color=black] (4.457011000932967,4.493901539343396) node {$m$};
                                    \draw[color=black] (0.14938230494912497,2.1711605758226997) node {$n$};
                                    \draw [fill=uuuuuu] (1.25,4.75) circle (1.5pt);
                                    \draw[color=uuuuuu] (1.585258536943739,5.5074612325160635) node {$L$};
                                    \draw [fill=uuuuuu] (5,3.7698396494843345) circle (1.5pt);
                                    \draw[color=uuuuuu] (5.343875732459052,4.493901539343396) node {$M$};
                                    \draw[color=black] (3.7390728849356605,3.8604267311104787) node {$r$};
                                    \draw[color=black] (0.3183089204779031,2.2556238835870883) node {$s$};
                                    \draw[color=black] (-2.300053620218158,3.902658384992673) node {$c_1$};
                                \end{scriptsize}
                            \end{axis}
                        \end{tikzpicture}

                    }
                \end{center}

                Now the arguments from before apply symmetrically.

            \end{mdframed}
            \begin{mdframed}
                This means that the circle must pass through
                A, B, L, and M. This implies that the arcs
                A to L, and B to M are equal, which implies the
                angles CAB and CBA are equal, which is the definition
                of an isoceles triangle.
                
                \begin{center}
                    \scalebox{0.5}{
                        \begin{tikzpicture}[line cap=round,line join=round,>=triangle 45,x=1cm,y=1cm]
                            \begin{axis}[
                                x=1cm,y=1cm,
                                axis lines=middle,
                                axis line style = {draw=none},
                                ymajorgrids=false,
                                xmajorgrids=false,
                                xmin=-8,
                                xmax=8,
                                ymin=-6,
                                ymax=11,
                                yticklabel = \empty,
                                xticklabel = \empty,
                                xtick={-16,-15,...,23},
                                ytick={-9,-8,...,12},]
                                \clip(-16.48988932463552,-9.548123376486268) rectangle (23.12340201686295,12.201178372843888);
                                \draw [line width=0.8pt] (-5,1)-- (0,8);
                                \draw [line width=0.8pt] (0,8)-- (5,1);
                                \draw [line width=0.8pt] (5,1)-- (-5,1);
                                \draw [line width=0.8pt] (5,1)-- (-2.3121639750819747,4.762970434885235);
                                \draw [line width=0.8pt] (-5,1)-- (2.3121639750819747,4.762970434885235);
                                \draw [line width=0.8pt] (0,0.2699980834847689) circle (5.053009281419926cm);
                                \begin{scriptsize}
                                    \draw [fill=qqqqff] (-5,1) circle (2.5pt);
                                    \draw[color=qqqqff] (-4.665026237621052,1.9177706525295326) node {$A$};
                                    \draw [fill=qqqqff] (5,1) circle (2.5pt);
                                    \draw[color=qqqqff] (5.343875732459052,1.9177706525295326) node {$B$};
                                    \draw [fill=qqqqff] (0,8) circle (2.5pt);
                                    \draw[color=qqqqff] (0.3183089204779031,8.928225196973816) node {$C$};
                                    \draw[color=black] (-1.7932737736318236,4.620596500989979) node {$l$};
                                    \draw[color=black] (2.0920383835300735,4.620596500989979) node {$m$};
                                    \draw[color=black] (0.14938230494912497,2.1711605758226997) node {$n$};
                                    \draw [fill=uuuuuu] (-2.3121639750819747,4.762970434885235) circle (1.5pt);
                                    \draw[color=uuuuuu] (-1.9622003891606017,5.5074612325160635) node {$L$};
                                    \draw [fill=uuuuuu] (2.3121639750819747,4.762970434885235) circle (1.5pt);
                                    \draw[color=uuuuuu] (2.6410498839986025,5.5074612325160635) node {$M$};
                                    \draw[color=black] (1.7964168063547117,3.987121692757062) node {$r$};
                                    \draw[color=black] (-0.9064090421057385,2.8046353840556164) node {$s$};
                                    \draw [fill=uuuuuu] (-2.312163975081976,4.7629704348852355) circle (1.5pt);
                                    \draw[color=uuuuuu] (-1.9622003891606017,5.5074612325160635) node {$O$};
                                    \draw[color=black] (-2.2578219663359635,4.3672065776968125) node {$c_1$};
                                \end{scriptsize}
                            \end{axis}
                        \end{tikzpicture}

                    }\\
                    {\Large Q.E.D}
                \end{center}
            \end{mdframed}

            I've added multiple pictures to this text, in addition
            I've added justification for some of the angle equivalences
            that I did not understand at first. I also added justification
            for the second case of O being outside of the triangle which
            the original proof glossed over. I believe the proof is correct
            however I am not particularly practiced when it comes to this
            style of proof so I likely would have missed any subtle errors.
        \end{mdframed}
\end{description}
\end{document}
