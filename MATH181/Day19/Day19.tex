\documentclass{report}
\usepackage[tmargin=2cm, rmargin=1in, lmargin=1in,margin=0.85in,bmargin=2cm,footskip=.2in]{geometry}
\usepackage{amsmath,amsfonts,amsthm,amssymb,mathtools}
\usepackage{enumitem}
\usepackage[]{mdframed}
\usepackage{tikz}
\renewcommand{\familydefault}{\sfdefault}

\title{\Huge{Math 181}\\Day 19 Notes}
\author{\huge{Elijah Hantman}}
\date{}

\begin{document}
\maketitle
\newpage

\begin{description}
    \item {\large Numerical Notation} 
        \begin{mdframed}
           \begin{itemize}
               \item Babylonian: Base 60, with horizontal and vertical symbols
               \item Greek Alphabetic: $\alpha, \beta, ... , \zeta$
               \item Western Arabaic: 1, 2, 3 ... 
           \end{itemize} 

           Anthropologist Steven Chrisomalis
           \begin{mdframed}
              A numerical notational system is a visual,
              long term, non-phonetic structured system
              representing numbers that involves numerical
              signs and rules for combining them into
              numeral phrases.
           \end{mdframed}

           This definition would preclude systems like tally
           marks as they have little structure, and cannot
           be combined into numerical phrases.

           This definition also excludes things like written
           English words since they are phonetic, at least
           in a loose sense.


           Of the systems we have seen so far.
           \begin{itemize}
               \item Babylonian positional: Base 60, with subbase 10 for ones place
               \item Greek Alphabetic: Base 10
               \item Western Arabaic: Base 10
           \end{itemize}
        \end{mdframed}
    \item {\large Analysing Number Systems}
        \begin{mdframed}
            \begin{itemize}
                \item Base, ie: Which groups of symbols are used
                \item Positional vs Additive
                    \begin{mdframed}
                        Greek and Roman systems are additive
                        (sort of, Roman is more complicated),
                        Modern Arabaic and Babylonian are
                        positional.

                        What does the value of a given symbol
                        depend on?

                        Can you repeat symbols?
                    \end{mdframed}
                \item Cumulative vs. Ciphered
                    \begin{mdframed}
                        Cumulative systems add or accumulate
                        the same symbol when it appears multiple
                        times (Babylonian system, Roman system).

                        Ciphered systems can only use the symbol
                        once (Greek, Western Arabaic)
                        \begin{mdframed}
                            Western Arabaic symbols each mean
                             a unique value. You cannot use
                             three of the same symbol to mean
                             three times the value.
                        \end{mdframed}
                    \end{mdframed}
            \end{itemize}
        \end{mdframed}
        \pagebreak
    \item {\large Babylonian Numerals In Practice}
        \begin{mdframed}
            Many systems used.
            \begin{itemize}
                \item Archaic systems, some used base 10,
                    others used combinations of 10 and 60.
                \item Cuneiform symbols used some additional
                    symbols.
                \item Sumerian System, Assyro-Babylonian,
                    etc. Added additional symbols for specific
                    values.
            \end{itemize}
            What is being recorded?
            \begin{itemize}
                \item Nearly impossible to know in most cases
                \item Good Archeology will give geographic
                    location, some context, approximate date,
                    approximate chronology.
                \item Bad case it was either forged or looted
                    and extracted from its original context.
            \end{itemize}
        \end{mdframed}
\end{description}


\end{document}
