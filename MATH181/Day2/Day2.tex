\documentclass{report}
\usepackage[tmargin=2cm, rmargin=1in, lmargin=1in,margin=0.85in,bmargin=2cm,footskip=.2in]{geometry}
\usepackage{amsmath,amsfonts,amsthm,amssymb,mathtools}
\usepackage{enumitem}
\usepackage[]{mdframed}
\usepackage{tikz}

\renewcommand{\familydefault}{\sfdefault}
\title{\Huge{Math 181}}
\author{\huge{Elijah Hantman}}
\date{}

\begin{document}
\maketitle
\newpage

\begin{description}
    \item Popular Math is partially about Historical Significance
        \begin{itemize}
            \item 1993 Andrew Wiles solved Fermat's Last theorem.
            \item Eureka reference to Archimedes.
                \begin{mdframed}
                    Archimedes lived on an island in the
                    mediterranian. King comissioned a crown
                    of gold, he thought that perhaps
                    the crown was adultered with lead and
                    bronze.

                    Asked Archimedes to determine whether the
                    crown was real or not. Was bathing and
                    realized that water was displace according
                    to his volume, got out of the bath and yelled
                    Eureka while running naked through the streets.
                \end{mdframed}
        \end{itemize}

    \item Timeline:
        \begin{enumerate}
            \item 1993: Andrew Wiles proves
                Fermat's Last Theorem.
                \begin{mdframed}
                    For all $a, b, c \in \mathbb{Z}^+$
                     \begin{displaymath}
                        a^n + b^n \neq c^n
                    \end{displaymath}
                    Given $n > 2$
                \end{mdframed}
            \item 1994: Wiles finds error in his proof
                \begin{mdframed}
                    Fermat's theorem was an open problem
                    for more than 350 years. Many of the
                    best names in Mathematics announced false
                    proofs.
                \end{mdframed}
            \item 1996: Wiles fixes his proof
                \begin{mdframed}
                    At the time it was believed only
                    20 people could understand his arguement.
                    Since then many have checked and become 
                    experts in this particular field.
                \end{mdframed}
        \end{enumerate}
    \item Whos is Fermat?
        \begin{itemize}
            \item Pierre De Fermat was born 1601
                in the south of France and lived
                there fore most of his life.
                \begin{mdframed}
                    Amateur mathematician, professional lawyer.
                    Was in corrospondance with the leading
                    mathematicians at the times.

                    Most of his known work was from letters and
                    corrospondance with other mathematicians.
                \end{mdframed}
                \begin{mdframed}
                    Wiles meanwhile was a professor at princeton,
                    and did mathematicians professionally.
                \end{mdframed}
            \item What did he do?
                \begin{itemize}
                    \item In 1657 he wrote a challenge to the
                        English mathematicians
                        \begin{enumerate}
                            \item John Wallis
                            \item William Brounecker
                        \end{enumerate}
                        \begin{mdframed}
                            \begin{displaymath}
                            x^3 + y^3 = z^3
                            \end{displaymath}
                            Has no positive integer
                            solutions
                        \end{mdframed}
                        \begin{mdframed}
                            In his letter he said
                            "One cube cannot be divided into two
                            cubes"
                        \end{mdframed}
                        \pagebreak
                    \item Why did he care?
                        \begin{mdframed}
                            Inspired by Greek mathematics
                            especially Arithmetica by
                            Diophantis which dates to
                            less than 300 AD.

                            Several hundred years after Plato
                            but still during a high point in
                            Greek culture.
                        \end{mdframed}
                        \begin{mdframed}
                            Diophantis was focused mostly on
                            solving polynomial equations.
                        \end{mdframed}
                        \begin{mdframed}
                            Fermat read a
                            1651 edition by Claude Gasper Bachet.
                            Diophantis' work had been rediscovered
                            around 1400.
                        \end{mdframed}
                        \begin{mdframed}
                            Most older Greek works were held in
                            libraries managed by the Catholic Church,
                            only in the 1400s were copies of older manuscripts
                            made avalible to the wealthy public. 
                        \end{mdframed}
                \end{itemize}
            \item How did we get Fermat's Last Theorem?
                \begin{mdframed}
                    Fermat's son collected and printed his father's
                    work and notes into the margins of texts like
                    Diophantis' and others.

                    \vspace{10pt}

                    Fermat's Last Theorem was posited in the margins
                    of his copy of Diophantis' book. He said he discovered
                    a proof that he didn't have space to write down.
                \end{mdframed}
                \begin{mdframed}
                    The later consensus was that Fermat was
                    incorrect about his proof, but it was
                    important to the later history and discoveries
                    of mathematics.
                \end{mdframed}
            \item Diophantis and Fermat built on math done
                on Pythagorean triples. $x^2 + y^2 = z^2$
                 \begin{mdframed}
                    Some example solutions,
                    \begin{itemize}
                        \item 3, 4, 5
                        \item 5, 12 ,13
                    \end{itemize}
                    There are infinitely many solutions
                \end{mdframed}
                \begin{mdframed}
                    Supposedly analyzed by Pythagoras, a Greek
                    Mathematician who predates Diophantis. (500BC)

                    However solutions almost certainly were not
                    originally found by Pythagoras.
                \end{mdframed}
                \begin{center}
                    {\huge "There is No More Evidence That Pythagoras
                    did all this math as Hercules defeated the Hydra"}
                \end{center}
                \begin{mdframed}
                    Our evidence of Pythagoras dates many years after
                    he would have died. This question was analyzed
                    by Greeks themselves, for example in Euclid's
                    Elements (250BC).
                \end{mdframed}
                \begin{mdframed}
                    Euclid is often called Greek, however he
                    is often refferred to as Euclid of Alexandria
                    which was a place in Egypt. Greek math is
                    not purely mediterranean but borrows from many
                    places around the world.
                \end{mdframed}
        \end{itemize}
        \pagebreak
    \item Moving Past Fermat\\
        {\large What Happened Between Fermat and Wiles?}
        \begin{itemize}
            \item Lots of Partial Results that work for some
                values of n, but not all.
                \begin{itemize}
                    \item 1816: Sophie Germain
                        \begin{mdframed}
                            Proved specific values of n,
                            and published in the Paris
                            Academy of Sciences.
                        \end{mdframed}
                    \item Mid 1800s: German Professor Ernest
                        Kummer
                        \begin{mdframed}
                            Proved Fermat's theorem with
                            the assumption that unique
                            factorization held for extensions
                            of the integers. Unfortunately this
                            property was limited to integers.

                            \vspace{10pt}

                            Also proved more cases of n.
                        \end{mdframed}
                \end{itemize}
            \item Wiles Proved the Stronger Modularity Conjecture
                of Yutaka Taniyama and Goro Shimura (from 1950s)
                \begin{mdframed}
                    Based on modular forms. Very technical and had
                    more mathematical importance.
                \end{mdframed}
                \begin{mdframed}
                    Known before Wiles to imply Fermat's last theorem.
                \end{mdframed}
                \begin{mdframed}
                    Taniyama committed suicide after the conjecture,
                    but Shimura was alive until Wiles time. Wiles
                    tried to converse with him to figure out the
                    name to go with, and he replied "The Shimura Theorem",
                    so they went with the Modularity Conjecture instead.
                \end{mdframed}
                \begin{mdframed}
                    Wiles Used techniques developed by 
                    Victor Koyvagin and Matthias Flachin in the
                    1980s.
                \end{mdframed}
                \begin{mdframed}
                    Popular stories jump from Diophantis to Fermat
                    to Wiles skipping over the rich history of
                    collaboration and teamwork in the meantime.
                \end{mdframed}
        \end{itemize}
    \item What are some Lenses we can apply to Fermat's
        Last Theorem?\\
        What Aspects of the Story could we study?
        \begin{itemize}
            \item View the historical development of
                relevant Mathematical ideas.
                \begin{mdframed}
                    A cartoon analysis would be to graph
                    exponents proved vs time.

                    We can see when Kummer or Germain
                    proved cases of n.
                \end{mdframed}
            \item We could look at it through a Biographical lens.
                \begin{mdframed}
                    The Nova documentary focused on Wiles'
                    personal story. He dedicated years of his
                    life working in a study room in his attic
                    trying to prove both Fermat's last theorem
                    and the Modularity Conjecture.

                    \vspace{10pt}

                    It put strain on his relationship and his
                    career. After 7 years he announces his result,
                    then announces its wrong, then 2 years later
                    announces the correct proof. During those
                    7 years he didn't share his work.
                \end{mdframed}
                \begin{mdframed}
                    Pierre De Fermat has a less accessible life.
                    We could still write a rough biography of his
                    life but it would be more difficult.
                \end{mdframed}
            \item History of Texts
                \begin{mdframed}
                    Interesting story of how Diophantis' work was
                    preserved until Fermat could read it. The
                    work of creating a readable and accurate
                    copy of his work was a scholarly achievement
                    in itself.
                \end{mdframed}
            \item History of Institutions
                \begin{mdframed}
                    Often mathematicians will convince themselves
                    that they've solved Fermat's last theorem
                    and get shot down.
                \end{mdframed}
                \begin{mdframed}
                    Wiles was different not only because he was
                    correct, but also because he had institutional
                    support from Princeton. He made the announcement
                    in front of Cambridge.

                    \vspace{10pt}

                    Wiles also had different circumstances,
                    he could work for 7 years without being
                    fired in an american research university.
                \end{mdframed}
                \begin{mdframed}
                    Kummer was working in a German university.\\
                    Fermat was working as a lawyer and did math
                    as a hobby.\\
                    Nobody knows how Euclid or Diophantis made
                    their living.
                \end{mdframed}
        \end{itemize}

    \item Other lenses
        \begin{description}
            \item Linguistically\\
                How math is written or notated.
            \item History of Mathematic tools\\
                Where each tool comes from?
            \item History of Geography\\
                How did geography affect culture?
            \item Calculating Devices\\
                The ease of calculations and tools has increased.
        \end{description}
\end{description}

\end{document}
