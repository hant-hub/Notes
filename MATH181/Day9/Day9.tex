\documentclass{report}
\usepackage[tmargin=2cm, rmargin=1in, lmargin=1in,margin=0.85in,bmargin=2cm,footskip=.2in]{geometry}
\usepackage{amsmath,amsfonts,amsthm,amssymb,mathtools}
\usepackage{enumitem}
\usepackage[]{mdframed}
\usepackage{tikz}
\renewcommand{\familydefault}{\sfdefault}

\title{\Huge{Math 181}}
\author{\huge{Elijah Hantman}}
\date{}

\begin{document}
\maketitle
\newpage

\begin{description}
    \item What are the Origins of Greek Theoretical Math? 
        \begin{mdframed}
            Famous Euclidian quote about the purity
            of learning mathematics for the purpose of
            knowledge. Greek mathematics was very unconcerned
            with application and focused on abstract
            gains in knowledge.

            \begin{itemize}
                \item Katz textbook has a confident guess
                    \begin{itemize}
                        \item From Thales (624-547 BC)
                            and Pythagoras (572-497 BC)
                        \item Brough from Egypt
                        \item Pythagoras was a mystic/religious
                            leader who preached "all is numbers"
                    \end{itemize}
                \item Pros:
                    \begin{itemize}
                        \item primary sources say it
                            \begin{mdframed}
                                Also comes with a great
                                story about the square
                                root of two and murder.
                            \end{mdframed}
                    \end{itemize}
                \item Cons:
                    \begin{itemize}
                        \item  sources are very late, hundreds
                            of years later and had clear 
                            agendas.
                        \item No mathematical sources before
                            400 BC, nothing from Pythagoras'
                            followers
                        \item Inconsistent with Greek culture
                            at the time.
                            \begin{mdframed}
                                One source, The mathematics
                                of Pythagoras. We don't have any
                                direct sources of Pythagoras.
                                The proofs Katz shows in his
                                textbook comes from:

                                \begin{itemize}
                                    \item Nichomachus' introduction
                                        to arithmetic
                                        \begin{mdframed}
                                            English translation
                                            avalible in a 1926
                                             translation.
                                        \end{mdframed}
                                        \begin{mdframed}
                                            Probably written
                                            only 100AD, which
                                            is hundreds of years
                                            after Pythagoras.
                                        \end{mdframed}
                                        \begin{mdframed}
                                            Nichomachus probably
                                            lived in Gerasa near
                                            the Jordanian city of
                                            Jerash
                                        \end{mdframed}
                                    \item 44 sources for Nichomachus'
                                        writing, earliest dates
                                        to 1000AD which is hundreds
                                        of years after he lived.
                                \end{itemize}
                            \end{mdframed}
                    \end{itemize}
            \end{itemize}
        \end{mdframed}
    \item Nichomachus
        \begin{mdframed}
            \begin{itemize}
                \item Even in the introduction of his
                    book he refers to Pythagoras as the
                    one of the ancients.
                \item Ancient Greeks saw mathematics
                    as encompassing several subjects
                    \begin{itemize}
                        \item Arithmetic
                        \item Geometry (focus on plane)
                        \item Music
                        \item Astronomy
                    \end{itemize}
                \item He attempts to raise mathematics and
                    philosophy above the common people,
                    people like handymen, artisans, etc.
                \item First Math Text
                    \begin{mdframed}
                        Nichomachus says that the cubes
                        can be made by summing n odd numbers.
                        In other words:

                        \begin{gather}
                           1, 8, 27, 125 
                           = 1, 3 + 5, 7 + 9 + 11, ...
                        \end{gather}

                        For a closed formula:
                        \begin{gather}
                            \sum_{i=0}^{n-1} (2(i + \frac{n(n+1)}{2})+1) = n^3 
                        \end{gather}

                        Which works for $n > 1$
                    \end{mdframed}
                    \begin{mdframed}
                        Nichomachus doesn't state where this
                        formula comes from, but he describes
                        cubes as a geometric object, and he
                        has many other formulas which he
                        proved via geometric reasoning,
                        which hints at what he did to solve.
                    \end{mdframed}
                    \begin{mdframed}
                        Looking at the trianglular numbers you
                        can form them by adding a row to the
                        previous term. You can do the same
                        thing with squares, showing that
                        the nth square is the (n-1)th square plus
                        an odd number.
                    \end{mdframed}
                    \begin{mdframed}
                        Nichomachus goes on to dicuss pentagonal
                        heptagonal, etc numbers.
                    \end{mdframed}
                    \begin{mdframed}
                        Nichomachus says that mathematics is the
                        work of Pythagoras, but that seems unlikely.
                        Modern consensus is that Pythagoras had
                        no relation to these results, but that
                        leaves the origin unknown.
                    \end{mdframed}
                    \begin{mdframed}
                        Current theory is that is comes from
                        common everyday mathematics, accounting,
                        carpentry, etc. The same kinds of people
                        Nichomachus denigrated in the introduction to
                        his book.
                    \end{mdframed}
            \end{itemize}
        \end{mdframed}
\end{description}


\end{document}
