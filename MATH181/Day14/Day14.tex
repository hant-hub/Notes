\documentclass{report}
\usepackage[tmargin=2cm, rmargin=1in, lmargin=1in,margin=0.85in,bmargin=2cm,footskip=.2in]{geometry}
\usepackage{amsmath,amsfonts,amsthm,amssymb,mathtools}
\usepackage{enumitem}
\usepackage[]{mdframed}
\usepackage{tikz}
\renewcommand{\familydefault}{\sfdefault}

\title{\Huge{Math 181}}
\author{\huge{Elijah Hantman}}
\date{}

\begin{document}
\maketitle
\newpage

\begin{description}
    \item {\large Cont.} 
        \begin{mdframed}
          Lemma 1:  
          \begin{mdframed}
              If $\frac{a}{b} = \frac{c}{d}$ then,
              $\frac{a+b}{b} = \frac{c+d}{d}$
          \end{mdframed}
          Lemma 2:
          \begin{mdframed}
              If $\frac{a}{b} = \frac{c}{d}$, then
              $\frac{a}{b} = \frac{a+c}{b+d}$ 
          \end{mdframed}

          \begin{mdframed}
              Something to note is both lemmas are usually
              considered in modern math to be simple algebraic
              manipulation, not worth noting.
          \end{mdframed}

          By hypothesis, the ratio between two consecutive
          terms of a geometric series is always the same.

          \begin{displaymath}
              \frac{a_{n+1}}{a_n} = \frac{a_{n}}{a_{n-1}} = ...
          \end{displaymath}

          By Lemma 1:

          \begin{displaymath}
              \frac{a_{n+1} - a_n}{a_n} = \frac{a_n - a_{n_1}}{a_{n-1}} = ...
          \end{displaymath}
          
          By Lemma 2:
          \begin{displaymath}
              \frac{a_2 - a_1}{a_1} = \frac{(a_{n+1} - a_n) + (a_n - a_{n-1}) + ... (a_2 - a_1)}{a_n + a_{n-1} + ... + a_2 + a_1}
          \end{displaymath}
          
          Then we simplify

          \begin{displaymath}
              \frac{a_2 - a_1}{a_1} = \frac{a_{n+1} - a_1}{a_n + a_{n-1} + ... + a_2 + a_1}
          \end{displaymath}
        \end{mdframed}
    \item {\large How did Euclid express this proof?}
        \begin{mdframed}
            Theorem: 
            \begin{mdframed}
               If as many numbers as we please be continued
               in proportion, and there be subtracted from the
               second and last numbers equal to the first,
               then, as the excess of the second is to the
               first, so will the excess of the last be to
               all those before it.
            \end{mdframed}
            Lemma 1:
            \begin{mdframed}
                If, as whole is to whole, is a number subtracted
                to a number subtracted, the remainder will also
                be to the remainder as whole is to whole.
            \end{mdframed}
            Lemma 2:
            \begin{mdframed}
                If there be as many numbers as we please
                in proportion, then as one of the antecedents
                is to one of the consequences, so are all 
                antecedents to all the consequences.

                \begin{mdframed}
                    If we have a bunch of ratios:
                    \begin{displaymath}
                        \frac{a_1}{b_1} = \frac{a_2}{b_2} = ...
                    \end{displaymath}

                    Then:

                    \begin{displaymath}
                        \frac{a_1 + a_2 + a_3 + ...}{ b_1 + b_2 + b_3 + ...}
                        = \frac{a_1}{b_1}
                    \end{displaymath}
                    
                \end{mdframed}

                Lemma 3:
                \begin{mdframed}
                    If four numbers be proportionate, they will
                    also be proportionate alternatively.

                    \begin{mdframed}
                        If $\frac{a}{b} = \frac{c}{d}$, then
                        $\frac{a}{c} = \frac{b}{d}$
                    \end{mdframed}
                \end{mdframed}

                The core concept of the proof is the same,
                however he only does $n = 3$, and he is
                unable to leverage modern algebraic manipulation.

                He must instead perform algebra geometrically.
            \end{mdframed}
            \begin{mdframed}
                The modern approach for understanding Greek
                mathematics is to combine anthropological approaches
                and mathematical approaches. The goal is to understand
                how they thought not just rewriting their results
                in terms of modern mathematics.
            \end{mdframed}
        \item {\large Archimedes}
            \begin{mdframed}
                Letter about the sand Reckoner. Most of his
                work was found in the form of letters with
                other mathematicians. The Sand Reckoner was
                written to one of the Kings of Syracuse.
                The question is framed as how many grains of
                sand can you fit in the universe, and the
                book is made to make Achimedes' mathematics
                more digestible by the non mathematician King.

                It touches astronomy and trigonometry, it also
                pushes the boundaries of Greek numerical notation.
                So Archimedes has to develop new notation to describe
                arbitarily large numbers.
            \end{mdframed}
        \end{mdframed}
\end{description}


\end{document}
