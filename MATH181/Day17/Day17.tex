\documentclass{report}
\usepackage[tmargin=2cm, rmargin=1in, lmargin=1in,margin=0.85in,bmargin=2cm,footskip=.2in]{geometry}
\usepackage{amsmath,amsfonts,amsthm,amssymb,mathtools}
\usepackage{enumitem}
\usepackage[]{mdframed}
\usepackage{tikz}
\renewcommand{\familydefault}{\sfdefault}

\title{\Huge{Math 181}\\Day 17 Notes}
\author{\huge{Elijah Hantman}}
\date{}

\begin{document}
\maketitle
\newpage

\begin{description}
    \item Writing System 
        \begin{mdframed}
            Archimedes creates a second order of numbers.

            By default you can count up to 10,000.
            You can use a number above the myriad symbol
            to indicate the number of Myriads. This caps
            out at $10^8$

            Each order indicates the number of $10^8$ sized units
            there are, and so on, maxing out at  ${10^8}^{10^8}$ 

            The order is the power of $10^8$.

            The second period is the number of ${10^8}^{10^8}$ units
            there are.

            The period is the powers of ${10^8}^{10^8}$ or $10^{8(10^8)}$

            The period maxes out at  $({10^8}^{10^8})^{10^8}$ 
            or $10^{8(10^{16})}$

        \end{mdframed}
\end{description}


\end{document}
