\documentclass{report}
\usepackage[tmargin=2cm, rmargin=1in, lmargin=1in,margin=0.85in,bmargin=2cm,footskip=.2in]{geometry}
\usepackage{amsmath,amsfonts,amsthm,amssymb,mathtools}
\usepackage{enumitem}
\usepackage[]{mdframed}
\usepackage{tikz}
\usepackage{listings}


\definecolor{codegreen}{rgb}{0,0.6,0}
\definecolor{codegray}{rgb}{0.5,0.5,0.5}
\definecolor{codepurple}{rgb}{0.58,0,0.82}
\definecolor{backcolour}{rgb}{0.95,0.95,0.92}

\lstdefinestyle{c_style}{
    language=C,
    backgroundcolor=\color{backcolour},   
    commentstyle=\color{codegreen},
    keywordstyle=\color{magenta},
    numberstyle=\tiny\color{codegray},
    stringstyle=\color{codepurple},
    basicstyle=\ttfamily\footnotesize,
    breakatwhitespace=false,         
    breaklines=true,                 
    captionpos=b,                    
    keepspaces=true,                 
    numbers=left,                    
    numbersep=5pt,                  
    showspaces=false,                
    showstringspaces=false,
    showtabs=false,                  
    tabsize=2
}

\lstdefinestyle{asm_style}{
    language=asm,
    backgroundcolor=\color{backcolour},   
    commentstyle=\color{codegreen},
    keywordstyle=\color{magenta},
    numberstyle=\tiny\color{codegray},
    stringstyle=\color{codepurple},
    basicstyle=\ttfamily\footnotesize,
    breakatwhitespace=false,         
    breaklines=true,                 
    captionpos=b,                    
    keepspaces=true,                 
    numbers=left,                    
    numbersep=5pt,                  
    showspaces=false,                
    showstringspaces=false,
    showtabs=false,                  
    tabsize=2
}




\title{\Huge{CSE 110A - Fundamentals Of Compiler Design}\\Day2}
\author{\huge{Elijah Hantman}}
\date{}

\begin{document}
\maketitle
\newpage

\begin{description}
    \item Class Format 
        \begin{itemize}
            \item 5:20 - 6:55 PM T/TH
            \item 10 mins of Q\&A after class if needed
        \end{itemize}
    \item Class Schedule
        \begin{enumerate}
            \item Announcements, homeworks, tests. Announcements sent out by email from
                Canvas
            \item Review last quiz.
            \item Review of latest material
            \item New Material
        \end{enumerate}
    \item Office Hours
        \begin{itemize}
            \item Appointment Basis
            \item Google sheet (10 min slots)
            \item Link posted on Canvas soon
            \item Don't waste slots for no reason
        \end{itemize}
    \item Class Content
        \begin{enumerate}
            \item Regular languages and Expressions
            \item Context free grammers and Parsing
                \begin{mdframed}
                    Bachus Naur Form, way to formally
                    describe context free grammer.
                \end{mdframed}
            \item Intermediate Representations
                \begin{mdframed}
                    Parse trees, converting complex expressions to
                    simple IR.
                \end{mdframed}
            \item Optimizations
                \begin{mdframed}
                    Simple optimizations, data flow optimizations,
                    etc.
                \end{mdframed}
        \end{enumerate}
        \begin{mdframed}
            Schedule on Website
        \end{mdframed}
    \item Assignments and Tests
        \begin{itemize}
            \item Assignments in Python
            \item Docker image used, don't rely too heavily on external libraries
            \item Must run on docker to be graded
            \item github classroom for automatic feedback
            \item May use around 3 pages of notes for exams
        \end{itemize}
\end{description}

\large{High Level Compiler Discussion Review}
\begin{description}
    \item What is a compiler
        \begin{mdframed}
            Input in a language\\
            Output in a different language
        \end{mdframed}
\end{description}

\large{Lexical Analysis}
\begin{itemize}
    \item Introduction
        \begin{itemize}
            \item Working with words we have various types of words
                \begin{itemize}
                    \item Nouns
                    \item Verbs
                    \item Articles
                    \item Preposition
                \end{itemize}

                Before being able to understand language we first
                need to break it apart into individual pieces which
                have specific structure and roles
        \end{itemize}
    \item Lexing
        \begin{mdframed}
            We take the string and break it into tokens.
            We replace each part of the string with a structure
            which contains more information about its meaning.

            One way this manifests is in a symbol table, we can
            keep track of the name, type, etc. of what an identifier
            means as we lex in order to generate tokens which refer to
            repeated objects.
        \end{mdframed}
    \item How can we define a language?
        \begin{itemize}
            \item ARTICLE: The, A, My, Your
            \item NOUN: Dog, Car, Computer
            \item VERB: Ran, Crashed, Accelerated
            \item ADJECTIVE: Purple, Spotted, Old
        \end{itemize}
        A sentence may be\\
        ARTICLE ADJECTIVE? NOUN VERB

        \begin{mdframed}
            We can generate valid elements of our language by substituting
            the various non terminal symbols for other symbols until we are left
            with only terminal symbols.
        \end{mdframed}
    \item How do we expand this language?
        \begin{mdframed}
           ARTICLE ADJECTIVE* NOUN VERB 
        \end{mdframed}

        The star is known as the Kleene star, and
        means repeat zero or more times. It corrosponds to this
        rule:

        \begin{mdframed}
            ADJECTIVE = NONE | ADJECTIVE ADJECTIVE
        \end{mdframed}

        As we can see we can either replace ADJECTIVE with NONE or we
        can replace it with ADJECTIVE ADJECTIVE, in addition to its other
        substitution rules.

        Stephen Cole Kleene formalized what a regular language was.

    \item Programs for Lexical Analysis
        \begin{mdframed}
            Start with definitions of tokens. What is valid for that token
            and what is invalid. In C a valid identifier can begin with a
            letter or an underscore, and it contains any number of letters,
            underscores, and numbers.
        \end{mdframed}

        \begin{mdframed}
            Soem programs can automatically generate Lexers. They are kinda
            cool but I personally am not particularly interested.
        \end{mdframed}
\end{itemize}

\end{document}

