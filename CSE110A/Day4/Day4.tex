\documentclass{report}
\usepackage[tmargin=2cm, rmargin=1in, lmargin=1in,margin=0.85in,bmargin=2cm,footskip=.2in]{geometry}
\usepackage{amsmath,amsfonts,amsthm,amssymb,mathtools}
\usepackage{enumitem}
\usepackage[]{mdframed}
\usepackage{tikz}
\usepackage{listings}


\definecolor{codegreen}{rgb}{0,0.6,0}
\definecolor{codegray}{rgb}{0.5,0.5,0.5}
\definecolor{codepurple}{rgb}{0.58,0,0.82}
\definecolor{backcolour}{rgb}{0.95,0.95,0.92}

\lstdefinestyle{c_style}{
    language=C,
    backgroundcolor=\color{backcolour},   
    commentstyle=\color{codegreen},
    keywordstyle=\color{magenta},
    numberstyle=\tiny\color{codegray},
    stringstyle=\color{codepurple},
    basicstyle=\ttfamily\footnotesize,
    breakatwhitespace=false,         
    breaklines=true,                 
    captionpos=b,                    
    keepspaces=true,                 
    numbers=left,                    
    numbersep=5pt,                  
    showspaces=false,                
    showstringspaces=false,
    showtabs=false,                  
    tabsize=2
}

\lstdefinestyle{asm_style}{
    language=asm,
    backgroundcolor=\color{backcolour},   
    commentstyle=\color{codegreen},
    keywordstyle=\color{magenta},
    numberstyle=\tiny\color{codegray},
    stringstyle=\color{codepurple},
    basicstyle=\ttfamily\footnotesize,
    breakatwhitespace=false,         
    breaklines=true,                 
    captionpos=b,                    
    keepspaces=true,                 
    numbers=left,                    
    numbersep=5pt,                  
    showspaces=false,                
    showstringspaces=false,
    showtabs=false,                  
    tabsize=2
}




\title{\Huge{CSE 110A - Fundamentals Of Compiler Design}}
\author{\huge{Elijah Hantman}}
\date{}

\begin{document}
\maketitle
\newpage

\begin{description}
    \item Naive Scanner Tokenizer
        \begin{itemize}
            \item Simple
            \item Common for early languages
            \item Can be Unwieldy for many types of tokens 
            \item Can be heavily nested
        \end{itemize}
    \item Regex
        \begin{itemize}
            \item Tokens are restricted to Regular languages 
            \item Extensible
            \item Modular
            \item Easy to add tokens
            \item Many regex implementations have issues with
                order dependency
            \item Not as powerful as context free grammers
        \end{itemize}
    \item Side Note on Chompsky Hierarchy\\
        From least to most powerful
        \begin{itemize}
            \item Regular
            \item Context free
            \item Context Sensitive
            \item Recursively Enumerable
        \end{itemize}
    \item What is Regex?
        \begin{itemize}
            \item Python Regex
                \begin{itemize}
                    \item Library re
                    \item re.fullmatch(pattern, string)\\
                        Only succeeds if entire string
                        matches the pattern
                \end{itemize}
            \item Recursive definition
                \begin{mdframed}
                    All characters are regular expressions.

                    These single character expressions only match
                    on strings which are equal to themselves.

                    If we write two regular expressions in sequence
                    they are concatenated. Which means that they
                    only match a string where some part of the
                    string matches the first regex, and the rest of
                    the string matches the second regex.

                    We have modifiers which modify how the regular expression
                    to the left matches.

                    The modifiers are as follows:
                    \begin{itemize}
                        \item *
                            \begin{mdframed}
                                The Kleene star, matches zero or more copies
                                of the expression to the left, equivalent
                                to zero or more copies of the expression
                                concatenated together.
                            \end{mdframed}
                        \item +
                            \begin{mdframed}
                                The plus matches one or more copies of
                                the expression to the left. Equal to one
                                or more copies concatenated together.
                            \end{mdframed}
                    \end{itemize}

                    There is one basic operator, choice (|).

                    What choice means is that the string matches if it matches the
                    expressin on the left, or the expression on the right.


                    There are rules of operator precedence, modifiers come before
                    choice, and concatenation comes before modifiers.

                    Concatenation can be specified via parentheses. These form groups.
                    Each group is a valid regular expression. Parentheses define a single
                    regular expression as a group which ignores precedence rules.
                    

                    As an extension, the square brackets mean match against any character inside
                    the bracket. [ab] is equivalent to a|b.

                    The \^{} carrot operator means the complement, so [\^{}ab] can match any character but
                    a or b.

                    Optional Modifier, equivalent to choice between regex with character and regex without
                    character. Same precedence as other modifiers. This is used to make regex non-greedy,
                    since it will always put the version without the string first.

                    The . character is used to denote matching any character.

                    Many languages also use '()' to indicate a capture group. This is an extension for
                    regex matchers that says to extract strings in groups and return them to the user.

                    When the \^{} operator is placed outside square brackets, it matches the NULL at
                    the start of the string. In Multiline mode, regex engines will match the beginning
                    of the line.

                    The \$ operator matches the NULL at the end of the string. In Multiline mode,
                    it will match the end of the line.

                \end{mdframed}
        \end{itemize}
    \item Token Actions
        \begin{itemize}
            \item Replacement
            \item Keywords
            \item Error reporting
            \item Symbol table Creation
        \end{itemize}
        \begin{mdframed}
            Actions are functions which run on and modify
            the token before it is returned. You can use
            them for keywords or marking them with additional
            metadata.
        \end{mdframed}
    \item First Class Functions
        \begin{itemize}
            \item Python has functional programming
            \item You can make them, pass them around, do whatever you want
            \item Idk if function pointers really count, like they are not the same.
        \end{itemize}
\end{description}

\end{document}

