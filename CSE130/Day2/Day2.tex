\documentclass{report}
\usepackage[tmargin=2cm, rmargin=1in, lmargin=1in,margin=0.85in,bmargin=2cm,footskip=.2in]{geometry}
\usepackage{amsmath,amsfonts,amsthm,amssymb,mathtools}
\usepackage{enumitem}
\usepackage[]{mdframed}
\usepackage{tikz}
\usepackage{listings}


\definecolor{codegreen}{rgb}{0,0.6,0}
\definecolor{codegray}{rgb}{0.5,0.5,0.5}
\definecolor{codepurple}{rgb}{0.58,0,0.82}
\definecolor{backcolour}{rgb}{0.95,0.95,0.92}

\lstdefinestyle{c_style}{
    language=C,
    backgroundcolor=\color{backcolour},   
    commentstyle=\color{codegreen},
    keywordstyle=\color{magenta},
    numberstyle=\tiny\color{codegray},
    stringstyle=\color{codepurple},
    basicstyle=\ttfamily\footnotesize,
    breakatwhitespace=false,         
    breaklines=true,                 
    captionpos=b,                    
    keepspaces=true,                 
    numbers=left,                    
    numbersep=5pt,                  
    showspaces=false,                
    showstringspaces=false,
    showtabs=false,                  
    tabsize=2
}

\lstdefinestyle{asm_style}{
    language=asm,
    backgroundcolor=\color{backcolour},   
    commentstyle=\color{codegreen},
    keywordstyle=\color{magenta},
    numberstyle=\tiny\color{codegray},
    stringstyle=\color{codepurple},
    basicstyle=\ttfamily\footnotesize,
    breakatwhitespace=false,         
    breaklines=true,                 
    captionpos=b,                    
    keepspaces=true,                 
    numbers=left,                    
    numbersep=5pt,                  
    showspaces=false,                
    showstringspaces=false,
    showtabs=false,                  
    tabsize=2
}




\title{\Huge{CSE 130}\\\huge{Principles of Computer System Design}}
\author{\huge{Elijah Hantman}}
\date{}

\begin{document}
\maketitle
\newpage

\large{Systems, Complexity and Fundamental Abstractions}
\begin{description}
    \item What is a System?
        \begin{itemize}
            \item Set of interconnected components
            \item What is a component depends on the problem and scope
            \item For hardware designers each individual device could be a component
            \item For CPU designers individual transistors may be component
            \item For Servers different stages of processing a request may be a component
        \end{itemize}
    \item Complexity
        \begin{itemize}
            \item Large systems with many components can make systems
                hard to understand
            \item No unified measure of complexity
            \item Complexity puts limits on what you can build and how
            \item Important to understand what is complex and how to
                mitigate or remove complexity
        \end{itemize}
    \item Common Issues:
        \begin{itemize}
            \item Emergent Properties
                \begin{itemize}
                    \item Properties which arise from the confluence of multiple components
                    \item Example: Millenium Bridge in London which ended up being extordinarily
                        unstable due to the fact that people walking on the bridge were pressured
                        to walk in sync which generated harmonic motion.
                    \item Example: Conway's Game of Life, simple rules give rise to a Turing complete
                        system which can compute any problem and simulate all kinds of unique and
                        new patterns
                \end{itemize}
            \item Propogation of Effects
                \begin{itemize}
                    \item Problems in one component can affect other components
                    \item Example: Electric companies, by connecting their systems they can
                        load balance and hopefully reduce failures. However it also means that
                        if one system is brought down by a failure all the systems go down
                    \item Example: Networking, if a network connection fails or a component
                        goes down, it can force changes in other components to deal with
                        the failure to send or recieve data from that particular host
                \end{itemize}
            \item Incommensurate Scaling
                \begin{itemize}
                    \item Different components in a system scale in different ways as the system
                        scales up.
                    \item Example: As a database grows the time to query data may increase faster
                        than the size leading to performance issues.
                    \item Example: As CPUs grew faster, memory grew faster slower, leading to
                        massive differences in computer speed vs. memory bandwidth.
                    \item Common in many systems especially among heterogenous components since
                        different types of work fundamentally have different costs.
                \end{itemize}
            \item Trade-Offs
                \begin{itemize}
                    \item Goals may conflict
                    \item Example: Bigger displays on phones look better but may cost more battery
                    \item Example: Merge Sort scales well, but a recursive implementation imposes a
                        burden on memory
                    \item Example: Insertion sort scales poorly, but has very small overhead making
                        it faster for small amounts of data.
                    \item Example: Spam filters have to balance rejecting spam emails and
                        accepting non spam emails, it can be very bad to reject too many
                        emails but the spam filters are pointless if they don't reject enough
                        emails
                \end{itemize}
        \end{itemize}
    \item Signs of Complexity
        \begin{itemize}
            \item No quantitative measure
            \item Irregular
        \end{itemize}
    \item Large Number of Components
        \begin{itemize}
            \item Size affects comprehensibility
            \item It is an indicator but not a direct
                measure
            \item Depends largely on how components are connected
        \end{itemize}
    \item Large Number of Interconnections
        \begin{itemize}
            \item Each interconnection places a burden on the person
                trying to understand the system
        \end{itemize}
    \item Many irregularities
        \begin{itemize}
            \item If there are consistency and patterns in the interconnections
                and components it can be easy to conceptualize
            \item The more exceptions the more likely the system is difficult
                to follow
            \item Each exception takes up some mental bandwidth
        \end{itemize}
    \item Long Descriptions
        \begin{itemize}
            \item If the best description requires many properties it may indicate
                complexity
            \item Kolmogorov Complexity - length of the shortest specification
            \item May be a reflection of other signs
        \end{itemize}
    \item Large Team of People
        \begin{itemize}
            \item The more people the more likely the system will be complex
            \item The boudaries between people are reflected in the product,
                leading to more components and more complexity
        \end{itemize}
    \item Sources of Complexity
        \begin{itemize}
            \item Cascading and interacting list of requirements
                \begin{itemize}
                    \item Each requirement may be minimal
                    \item The Problem is that each requirement may interact with some
                        number of previous requirements causing exponential growth
                        in the possible interactions
                    \item Example: Phones
                        \begin{itemize}
                            \item corded phones have a single purpose
                            \item flip phones have basic apps
                            \item smart phones are full computers and can do hundreds of tasks
                        \end{itemize}
                    \item Must avoid excessive Generality, systems that can do anything are not
                        particularly useful for anything. The more general the more complicated
                        and more difficult it is to create a good tool.
                \end{itemize}
            \item Maintaining high utilization
                \begin{itemize}
                    \item If we desire high utilization we need more complexity
                    \item Law of Diminishing Returns: the closer we get to some maximum
                        the more effort it requires
                    \item Utilization is less percentage and more odds, if we double
                        the amount of complexity and utlization, it halves the idle
                        time. This means that by doubling the complexity we get
                        highly diminishing returns.
                \end{itemize}
        \end{itemize}
    \item How to mitigate Complexity
        \begin{itemize}
            \item Modularity
                \begin{itemize}
                    \item Divide and conquer
                    \item design the system as a small number of interacting subsystems,
                        reduces the amount of the system under consideration at any
                        one point
                    \item Reduces the global problem to small set of interactions,
                        allows ignoring global problem when solving sub problems
                        (at least to some extent)
                \end{itemize}
            \item Abstraction
                \begin{itemize}
                    \item abstraction is the separation of interface from internals
                    \item It is the finding of logical boundaries for modularity
                    \item Abstraction is the process of modularizing and then
                        chunking components
                    \item About the property of modules being able to only
                        consider the interface of a component rather than
                        considering the internal implementations
                    \item Abstraction is a tool to reduce \textit{granularity}
                \end{itemize}
            \item Layering
                \begin{itemize}
                    \item One way of limiting or controlling how modules interact
                    \item Layered systems have modules which only communication with
                        layers directly below or above them
                    \item Another way of creating lower granularity interfaces
                    \item Most programs use this to some degree to abstract
                        APIs or tasks which have high numbers of repeated tasks
                \end{itemize}
            \item Hierarchy
                \begin{itemize}
                    \item This is another way of organizing modules
                    \item Modules only interact in their subsystems
                    \item Subsystems only have a single module as their interface
                    \item The goal is to reduce the number of modules while allowing
                        for more complex interconnections within subsystems
                    \item Limits the visibility of each level of abstraction to limit
                        complexity (aids chunking)
                \end{itemize}
        \end{itemize}
    \item Names Make Connections
        \begin{itemize}
            \item The four techniques decompose systems into smaller systems
            \item Naming modules allows us to begin recomposing small subsystems back
                into the larger system
            \item Naming allows for
                \begin{itemize}
                    \item postponing decisions (allowing for better decisions with more information)
                    \item Easy replacement of modules (pointing modules to different places)
                    \item Sharing of modules (code reuse, maybe even process reuse)
                \end{itemize}
        \end{itemize}
    \item Unique Problem of Computer Systems
        \begin{itemize}
            \item The complexity is not limited by physical constraints
            \item Rate of technology change is unprecedented
            \item High hardware limits allow for extremely unprecedented complexity
                in systems, especially in systems which are made to take advantage
                of various types of scaling.
        \end{itemize}
    \item Coping with Complexity
        \begin{itemize}
            \item Deal with complexity via iteration
            \item Keep it simple
            \item By rapidly iterating we learn where the logical boundaries
                are, and how to best deliver constraints
            \item By keeping things simple we only add the minimum
                amount of abstraction, modularity, hierarchy, or layering, since
                each additional abstraction adds a constant amount of complexity.
            \item By getting feedback we can better define the requirements for systems
                in a sensible and practical manner, the goal is to build something reliable
                and useful, not to make something abstractly beautiful.
        \end{itemize}
    \item Rest of the Course
        \begin{itemize}
            \item How to use complexity management principles
                \begin{itemize}
                    \item Modularity
                    \item Abstractions
                    \item Client/Server and Virtualization
                \end{itemize}
        \end{itemize}
    \item Fundamental Abstractions
        \begin{enumerate}
            \item Memory (objects by name)
                \begin{itemize}
                    \item Storage
                    \item Wide ranging technologies
                        \begin{itemize}
                            \item Volitile vs non-Volitile
                            \item Latency vs cost
                        \end{itemize}
                    \item All memory has two fundamental operations
                        \begin{itemize}
                            \item Write(name, value)
                            \item Read(name) $\to$ value
                        \end{itemize}
                    \item Memory devices read and write contiguous bits
                    \item Two Useful properties
                        \begin{itemize}
                            \item Read/write coherence, read always returns the most recent write
                            \item Before or after atomicity, all read or writes behave as if they
                                occured in a single linear order, ie: reads and writes have a cannonical
                                order
                        \end{itemize}
                    \item Problems
                        \begin{itemize}
                            \item Concurrency (non-linear or undefined access order)
                            \item Remote Storage (amplified delays in Reads/Writes)
                            \item Replicated Storage (to improve reliability at cost of synchronization)
                        \end{itemize}
                \end{itemize}
            \item Interpreter (manipulate names)
                \begin{itemize}
                    \item Active elements of computer, the things that do the computations
                    \item Wide-ranging and layered
                        \begin{itemize}
                            \item Example: Human generates requests to calendar, Java interpreter,
                                 byte-code interpreter, hardware interpreter
                        \end{itemize}
                    \item Simple abstraction
                        \begin{itemize}
                            \item Instruction reference, where is the next instruction
                            \item reprotoire, what actions are avalible
                            \item Environment, what is the current state?
                        \end{itemize}
                    \item Many systems have multiple interpreters
                        \begin{itemize}
                            \item usually asynchronous
                            \item different progression rates
                            \item assume no control over exact order of execution
                            \item design alogithms to explicitly control timing and order
                        \end{itemize}
                \end{itemize}
            \item Communication (move data to and from names)
                \begin{itemize}
                    \item Allows for data movement between physically separate components
                    \item Fundamental interface
                        \begin{itemize}
                            \item SEND(name, msg)
                            \item Recieve(name) $\to$ msg
                        \end{itemize}
                    \item Properties
                        \begin{itemize}
                            \item Unpredictable time to complete SEND and RECIEVE
                            \item Hostile environment leads to data corruption
                            \item Data loss
                            \item Asynchronous timing, no garuntee of order
                        \end{itemize}
                \end{itemize}
        \end{enumerate}
\end{description}



\end{document}

