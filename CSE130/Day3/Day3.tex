\documentclass{report}
\usepackage[tmargin=2cm, rmargin=1in, lmargin=1in,margin=0.85in,bmargin=2cm,footskip=.2in]{geometry}
\usepackage{amsmath,amsfonts,amsthm,amssymb,mathtools}
\usepackage{enumitem}
\usepackage[]{mdframed}
\usepackage{tikz}
\usepackage{listings}


\definecolor{codegreen}{rgb}{0,0.6,0}
\definecolor{codegray}{rgb}{0.5,0.5,0.5}
\definecolor{codepurple}{rgb}{0.58,0,0.82}
\definecolor{backcolour}{rgb}{0.95,0.95,0.92}

\lstdefinestyle{c_style}{
    language=C,
    backgroundcolor=\color{backcolour},   
    commentstyle=\color{codegreen},
    keywordstyle=\color{magenta},
    numberstyle=\tiny\color{codegray},
    stringstyle=\color{codepurple},
    basicstyle=\ttfamily\footnotesize,
    breakatwhitespace=false,         
    breaklines=true,                 
    captionpos=b,                    
    keepspaces=true,                 
    numbers=left,                    
    numbersep=5pt,                  
    showspaces=false,                
    showstringspaces=false,
    showtabs=false,                  
    tabsize=2
}

\lstdefinestyle{asm_style}{
    language=asm,
    backgroundcolor=\color{backcolour},   
    commentstyle=\color{codegreen},
    keywordstyle=\color{magenta},
    numberstyle=\tiny\color{codegray},
    stringstyle=\color{codepurple},
    basicstyle=\ttfamily\footnotesize,
    breakatwhitespace=false,         
    breaklines=true,                 
    captionpos=b,                    
    keepspaces=true,                 
    numbers=left,                    
    numbersep=5pt,                  
    showspaces=false,                
    showstringspaces=false,
    showtabs=false,                  
    tabsize=2
}




\title{\Huge{CSE 130}\\\huge{Principles of Computer System Design}}
\author{\huge{Elijah Hantman}}
\date{}

\begin{document}
\maketitle
\newpage

\large{Intro To Processes}
\begin{description}
    \item Modularity, Virtualization, etc.
    \item What is a Process
        \begin{mdframed}
            An instance of a program to be executed.
            Refers to the program being run, the register and variable
            state, the I/O it has access to, the address space
            which it owns, etc.
        \end{mdframed}
    \item OS view
        \begin{itemize}
            \item Process table, each entry is a process and contains a PCB
            \item OS scheduler picks processes to be executed at any given time
            \item When a process is suspended it's state is saved in the PCB
            \item Its memory is not touched, it is spatially concurrent.
                \begin{mdframed}
                    This is possible via virtualization. No process can directly
                    access physical memory, instead each memory access goes through
                    a memory mapper which converts the address into a physical address
                    via a page table.

                    The page table data for each process is stored in the PCB and
                    is controlled by the OS.
                \end{mdframed}
                \begin{mdframed}
                    Old way of doing this before OS support for memory virtualization
                    was to use PIE executables which are address agnostic. PIE is still
                    used for DLLs or code which may be loaded in at runtime, since there
                    is no way to garuntee where these blocks of code where be loaded in.
                \end{mdframed}
        \end{itemize}
    \item User View
        \begin{itemize}
            \item Process Creation
                \begin{itemize}
                    \item System initialization
                    \item User request via shell or in user space program
                    \item shell scripts
                \end{itemize}
            \item Process Termination
                \begin{itemize}
                    \item Program exit
                    \item Error Exit
                    \item Fatal error ex: segfault, sigkill, etc.
                    \item Killed by user command: SIGQuit, SigInt, etc.
                \end{itemize}
        \end{itemize}
    \item Programmer View
        \begin{itemize}
            \item Processes start as clones
            \item Processes can load and switch to different program
            \item Child processes can create their own child processes
            \item Eventually process terminates, calls exit or abort, or is killed by the OS
        \end{itemize}
        
\end{description}


\end{document}

